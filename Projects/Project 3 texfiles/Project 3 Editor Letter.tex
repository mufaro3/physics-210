\documentclass[11pt]{article}

    \usepackage[breakable]{tcolorbox}
    \usepackage{parskip} % Stop auto-indenting (to mimic markdown behaviour)
    

    % Basic figure setup, for now with no caption control since it's done
    % automatically by Pandoc (which extracts ![](path) syntax from Markdown).
    \usepackage{graphicx}
    % Maintain compatibility with old templates. Remove in nbconvert 6.0
    \let\Oldincludegraphics\includegraphics
    % Ensure that by default, figures have no caption (until we provide a
    % proper Figure object with a Caption API and a way to capture that
    % in the conversion process - todo).
    \usepackage{caption}
    \DeclareCaptionFormat{nocaption}{}
    \captionsetup{format=nocaption,aboveskip=0pt,belowskip=0pt}

    \usepackage{float}
    \floatplacement{figure}{H} % forces figures to be placed at the correct location
    \usepackage{xcolor} % Allow colors to be defined
    \usepackage{enumerate} % Needed for markdown enumerations to work
    \usepackage{geometry} % Used to adjust the document margins
    \usepackage{amsmath} % Equations
    \usepackage{amssymb} % Equations
    \usepackage{textcomp} % defines textquotesingle
    % Hack from http://tex.stackexchange.com/a/47451/13684:
    \AtBeginDocument{%
        \def\PYZsq{\textquotesingle}% Upright quotes in Pygmentized code
    }
    \usepackage{upquote} % Upright quotes for verbatim code
    \usepackage{eurosym} % defines \euro

    \usepackage{iftex}
    \ifPDFTeX
        \usepackage[T1]{fontenc}
        \IfFileExists{alphabeta.sty}{
              \usepackage{alphabeta}
          }{
              \usepackage[mathletters]{ucs}
              \usepackage[utf8x]{inputenc}
          }
    \else
        \usepackage{fontspec}
        \usepackage{unicode-math}
    \fi

    \usepackage{fancyvrb} % verbatim replacement that allows latex
    \usepackage[Export]{adjustbox} % Used to constrain images to a maximum size
    \adjustboxset{max size={0.9\linewidth}{0.9\paperheight}}

    % The hyperref package gives us a pdf with properly built
    % internal navigation ('pdf bookmarks' for the table of contents,
    % internal cross-reference links, web links for URLs, etc.)
    \usepackage{hyperref}
    % The default LaTeX title has an obnoxious amount of whitespace. By default,
    % titling removes some of it. It also provides customization options.
    \usepackage{titling}
    \usepackage{longtable} % longtable support required by pandoc >1.10
    \usepackage{booktabs}  % table support for pandoc > 1.12.2
    \usepackage{array}     % table support for pandoc >= 2.11.3
    \usepackage{calc}      % table minipage width calculation for pandoc >= 2.11.1
    \usepackage[inline]{enumitem} % IRkernel/repr support (it uses the enumerate* environment)
    \usepackage[normalem]{ulem} % ulem is needed to support strikethroughs (\sout)
                                % normalem makes italics be italics, not underlines
    \usepackage{mathrsfs}
    

    
    % Colors for the hyperref package
    \definecolor{urlcolor}{rgb}{0,.145,.698}
    \definecolor{linkcolor}{rgb}{.71,0.21,0.01}
    \definecolor{citecolor}{rgb}{.12,.54,.11}

    % ANSI colors
    \definecolor{ansi-black}{HTML}{3E424D}
    \definecolor{ansi-black-intense}{HTML}{282C36}
    \definecolor{ansi-red}{HTML}{E75C58}
    \definecolor{ansi-red-intense}{HTML}{B22B31}
    \definecolor{ansi-green}{HTML}{00A250}
    \definecolor{ansi-green-intense}{HTML}{007427}
    \definecolor{ansi-yellow}{HTML}{DDB62B}
    \definecolor{ansi-yellow-intense}{HTML}{B27D12}
    \definecolor{ansi-blue}{HTML}{208FFB}
    \definecolor{ansi-blue-intense}{HTML}{0065CA}
    \definecolor{ansi-magenta}{HTML}{D160C4}
    \definecolor{ansi-magenta-intense}{HTML}{A03196}
    \definecolor{ansi-cyan}{HTML}{60C6C8}
    \definecolor{ansi-cyan-intense}{HTML}{258F8F}
    \definecolor{ansi-white}{HTML}{C5C1B4}
    \definecolor{ansi-white-intense}{HTML}{A1A6B2}
    \definecolor{ansi-default-inverse-fg}{HTML}{FFFFFF}
    \definecolor{ansi-default-inverse-bg}{HTML}{000000}

    % common color for the border for error outputs.
    \definecolor{outerrorbackground}{HTML}{FFDFDF}

    % commands and environments needed by pandoc snippets
    % extracted from the output of `pandoc -s`
    \providecommand{\tightlist}{%
      \setlength{\itemsep}{0pt}\setlength{\parskip}{0pt}}
    \DefineVerbatimEnvironment{Highlighting}{Verbatim}{commandchars=\\\{\}}
    % Add ',fontsize=\small' for more characters per line
    \newenvironment{Shaded}{}{}
    \newcommand{\KeywordTok}[1]{\textcolor[rgb]{0.00,0.44,0.13}{\textbf{{#1}}}}
    \newcommand{\DataTypeTok}[1]{\textcolor[rgb]{0.56,0.13,0.00}{{#1}}}
    \newcommand{\DecValTok}[1]{\textcolor[rgb]{0.25,0.63,0.44}{{#1}}}
    \newcommand{\BaseNTok}[1]{\textcolor[rgb]{0.25,0.63,0.44}{{#1}}}
    \newcommand{\FloatTok}[1]{\textcolor[rgb]{0.25,0.63,0.44}{{#1}}}
    \newcommand{\CharTok}[1]{\textcolor[rgb]{0.25,0.44,0.63}{{#1}}}
    \newcommand{\StringTok}[1]{\textcolor[rgb]{0.25,0.44,0.63}{{#1}}}
    \newcommand{\CommentTok}[1]{\textcolor[rgb]{0.38,0.63,0.69}{\textit{{#1}}}}
    \newcommand{\OtherTok}[1]{\textcolor[rgb]{0.00,0.44,0.13}{{#1}}}
    \newcommand{\AlertTok}[1]{\textcolor[rgb]{1.00,0.00,0.00}{\textbf{{#1}}}}
    \newcommand{\FunctionTok}[1]{\textcolor[rgb]{0.02,0.16,0.49}{{#1}}}
    \newcommand{\RegionMarkerTok}[1]{{#1}}
    \newcommand{\ErrorTok}[1]{\textcolor[rgb]{1.00,0.00,0.00}{\textbf{{#1}}}}
    \newcommand{\NormalTok}[1]{{#1}}

    % Additional commands for more recent versions of Pandoc
    \newcommand{\ConstantTok}[1]{\textcolor[rgb]{0.53,0.00,0.00}{{#1}}}
    \newcommand{\SpecialCharTok}[1]{\textcolor[rgb]{0.25,0.44,0.63}{{#1}}}
    \newcommand{\VerbatimStringTok}[1]{\textcolor[rgb]{0.25,0.44,0.63}{{#1}}}
    \newcommand{\SpecialStringTok}[1]{\textcolor[rgb]{0.73,0.40,0.53}{{#1}}}
    \newcommand{\ImportTok}[1]{{#1}}
    \newcommand{\DocumentationTok}[1]{\textcolor[rgb]{0.73,0.13,0.13}{\textit{{#1}}}}
    \newcommand{\AnnotationTok}[1]{\textcolor[rgb]{0.38,0.63,0.69}{\textbf{\textit{{#1}}}}}
    \newcommand{\CommentVarTok}[1]{\textcolor[rgb]{0.38,0.63,0.69}{\textbf{\textit{{#1}}}}}
    \newcommand{\VariableTok}[1]{\textcolor[rgb]{0.10,0.09,0.49}{{#1}}}
    \newcommand{\ControlFlowTok}[1]{\textcolor[rgb]{0.00,0.44,0.13}{\textbf{{#1}}}}
    \newcommand{\OperatorTok}[1]{\textcolor[rgb]{0.40,0.40,0.40}{{#1}}}
    \newcommand{\BuiltInTok}[1]{{#1}}
    \newcommand{\ExtensionTok}[1]{{#1}}
    \newcommand{\PreprocessorTok}[1]{\textcolor[rgb]{0.74,0.48,0.00}{{#1}}}
    \newcommand{\AttributeTok}[1]{\textcolor[rgb]{0.49,0.56,0.16}{{#1}}}
    \newcommand{\InformationTok}[1]{\textcolor[rgb]{0.38,0.63,0.69}{\textbf{\textit{{#1}}}}}
    \newcommand{\WarningTok}[1]{\textcolor[rgb]{0.38,0.63,0.69}{\textbf{\textit{{#1}}}}}


    % Define a nice break command that doesn't care if a line doesn't already
    % exist.
    \def\br{\hspace*{\fill} \\* }
    % Math Jax compatibility definitions
    \def\gt{>}
    \def\lt{<}
    \let\Oldtex\TeX
    \let\Oldlatex\LaTeX
    \renewcommand{\TeX}{\textrm{\Oldtex}}
    \renewcommand{\LaTeX}{\textrm{\Oldlatex}}
    % Document parameters
    % Document title
    \title{Project 3 Editor Letter}
    
    
    
    
    
% Pygments definitions
\makeatletter
\def\PY@reset{\let\PY@it=\relax \let\PY@bf=\relax%
    \let\PY@ul=\relax \let\PY@tc=\relax%
    \let\PY@bc=\relax \let\PY@ff=\relax}
\def\PY@tok#1{\csname PY@tok@#1\endcsname}
\def\PY@toks#1+{\ifx\relax#1\empty\else%
    \PY@tok{#1}\expandafter\PY@toks\fi}
\def\PY@do#1{\PY@bc{\PY@tc{\PY@ul{%
    \PY@it{\PY@bf{\PY@ff{#1}}}}}}}
\def\PY#1#2{\PY@reset\PY@toks#1+\relax+\PY@do{#2}}

\@namedef{PY@tok@w}{\def\PY@tc##1{\textcolor[rgb]{0.73,0.73,0.73}{##1}}}
\@namedef{PY@tok@c}{\let\PY@it=\textit\def\PY@tc##1{\textcolor[rgb]{0.24,0.48,0.48}{##1}}}
\@namedef{PY@tok@cp}{\def\PY@tc##1{\textcolor[rgb]{0.61,0.40,0.00}{##1}}}
\@namedef{PY@tok@k}{\let\PY@bf=\textbf\def\PY@tc##1{\textcolor[rgb]{0.00,0.50,0.00}{##1}}}
\@namedef{PY@tok@kp}{\def\PY@tc##1{\textcolor[rgb]{0.00,0.50,0.00}{##1}}}
\@namedef{PY@tok@kt}{\def\PY@tc##1{\textcolor[rgb]{0.69,0.00,0.25}{##1}}}
\@namedef{PY@tok@o}{\def\PY@tc##1{\textcolor[rgb]{0.40,0.40,0.40}{##1}}}
\@namedef{PY@tok@ow}{\let\PY@bf=\textbf\def\PY@tc##1{\textcolor[rgb]{0.67,0.13,1.00}{##1}}}
\@namedef{PY@tok@nb}{\def\PY@tc##1{\textcolor[rgb]{0.00,0.50,0.00}{##1}}}
\@namedef{PY@tok@nf}{\def\PY@tc##1{\textcolor[rgb]{0.00,0.00,1.00}{##1}}}
\@namedef{PY@tok@nc}{\let\PY@bf=\textbf\def\PY@tc##1{\textcolor[rgb]{0.00,0.00,1.00}{##1}}}
\@namedef{PY@tok@nn}{\let\PY@bf=\textbf\def\PY@tc##1{\textcolor[rgb]{0.00,0.00,1.00}{##1}}}
\@namedef{PY@tok@ne}{\let\PY@bf=\textbf\def\PY@tc##1{\textcolor[rgb]{0.80,0.25,0.22}{##1}}}
\@namedef{PY@tok@nv}{\def\PY@tc##1{\textcolor[rgb]{0.10,0.09,0.49}{##1}}}
\@namedef{PY@tok@no}{\def\PY@tc##1{\textcolor[rgb]{0.53,0.00,0.00}{##1}}}
\@namedef{PY@tok@nl}{\def\PY@tc##1{\textcolor[rgb]{0.46,0.46,0.00}{##1}}}
\@namedef{PY@tok@ni}{\let\PY@bf=\textbf\def\PY@tc##1{\textcolor[rgb]{0.44,0.44,0.44}{##1}}}
\@namedef{PY@tok@na}{\def\PY@tc##1{\textcolor[rgb]{0.41,0.47,0.13}{##1}}}
\@namedef{PY@tok@nt}{\let\PY@bf=\textbf\def\PY@tc##1{\textcolor[rgb]{0.00,0.50,0.00}{##1}}}
\@namedef{PY@tok@nd}{\def\PY@tc##1{\textcolor[rgb]{0.67,0.13,1.00}{##1}}}
\@namedef{PY@tok@s}{\def\PY@tc##1{\textcolor[rgb]{0.73,0.13,0.13}{##1}}}
\@namedef{PY@tok@sd}{\let\PY@it=\textit\def\PY@tc##1{\textcolor[rgb]{0.73,0.13,0.13}{##1}}}
\@namedef{PY@tok@si}{\let\PY@bf=\textbf\def\PY@tc##1{\textcolor[rgb]{0.64,0.35,0.47}{##1}}}
\@namedef{PY@tok@se}{\let\PY@bf=\textbf\def\PY@tc##1{\textcolor[rgb]{0.67,0.36,0.12}{##1}}}
\@namedef{PY@tok@sr}{\def\PY@tc##1{\textcolor[rgb]{0.64,0.35,0.47}{##1}}}
\@namedef{PY@tok@ss}{\def\PY@tc##1{\textcolor[rgb]{0.10,0.09,0.49}{##1}}}
\@namedef{PY@tok@sx}{\def\PY@tc##1{\textcolor[rgb]{0.00,0.50,0.00}{##1}}}
\@namedef{PY@tok@m}{\def\PY@tc##1{\textcolor[rgb]{0.40,0.40,0.40}{##1}}}
\@namedef{PY@tok@gh}{\let\PY@bf=\textbf\def\PY@tc##1{\textcolor[rgb]{0.00,0.00,0.50}{##1}}}
\@namedef{PY@tok@gu}{\let\PY@bf=\textbf\def\PY@tc##1{\textcolor[rgb]{0.50,0.00,0.50}{##1}}}
\@namedef{PY@tok@gd}{\def\PY@tc##1{\textcolor[rgb]{0.63,0.00,0.00}{##1}}}
\@namedef{PY@tok@gi}{\def\PY@tc##1{\textcolor[rgb]{0.00,0.52,0.00}{##1}}}
\@namedef{PY@tok@gr}{\def\PY@tc##1{\textcolor[rgb]{0.89,0.00,0.00}{##1}}}
\@namedef{PY@tok@ge}{\let\PY@it=\textit}
\@namedef{PY@tok@gs}{\let\PY@bf=\textbf}
\@namedef{PY@tok@ges}{\let\PY@bf=\textbf\let\PY@it=\textit}
\@namedef{PY@tok@gp}{\let\PY@bf=\textbf\def\PY@tc##1{\textcolor[rgb]{0.00,0.00,0.50}{##1}}}
\@namedef{PY@tok@go}{\def\PY@tc##1{\textcolor[rgb]{0.44,0.44,0.44}{##1}}}
\@namedef{PY@tok@gt}{\def\PY@tc##1{\textcolor[rgb]{0.00,0.27,0.87}{##1}}}
\@namedef{PY@tok@err}{\def\PY@bc##1{{\setlength{\fboxsep}{\string -\fboxrule}\fcolorbox[rgb]{1.00,0.00,0.00}{1,1,1}{\strut ##1}}}}
\@namedef{PY@tok@kc}{\let\PY@bf=\textbf\def\PY@tc##1{\textcolor[rgb]{0.00,0.50,0.00}{##1}}}
\@namedef{PY@tok@kd}{\let\PY@bf=\textbf\def\PY@tc##1{\textcolor[rgb]{0.00,0.50,0.00}{##1}}}
\@namedef{PY@tok@kn}{\let\PY@bf=\textbf\def\PY@tc##1{\textcolor[rgb]{0.00,0.50,0.00}{##1}}}
\@namedef{PY@tok@kr}{\let\PY@bf=\textbf\def\PY@tc##1{\textcolor[rgb]{0.00,0.50,0.00}{##1}}}
\@namedef{PY@tok@bp}{\def\PY@tc##1{\textcolor[rgb]{0.00,0.50,0.00}{##1}}}
\@namedef{PY@tok@fm}{\def\PY@tc##1{\textcolor[rgb]{0.00,0.00,1.00}{##1}}}
\@namedef{PY@tok@vc}{\def\PY@tc##1{\textcolor[rgb]{0.10,0.09,0.49}{##1}}}
\@namedef{PY@tok@vg}{\def\PY@tc##1{\textcolor[rgb]{0.10,0.09,0.49}{##1}}}
\@namedef{PY@tok@vi}{\def\PY@tc##1{\textcolor[rgb]{0.10,0.09,0.49}{##1}}}
\@namedef{PY@tok@vm}{\def\PY@tc##1{\textcolor[rgb]{0.10,0.09,0.49}{##1}}}
\@namedef{PY@tok@sa}{\def\PY@tc##1{\textcolor[rgb]{0.73,0.13,0.13}{##1}}}
\@namedef{PY@tok@sb}{\def\PY@tc##1{\textcolor[rgb]{0.73,0.13,0.13}{##1}}}
\@namedef{PY@tok@sc}{\def\PY@tc##1{\textcolor[rgb]{0.73,0.13,0.13}{##1}}}
\@namedef{PY@tok@dl}{\def\PY@tc##1{\textcolor[rgb]{0.73,0.13,0.13}{##1}}}
\@namedef{PY@tok@s2}{\def\PY@tc##1{\textcolor[rgb]{0.73,0.13,0.13}{##1}}}
\@namedef{PY@tok@sh}{\def\PY@tc##1{\textcolor[rgb]{0.73,0.13,0.13}{##1}}}
\@namedef{PY@tok@s1}{\def\PY@tc##1{\textcolor[rgb]{0.73,0.13,0.13}{##1}}}
\@namedef{PY@tok@mb}{\def\PY@tc##1{\textcolor[rgb]{0.40,0.40,0.40}{##1}}}
\@namedef{PY@tok@mf}{\def\PY@tc##1{\textcolor[rgb]{0.40,0.40,0.40}{##1}}}
\@namedef{PY@tok@mh}{\def\PY@tc##1{\textcolor[rgb]{0.40,0.40,0.40}{##1}}}
\@namedef{PY@tok@mi}{\def\PY@tc##1{\textcolor[rgb]{0.40,0.40,0.40}{##1}}}
\@namedef{PY@tok@il}{\def\PY@tc##1{\textcolor[rgb]{0.40,0.40,0.40}{##1}}}
\@namedef{PY@tok@mo}{\def\PY@tc##1{\textcolor[rgb]{0.40,0.40,0.40}{##1}}}
\@namedef{PY@tok@ch}{\let\PY@it=\textit\def\PY@tc##1{\textcolor[rgb]{0.24,0.48,0.48}{##1}}}
\@namedef{PY@tok@cm}{\let\PY@it=\textit\def\PY@tc##1{\textcolor[rgb]{0.24,0.48,0.48}{##1}}}
\@namedef{PY@tok@cpf}{\let\PY@it=\textit\def\PY@tc##1{\textcolor[rgb]{0.24,0.48,0.48}{##1}}}
\@namedef{PY@tok@c1}{\let\PY@it=\textit\def\PY@tc##1{\textcolor[rgb]{0.24,0.48,0.48}{##1}}}
\@namedef{PY@tok@cs}{\let\PY@it=\textit\def\PY@tc##1{\textcolor[rgb]{0.24,0.48,0.48}{##1}}}

\def\PYZbs{\char`\\}
\def\PYZus{\char`\_}
\def\PYZob{\char`\{}
\def\PYZcb{\char`\}}
\def\PYZca{\char`\^}
\def\PYZam{\char`\&}
\def\PYZlt{\char`\<}
\def\PYZgt{\char`\>}
\def\PYZsh{\char`\#}
\def\PYZpc{\char`\%}
\def\PYZdl{\char`\$}
\def\PYZhy{\char`\-}
\def\PYZsq{\char`\'}
\def\PYZdq{\char`\"}
\def\PYZti{\char`\~}
% for compatibility with earlier versions
\def\PYZat{@}
\def\PYZlb{[}
\def\PYZrb{]}
\makeatother


    % For linebreaks inside Verbatim environment from package fancyvrb.
    \makeatletter
        \newbox\Wrappedcontinuationbox
        \newbox\Wrappedvisiblespacebox
        \newcommand*\Wrappedvisiblespace {\textcolor{red}{\textvisiblespace}}
        \newcommand*\Wrappedcontinuationsymbol {\textcolor{red}{\llap{\tiny$\m@th\hookrightarrow$}}}
        \newcommand*\Wrappedcontinuationindent {3ex }
        \newcommand*\Wrappedafterbreak {\kern\Wrappedcontinuationindent\copy\Wrappedcontinuationbox}
        % Take advantage of the already applied Pygments mark-up to insert
        % potential linebreaks for TeX processing.
        %        {, <, #, %, $, ' and ": go to next line.
        %        _, }, ^, &, >, - and ~: stay at end of broken line.
        % Use of \textquotesingle for straight quote.
        \newcommand*\Wrappedbreaksatspecials {%
            \def\PYGZus{\discretionary{\char`\_}{\Wrappedafterbreak}{\char`\_}}%
            \def\PYGZob{\discretionary{}{\Wrappedafterbreak\char`\{}{\char`\{}}%
            \def\PYGZcb{\discretionary{\char`\}}{\Wrappedafterbreak}{\char`\}}}%
            \def\PYGZca{\discretionary{\char`\^}{\Wrappedafterbreak}{\char`\^}}%
            \def\PYGZam{\discretionary{\char`\&}{\Wrappedafterbreak}{\char`\&}}%
            \def\PYGZlt{\discretionary{}{\Wrappedafterbreak\char`\<}{\char`\<}}%
            \def\PYGZgt{\discretionary{\char`\>}{\Wrappedafterbreak}{\char`\>}}%
            \def\PYGZsh{\discretionary{}{\Wrappedafterbreak\char`\#}{\char`\#}}%
            \def\PYGZpc{\discretionary{}{\Wrappedafterbreak\char`\%}{\char`\%}}%
            \def\PYGZdl{\discretionary{}{\Wrappedafterbreak\char`\$}{\char`\$}}%
            \def\PYGZhy{\discretionary{\char`\-}{\Wrappedafterbreak}{\char`\-}}%
            \def\PYGZsq{\discretionary{}{\Wrappedafterbreak\textquotesingle}{\textquotesingle}}%
            \def\PYGZdq{\discretionary{}{\Wrappedafterbreak\char`\"}{\char`\"}}%
            \def\PYGZti{\discretionary{\char`\~}{\Wrappedafterbreak}{\char`\~}}%
        }
        % Some characters . , ; ? ! / are not pygmentized.
        % This macro makes them "active" and they will insert potential linebreaks
        \newcommand*\Wrappedbreaksatpunct {%
            \lccode`\~`\.\lowercase{\def~}{\discretionary{\hbox{\char`\.}}{\Wrappedafterbreak}{\hbox{\char`\.}}}%
            \lccode`\~`\,\lowercase{\def~}{\discretionary{\hbox{\char`\,}}{\Wrappedafterbreak}{\hbox{\char`\,}}}%
            \lccode`\~`\;\lowercase{\def~}{\discretionary{\hbox{\char`\;}}{\Wrappedafterbreak}{\hbox{\char`\;}}}%
            \lccode`\~`\:\lowercase{\def~}{\discretionary{\hbox{\char`\:}}{\Wrappedafterbreak}{\hbox{\char`\:}}}%
            \lccode`\~`\?\lowercase{\def~}{\discretionary{\hbox{\char`\?}}{\Wrappedafterbreak}{\hbox{\char`\?}}}%
            \lccode`\~`\!\lowercase{\def~}{\discretionary{\hbox{\char`\!}}{\Wrappedafterbreak}{\hbox{\char`\!}}}%
            \lccode`\~`\/\lowercase{\def~}{\discretionary{\hbox{\char`\/}}{\Wrappedafterbreak}{\hbox{\char`\/}}}%
            \catcode`\.\active
            \catcode`\,\active
            \catcode`\;\active
            \catcode`\:\active
            \catcode`\?\active
            \catcode`\!\active
            \catcode`\/\active
            \lccode`\~`\~
        }
    \makeatother

    \let\OriginalVerbatim=\Verbatim
    \makeatletter
    \renewcommand{\Verbatim}[1][1]{%
        %\parskip\z@skip
        \sbox\Wrappedcontinuationbox {\Wrappedcontinuationsymbol}%
        \sbox\Wrappedvisiblespacebox {\FV@SetupFont\Wrappedvisiblespace}%
        \def\FancyVerbFormatLine ##1{\hsize\linewidth
            \vtop{\raggedright\hyphenpenalty\z@\exhyphenpenalty\z@
                \doublehyphendemerits\z@\finalhyphendemerits\z@
                \strut ##1\strut}%
        }%
        % If the linebreak is at a space, the latter will be displayed as visible
        % space at end of first line, and a continuation symbol starts next line.
        % Stretch/shrink are however usually zero for typewriter font.
        \def\FV@Space {%
            \nobreak\hskip\z@ plus\fontdimen3\font minus\fontdimen4\font
            \discretionary{\copy\Wrappedvisiblespacebox}{\Wrappedafterbreak}
            {\kern\fontdimen2\font}%
        }%

        % Allow breaks at special characters using \PYG... macros.
        \Wrappedbreaksatspecials
        % Breaks at punctuation characters . , ; ? ! and / need catcode=\active
        \OriginalVerbatim[#1,codes*=\Wrappedbreaksatpunct]%
    }
    \makeatother

    % Exact colors from NB
    \definecolor{incolor}{HTML}{303F9F}
    \definecolor{outcolor}{HTML}{D84315}
    \definecolor{cellborder}{HTML}{CFCFCF}
    \definecolor{cellbackground}{HTML}{F7F7F7}

    % prompt
    \makeatletter
    \newcommand{\boxspacing}{\kern\kvtcb@left@rule\kern\kvtcb@boxsep}
    \makeatother
    \newcommand{\prompt}[4]{
        {\ttfamily\llap{{\color{#2}[#3]:\hspace{3pt}#4}}\vspace{-\baselineskip}}
    }
    

    
    % Prevent overflowing lines due to hard-to-break entities
    \sloppy
    % Setup hyperref package
    \hypersetup{
      breaklinks=true,  % so long urls are correctly broken across lines
      colorlinks=true,
      urlcolor=urlcolor,
      linkcolor=linkcolor,
      citecolor=citecolor,
      }
    % Slightly bigger margins than the latex defaults
    
    \geometry{verbose,tmargin=1in,bmargin=1in,lmargin=1in,rmargin=1in}
    
    

\begin{document}
    
    \maketitle
    
    

    
    Dear Allen Zhao,

Thank you for your feedback on my Project 3 submission. Here are my
responses. I hope that they are sufficient.

    \begin{quote}
This is quite a deviation from the array-based Monte Carlo framework we
introduced in class. If you are intent on using this, I would like to
see a straightforward comparison with the method we've laid out for you
which makes it clear that your implementation is equivalent and robust
enough for the purposes of this project.
\end{quote}

    I spoke to Joss about this. I understand your concern, but I'm
completely unsure what you mean by the ``array-based \ldots{} method
we've laid out for you,'' as I never followed any sort of tutorial or
outline for performing this project. The starter code for this project
(such as Homework 17) doesn't include a full process for this and
Homework 16 isn't immediately applicable for these kinds of trees, so
I'm not sure what you want me to compare against (as I've been using
tree generation from the beginning of Homework 17). With that, Joss
simply told me to ignore this.

However, despite that, I would still like to provide some justification
for my method to qualitatively defend it regardless. I actually made a
major mistake in my prior assumption that using the
exponentially-growing tree would result in modeling being effectively
impossible for generational depths that are too high. While it is true
that a generational depth of, say, 20 could result in
\(2^{20} \approx 1.05 \times 10^{6}\) neutrons that have to be
simultaneously modeled, this assumes that \emph{no neutrons are lost},
and this is completely unrealistic for the simulation. Even the most
stable simulations I'm modelling will lose large amounts of neutrons, so
these systems certainly aren't going to be reaching such large numbers
of objects being handled at the same time. In particular, the effective
multiplication rate for stable systems has experimentally trended
towards being just above \(k=1\) (reactor criticality), so the number of
objects being managed is generally growing, but not by a considerable
amount each generation.

    \begin{quote}
One major concern I have is that you are currently only modeling 3--4
generations of neutrons, which might not be enough to give statistically
sound results (you need to verify this). If you end up needing to
iterate over many more generations, I am not sure how performant an
exponentially-growing tree of Python objects will be.
\end{quote}

    Honestly, you're right. I spoke to Joss about this as well, and the
solution I've come to is that I could refocus my project to model
long-run stable \(k\) values and produce a singular contour, and this
would require determining the maximum generation depth required for a
system to be stable across all of the systems.

    \begin{quote}
You have very well laid-out unit tests. This is great practice, but for
the purposes of this project you can depend on ``trusting the math'' for
simple expressions.
\end{quote}

To keep this simple (and because of Joss' advice on my Project 2), I'll
move all of the tests under validation to declutter the main section of
the project.

\begin{quote}
While your printed trees are a great debug tool, they're a bit cluttered
to be included in their entirety in your writeup. They're definitely
closer to debugging text, and it isn't very useful for the reader to
interpret the larger ones. Maybe leave a small example at most.
\end{quote}

Yeah, this is probably a good idea. I'll remove the longer ones, and
just keep the example.

\begin{quote}
Not sure how I feel about using ChatGPT to write unit tests\ldots{}
\end{quote}

I'm only using it for tests where I need to verify very basic things
(like the shape of output vectors and whatnot), so the code for doing it
is tedious rather than complex. If it works, it works.

    \begin{quote}
While your code and description of its implementation is great, this
currently almost feels like the main purpose of your writeup rather than
the investigation you should be carrying out for Project 3.
\end{quote}

    Honestly, you are right, and I feel as though the best way to rectify
this is to apply the additions to the introduction that you request in
your next point alongside some deeper consideration sprinkled throughout
the document. This message is a bit vague (in comparison to your next
point about the introduction) so I'm going to take the liberty of just
assuming that the most critical areas to increase the level of
specificity will still remain at the investigative sections (so the
strictly technical sections, such as function definitions, will remain
mostly just technical descriptions).

    \begin{quote}
Your project almost immediately launches into a technical review of
code\ldots{} without giving the reader any of the necessary context
(what is a monte carlo simulation? what is the physical scenario you are
investigating? how are you modeling your system?) to properly follow it.
Imagine your writeup as a lab report. Prior to making any nontrivial
calculations in your ``Methods'' section, you would want to first
introduce the relevant formalism and equations you are working with.
\end{quote}

    To begin, I'm going to address this point by considering each of the
questions that you're explicitly asking here one-by-one, and then I'll
use my answers to each of those questions to produce the basis statement
that you're asking for as one succinct description.

\begin{quote}
What is a Monte Carlo Simulation?
\end{quote}

``A Monte Carlo simulation is a general class of stochastic (random)
methods for performing computational calculations. In this case, we're
calculating a value through performing a high volume of random
simulations to produce a distribution of values (multiplication rates),
then selecting the mean and standard error of that distribution to act
as the final calculation of the value.''

\begin{quote}
What is the physical scenario you are investigating? / How are you
modelling your system?
\end{quote}

``\textbf{General Background Information}

Consider the core of a nuclear reactor. neutrons are, obviously, much
smaller than the gaps between atoms and their neutral charge causes them
to not have attractive force to free electrons or any repellant force,
meaning that they can fairly easily escape the metallic bounds of
reactor cores once atoms are split through fission. Therefore, it's
fairly accurate to model the neutrons in a reactor as having no physical
bounds whatsoever (in the form of a potential field blocking them from
leaving), and this simulation seeks to investigate how often neutrons
continue to replicate within this reactor.

As a nuclear reaction progresses, neutrons ``replicate'' into several
more by triggering further fission reactions in other atoms, and the
amount of free neutrons that are ``produced'' from the triggered fission
by the previous neutron are considered to be successfully replicated.
However, as the neutrons leak out of the reactor, they're unlikely to
return within the core to continue replications, and it's also unlikely
for them to replicate outside of the core, so for the purposes of this
reactor, we're only interested in the neutrons that replicate within the
core, and any neutrons that leak out of the reactor can be said to be
effectively lost to the reactor altogether.

Therefore, we can model this system using a non-physical bound for the
reactor core and a growing tree of randomly-generated neutrons for the
replications: - The reactor core is like a ``line in the sand'' rather
than a wall: it has no power to stop neutrons from exiting, and they
will do so randomly, but once they cross the boundary, further progress
is entirely not considered. - The neutrons are assumed to always induce
a chain reaction that produces exactly two further free neutrons every
time, but each neutron that is produced is then sent off in a
randomly-determined direction and distance from the reaction that
produced it, and if it ends up outside of the reactor core, then that
neutron is considered ``lost.''

This, of course, requires an important distinction: each neutron is
implicitly defined to replicate two further neutrons each time, but the
effective replication rate is only considering the neutrons that remain
in the reactor, so for example, a neutron could induce a reaction to
produce two free neutrons, but if one neutron exits the reactor and the
other ends within the reactor, then that neutron is said to have only
replicated a single neutron, and the single neutron that remained in the
reactor will induce another reaction.

\textbf{Research Question}

With all of the above information in mind, for a given core shape, the
reactor core is going to eventually reach a specific replication rate.
Unstable reactor geometries would have replication rates approach zero
due to too much leakage, and stable reactor geometries will have
replication rates that stay roughly constant at some non-zero value. The
focus of this simulation, in particular, is to see how this long-run
replication rate changes as the geometry of the core changes.

To do this, I've defined the reactor core's geometry as a rectangular
prism, so for simplicity, we can call it a ``box.'' This means that the
box's geometry has three parameters: a length \(\ell\), width \(w\), and
height \(h\). To keep this simulation simple, we'll make the length
\(\ell\) a function of the width and the height:
\[\ell = \frac{1}{2} (w + h),\] Then, we can just vary the volume \(V\)
and the aspect ratio \(R\) with each being explicit functions of only
width and height: \begin{align}
R &= \frac{w}{h} \\
V &= w \ell h = wh\frac{1}{2}(w + h)
\end{align} then each of the parameters are only a function of \(R\)
and \(V\): $$\begin{cases}
w &= R \left(\frac{2V}{R(R+1)}\right)^\frac{1}{3} \\
h &= \left(\frac{2V}{R(R+1)}\right)^\frac{1}{3} \\
\ell &= \frac{1}{2} (R + 1) \left(\frac{2V}{R(R+1)}\right)^\frac{1}{3}
\end{cases}.$$

Now we have a two-dimensional phase space for the simulation, aspect
ratio \(R\) and volume \(V\), and thus, each box geometry can be defined
as a phase coordinate \((R,V)\). For each geometry, we can perform a
singular simulation with a random starting position, then record the
replication rate \(k_i\) (for an individual simulation \(i\)) after a
certain number of generations, \(g\). (It's incredibly important that we
pick \(g\) correctly to best characterize our system, but for now, we'll
just vaguely state that it's a large enough number to characterize the
long-term behavior of the system.) We can then perform this individual
simulation a large number of times with more random starting positions,
making the overall simulation \(f\) a Monte-Carlo simulation that
estimates the long-term multiplication rate as \(\mathbf{k}\) based on
the mean and standard error, \(\overline{k} + \delta k\). Therefore,
thus far, we're producing a mapping using our simulation \(f\):
\[f(R,V) \mapsto \mathbf{k} = \overline{k} + \delta k.\]

Thus, simply put, the overarching goal is to calculate this mapping for
each \((R,V)\) over some two-dimensional phase domain \(D\)
\[D = \{ R_{\min} \le R \le R_{\max} \} \otimes \{ V_{\min} \le V \le V_{\max} \}\]
and then analyze the results we observe.

As for the hypothesis, one can safely assume from basic intuition that
smaller volumes should be less likely to retain neutrons, as at an
infinite volume of the reactor core (and also assuming the entirety of
all space is filled with fuel in the realistic case), the reaction will
proceed forever and grow over all space. However, the aspect ratio is
much more interesting. We can reasonably assume that there should be the
greatest preference to more even aspect ratios (closer to 1:1), because
if the width, height, or length of the box is approximately zero, any
movement of a neutron along the approximately zero axis in
three-dimensional space will result in ejection.''

    \begin{quote}
Fig2/3: There seems to be a misconception of what \(k\) is. You should
treat \(k\) as a macroscopic constant intrinsic to a given test
structure - not neutrons. By performing this Monte Carlo simulation, you
are ultimately trying to compute the best possible ``measurement'' of k
by averaging out the statistically-distributed results of many
``experimental'' simulations.
\end{quote}

    I understand what you mean. I think the major error in my work is in
Figure 2 with the line ``The first generation can only ever replicate 0,
1, or 2 neutrons exactly, so an individual neutron \(i\) will have
exactly \(k_i = 0, 1, 2\): a discrete range. So, in short, because the
range of possible \(k\) values for a given neutron is so limited, the
negative contribution from low \(k\) values will be much higher.'' and
in figure 3, with the line ``The first generation will usually have a
replication rate of \(k=2\) or \(k=1\)''. I think the best remedy for
this is just switching the entire project to use long-run \(k\)-values,
as stated previously.

    \begin{quote}
If your k values appear to still change significantly across generations
(e.g.~in Fig2), what does this tell you about the accuracy of your
current measurement? How can you ensure this value of k has stabilized
before taking a measurement? Try running your simulation for additional
generations and plotting k vs generation number. Consider the relative
uncertainty of your average k measurements in Figures 3--4. Does this
change with volume and aspect ratio?
\end{quote}

    Yet again, there is a lot to be addressed in this point, so I'm going to
answer each of your individual questions, but the overall remark is that
my simulation, as it stands, isn't stabilized for the number of
generations that I'm considering.

\begin{quote}
If your k values appear to still change significantly across generations
(e.g.~in Fig2), what does this tell you about the accuracy of your
current measurement?
\end{quote}

Very simple: my simulations are just not stabilized. High relative error
on \(k\) is an extremely good indicator that the simulation is not yet
long-run stable, so I'll need to increase the number of generations
significantly.

\begin{quote}
How can you ensure this value of k has stabilized before taking a
measurement?
\end{quote}

I discussed this with Joss, and the solution I've reached is to pick a
minimum-stable-generation \(g\), justify this based on empirically
showing which reactor geometries require longer generational depths to
stabilize, then simply using that generation depth for all of the future
Monte Carlo simulations.

\begin{quote}
Consider the relative uncertainty of your average k measurements in
Figures 3--4. Does this change with volume and aspect ratio?
\end{quote}

This is a difficult question to answer, because the uncertainty in the
\(k\) value is not something that is consistent from generation to
generation. For example, in Figure 6,

\begin{figure}
\centering
\includegraphics{example_image.png}
\caption{image.png}
\end{figure}

the uncertainty spikes just before all systems completely destabilize
and both the multiplication rate and the uncertainty go to zero. This is
interesting behavior for sure, and it's insightful for understanding the
oddly random trends seen in the final contour, but it's not something
that's particularly consistent enough or really representative to make a
greater overall claim about how the uncertainty in all of the \(k\)
measurements is a function of the geometry of the box. I'll make sure to
discuss the role that uncertainty is playing here in interpreting random
behavior from the simulation, but fundamentally, the goal with the
uncertainty through my investigation is simply as a guidepost for
understanding what parameters need to be given to minimize this
uncertainty enough that our results are accurate.

    \begin{quote}
You need a much more thorough discussion of the uncertainties in your
data and their implications on how you can interpret your results. Is
20--40 replications enough to average out statistical differences in
this case?
\end{quote}

    This is a good point. I should take proper care in picking a standard
replication rate alongside the generational depth to minimize standard
error as well, so after determining \(g\), I'll also determine a good
minimal replication rate for producing a relative error for in the
long-run multiplication rate less than some very small maximal
threshold.

I've added this section as ``Determining the Minimal Replication Rate
for Producing Accurate Aggregate Multiplication Rates.''

    \begin{quote}
You are also making a lot of assertions about global minima and maxima
in your case studies without actually modeling the data you are working
with.
\end{quote}

    I'm slightly confused by what you mean by my not ``actually modeling the
data {[}I'm{]} working with,'' but I assume that the problem is my
referring to these points as ``global maxima'' or ``global minima''
without explicitly finding the global maxima and minima directly from my
calculated data. In any case, it's probably more accurate to examine
these points in a much broader sense, considering the point
\((R,V)=(1,200)\) as reflective of the broader class of even aspect
ratios and large volumes and the point \((R,V)=(10,1)\) as reflective of
the broader class of skewed aspect ratios and small volumes, and I can
safely make much more general statements about how this relates to the
overall trend of reactor stabilities without decreasing the quality of
my conclusions/reflections.

    \begin{quote}
Beyond your heatmaps, the majority of your project focuses more on the
properties of the process we are modeling rather than a study of system
behavior vs your independent variables. You should perform a thorough
and ideally quantitative analysis of how k scales with volume and aspect
ratio. One way to do this is to fit cross-sections of your heat map data
to a model of your devising.
\end{quote}

    I see what you mean, and your idea is actually perfect as-is. I'll just
do exactly that, and add a dedicated section for it. It's quite long, so
I won't be adding it to this document, but it can be found under ``Phase
Exploration: Cross-Sectional Analysis.''

    \begin{quote}
Appendix: While it's great that you discuss the various validation and
unit-testing techniques used to develop your project, you should also
include some concrete and followable instances of code validation. An
example of something you could check here is whether or not your neutron
population behaves as expected for a given multiplication factor.
\end{quote}

    I understand what you're looking for here. Now, I'm a bit confused how I
would implement your given example as the multiplication rate is what we
calculate rather than the parameter I'm using. What would realistically
make far more sense here is a situational validation similar to the kind
I performed on project 2, where I begin by considering trivial cases
where their behavior should be known and expected then see how neutrons
develop from there. I did a very minor example of this through the
ending of the phase investigation, but what in this case, I can consider
points far outside of my phase investigation to produce truly trivial
situations (like the aforementioned ``infinite box'' with a size much
greater than the mean-free-path or a box with a volume approaching
zero).

    \begin{quote}
I'm unsure if your current experiment and its results generalize well
enough to assert that a sphere would be ideal. What additional testing
could you do to solidify this claim?
\end{quote}

    You're absolutely correct. I definitely don't have enough to assert that
a sphere would be ideal, and I understand that it would be possible to
directly verify this claim through directly modelling it at the end of
my simulation, but I actually no longer believe that it is helpful to my
investigation whatsoever to add this claim. It makes much more sense to
only directly state exactly what was proven through my investigation
under my claims then move this section of the discussion about
non-rectangular geometry as a future hypothesis/research question under
``future research'' with the basis that this study serves as a good
starting-point for trying to undergo this research.

    I hope that this is good enough! Thanks again for the notes, and thank
you for the great term.

Regards,

Mufaro Machaya

    \begin{tcolorbox}[breakable, size=fbox, boxrule=1pt, pad at break*=1mm,colback=cellbackground, colframe=cellborder]
\prompt{In}{incolor}{ }{\boxspacing}
\begin{Verbatim}[commandchars=\\\{\}]

\end{Verbatim}
\end{tcolorbox}


    % Add a bibliography block to the postdoc
    
    
    
\end{document}
