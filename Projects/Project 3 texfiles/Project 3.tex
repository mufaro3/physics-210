\documentclass[11pt]{article}

    \usepackage[breakable]{tcolorbox}
    \usepackage{parskip} % Stop auto-indenting (to mimic markdown behaviour)
    

    % Basic figure setup, for now with no caption control since it's done
    % automatically by Pandoc (which extracts ![](path) syntax from Markdown).
    \usepackage{graphicx}
    % Maintain compatibility with old templates. Remove in nbconvert 6.0
    \let\Oldincludegraphics\includegraphics
    % Ensure that by default, figures have no caption (until we provide a
    % proper Figure object with a Caption API and a way to capture that
    % in the conversion process - todo).
    \usepackage{caption}
    \DeclareCaptionFormat{nocaption}{}
    \captionsetup{format=nocaption,aboveskip=0pt,belowskip=0pt}

    \usepackage{float}
    \floatplacement{figure}{H} % forces figures to be placed at the correct location
    \usepackage{xcolor} % Allow colors to be defined
    \usepackage{enumerate} % Needed for markdown enumerations to work
    \usepackage{geometry} % Used to adjust the document margins
    \usepackage{amsmath} % Equations
    \usepackage{amssymb} % Equations
    \usepackage{textcomp} % defines textquotesingle
    % Hack from http://tex.stackexchange.com/a/47451/13684:
    \AtBeginDocument{%
        \def\PYZsq{\textquotesingle}% Upright quotes in Pygmentized code
    }
    \usepackage{upquote} % Upright quotes for verbatim code
    \usepackage{eurosym} % defines \euro

    \usepackage{iftex}
    \ifPDFTeX
        \usepackage[T1]{fontenc}
        \IfFileExists{alphabeta.sty}{
              \usepackage{alphabeta}
          }{
              \usepackage[mathletters]{ucs}
              \usepackage[utf8x]{inputenc}
          }
    \else
        \usepackage{fontspec}
        \usepackage{unicode-math}
    \fi

    \usepackage{fancyvrb} % verbatim replacement that allows latex
    \usepackage[Export]{adjustbox} % Used to constrain images to a maximum size
    \adjustboxset{max size={0.9\linewidth}{0.9\paperheight}}

    % The hyperref package gives us a pdf with properly built
    % internal navigation ('pdf bookmarks' for the table of contents,
    % internal cross-reference links, web links for URLs, etc.)
    \usepackage{hyperref}
    % The default LaTeX title has an obnoxious amount of whitespace. By default,
    % titling removes some of it. It also provides customization options.
    \usepackage{titling}
    \usepackage{longtable} % longtable support required by pandoc >1.10
    \usepackage{booktabs}  % table support for pandoc > 1.12.2
    \usepackage{array}     % table support for pandoc >= 2.11.3
    \usepackage{calc}      % table minipage width calculation for pandoc >= 2.11.1
    \usepackage[inline]{enumitem} % IRkernel/repr support (it uses the enumerate* environment)
    \usepackage[normalem]{ulem} % ulem is needed to support strikethroughs (\sout)
                                % normalem makes italics be italics, not underlines
    \usepackage{mathrsfs}
    

    
    % Colors for the hyperref package
    \definecolor{urlcolor}{rgb}{0,.145,.698}
    \definecolor{linkcolor}{rgb}{.71,0.21,0.01}
    \definecolor{citecolor}{rgb}{.12,.54,.11}

    % ANSI colors
    \definecolor{ansi-black}{HTML}{3E424D}
    \definecolor{ansi-black-intense}{HTML}{282C36}
    \definecolor{ansi-red}{HTML}{E75C58}
    \definecolor{ansi-red-intense}{HTML}{B22B31}
    \definecolor{ansi-green}{HTML}{00A250}
    \definecolor{ansi-green-intense}{HTML}{007427}
    \definecolor{ansi-yellow}{HTML}{DDB62B}
    \definecolor{ansi-yellow-intense}{HTML}{B27D12}
    \definecolor{ansi-blue}{HTML}{208FFB}
    \definecolor{ansi-blue-intense}{HTML}{0065CA}
    \definecolor{ansi-magenta}{HTML}{D160C4}
    \definecolor{ansi-magenta-intense}{HTML}{A03196}
    \definecolor{ansi-cyan}{HTML}{60C6C8}
    \definecolor{ansi-cyan-intense}{HTML}{258F8F}
    \definecolor{ansi-white}{HTML}{C5C1B4}
    \definecolor{ansi-white-intense}{HTML}{A1A6B2}
    \definecolor{ansi-default-inverse-fg}{HTML}{FFFFFF}
    \definecolor{ansi-default-inverse-bg}{HTML}{000000}

    % common color for the border for error outputs.
    \definecolor{outerrorbackground}{HTML}{FFDFDF}

    % commands and environments needed by pandoc snippets
    % extracted from the output of `pandoc -s`
    \providecommand{\tightlist}{%
      \setlength{\itemsep}{0pt}\setlength{\parskip}{0pt}}
    \DefineVerbatimEnvironment{Highlighting}{Verbatim}{commandchars=\\\{\}}
    % Add ',fontsize=\small' for more characters per line
    \newenvironment{Shaded}{}{}
    \newcommand{\KeywordTok}[1]{\textcolor[rgb]{0.00,0.44,0.13}{\textbf{{#1}}}}
    \newcommand{\DataTypeTok}[1]{\textcolor[rgb]{0.56,0.13,0.00}{{#1}}}
    \newcommand{\DecValTok}[1]{\textcolor[rgb]{0.25,0.63,0.44}{{#1}}}
    \newcommand{\BaseNTok}[1]{\textcolor[rgb]{0.25,0.63,0.44}{{#1}}}
    \newcommand{\FloatTok}[1]{\textcolor[rgb]{0.25,0.63,0.44}{{#1}}}
    \newcommand{\CharTok}[1]{\textcolor[rgb]{0.25,0.44,0.63}{{#1}}}
    \newcommand{\StringTok}[1]{\textcolor[rgb]{0.25,0.44,0.63}{{#1}}}
    \newcommand{\CommentTok}[1]{\textcolor[rgb]{0.38,0.63,0.69}{\textit{{#1}}}}
    \newcommand{\OtherTok}[1]{\textcolor[rgb]{0.00,0.44,0.13}{{#1}}}
    \newcommand{\AlertTok}[1]{\textcolor[rgb]{1.00,0.00,0.00}{\textbf{{#1}}}}
    \newcommand{\FunctionTok}[1]{\textcolor[rgb]{0.02,0.16,0.49}{{#1}}}
    \newcommand{\RegionMarkerTok}[1]{{#1}}
    \newcommand{\ErrorTok}[1]{\textcolor[rgb]{1.00,0.00,0.00}{\textbf{{#1}}}}
    \newcommand{\NormalTok}[1]{{#1}}

    % Additional commands for more recent versions of Pandoc
    \newcommand{\ConstantTok}[1]{\textcolor[rgb]{0.53,0.00,0.00}{{#1}}}
    \newcommand{\SpecialCharTok}[1]{\textcolor[rgb]{0.25,0.44,0.63}{{#1}}}
    \newcommand{\VerbatimStringTok}[1]{\textcolor[rgb]{0.25,0.44,0.63}{{#1}}}
    \newcommand{\SpecialStringTok}[1]{\textcolor[rgb]{0.73,0.40,0.53}{{#1}}}
    \newcommand{\ImportTok}[1]{{#1}}
    \newcommand{\DocumentationTok}[1]{\textcolor[rgb]{0.73,0.13,0.13}{\textit{{#1}}}}
    \newcommand{\AnnotationTok}[1]{\textcolor[rgb]{0.38,0.63,0.69}{\textbf{\textit{{#1}}}}}
    \newcommand{\CommentVarTok}[1]{\textcolor[rgb]{0.38,0.63,0.69}{\textbf{\textit{{#1}}}}}
    \newcommand{\VariableTok}[1]{\textcolor[rgb]{0.10,0.09,0.49}{{#1}}}
    \newcommand{\ControlFlowTok}[1]{\textcolor[rgb]{0.00,0.44,0.13}{\textbf{{#1}}}}
    \newcommand{\OperatorTok}[1]{\textcolor[rgb]{0.40,0.40,0.40}{{#1}}}
    \newcommand{\BuiltInTok}[1]{{#1}}
    \newcommand{\ExtensionTok}[1]{{#1}}
    \newcommand{\PreprocessorTok}[1]{\textcolor[rgb]{0.74,0.48,0.00}{{#1}}}
    \newcommand{\AttributeTok}[1]{\textcolor[rgb]{0.49,0.56,0.16}{{#1}}}
    \newcommand{\InformationTok}[1]{\textcolor[rgb]{0.38,0.63,0.69}{\textbf{\textit{{#1}}}}}
    \newcommand{\WarningTok}[1]{\textcolor[rgb]{0.38,0.63,0.69}{\textbf{\textit{{#1}}}}}


    % Define a nice break command that doesn't care if a line doesn't already
    % exist.
    \def\br{\hspace*{\fill} \\* }
    % Math Jax compatibility definitions
    \def\gt{>}
    \def\lt{<}
    \let\Oldtex\TeX
    \let\Oldlatex\LaTeX
    \renewcommand{\TeX}{\textrm{\Oldtex}}
    \renewcommand{\LaTeX}{\textrm{\Oldlatex}}
    % Document parameters
    % Document title
    \title{Project 3}
    
    
    
    
    
% Pygments definitions
\makeatletter
\def\PY@reset{\let\PY@it=\relax \let\PY@bf=\relax%
    \let\PY@ul=\relax \let\PY@tc=\relax%
    \let\PY@bc=\relax \let\PY@ff=\relax}
\def\PY@tok#1{\csname PY@tok@#1\endcsname}
\def\PY@toks#1+{\ifx\relax#1\empty\else%
    \PY@tok{#1}\expandafter\PY@toks\fi}
\def\PY@do#1{\PY@bc{\PY@tc{\PY@ul{%
    \PY@it{\PY@bf{\PY@ff{#1}}}}}}}
\def\PY#1#2{\PY@reset\PY@toks#1+\relax+\PY@do{#2}}

\@namedef{PY@tok@w}{\def\PY@tc##1{\textcolor[rgb]{0.73,0.73,0.73}{##1}}}
\@namedef{PY@tok@c}{\let\PY@it=\textit\def\PY@tc##1{\textcolor[rgb]{0.24,0.48,0.48}{##1}}}
\@namedef{PY@tok@cp}{\def\PY@tc##1{\textcolor[rgb]{0.61,0.40,0.00}{##1}}}
\@namedef{PY@tok@k}{\let\PY@bf=\textbf\def\PY@tc##1{\textcolor[rgb]{0.00,0.50,0.00}{##1}}}
\@namedef{PY@tok@kp}{\def\PY@tc##1{\textcolor[rgb]{0.00,0.50,0.00}{##1}}}
\@namedef{PY@tok@kt}{\def\PY@tc##1{\textcolor[rgb]{0.69,0.00,0.25}{##1}}}
\@namedef{PY@tok@o}{\def\PY@tc##1{\textcolor[rgb]{0.40,0.40,0.40}{##1}}}
\@namedef{PY@tok@ow}{\let\PY@bf=\textbf\def\PY@tc##1{\textcolor[rgb]{0.67,0.13,1.00}{##1}}}
\@namedef{PY@tok@nb}{\def\PY@tc##1{\textcolor[rgb]{0.00,0.50,0.00}{##1}}}
\@namedef{PY@tok@nf}{\def\PY@tc##1{\textcolor[rgb]{0.00,0.00,1.00}{##1}}}
\@namedef{PY@tok@nc}{\let\PY@bf=\textbf\def\PY@tc##1{\textcolor[rgb]{0.00,0.00,1.00}{##1}}}
\@namedef{PY@tok@nn}{\let\PY@bf=\textbf\def\PY@tc##1{\textcolor[rgb]{0.00,0.00,1.00}{##1}}}
\@namedef{PY@tok@ne}{\let\PY@bf=\textbf\def\PY@tc##1{\textcolor[rgb]{0.80,0.25,0.22}{##1}}}
\@namedef{PY@tok@nv}{\def\PY@tc##1{\textcolor[rgb]{0.10,0.09,0.49}{##1}}}
\@namedef{PY@tok@no}{\def\PY@tc##1{\textcolor[rgb]{0.53,0.00,0.00}{##1}}}
\@namedef{PY@tok@nl}{\def\PY@tc##1{\textcolor[rgb]{0.46,0.46,0.00}{##1}}}
\@namedef{PY@tok@ni}{\let\PY@bf=\textbf\def\PY@tc##1{\textcolor[rgb]{0.44,0.44,0.44}{##1}}}
\@namedef{PY@tok@na}{\def\PY@tc##1{\textcolor[rgb]{0.41,0.47,0.13}{##1}}}
\@namedef{PY@tok@nt}{\let\PY@bf=\textbf\def\PY@tc##1{\textcolor[rgb]{0.00,0.50,0.00}{##1}}}
\@namedef{PY@tok@nd}{\def\PY@tc##1{\textcolor[rgb]{0.67,0.13,1.00}{##1}}}
\@namedef{PY@tok@s}{\def\PY@tc##1{\textcolor[rgb]{0.73,0.13,0.13}{##1}}}
\@namedef{PY@tok@sd}{\let\PY@it=\textit\def\PY@tc##1{\textcolor[rgb]{0.73,0.13,0.13}{##1}}}
\@namedef{PY@tok@si}{\let\PY@bf=\textbf\def\PY@tc##1{\textcolor[rgb]{0.64,0.35,0.47}{##1}}}
\@namedef{PY@tok@se}{\let\PY@bf=\textbf\def\PY@tc##1{\textcolor[rgb]{0.67,0.36,0.12}{##1}}}
\@namedef{PY@tok@sr}{\def\PY@tc##1{\textcolor[rgb]{0.64,0.35,0.47}{##1}}}
\@namedef{PY@tok@ss}{\def\PY@tc##1{\textcolor[rgb]{0.10,0.09,0.49}{##1}}}
\@namedef{PY@tok@sx}{\def\PY@tc##1{\textcolor[rgb]{0.00,0.50,0.00}{##1}}}
\@namedef{PY@tok@m}{\def\PY@tc##1{\textcolor[rgb]{0.40,0.40,0.40}{##1}}}
\@namedef{PY@tok@gh}{\let\PY@bf=\textbf\def\PY@tc##1{\textcolor[rgb]{0.00,0.00,0.50}{##1}}}
\@namedef{PY@tok@gu}{\let\PY@bf=\textbf\def\PY@tc##1{\textcolor[rgb]{0.50,0.00,0.50}{##1}}}
\@namedef{PY@tok@gd}{\def\PY@tc##1{\textcolor[rgb]{0.63,0.00,0.00}{##1}}}
\@namedef{PY@tok@gi}{\def\PY@tc##1{\textcolor[rgb]{0.00,0.52,0.00}{##1}}}
\@namedef{PY@tok@gr}{\def\PY@tc##1{\textcolor[rgb]{0.89,0.00,0.00}{##1}}}
\@namedef{PY@tok@ge}{\let\PY@it=\textit}
\@namedef{PY@tok@gs}{\let\PY@bf=\textbf}
\@namedef{PY@tok@ges}{\let\PY@bf=\textbf\let\PY@it=\textit}
\@namedef{PY@tok@gp}{\let\PY@bf=\textbf\def\PY@tc##1{\textcolor[rgb]{0.00,0.00,0.50}{##1}}}
\@namedef{PY@tok@go}{\def\PY@tc##1{\textcolor[rgb]{0.44,0.44,0.44}{##1}}}
\@namedef{PY@tok@gt}{\def\PY@tc##1{\textcolor[rgb]{0.00,0.27,0.87}{##1}}}
\@namedef{PY@tok@err}{\def\PY@bc##1{{\setlength{\fboxsep}{\string -\fboxrule}\fcolorbox[rgb]{1.00,0.00,0.00}{1,1,1}{\strut ##1}}}}
\@namedef{PY@tok@kc}{\let\PY@bf=\textbf\def\PY@tc##1{\textcolor[rgb]{0.00,0.50,0.00}{##1}}}
\@namedef{PY@tok@kd}{\let\PY@bf=\textbf\def\PY@tc##1{\textcolor[rgb]{0.00,0.50,0.00}{##1}}}
\@namedef{PY@tok@kn}{\let\PY@bf=\textbf\def\PY@tc##1{\textcolor[rgb]{0.00,0.50,0.00}{##1}}}
\@namedef{PY@tok@kr}{\let\PY@bf=\textbf\def\PY@tc##1{\textcolor[rgb]{0.00,0.50,0.00}{##1}}}
\@namedef{PY@tok@bp}{\def\PY@tc##1{\textcolor[rgb]{0.00,0.50,0.00}{##1}}}
\@namedef{PY@tok@fm}{\def\PY@tc##1{\textcolor[rgb]{0.00,0.00,1.00}{##1}}}
\@namedef{PY@tok@vc}{\def\PY@tc##1{\textcolor[rgb]{0.10,0.09,0.49}{##1}}}
\@namedef{PY@tok@vg}{\def\PY@tc##1{\textcolor[rgb]{0.10,0.09,0.49}{##1}}}
\@namedef{PY@tok@vi}{\def\PY@tc##1{\textcolor[rgb]{0.10,0.09,0.49}{##1}}}
\@namedef{PY@tok@vm}{\def\PY@tc##1{\textcolor[rgb]{0.10,0.09,0.49}{##1}}}
\@namedef{PY@tok@sa}{\def\PY@tc##1{\textcolor[rgb]{0.73,0.13,0.13}{##1}}}
\@namedef{PY@tok@sb}{\def\PY@tc##1{\textcolor[rgb]{0.73,0.13,0.13}{##1}}}
\@namedef{PY@tok@sc}{\def\PY@tc##1{\textcolor[rgb]{0.73,0.13,0.13}{##1}}}
\@namedef{PY@tok@dl}{\def\PY@tc##1{\textcolor[rgb]{0.73,0.13,0.13}{##1}}}
\@namedef{PY@tok@s2}{\def\PY@tc##1{\textcolor[rgb]{0.73,0.13,0.13}{##1}}}
\@namedef{PY@tok@sh}{\def\PY@tc##1{\textcolor[rgb]{0.73,0.13,0.13}{##1}}}
\@namedef{PY@tok@s1}{\def\PY@tc##1{\textcolor[rgb]{0.73,0.13,0.13}{##1}}}
\@namedef{PY@tok@mb}{\def\PY@tc##1{\textcolor[rgb]{0.40,0.40,0.40}{##1}}}
\@namedef{PY@tok@mf}{\def\PY@tc##1{\textcolor[rgb]{0.40,0.40,0.40}{##1}}}
\@namedef{PY@tok@mh}{\def\PY@tc##1{\textcolor[rgb]{0.40,0.40,0.40}{##1}}}
\@namedef{PY@tok@mi}{\def\PY@tc##1{\textcolor[rgb]{0.40,0.40,0.40}{##1}}}
\@namedef{PY@tok@il}{\def\PY@tc##1{\textcolor[rgb]{0.40,0.40,0.40}{##1}}}
\@namedef{PY@tok@mo}{\def\PY@tc##1{\textcolor[rgb]{0.40,0.40,0.40}{##1}}}
\@namedef{PY@tok@ch}{\let\PY@it=\textit\def\PY@tc##1{\textcolor[rgb]{0.24,0.48,0.48}{##1}}}
\@namedef{PY@tok@cm}{\let\PY@it=\textit\def\PY@tc##1{\textcolor[rgb]{0.24,0.48,0.48}{##1}}}
\@namedef{PY@tok@cpf}{\let\PY@it=\textit\def\PY@tc##1{\textcolor[rgb]{0.24,0.48,0.48}{##1}}}
\@namedef{PY@tok@c1}{\let\PY@it=\textit\def\PY@tc##1{\textcolor[rgb]{0.24,0.48,0.48}{##1}}}
\@namedef{PY@tok@cs}{\let\PY@it=\textit\def\PY@tc##1{\textcolor[rgb]{0.24,0.48,0.48}{##1}}}

\def\PYZbs{\char`\\}
\def\PYZus{\char`\_}
\def\PYZob{\char`\{}
\def\PYZcb{\char`\}}
\def\PYZca{\char`\^}
\def\PYZam{\char`\&}
\def\PYZlt{\char`\<}
\def\PYZgt{\char`\>}
\def\PYZsh{\char`\#}
\def\PYZpc{\char`\%}
\def\PYZdl{\char`\$}
\def\PYZhy{\char`\-}
\def\PYZsq{\char`\'}
\def\PYZdq{\char`\"}
\def\PYZti{\char`\~}
% for compatibility with earlier versions
\def\PYZat{@}
\def\PYZlb{[}
\def\PYZrb{]}
\makeatother


    % For linebreaks inside Verbatim environment from package fancyvrb.
    \makeatletter
        \newbox\Wrappedcontinuationbox
        \newbox\Wrappedvisiblespacebox
        \newcommand*\Wrappedvisiblespace {\textcolor{red}{\textvisiblespace}}
        \newcommand*\Wrappedcontinuationsymbol {\textcolor{red}{\llap{\tiny$\m@th\hookrightarrow$}}}
        \newcommand*\Wrappedcontinuationindent {3ex }
        \newcommand*\Wrappedafterbreak {\kern\Wrappedcontinuationindent\copy\Wrappedcontinuationbox}
        % Take advantage of the already applied Pygments mark-up to insert
        % potential linebreaks for TeX processing.
        %        {, <, #, %, $, ' and ": go to next line.
        %        _, }, ^, &, >, - and ~: stay at end of broken line.
        % Use of \textquotesingle for straight quote.
        \newcommand*\Wrappedbreaksatspecials {%
            \def\PYGZus{\discretionary{\char`\_}{\Wrappedafterbreak}{\char`\_}}%
            \def\PYGZob{\discretionary{}{\Wrappedafterbreak\char`\{}{\char`\{}}%
            \def\PYGZcb{\discretionary{\char`\}}{\Wrappedafterbreak}{\char`\}}}%
            \def\PYGZca{\discretionary{\char`\^}{\Wrappedafterbreak}{\char`\^}}%
            \def\PYGZam{\discretionary{\char`\&}{\Wrappedafterbreak}{\char`\&}}%
            \def\PYGZlt{\discretionary{}{\Wrappedafterbreak\char`\<}{\char`\<}}%
            \def\PYGZgt{\discretionary{\char`\>}{\Wrappedafterbreak}{\char`\>}}%
            \def\PYGZsh{\discretionary{}{\Wrappedafterbreak\char`\#}{\char`\#}}%
            \def\PYGZpc{\discretionary{}{\Wrappedafterbreak\char`\%}{\char`\%}}%
            \def\PYGZdl{\discretionary{}{\Wrappedafterbreak\char`\$}{\char`\$}}%
            \def\PYGZhy{\discretionary{\char`\-}{\Wrappedafterbreak}{\char`\-}}%
            \def\PYGZsq{\discretionary{}{\Wrappedafterbreak\textquotesingle}{\textquotesingle}}%
            \def\PYGZdq{\discretionary{}{\Wrappedafterbreak\char`\"}{\char`\"}}%
            \def\PYGZti{\discretionary{\char`\~}{\Wrappedafterbreak}{\char`\~}}%
        }
        % Some characters . , ; ? ! / are not pygmentized.
        % This macro makes them "active" and they will insert potential linebreaks
        \newcommand*\Wrappedbreaksatpunct {%
            \lccode`\~`\.\lowercase{\def~}{\discretionary{\hbox{\char`\.}}{\Wrappedafterbreak}{\hbox{\char`\.}}}%
            \lccode`\~`\,\lowercase{\def~}{\discretionary{\hbox{\char`\,}}{\Wrappedafterbreak}{\hbox{\char`\,}}}%
            \lccode`\~`\;\lowercase{\def~}{\discretionary{\hbox{\char`\;}}{\Wrappedafterbreak}{\hbox{\char`\;}}}%
            \lccode`\~`\:\lowercase{\def~}{\discretionary{\hbox{\char`\:}}{\Wrappedafterbreak}{\hbox{\char`\:}}}%
            \lccode`\~`\?\lowercase{\def~}{\discretionary{\hbox{\char`\?}}{\Wrappedafterbreak}{\hbox{\char`\?}}}%
            \lccode`\~`\!\lowercase{\def~}{\discretionary{\hbox{\char`\!}}{\Wrappedafterbreak}{\hbox{\char`\!}}}%
            \lccode`\~`\/\lowercase{\def~}{\discretionary{\hbox{\char`\/}}{\Wrappedafterbreak}{\hbox{\char`\/}}}%
            \catcode`\.\active
            \catcode`\,\active
            \catcode`\;\active
            \catcode`\:\active
            \catcode`\?\active
            \catcode`\!\active
            \catcode`\/\active
            \lccode`\~`\~
        }
    \makeatother

    \let\OriginalVerbatim=\Verbatim
    \makeatletter
    \renewcommand{\Verbatim}[1][1]{%
        %\parskip\z@skip
        \sbox\Wrappedcontinuationbox {\Wrappedcontinuationsymbol}%
        \sbox\Wrappedvisiblespacebox {\FV@SetupFont\Wrappedvisiblespace}%
        \def\FancyVerbFormatLine ##1{\hsize\linewidth
            \vtop{\raggedright\hyphenpenalty\z@\exhyphenpenalty\z@
                \doublehyphendemerits\z@\finalhyphendemerits\z@
                \strut ##1\strut}%
        }%
        % If the linebreak is at a space, the latter will be displayed as visible
        % space at end of first line, and a continuation symbol starts next line.
        % Stretch/shrink are however usually zero for typewriter font.
        \def\FV@Space {%
            \nobreak\hskip\z@ plus\fontdimen3\font minus\fontdimen4\font
            \discretionary{\copy\Wrappedvisiblespacebox}{\Wrappedafterbreak}
            {\kern\fontdimen2\font}%
        }%

        % Allow breaks at special characters using \PYG... macros.
        \Wrappedbreaksatspecials
        % Breaks at punctuation characters . , ; ? ! and / need catcode=\active
        \OriginalVerbatim[#1,codes*=\Wrappedbreaksatpunct]%
    }
    \makeatother

    % Exact colors from NB
    \definecolor{incolor}{HTML}{303F9F}
    \definecolor{outcolor}{HTML}{D84315}
    \definecolor{cellborder}{HTML}{CFCFCF}
    \definecolor{cellbackground}{HTML}{F7F7F7}

    % prompt
    \makeatletter
    \newcommand{\boxspacing}{\kern\kvtcb@left@rule\kern\kvtcb@boxsep}
    \makeatother
    \newcommand{\prompt}[4]{
        {\ttfamily\llap{{\color{#2}[#3]:\hspace{3pt}#4}}\vspace{-\baselineskip}}
    }
    

    
    % Prevent overflowing lines due to hard-to-break entities
    \sloppy
    % Setup hyperref package
    \hypersetup{
      breaklinks=true,  % so long urls are correctly broken across lines
      colorlinks=true,
      urlcolor=urlcolor,
      linkcolor=linkcolor,
      citecolor=citecolor,
      }
    % Slightly bigger margins than the latex defaults
    
    \geometry{verbose,tmargin=1in,bmargin=1in,lmargin=1in,rmargin=1in}
    
    

\begin{document}
    
    \maketitle
    
    

    
    \hypertarget{project-3-multiplication-rate-across-generations-as-a-function-of-aspect-ratio-and-volume}{%
\section{Project 3: Multiplication rate across generations as a function
of Aspect Ratio and
Volume}\label{project-3-multiplication-rate-across-generations-as-a-function-of-aspect-ratio-and-volume}}

    \textbf{General Background Information}

Consider the core of a nuclear reactor. neutrons are, obviously, much
smaller than the gaps between atoms and their neutral charge causes them
to not have attractive force to free electrons or any repellant force,
meaning that they can fairly easily escape the metallic bounds of
reactor cores once atoms are split through fission. Therefore, it's
fairly accurate to model the neutrons in a reactor as having no physical
bounds whatsoever (in the form of a potential field blocking them from
leaving), and this simulation seeks to investigate how often neutrons
continue to replicate within this reactor.

As a nuclear reaction progresses, neutrons ``multiply'' into several
more by triggering further fission reactions in other atoms, and the
amount of free neutrons that are produced from the triggered fission by
the previous neutron are considered to be successfully replicated.
However, as the neutrons leak out of the reactor, they're unlikely to
return within the core to continue replications, and it's also unlikely
for them to replicate outside of the core, so for the purposes of this
reactor, we're only interested in the neutrons that replicate within the
core, and any neutrons that leak out of the reactor can be said to be
effectively lost to the reactor altogether.

Therefore, we can model this system using a non-physical bound for the
reactor core and a growing tree of randomly-generated neutrons for the
replications: - The reactor core is like a ``line in the sand'' rather
than a wall: it has no power to stop neutrons from exiting, and they
will do so randomly, but once they cross the boundary, further progress
is entirely not considered. - The neutrons are assumed to always induce
a chain reaction that produces exactly two further free neutrons every
time, but each neutron that is produced is then sent off in a
randomly-determined direction and distance from the reaction that
produced it, and if it ends up outside of the reactor core, then that
neutron is considered ``lost.''

This, of course, requires an important distinction: each neutron is
implicitly defined to replicate two further neutrons each time, but the
effective replication rate is only considering the neutrons that remain
in the reactor, so for example, a neutron could induce a reaction to
produce two free neutrons, but if one neutron exits the reactor and the
other ends within the reactor, then that neutron is said to have only
replicated a single neutron, and the single neutron that remained in the
reactor will induce another reaction.

\textbf{Research Question}

With all of the above information in mind, for a given core shape, the
reactor core is going to eventually reach a specific replication rate.
Unstable reactor geometries would have replication rates approach zero
due to too much leakage, and stable reactor geometries will have
replication rates that stay roughly constant at some non-zero value. The
focus of this simulation, in particular, is to see how this long-run
replication rate changes as the geometry of the core changes.

To do this, I've defined the reactor core's geometry as a rectangular
prism, so for simplicity, we can call it a ``box.'' This means that the
box's geometry has three parameters: a length \(\ell\), width \(w\), and
height \(h\). To keep this simulation simple, we'll make the length
\(\ell\) a function of the width and the height:
\[\ell = \frac{1}{2} (w + h),\] Then, we can just vary the volume \(V\)
and the aspect ratio \(R\) with each being explicit functions of only
width and height: \begin{align}
R &= \frac{w}{h} \\
V &= w \ell h = wh\frac{1}{2}(w + h)
\end{align} then each of the parameters are only a function of \(R\)
and \(V\): \[\begin{cases}
w &= R \left(\frac{2V}{R(R+1)}\right)^\frac{1}{3} \\
h &= \left(\frac{2V}{R(R+1)}\right)^\frac{1}{3} \\
\ell &= \frac{1}{2} (R + 1) \left(\frac{2V}{R(R+1)}\right)^\frac{1}{3}
\end{cases}.\]

Now we have a two-dimensional phase space for the simulation, aspect
ratio \(R\) and volume \(V\), and thus, each box geometry can be defined
as a phase coordinate \((R,V)\). For each geometry, we can perform a
singular simulation with a random starting position, then record the
replication rate \(k_i\) (for an individual simulation \(i\)) after a
certain number of generations, \(g\). (It's incredibly important that we
pick \(g\) correctly to best characterize our system, but for now, we'll
just vaguely state that it's a large enough number to characterize the
long-term behavior of the system.) We can then perform this individual
simulation a large number of times with more random starting positions,
making the overall simulation \(f\) a Monte-Carlo simulation that
estimates the long-term multiplication rate as \(\mathbf{k}\) based on
the mean and standard error, \(\overline{k} + \delta k\).

(To be clear, a Monte Carlo simulation is a general class of stochastic
(random) methods for performing computational calculations. In this
case, we're calculating a value through performing a high volume of
random simulations to produce a distribution of values (multiplication
rates), then selecting the mean and standard error of that distribution
to act as the final calculation of the value.)

Therefore, thus far, we're producing a mapping using our simulation
\(f\): \[f(R,V) \mapsto \mathbf{k} = \overline{k} + \delta k.\] In
particular, however, we'll actually do some preliminary work to ensure
that the relative error \(\delta k/k\) is sufficiently small, and then,
we'll only really care about the mean multiplication rate as we'll know
that it is accurate, reducing this mapping down to just
\[f(R,V) \mapsto \overline{k}\]

Thus, simply put, the overarching goal is to calculate this mapping for
each \((R,V)\) over some two-dimensional phase domain \(D\)
\[D = \{ R_{\min} \le R \le R_{\max} \} \otimes \{ V_{\min} \le V \le V_{\max} \}\]
and then analyze the results we observe.

In plain language, the research question is ``\emph{How does the average
multiplication rate across generations change as a function of the
aspect ratio and area of the containing box?}''

For the hypothesis, one can safely assume from basic intuition that
smaller volumes should be less likely to retain neutrons, as at an
infinite volume of the reactor core (and also assuming the entirety of
all space is filled with fuel in the realistic case), the reaction will
proceed forever and grow over all space. However, the aspect ratio is
much more interesting. We can reasonably assume that there should be the
greatest preference to more even aspect ratios (closer to 1:1), because
if the width, height, or length of the box is approximately zero, any
movement of a neutron along the approximately zero axis in
three-dimensional space will result in ejection.

\textbf{A Quick Note on Lazy Naming}

Before continuing, it's important to note one core stylistic choice that
I've made: the starting neutron(s) is technically the first generation,
but I'm referring to it as the zero-th generation. For generations
beyond the first, I may use the terms ``X-th generation'' and
``generation X'' interchangably, so for example, I may refer to the
generation two as the second generation when it's really the third, as
the only generation where it is really important to make this
distinction is when making the first generation (generation zero) and
generation one distinct (and I will not refer to generation one as the
first generation).

    \textbf{Code Block Summary:} Standard Imports.

    \begin{tcolorbox}[breakable, size=fbox, boxrule=1pt, pad at break*=1mm,colback=cellbackground, colframe=cellborder]
\prompt{In}{incolor}{1}{\boxspacing}
\begin{Verbatim}[commandchars=\\\{\}]
\PY{k+kn}{import} \PY{n+nn}{numpy} \PY{k}{as} \PY{n+nn}{np}
\PY{k+kn}{from} \PY{n+nn}{matplotlib} \PY{k+kn}{import} \PY{n}{pyplot} \PY{k}{as} \PY{n}{plt}
\end{Verbatim}
\end{tcolorbox}

    \textbf{Code Block Summary:} Non-standard Imports. This project utilizes
dataclasses \href{https://docs.python.org/3/library/dataclasses.html}{1}
for data storage, the collections deque
\href{https://docs.python.org/3/library/collections.html\#collections.deque}{2}
for non-recursive breadth-first-search, and pillow
\href{https://pypi.org/project/pillow/}{3} with io
\href{https://docs.python.org/3/library/io.html}{4} for image storage.
Additionally, I'm using itertools
\href{https://docs.python.org/3/library/itertools.html}{6} to produce
the list of coordinates making up a 3-D box from a list of dimensions
and matplotlib's Axes3D
\href{https://matplotlib.org/3.3.3/api/_as_gen/mpl_toolkits.mplot3d.axes3d.Axes3D.html}{7}
for plotting said boxes. Lastly, there is time
\href{https://docs.python.org/3/library/time.html}{9} for function
timing (debugging information).

    \begin{tcolorbox}[breakable, size=fbox, boxrule=1pt, pad at break*=1mm,colback=cellbackground, colframe=cellborder]
\prompt{In}{incolor}{2}{\boxspacing}
\begin{Verbatim}[commandchars=\\\{\}]
\PY{k+kn}{from} \PY{n+nn}{dataclasses} \PY{k+kn}{import} \PY{n}{dataclass}
\PY{k+kn}{from} \PY{n+nn}{collections} \PY{k+kn}{import} \PY{n}{deque}
\PY{k+kn}{from} \PY{n+nn}{PIL} \PY{k+kn}{import} \PY{n}{Image}
\PY{k+kn}{import} \PY{n+nn}{io}

\PY{k+kn}{from} \PY{n+nn}{itertools} \PY{k+kn}{import} \PY{n}{product}\PY{p}{,} \PY{n}{combinations}
\PY{k+kn}{from} \PY{n+nn}{mpl\PYZus{}toolkits}\PY{n+nn}{.}\PY{n+nn}{mplot3d} \PY{k+kn}{import} \PY{n}{Axes3D}

\PY{k+kn}{import} \PY{n+nn}{time}
\end{Verbatim}
\end{tcolorbox}

    \hypertarget{non-tree-related-helper-functions}{%
\subsection{(Non-Tree-Related) Helper
Functions}\label{non-tree-related-helper-functions}}

    \textbf{Code Block Summary:} (\texttt{calculate\_dimensions}) This is a
helper function to obtain the dimensions of the box given an area and
aspect ratio. This can be done mathematically.

Let \(V\) be the area and \(R\) be the aspect ratio, and \(w\) be the
width and \(h\) be the height. As a somewhat trivial definition, to keep
the phase only in terms of \(w\) and \(h\), we can define the length as
\(\ell = \frac{1}{2}(w+h)\) (the average between the two values).

Therefore, the aspect ratio (which can still be kept 2-dimensional) is
\(R = \frac{w}{h}\) and the volume is \begin{equation}
V = wh\ell = wh\frac{1}{2}(w + h).
\end{equation} and knowing that \(w = Rh\), this becomes
\[V = \frac{1}{2} Rh^3 (R + 1)\] which can be rewritten as
\begin{equation}
h = \left(\frac{2V}{R(R+1)}\right)^\frac{1}{3}
\end{equation}

    \begin{tcolorbox}[breakable, size=fbox, boxrule=1pt, pad at break*=1mm,colback=cellbackground, colframe=cellborder]
\prompt{In}{incolor}{3}{\boxspacing}
\begin{Verbatim}[commandchars=\\\{\}]
\PY{k}{def} \PY{n+nf}{calculate\PYZus{}dimensions}\PY{p}{(}\PY{n}{aspect\PYZus{}ratio}\PY{p}{:} \PY{n+nb}{float}\PY{p}{,} \PY{n}{volume}\PY{p}{:} \PY{n+nb}{float}\PY{p}{)}\PY{p}{:}
    \PY{n}{height} \PY{o}{=} \PY{p}{(}\PY{l+m+mi}{2} \PY{o}{*} \PY{n}{volume} \PY{o}{/} \PY{p}{(}\PY{n}{aspect\PYZus{}ratio} \PY{o}{*} \PY{p}{(}\PY{n}{aspect\PYZus{}ratio} \PY{o}{+} \PY{l+m+mi}{1}\PY{p}{)}\PY{p}{)}\PY{p}{)} \PY{o}{*}\PY{o}{*} \PY{p}{(}\PY{l+m+mi}{1}\PY{o}{/}\PY{l+m+mi}{3}\PY{p}{)}
    \PY{n}{width}  \PY{o}{=} \PY{n}{aspect\PYZus{}ratio} \PY{o}{*} \PY{n}{height}
    \PY{n}{length} \PY{o}{=} \PY{p}{(}\PY{n}{width} \PY{o}{+} \PY{n}{height}\PY{p}{)} \PY{o}{/} \PY{l+m+mi}{2}
    
    \PY{k}{return} \PY{p}{(}\PY{n}{width}\PY{p}{,} \PY{n}{length}\PY{p}{,} \PY{n}{height}\PY{p}{)}
\end{Verbatim}
\end{tcolorbox}

    \textbf{Code Block Summary:} (\texttt{save\_graph\_and\_close}) Standard
graph saving function. This is the description of this function from
Project 2: \textgreater{} This function saves graphs to memory to be
stored into variables, and from there, I can called them again using
\texttt{display}. This workflow makes managing the Jupyter notebook
easy, and it also allows me to define all of my graphs together but
display them in separate cells.

    \begin{tcolorbox}[breakable, size=fbox, boxrule=1pt, pad at break*=1mm,colback=cellbackground, colframe=cellborder]
\prompt{In}{incolor}{4}{\boxspacing}
\begin{Verbatim}[commandchars=\\\{\}]
\PY{k}{def} \PY{n+nf}{save\PYZus{}graph\PYZus{}and\PYZus{}close}\PY{p}{(}\PY{p}{)}\PY{p}{:}
    \PY{n}{buf} \PY{o}{=} \PY{n}{io}\PY{o}{.}\PY{n}{BytesIO}\PY{p}{(}\PY{p}{)}
    \PY{n}{plt}\PY{o}{.}\PY{n}{savefig}\PY{p}{(}\PY{n}{buf}\PY{p}{,} \PY{n+nb}{format}\PY{o}{=}\PY{l+s+s1}{\PYZsq{}}\PY{l+s+s1}{png}\PY{l+s+s1}{\PYZsq{}}\PY{p}{)}
    \PY{n}{buf}\PY{o}{.}\PY{n}{seek}\PY{p}{(}\PY{l+m+mi}{0}\PY{p}{)}
    \PY{n}{graph} \PY{o}{=} \PY{n}{Image}\PY{o}{.}\PY{n}{open}\PY{p}{(}\PY{n}{buf}\PY{p}{)}
    \PY{n}{plt}\PY{o}{.}\PY{n}{close}\PY{p}{(}\PY{p}{)}
    \PY{k}{return} \PY{n}{graph}
\end{Verbatim}
\end{tcolorbox}

    \hypertarget{tree-generation}{%
\subsection{Tree Generation}\label{tree-generation}}

    \textbf{Code Block Summary:} (\texttt{get\_next\_neutron\_position})
This function will randomly determine the ending position of a neutron
given a starting position. It does this by first determining a random
angular direction \(\theta \in [0, 2\pi)\) and an azimuthal direction
\(\phi \in [0, \pi]\) to propagate in alongside a random distance
calculated as \(d \propto \exp(\lambda)\). From there, the new position
can be calculated using the old position \(\mathbf{r}_0\) as
\[\mathbf{r} = \mathbf{r}_0 + d \langle \sin \phi \cos \theta, \sin \phi \sin \theta, \cos \phi \rangle\]

    \begin{tcolorbox}[breakable, size=fbox, boxrule=1pt, pad at break*=1mm,colback=cellbackground, colframe=cellborder]
\prompt{In}{incolor}{5}{\boxspacing}
\begin{Verbatim}[commandchars=\\\{\}]
\PY{n}{MEAN\PYZus{}FREE\PYZus{}PATH} \PY{o}{=} \PY{l+m+mf}{2.65}

\PY{k}{def} \PY{n+nf}{get\PYZus{}next\PYZus{}neutron\PYZus{}position}\PY{p}{(}\PY{n}{current\PYZus{}position}\PY{p}{:} \PY{n}{np}\PY{o}{.}\PY{n}{array}\PY{p}{)} \PY{o}{\PYZhy{}}\PY{o}{\PYZgt{}} \PY{n}{np}\PY{o}{.}\PY{n}{array}\PY{p}{:}
\PY{+w}{    }\PY{l+s+sd}{\PYZdq{}\PYZdq{}\PYZdq{}}
\PY{l+s+sd}{    Generates a random direction for the neutron }
\PY{l+s+sd}{    and uses the mean\PYZhy{}free\PYZhy{}path to calculate the}
\PY{l+s+sd}{    next position for the neutron.}
\PY{l+s+sd}{    \PYZdq{}\PYZdq{}\PYZdq{}}
    \PY{n}{theta} \PY{o}{=} \PY{n}{np}\PY{o}{.}\PY{n}{random}\PY{o}{.}\PY{n}{uniform}\PY{p}{(}\PY{l+m+mi}{0}\PY{p}{,} \PY{l+m+mi}{2} \PY{o}{*} \PY{n}{np}\PY{o}{.}\PY{n}{pi}\PY{p}{)}
    \PY{n}{phi}   \PY{o}{=} \PY{n}{np}\PY{o}{.}\PY{n}{random}\PY{o}{.}\PY{n}{uniform}\PY{p}{(}\PY{l+m+mi}{0}\PY{p}{,} \PY{n}{np}\PY{o}{.}\PY{n}{pi}\PY{p}{)}
    \PY{n}{direction} \PY{o}{=} \PY{n}{np}\PY{o}{.}\PY{n}{array}\PY{p}{(}\PY{p}{[} \PY{n}{np}\PY{o}{.}\PY{n}{sin}\PY{p}{(}\PY{n}{phi}\PY{p}{)} \PY{o}{*} \PY{n}{np}\PY{o}{.}\PY{n}{cos}\PY{p}{(}\PY{n}{theta}\PY{p}{)}\PY{p}{,} \PY{n}{np}\PY{o}{.}\PY{n}{sin}\PY{p}{(}\PY{n}{phi}\PY{p}{)} \PY{o}{*} \PY{n}{np}\PY{o}{.}\PY{n}{sin}\PY{p}{(}\PY{n}{theta}\PY{p}{)}\PY{p}{,} \PY{n}{np}\PY{o}{.}\PY{n}{cos}\PY{p}{(}\PY{n}{phi}\PY{p}{)} \PY{p}{]}\PY{p}{)}
    \PY{n}{distance}  \PY{o}{=} \PY{n}{np}\PY{o}{.}\PY{n}{random}\PY{o}{.}\PY{n}{exponential}\PY{p}{(}\PY{n}{MEAN\PYZus{}FREE\PYZus{}PATH}\PY{p}{)}
    
    \PY{k}{return} \PY{n}{current\PYZus{}position} \PY{o}{+} \PY{n}{distance} \PY{o}{*} \PY{n}{direction}
\end{Verbatim}
\end{tcolorbox}

    \textbf{Code Block Summary:} (\texttt{class\ NeutronNode}) This is a
structure to store Neutron information as a tree element. Each Neutron
stores its own information (that being, it's starting position
\(\mathbf{r}_0\), ending position \(\mathbf{r}_{1}\), and then whether
or not it was lost) alongside a list of pointers to each of its
``children'' Neutrons.

    \begin{tcolorbox}[breakable, size=fbox, boxrule=1pt, pad at break*=1mm,colback=cellbackground, colframe=cellborder]
\prompt{In}{incolor}{6}{\boxspacing}
\begin{Verbatim}[commandchars=\\\{\}]
\PY{n+nd}{@dataclass}
\PY{k}{class} \PY{n+nc}{NeutronNode}\PY{p}{:}
    \PY{n}{start\PYZus{}pos}\PY{p}{:} \PY{n}{np}\PY{o}{.}\PY{n}{array}          \PY{o}{=} \PY{k+kc}{None}
    \PY{n}{end\PYZus{}pos}\PY{p}{:}   \PY{n}{np}\PY{o}{.}\PY{n}{array}          \PY{o}{=} \PY{k+kc}{None}
    \PY{n}{children}\PY{p}{:}  \PY{n+nb}{list}              \PY{o}{=} \PY{k+kc}{None}
    \PY{n}{lost}\PY{p}{:}      \PY{n+nb}{bool}              \PY{o}{=} \PY{k+kc}{False}
\end{Verbatim}
\end{tcolorbox}

    \textbf{Code Block Summary:} (\texttt{EXAMPLE\_TREE}) This is an example
tree used for testing purposes, and it shows the power of the tree-based
data structure for storing Neutron information.

    \begin{tcolorbox}[breakable, size=fbox, boxrule=1pt, pad at break*=1mm,colback=cellbackground, colframe=cellborder]
\prompt{In}{incolor}{7}{\boxspacing}
\begin{Verbatim}[commandchars=\\\{\}]
\PY{n}{EXAMPLE\PYZus{}TREE} \PY{o}{=} \PY{n}{NeutronNode}\PY{p}{(}
        \PY{n}{start\PYZus{}pos} \PY{o}{=} \PY{n}{np}\PY{o}{.}\PY{n}{array}\PY{p}{(}\PY{p}{[}\PY{l+m+mi}{0}\PY{p}{,} \PY{l+m+mi}{0}\PY{p}{,} \PY{l+m+mi}{0}\PY{p}{]}\PY{p}{)}\PY{p}{,} 
        \PY{n}{end\PYZus{}pos}   \PY{o}{=} \PY{n}{np}\PY{o}{.}\PY{n}{array}\PY{p}{(}\PY{p}{[}\PY{l+m+mi}{1}\PY{p}{,} \PY{l+m+mi}{1}\PY{p}{,} \PY{l+m+mi}{0}\PY{p}{]}\PY{p}{)}\PY{p}{,}
        
        \PY{n}{children}  \PY{o}{=} \PY{p}{[}
            \PY{n}{NeutronNode}\PY{p}{(}
                \PY{n}{start\PYZus{}pos} \PY{o}{=} \PY{n}{np}\PY{o}{.}\PY{n}{array}\PY{p}{(}\PY{p}{[}\PY{l+m+mi}{1}\PY{p}{,} \PY{l+m+mi}{1}\PY{p}{,} \PY{l+m+mi}{0}\PY{p}{]}\PY{p}{)}\PY{p}{,} 
                \PY{n}{end\PYZus{}pos}   \PY{o}{=} \PY{n}{np}\PY{o}{.}\PY{n}{array}\PY{p}{(}\PY{p}{[}\PY{l+m+mi}{2}\PY{p}{,} \PY{l+m+mi}{1}\PY{p}{,} \PY{l+m+mi}{0}\PY{p}{]}\PY{p}{)}\PY{p}{,}
                
                \PY{n}{children}  \PY{o}{=} \PY{p}{[}
                    \PY{n}{NeutronNode}\PY{p}{(}
                        \PY{n}{start\PYZus{}pos} \PY{o}{=} \PY{n}{np}\PY{o}{.}\PY{n}{array}\PY{p}{(}\PY{p}{[}\PY{l+m+mi}{2}\PY{p}{,} \PY{l+m+mi}{1}\PY{p}{,} \PY{l+m+mi}{0}\PY{p}{]}\PY{p}{)}\PY{p}{,}
                        \PY{n}{end\PYZus{}pos}   \PY{o}{=} \PY{n}{np}\PY{o}{.}\PY{n}{array}\PY{p}{(}\PY{p}{[}\PY{l+m+mi}{4}\PY{p}{,} \PY{l+m+mi}{3}\PY{p}{,} \PY{l+m+mi}{0}\PY{p}{]}\PY{p}{)}
                    \PY{p}{)}\PY{p}{,}
                    
                    \PY{n}{NeutronNode}\PY{p}{(}
                        \PY{n}{start\PYZus{}pos} \PY{o}{=} \PY{n}{np}\PY{o}{.}\PY{n}{array}\PY{p}{(}\PY{p}{[}\PY{l+m+mi}{2}\PY{p}{,} \PY{l+m+mi}{1}\PY{p}{,} \PY{l+m+mi}{0}\PY{p}{]}\PY{p}{)}\PY{p}{,}
                        \PY{n}{end\PYZus{}pos}   \PY{o}{=} \PY{n}{np}\PY{o}{.}\PY{n}{array}\PY{p}{(}\PY{p}{[}\PY{l+m+mi}{3}\PY{p}{,} \PY{l+m+mi}{2}\PY{p}{,} \PY{l+m+mi}{0}\PY{p}{]}\PY{p}{)}
                    \PY{p}{)}
                \PY{p}{]}
            \PY{p}{)}\PY{p}{,}
            
            \PY{n}{NeutronNode}\PY{p}{(}
                \PY{n}{start\PYZus{}pos} \PY{o}{=} \PY{n}{np}\PY{o}{.}\PY{n}{array}\PY{p}{(}\PY{p}{[}\PY{l+m+mi}{1}\PY{p}{,} \PY{l+m+mi}{1}\PY{p}{,} \PY{l+m+mi}{0}\PY{p}{]}\PY{p}{)}\PY{p}{,} 
                \PY{n}{end\PYZus{}pos}   \PY{o}{=} \PY{n}{np}\PY{o}{.}\PY{n}{array}\PY{p}{(}\PY{p}{[}\PY{l+m+mi}{1}\PY{p}{,} \PY{l+m+mi}{2}\PY{p}{,} \PY{l+m+mi}{0}\PY{p}{]}\PY{p}{)}
            \PY{p}{)}
        \PY{p}{]}
    \PY{p}{)}
\end{Verbatim}
\end{tcolorbox}

    \textbf{Code Block Summary:} (\texttt{print\_tree}) This is a simple
recursive helper method to print a tree in a file-tree format, allowing
for quick debugging of Neutron Trees.

    \begin{tcolorbox}[breakable, size=fbox, boxrule=1pt, pad at break*=1mm,colback=cellbackground, colframe=cellborder]
\prompt{In}{incolor}{8}{\boxspacing}
\begin{Verbatim}[commandchars=\\\{\}]
\PY{k}{def} \PY{n+nf}{print\PYZus{}tree}\PY{p}{(}\PY{n}{node}\PY{p}{:} \PY{n}{NeutronNode}\PY{p}{,} \PY{n}{indent}\PY{p}{:} \PY{n+nb}{int} \PY{o}{=} \PY{l+m+mi}{0}\PY{p}{)}\PY{p}{:}
\PY{+w}{    }\PY{l+s+sd}{\PYZdq{}\PYZdq{}\PYZdq{}}
\PY{l+s+sd}{    Recursively pretty\PYZhy{}print a neutron trajectory tree.}
\PY{l+s+sd}{    \PYZdq{}\PYZdq{}\PYZdq{}}
    \PY{k}{if} \PY{n}{node} \PY{o+ow}{is} \PY{k+kc}{None}\PY{p}{:}
        \PY{n+nb}{print}\PY{p}{(}\PY{l+s+s2}{\PYZdq{}}\PY{l+s+s2}{ }\PY{l+s+s2}{\PYZdq{}} \PY{o}{*} \PY{n}{indent} \PY{o}{+} \PY{l+s+s2}{\PYZdq{}}\PY{l+s+s2}{(empty node)}\PY{l+s+s2}{\PYZdq{}}\PY{p}{)}
        \PY{k}{return}
    
    \PY{n}{space} \PY{o}{=} \PY{l+s+s2}{\PYZdq{}}\PY{l+s+s2}{  }\PY{l+s+s2}{\PYZdq{}} \PY{o}{*} \PY{n}{indent}
    \PY{n}{start} \PY{o}{=} \PY{n+nb}{tuple}\PY{p}{(}\PY{n}{np}\PY{o}{.}\PY{n}{round}\PY{p}{(}\PY{n}{node}\PY{o}{.}\PY{n}{start\PYZus{}pos}\PY{p}{,} \PY{l+m+mi}{2}\PY{p}{)}\PY{o}{.}\PY{n}{tolist}\PY{p}{(}\PY{p}{)}\PY{p}{)} \PY{k}{if} \PY{n}{node}\PY{o}{.}\PY{n}{start\PYZus{}pos} \PY{o+ow}{is} \PY{o+ow}{not} \PY{k+kc}{None} \PY{k}{else} \PY{k+kc}{None}
    \PY{n}{end}   \PY{o}{=} \PY{n+nb}{tuple}\PY{p}{(}\PY{n}{np}\PY{o}{.}\PY{n}{round}\PY{p}{(}\PY{n}{node}\PY{o}{.}\PY{n}{end\PYZus{}pos}\PY{p}{,} \PY{l+m+mi}{2}\PY{p}{)}\PY{o}{.}\PY{n}{tolist}\PY{p}{(}\PY{p}{)}\PY{p}{)} \PY{k}{if} \PY{n}{node}\PY{o}{.}\PY{n}{end\PYZus{}pos} \PY{o+ow}{is} \PY{o+ow}{not} \PY{k+kc}{None} \PY{k}{else} \PY{k+kc}{None}
    \PY{n}{status} \PY{o}{=} \PY{l+s+s2}{\PYZdq{}}\PY{l+s+s2}{lost}\PY{l+s+s2}{\PYZdq{}} \PY{k}{if} \PY{n}{node}\PY{o}{.}\PY{n}{lost} \PY{k}{else} \PY{l+s+s2}{\PYZdq{}}\PY{l+s+s2}{active}\PY{l+s+s2}{\PYZdq{}}
    
    \PY{n+nb}{print}\PY{p}{(}\PY{l+s+sa}{f}\PY{l+s+s2}{\PYZdq{}}\PY{l+s+si}{\PYZob{}}\PY{n}{space}\PY{l+s+si}{\PYZcb{}}\PY{l+s+s2}{-> }\PY{l+s+si}{\PYZob{}}\PY{n}{status}\PY{o}{.}\PY{n}{upper}\PY{p}{(}\PY{p}{)}\PY{l+s+si}{\PYZcb{}}\PY{l+s+s2}{  start=}\PY{l+s+si}{\PYZob{}}\PY{n}{start}\PY{l+s+si}{\PYZcb{}}\PY{l+s+s2}{  ->  end=}\PY{l+s+si}{\PYZob{}}\PY{n}{end}\PY{l+s+si}{\PYZcb{}}\PY{l+s+s2}{\PYZdq{}}\PY{p}{)}
    
    \PY{k}{if} \PY{n}{node}\PY{o}{.}\PY{n}{children}\PY{p}{:}
        \PY{k}{for} \PY{n}{child} \PY{o+ow}{in} \PY{n}{node}\PY{o}{.}\PY{n}{children}\PY{p}{:}
            \PY{n}{print\PYZus{}tree}\PY{p}{(}\PY{n}{child}\PY{p}{,} \PY{n}{indent} \PY{o}{+} \PY{l+m+mi}{1}\PY{p}{)}
            
\PY{n}{print\PYZus{}tree}\PY{p}{(}\PY{n}{EXAMPLE\PYZus{}TREE}\PY{p}{)}
\end{Verbatim}
\end{tcolorbox}

    \begin{Verbatim}[commandchars=\\\{\}]
-> ACTIVE  start=(0, 0, 0)  ->  end=(1, 1, 0)
  -> ACTIVE  start=(1, 1, 0)  ->  end=(2, 1, 0)
    -> ACTIVE  start=(2, 1, 0)  ->  end=(4, 3, 0)
    -> ACTIVE  start=(2, 1, 0)  ->  end=(3, 2, 0)
  -> ACTIVE  start=(1, 1, 0)  ->  end=(1, 2, 0)
    \end{Verbatim}

    \textbf{Code Block Summary:} (\texttt{traverse\_tree}) A non-recursive
helper method for traversing a tree. It's used to simplify tree
traversal from having to implement a tree traversal function each time
with using Python's built-in collection iteration syntax, i.e.,

\begin{Shaded}
\begin{Highlighting}[]
\ControlFlowTok{for}\NormalTok{ start\_pos, end\_pos, generation }\KeywordTok{in}\NormalTok{ traverse\_tree(tree):}
\NormalTok{    do\_something(start\_pos, end\_pos, generation)}
\end{Highlighting}
\end{Shaded}

This helper is implemented using non-recursive Breadth-first search
\href{https://en.wikipedia.org/wiki/Breadth-first_search}{6}.

    \begin{tcolorbox}[breakable, size=fbox, boxrule=1pt, pad at break*=1mm,colback=cellbackground, colframe=cellborder]
\prompt{In}{incolor}{9}{\boxspacing}
\begin{Verbatim}[commandchars=\\\{\}]
\PY{k}{def} \PY{n+nf}{traverse\PYZus{}tree}\PY{p}{(}\PY{n}{root}\PY{p}{:} \PY{n}{NeutronNode}\PY{p}{)}\PY{p}{:}
\PY{+w}{    }\PY{l+s+sd}{\PYZdq{}\PYZdq{}\PYZdq{}}
\PY{l+s+sd}{    A simple method for traversing the tree using BFS}
\PY{l+s+sd}{    without having to re\PYZhy{}implement it each time.}
\PY{l+s+sd}{    \PYZdq{}\PYZdq{}\PYZdq{}}
    \PY{k}{if} \PY{n}{root} \PY{o+ow}{is} \PY{k+kc}{None}\PY{p}{:}
        \PY{k}{return}

    \PY{c+c1}{\PYZsh{} (node, generation)}
    \PY{n}{queue} \PY{o}{=} \PY{n}{deque}\PY{p}{(}\PY{p}{[}\PY{p}{(}\PY{n}{root}\PY{p}{,} \PY{l+m+mi}{0}\PY{p}{)}\PY{p}{]}\PY{p}{)}

    \PY{k}{while} \PY{n}{queue}\PY{p}{:}
        \PY{n}{node}\PY{p}{,} \PY{n}{generation} \PY{o}{=} \PY{n}{queue}\PY{o}{.}\PY{n}{popleft}\PY{p}{(}\PY{p}{)}

        \PY{c+c1}{\PYZsh{} yield data for the current node}
        \PY{k}{yield} \PY{n}{node}\PY{p}{,} \PY{n}{generation}

        \PY{c+c1}{\PYZsh{} enqueue children if they exist}
        \PY{k}{if} \PY{n}{node}\PY{o}{.}\PY{n}{children}\PY{p}{:}
            \PY{k}{for} \PY{n}{child} \PY{o+ow}{in} \PY{n}{node}\PY{o}{.}\PY{n}{children}\PY{p}{:}
                \PY{n}{queue}\PY{o}{.}\PY{n}{append}\PY{p}{(}\PY{p}{(}\PY{n}{child}\PY{p}{,} \PY{n}{generation} \PY{o}{+} \PY{l+m+mi}{1}\PY{p}{)}\PY{p}{)}
\end{Verbatim}
\end{tcolorbox}

    \begin{tcolorbox}[breakable, size=fbox, boxrule=1pt, pad at break*=1mm,colback=cellbackground, colframe=cellborder]
\prompt{In}{incolor}{10}{\boxspacing}
\begin{Verbatim}[commandchars=\\\{\}]
\PY{k}{def} \PY{n+nf}{in\PYZus{}bound}\PY{p}{(}\PY{n}{value}\PY{p}{,} \PY{n}{bound}\PY{p}{)}\PY{p}{:}
\PY{+w}{    }\PY{l+s+sd}{\PYZdq{}\PYZdq{}\PYZdq{}}
\PY{l+s+sd}{    Helper function to simplify bounds checking}
\PY{l+s+sd}{    \PYZdq{}\PYZdq{}\PYZdq{}}
    \PY{k}{return} \PY{l+m+mi}{0} \PY{o}{\PYZlt{}}\PY{o}{=} \PY{n}{value} \PY{o}{\PYZlt{}}\PY{o}{=} \PY{n}{bound}

\PY{k}{def} \PY{n+nf}{bounded\PYZus{}in\PYZus{}dimensions}\PY{p}{(}\PY{n}{pos}\PY{p}{,} \PY{n}{dimensions}\PY{p}{)}\PY{p}{:}
\PY{+w}{    }\PY{l+s+sd}{\PYZdq{}\PYZdq{}\PYZdq{}}
\PY{l+s+sd}{    Checks to see if a position is within the given dimensions}
\PY{l+s+sd}{    \PYZdq{}\PYZdq{}\PYZdq{}}
    \PY{n}{bounded} \PY{o}{=} \PY{k+kc}{True}
    \PY{k}{for} \PY{n}{i}\PY{p}{,} \PY{n}{dim} \PY{o+ow}{in} \PY{n+nb}{enumerate}\PY{p}{(}\PY{n}{dimensions}\PY{p}{)}\PY{p}{:}
        \PY{n}{bounded} \PY{o}{=} \PY{n}{bounded} \PY{o+ow}{and} \PY{l+m+mi}{0} \PY{o}{\PYZlt{}}\PY{o}{=} \PY{n}{pos}\PY{p}{[}\PY{n}{i}\PY{p}{]} \PY{o}{\PYZlt{}}\PY{o}{=} \PY{n}{dim}
    \PY{k}{return} \PY{n}{bounded}
\end{Verbatim}
\end{tcolorbox}

    \textbf{Code Block Summary} (\texttt{random\_position\_in\_box}) This is
a helper function for obtaining a random position within the box based
on its dimensions \((w,h)\) for the sake of initial Neutron placement at
the beginning of a tree.

    \begin{tcolorbox}[breakable, size=fbox, boxrule=1pt, pad at break*=1mm,colback=cellbackground, colframe=cellborder]
\prompt{In}{incolor}{11}{\boxspacing}
\begin{Verbatim}[commandchars=\\\{\}]
\PY{k}{def} \PY{n+nf}{random\PYZus{}position\PYZus{}in\PYZus{}box}\PY{p}{(}\PY{n}{dimensions}\PY{p}{:} \PY{n+nb}{tuple}\PY{p}{[}\PY{n+nb}{float}\PY{p}{]}\PY{p}{)} \PY{o}{\PYZhy{}}\PY{o}{\PYZgt{}} \PY{n}{np}\PY{o}{.}\PY{n}{array}\PY{p}{:}
\PY{+w}{    }\PY{l+s+sd}{\PYZdq{}\PYZdq{}\PYZdq{}}
\PY{l+s+sd}{    Given `dimensions` = (width, length, height), returns}
\PY{l+s+sd}{    a random (x,y,z) where 0 \PYZlt{}= x \PYZlt{}= width and}
\PY{l+s+sd}{    0 \PYZlt{}= y \PYZlt{}= height.}
\PY{l+s+sd}{    \PYZdq{}\PYZdq{}\PYZdq{}}
    \PY{k}{return} \PY{n}{np}\PY{o}{.}\PY{n}{array}\PY{p}{(}\PY{p}{[}\PY{n}{np}\PY{o}{.}\PY{n}{random}\PY{o}{.}\PY{n}{uniform}\PY{p}{(}\PY{l+m+mi}{0}\PY{p}{,} \PY{n}{dim}\PY{p}{)} \PY{k}{for} \PY{n}{dim} \PY{o+ow}{in} \PY{n}{dimensions}\PY{p}{]}\PY{p}{)}
\end{Verbatim}
\end{tcolorbox}

    \textbf{Code Block Summary:} (\texttt{is\_outside\_box}) This is a
helper method for determining whether a position \((x,y)\) is outside of
the box given dimensions \((w,h)\), by checking if it is: - Horizontally
outside the box, to the left: \(x < 0\) - Horizontally outside the box,
to the right: \(x > w\) - Vertically outside the box, below: \(y < 0\) -
Vertically outside the box, above: \(y > h\)

    \begin{tcolorbox}[breakable, size=fbox, boxrule=1pt, pad at break*=1mm,colback=cellbackground, colframe=cellborder]
\prompt{In}{incolor}{12}{\boxspacing}
\begin{Verbatim}[commandchars=\\\{\}]
\PY{k}{def} \PY{n+nf}{is\PYZus{}outside\PYZus{}box}\PY{p}{(}\PY{n}{pos}\PY{p}{:} \PY{n}{np}\PY{o}{.}\PY{n}{array}\PY{p}{,} \PY{n}{dimensions}\PY{p}{:} \PY{n+nb}{tuple}\PY{p}{[}\PY{n+nb}{float}\PY{p}{]}\PY{p}{)} \PY{o}{\PYZhy{}}\PY{o}{\PYZgt{}} \PY{n+nb}{bool}\PY{p}{:}
\PY{+w}{    }\PY{l+s+sd}{\PYZdq{}\PYZdq{}\PYZdq{}}
\PY{l+s+sd}{    Given pos        = (x, y, z), }
\PY{l+s+sd}{          dimensions = (w, l, h),}
\PY{l+s+sd}{    returns if the position is outside of the dimensions.}
\PY{l+s+sd}{    \PYZdq{}\PYZdq{}\PYZdq{}}
    \PY{k}{return} \PY{o+ow}{not} \PY{n}{bounded\PYZus{}in\PYZus{}dimensions}\PY{p}{(}\PY{n}{pos}\PY{p}{,} \PY{n}{dimensions}\PY{p}{)}
\end{Verbatim}
\end{tcolorbox}

    \textbf{Code Block Summary:} (\texttt{generate\_children}) This function
generates the list of children for a given node based on the starting
position of the children (the ending position of the parent)
\(\mathbf{r}_0\), the number of children to be generated \(n\), and the
dimensions of the box \((w,h)\).

    \begin{tcolorbox}[breakable, size=fbox, boxrule=1pt, pad at break*=1mm,colback=cellbackground, colframe=cellborder]
\prompt{In}{incolor}{13}{\boxspacing}
\begin{Verbatim}[commandchars=\\\{\}]
\PY{k}{def} \PY{n+nf}{generate\PYZus{}children}\PY{p}{(}\PY{n}{start\PYZus{}pos}\PY{p}{:} \PY{n}{np}\PY{o}{.}\PY{n}{array}\PY{p}{,} \PY{n}{n}\PY{p}{:} \PY{n+nb}{int}\PY{p}{,} \PY{n}{dimensions}\PY{p}{:} \PY{n+nb}{tuple}\PY{p}{[}\PY{n+nb}{float}\PY{p}{]}\PY{p}{)} \PY{o}{\PYZhy{}}\PY{o}{\PYZgt{}} \PY{n+nb}{list}\PY{p}{[}\PY{n}{NeutronNode}\PY{p}{]}\PY{p}{:}
\PY{+w}{    }\PY{l+s+sd}{\PYZdq{}\PYZdq{}\PYZdq{}}
\PY{l+s+sd}{    Given start\PYZus{}pos}
\PY{l+s+sd}{    \PYZdq{}\PYZdq{}\PYZdq{}}
    \PY{n}{children} \PY{o}{=} \PY{p}{[}\PY{p}{]}
    \PY{k}{for} \PY{n}{\PYZus{}} \PY{o+ow}{in} \PY{n+nb}{range}\PY{p}{(}\PY{n}{n}\PY{p}{)}\PY{p}{:}
        \PY{n}{end\PYZus{}pos} \PY{o}{=} \PY{n}{get\PYZus{}next\PYZus{}neutron\PYZus{}position}\PY{p}{(}\PY{n}{start\PYZus{}pos}\PY{p}{)}
        
        \PY{n}{children}\PY{o}{.}\PY{n}{append}\PY{p}{(}\PY{n}{NeutronNode}\PY{p}{(}
            \PY{n}{start\PYZus{}pos} \PY{o}{=} \PY{n}{start\PYZus{}pos}\PY{p}{,}
            \PY{n}{end\PYZus{}pos}   \PY{o}{=} \PY{n}{end\PYZus{}pos}\PY{p}{,}
            \PY{n}{lost}      \PY{o}{=} \PY{n}{is\PYZus{}outside\PYZus{}box}\PY{p}{(}\PY{n}{end\PYZus{}pos}\PY{p}{,} \PY{n}{dimensions}\PY{p}{)}
        \PY{p}{)}\PY{p}{)}
    \PY{k}{return} \PY{n}{children}
\end{Verbatim}
\end{tcolorbox}

    \textbf{Code Block Summary:} (\texttt{generate\_tree\_recursive}) This
is a recursive helper function for performing the generative recursion
necessary to build this tree. It was produced using the generative
recursion template from CPSC 110
\href{https://cs110.students.cs.ubc.ca/reference/design-recipes.html\#GenRec}{5}.
Generation terminates for all nodes once reaching the generation limit
and for any node that ends up becoming ``lost'' (or falls outside of the
box on its end position).

This is an in-place generative recursion, so no values are returned in
this function, pointers are just appended to the end of the same tree.

    \begin{tcolorbox}[breakable, size=fbox, boxrule=1pt, pad at break*=1mm,colback=cellbackground, colframe=cellborder]
\prompt{In}{incolor}{14}{\boxspacing}
\begin{Verbatim}[commandchars=\\\{\}]
\PY{k}{def} \PY{n+nf}{generate\PYZus{}tree\PYZus{}recursive}\PY{p}{(}\PY{n}{parent}\PY{p}{:} \PY{n}{NeutronNode}\PY{p}{,} \PY{n}{gen}\PY{p}{:} \PY{n+nb}{int}\PY{p}{,} \PY{n}{max\PYZus{}generations}\PY{p}{:} \PY{n+nb}{int}\PY{p}{,} \PY{n}{dimensions}\PY{p}{:} \PY{n+nb}{tuple}\PY{p}{[}\PY{n+nb}{float}\PY{p}{]}\PY{p}{)}\PY{p}{:}
\PY{+w}{    }\PY{l+s+sd}{\PYZdq{}\PYZdq{}\PYZdq{}}
\PY{l+s+sd}{    Generates the neutron tree using in place tail recursion }
\PY{l+s+sd}{    \PYZdq{}\PYZdq{}\PYZdq{}}
    
    \PY{k}{if} \PY{n}{gen} \PY{o}{\PYZgt{}}\PY{o}{=} \PY{n}{max\PYZus{}generations} \PY{o+ow}{or} \PY{n}{parent}\PY{o}{.}\PY{n}{lost}\PY{p}{:}
        \PY{n}{parent}\PY{o}{.}\PY{n}{children} \PY{o}{=} \PY{p}{[}\PY{p}{]}
        \PY{k}{return}
    
    \PY{n}{parent}\PY{o}{.}\PY{n}{children} \PY{o}{=} \PY{n}{generate\PYZus{}children}\PY{p}{(}\PY{n}{parent}\PY{o}{.}\PY{n}{end\PYZus{}pos}\PY{p}{,} \PY{l+m+mi}{2}\PY{p}{,} \PY{n}{dimensions}\PY{p}{)}
    \PY{k}{for} \PY{n}{child} \PY{o+ow}{in} \PY{n}{parent}\PY{o}{.}\PY{n}{children}\PY{p}{:}
        \PY{n}{generate\PYZus{}tree\PYZus{}recursive}\PY{p}{(}\PY{n}{child}\PY{p}{,} \PY{n}{gen} \PY{o}{+} \PY{l+m+mi}{1}\PY{p}{,} \PY{n}{max\PYZus{}generations}\PY{p}{,} \PY{n}{dimensions}\PY{p}{)}
\end{Verbatim}
\end{tcolorbox}

    \textbf{Code Block Summary:} (\texttt{generate\_tree}) This is the
non-recursive root function used to begin the recursion necessary to
build the Neutron tree using by initiating
\texttt{generate\_tree\_recursive} given the number of generations and
the dimensions of the box.

    \begin{tcolorbox}[breakable, size=fbox, boxrule=1pt, pad at break*=1mm,colback=cellbackground, colframe=cellborder]
\prompt{In}{incolor}{15}{\boxspacing}
\begin{Verbatim}[commandchars=\\\{\}]
\PY{k}{def} \PY{n+nf}{generate\PYZus{}tree}\PY{p}{(}\PY{n}{num\PYZus{}generations}\PY{p}{:} \PY{n+nb}{int}\PY{p}{,} \PY{n}{dimensions}\PY{p}{:} \PY{n+nb}{tuple}\PY{p}{[}\PY{n+nb}{float}\PY{p}{]}\PY{p}{)} \PY{o}{\PYZhy{}}\PY{o}{\PYZgt{}} \PY{n}{NeutronNode}\PY{p}{:}
\PY{+w}{    }\PY{l+s+sd}{\PYZdq{}\PYZdq{}\PYZdq{}}
\PY{l+s+sd}{    Calculates the trajectories (start and end points) for each }
\PY{l+s+sd}{    Neutron Generation by building a Neutron tree with BFS alongside}
\PY{l+s+sd}{    the replication numbers for each generation transition.}
\PY{l+s+sd}{    \PYZdq{}\PYZdq{}\PYZdq{}}
    \PY{n}{initial\PYZus{}pos} \PY{o}{=} \PY{n}{random\PYZus{}position\PYZus{}in\PYZus{}box}\PY{p}{(}\PY{n}{dimensions}\PY{p}{)}
    \PY{n}{end\PYZus{}pos}     \PY{o}{=} \PY{n}{get\PYZus{}next\PYZus{}neutron\PYZus{}position}\PY{p}{(}\PY{n}{initial\PYZus{}pos}\PY{p}{)}
    
    \PY{n}{root} \PY{o}{=} \PY{n}{NeutronNode}\PY{p}{(}
        \PY{n}{start\PYZus{}pos} \PY{o}{=} \PY{n}{initial\PYZus{}pos}\PY{p}{,}
        \PY{n}{end\PYZus{}pos}   \PY{o}{=} \PY{n}{end\PYZus{}pos}\PY{p}{,}
        \PY{n}{lost}      \PY{o}{=} \PY{n}{is\PYZus{}outside\PYZus{}box}\PY{p}{(}\PY{n}{end\PYZus{}pos}\PY{p}{,} \PY{n}{dimensions}\PY{p}{)}
    \PY{p}{)}
    
    \PY{n}{generate\PYZus{}tree\PYZus{}recursive}\PY{p}{(}\PY{n}{root}\PY{p}{,} \PY{l+m+mi}{0}\PY{p}{,} \PY{n}{num\PYZus{}generations}\PY{p}{,} \PY{n}{dimensions}\PY{p}{)}
    
    \PY{k}{return} \PY{n}{root}
\end{Verbatim}
\end{tcolorbox}

    \hypertarget{plotting}{%
\subsection{Plotting}\label{plotting}}

    \textbf{Code Block Summary:} (\texttt{class\ SimulationResult}) This is
a structure for storing any graphs generated from a given simulation (if
such graphs are generated under \texttt{simulate}) alongside the final
Neutron tree(s) that are generated.

    \begin{tcolorbox}[breakable, size=fbox, boxrule=1pt, pad at break*=1mm,colback=cellbackground, colframe=cellborder]
\prompt{In}{incolor}{16}{\boxspacing}
\begin{Verbatim}[commandchars=\\\{\}]
\PY{n+nd}{@dataclass}
\PY{k}{class} \PY{n+nc}{SimulationResult}\PY{p}{:}
    \PY{n}{trajectories\PYZus{}graph}\PY{p}{:} \PY{n}{Image}
    \PY{n}{result\PYZus{}trees}\PY{p}{:}       \PY{n}{NeutronNode}
\end{Verbatim}
\end{tcolorbox}

    \textbf{Code Block Summary} (\texttt{draw\_3d\_box}) This is a method to
greatly simplify plotting a wireframe of the box on a 3-dimensional plot
using only two sets of 3-dimensional coordinates (and origin point and
the dimensions of the box).

    \begin{tcolorbox}[breakable, size=fbox, boxrule=1pt, pad at break*=1mm,colback=cellbackground, colframe=cellborder]
\prompt{In}{incolor}{17}{\boxspacing}
\begin{Verbatim}[commandchars=\\\{\}]
\PY{k}{def} \PY{n+nf}{draw\PYZus{}3d\PYZus{}box}\PY{p}{(}\PY{n}{ax}\PY{p}{,} \PY{n}{origin}\PY{p}{,} \PY{n}{dimensions}\PY{p}{,} \PY{n}{edgecolor}\PY{o}{=}\PY{l+s+s1}{\PYZsq{}}\PY{l+s+s1}{red}\PY{l+s+s1}{\PYZsq{}}\PY{p}{,} \PY{n}{linestyle}\PY{o}{=}\PY{l+s+s1}{\PYZsq{}}\PY{l+s+s1}{dashed}\PY{l+s+s1}{\PYZsq{}}\PY{p}{,} \PY{n}{linewidth}\PY{o}{=}\PY{l+m+mi}{1}\PY{p}{)}\PY{p}{:}
\PY{+w}{    }\PY{l+s+sd}{\PYZdq{}\PYZdq{}\PYZdq{}}
\PY{l+s+sd}{    Draws a 3D wireframe box on a given 3D axes.}
\PY{l+s+sd}{    \PYZdq{}\PYZdq{}\PYZdq{}}
    \PY{c+c1}{\PYZsh{} Define the 8 vertices of the box relative to the origin}
    \PY{n}{x\PYZus{}min}\PY{p}{,} \PY{n}{y\PYZus{}min}\PY{p}{,} \PY{n}{z\PYZus{}min} \PY{o}{=} \PY{n}{origin}
    \PY{n}{width}\PY{p}{,} \PY{n}{height}\PY{p}{,} \PY{n}{depth} \PY{o}{=} \PY{n}{dimensions}
    \PY{n}{x\PYZus{}max}\PY{p}{,} \PY{n}{y\PYZus{}max}\PY{p}{,} \PY{n}{z\PYZus{}max} \PY{o}{=} \PY{n}{x\PYZus{}min} \PY{o}{+} \PY{n}{width}\PY{p}{,} \PY{n}{y\PYZus{}min} \PY{o}{+} \PY{n}{height}\PY{p}{,} \PY{n}{z\PYZus{}min} \PY{o}{+} \PY{n}{depth}

    \PY{c+c1}{\PYZsh{} Create a list of all 8 vertex coordinates}
    \PY{n}{vertices} \PY{o}{=} \PY{n}{np}\PY{o}{.}\PY{n}{array}\PY{p}{(}\PY{n+nb}{list}\PY{p}{(}\PY{n}{product}\PY{p}{(}\PY{p}{[}\PY{n}{x\PYZus{}min}\PY{p}{,} \PY{n}{x\PYZus{}max}\PY{p}{]}\PY{p}{,} \PY{p}{[}\PY{n}{y\PYZus{}min}\PY{p}{,} \PY{n}{y\PYZus{}max}\PY{p}{]}\PY{p}{,} \PY{p}{[}\PY{n}{z\PYZus{}min}\PY{p}{,} \PY{n}{z\PYZus{}max}\PY{p}{]}\PY{p}{)}\PY{p}{)}\PY{p}{)}

    \PY{c+c1}{\PYZsh{} Draw the 12 edges by iterating over combinations of vertices}
    \PY{k}{for} \PY{n}{s}\PY{p}{,} \PY{n}{e} \PY{o+ow}{in} \PY{n}{combinations}\PY{p}{(}\PY{n}{vertices}\PY{p}{,} \PY{l+m+mi}{2}\PY{p}{)}\PY{p}{:}
        \PY{c+c1}{\PYZsh{} Only draw an edge if the vertices are connected along a single axis}
        \PY{k}{if} \PY{n}{np}\PY{o}{.}\PY{n}{sum}\PY{p}{(}\PY{n}{np}\PY{o}{.}\PY{n}{abs}\PY{p}{(}\PY{n}{s} \PY{o}{\PYZhy{}} \PY{n}{e}\PY{p}{)}\PY{p}{)} \PY{o}{==} \PY{n}{width} \PY{o+ow}{or} \PY{n}{np}\PY{o}{.}\PY{n}{sum}\PY{p}{(}\PY{n}{np}\PY{o}{.}\PY{n}{abs}\PY{p}{(}\PY{n}{s} \PY{o}{\PYZhy{}} \PY{n}{e}\PY{p}{)}\PY{p}{)} \PY{o}{==} \PY{n}{height} \PY{o+ow}{or} \PY{n}{np}\PY{o}{.}\PY{n}{sum}\PY{p}{(}\PY{n}{np}\PY{o}{.}\PY{n}{abs}\PY{p}{(}\PY{n}{s} \PY{o}{\PYZhy{}} \PY{n}{e}\PY{p}{)}\PY{p}{)} \PY{o}{==} \PY{n}{depth}\PY{p}{:}
            \PY{n}{ax}\PY{o}{.}\PY{n}{plot3D}\PY{p}{(}\PY{o}{*}\PY{n+nb}{zip}\PY{p}{(}\PY{n}{s}\PY{p}{,} \PY{n}{e}\PY{p}{)}\PY{p}{,} \PY{n}{color}\PY{o}{=}\PY{n}{edgecolor}\PY{p}{,} \PY{n}{linestyle}\PY{o}{=}\PY{n}{linestyle}\PY{p}{,} \PY{n}{linewidth}\PY{o}{=}\PY{n}{linewidth}\PY{p}{)}
\end{Verbatim}
\end{tcolorbox}

    \textbf{Code Block Summary:} (\texttt{plot\_trajectories}) This function
plots a 2-D graph of the trajectories of the Neutrons, producing the
final tree-like structure colored and labelled by generation.

    \begin{tcolorbox}[breakable, size=fbox, boxrule=1pt, pad at break*=1mm,colback=cellbackground, colframe=cellborder]
\prompt{In}{incolor}{18}{\boxspacing}
\begin{Verbatim}[commandchars=\\\{\}]
\PY{k}{def} \PY{n+nf}{plot\PYZus{}trajectories}\PY{p}{(}\PY{n}{trees}\PY{p}{:} \PY{n+nb}{list}\PY{p}{[}\PY{n}{NeutronNode}\PY{p}{]}\PY{p}{,} \PY{n}{num\PYZus{}generations}\PY{p}{:} \PY{n+nb}{int}\PY{p}{,} \PY{n}{dimensions}\PY{p}{:} \PY{n+nb}{tuple}\PY{p}{[}\PY{n+nb}{float}\PY{p}{]}\PY{p}{,} \PY{n}{padding}\PY{p}{:} \PY{n+nb}{int} \PY{o}{=} \PY{l+m+mi}{5}\PY{p}{)}\PY{p}{:}
\PY{+w}{    }\PY{l+s+sd}{\PYZdq{}\PYZdq{}\PYZdq{}}
\PY{l+s+sd}{    Plots the final trajectories of each neutron tree }
\PY{l+s+sd}{    simulation, colored and labeled by generation.}
\PY{l+s+sd}{    \PYZdq{}\PYZdq{}\PYZdq{}}
    \PY{n}{fig} \PY{o}{=} \PY{n}{plt}\PY{o}{.}\PY{n}{figure}\PY{p}{(}\PY{p}{)}
    \PY{n}{ax} \PY{o}{=} \PY{n}{fig}\PY{o}{.}\PY{n}{add\PYZus{}subplot}\PY{p}{(}\PY{n}{projection}\PY{o}{=}\PY{l+s+s1}{\PYZsq{}}\PY{l+s+s1}{3d}\PY{l+s+s1}{\PYZsq{}}\PY{p}{)}
    
    \PY{c+c1}{\PYZsh{} For legend handles (store one Line2D per generation)}
    \PY{n}{generation\PYZus{}handles} \PY{o}{=} \PY{p}{\PYZob{}}\PY{p}{\PYZcb{}}
    \PY{n}{colors} \PY{o}{=} \PY{p}{[}\PY{l+s+s1}{\PYZsq{}}\PY{l+s+s1}{blue}\PY{l+s+s1}{\PYZsq{}}\PY{p}{,} \PY{l+s+s1}{\PYZsq{}}\PY{l+s+s1}{red}\PY{l+s+s1}{\PYZsq{}}\PY{p}{,} \PY{l+s+s1}{\PYZsq{}}\PY{l+s+s1}{green}\PY{l+s+s1}{\PYZsq{}}\PY{p}{,} \PY{l+s+s1}{\PYZsq{}}\PY{l+s+s1}{orange}\PY{l+s+s1}{\PYZsq{}}\PY{p}{,} 
              \PY{l+s+s1}{\PYZsq{}}\PY{l+s+s1}{magenta}\PY{l+s+s1}{\PYZsq{}}\PY{p}{,} \PY{l+s+s1}{\PYZsq{}}\PY{l+s+s1}{tab:gray}\PY{l+s+s1}{\PYZsq{}}\PY{p}{,} \PY{l+s+s1}{\PYZsq{}}\PY{l+s+s1}{tab:olive}\PY{l+s+s1}{\PYZsq{}}\PY{p}{,} 
              \PY{l+s+s1}{\PYZsq{}}\PY{l+s+s1}{tab:brown}\PY{l+s+s1}{\PYZsq{}}\PY{p}{,} \PY{l+s+s1}{\PYZsq{}}\PY{l+s+s1}{tab:pink}\PY{l+s+s1}{\PYZsq{}}\PY{p}{,} \PY{l+s+s1}{\PYZsq{}}\PY{l+s+s1}{tab:cyan}\PY{l+s+s1}{\PYZsq{}}\PY{p}{,} 
              \PY{l+s+s1}{\PYZsq{}}\PY{l+s+s1}{xkcd:algae}\PY{l+s+s1}{\PYZsq{}}\PY{p}{,} \PY{l+s+s1}{\PYZsq{}}\PY{l+s+s1}{xkcd:crimson}\PY{l+s+s1}{\PYZsq{}}\PY{p}{,} \PY{l+s+s1}{\PYZsq{}}\PY{l+s+s1}{xkcd:ochre}\PY{l+s+s1}{\PYZsq{}}\PY{p}{,}
              \PY{l+s+s1}{\PYZsq{}}\PY{l+s+s1}{xkcd:leaf green}\PY{l+s+s1}{\PYZsq{}}\PY{p}{,} \PY{l+s+s1}{\PYZsq{}}\PY{l+s+s1}{xkcd:sea blue}\PY{l+s+s1}{\PYZsq{}}\PY{p}{,} \PY{l+s+s1}{\PYZsq{}}\PY{l+s+s1}{xkcd:sage}\PY{l+s+s1}{\PYZsq{}}\PY{p}{]}
    
    \PY{c+c1}{\PYZsh{} Traverse the trees}
    \PY{k}{for} \PY{n}{tree} \PY{o+ow}{in} \PY{n}{trees}\PY{p}{:}
        \PY{k}{for} \PY{n}{neutron}\PY{p}{,} \PY{n}{generation} \PY{o+ow}{in} \PY{n}{traverse\PYZus{}tree}\PY{p}{(}\PY{n}{tree}\PY{p}{)}\PY{p}{:}
            \PY{c+c1}{\PYZsh{} Just in case of errors}
            \PY{k}{if} \PY{n}{neutron}\PY{o}{.}\PY{n}{start\PYZus{}pos} \PY{o+ow}{is} \PY{k+kc}{None} \PY{o+ow}{or} \PY{n}{neutron}\PY{o}{.}\PY{n}{end\PYZus{}pos} \PY{o+ow}{is} \PY{k+kc}{None}\PY{p}{:}
                \PY{k}{continue}
                
            \PY{n}{color} \PY{o}{=} \PY{n}{colors}\PY{p}{[}\PY{n}{generation}\PY{p}{]}
            
            \PY{c+c1}{\PYZsh{} Draw the trajectory line}
            \PY{p}{(}\PY{n}{line}\PY{p}{,}\PY{p}{)} \PY{o}{=} \PY{n}{ax}\PY{o}{.}\PY{n}{plot}\PY{p}{(}
                \PY{p}{[}\PY{n}{neutron}\PY{o}{.}\PY{n}{start\PYZus{}pos}\PY{p}{[}\PY{l+m+mi}{0}\PY{p}{]}\PY{p}{,} \PY{n}{neutron}\PY{o}{.}\PY{n}{end\PYZus{}pos}\PY{p}{[}\PY{l+m+mi}{0}\PY{p}{]}\PY{p}{]}\PY{p}{,}
                \PY{p}{[}\PY{n}{neutron}\PY{o}{.}\PY{n}{start\PYZus{}pos}\PY{p}{[}\PY{l+m+mi}{1}\PY{p}{]}\PY{p}{,} \PY{n}{neutron}\PY{o}{.}\PY{n}{end\PYZus{}pos}\PY{p}{[}\PY{l+m+mi}{1}\PY{p}{]}\PY{p}{]}\PY{p}{,}
                \PY{p}{[}\PY{n}{neutron}\PY{o}{.}\PY{n}{start\PYZus{}pos}\PY{p}{[}\PY{l+m+mi}{2}\PY{p}{]}\PY{p}{,} \PY{n}{neutron}\PY{o}{.}\PY{n}{end\PYZus{}pos}\PY{p}{[}\PY{l+m+mi}{2}\PY{p}{]}\PY{p}{]}\PY{p}{,}
                \PY{n}{color}\PY{o}{=}\PY{n}{color}\PY{p}{,}
                \PY{n}{alpha}\PY{o}{=}\PY{l+m+mf}{0.8}\PY{p}{,}
                \PY{n}{linewidth}\PY{o}{=}\PY{l+m+mf}{1.2}\PY{p}{,}
                \PY{n}{label}\PY{o}{=}\PY{l+s+sa}{f}\PY{l+s+s2}{\PYZdq{}}\PY{l+s+s2}{Gen }\PY{l+s+si}{\PYZob{}}\PY{n}{generation}\PY{l+s+si}{\PYZcb{}}\PY{l+s+s2}{\PYZdq{}}
            \PY{p}{)}

            \PY{c+c1}{\PYZsh{} Mark start (circle) and end (X)}
            \PY{n}{ax}\PY{o}{.}\PY{n}{scatter}\PY{p}{(}\PY{o}{*}\PY{n}{neutron}\PY{o}{.}\PY{n}{end\PYZus{}pos}\PY{p}{,} \PY{n}{color}\PY{o}{=}\PY{n}{color}\PY{p}{,} \PY{n}{marker}\PY{o}{=}\PY{l+s+s1}{\PYZsq{}}\PY{l+s+s1}{o}\PY{l+s+s1}{\PYZsq{}}\PY{p}{,} \PY{n}{s}\PY{o}{=}\PY{l+m+mf}{12.5}\PY{p}{,} \PY{n}{alpha}\PY{o}{=}\PY{l+m+mf}{0.7}\PY{p}{)}

            \PY{c+c1}{\PYZsh{} Save one handle per generation for legend}
            \PY{k}{if} \PY{n}{generation} \PY{o+ow}{not} \PY{o+ow}{in} \PY{n}{generation\PYZus{}handles}\PY{p}{:}
                \PY{n}{generation\PYZus{}handles}\PY{p}{[}\PY{n}{generation}\PY{p}{]} \PY{o}{=} \PY{n}{line}

    \PY{c+c1}{\PYZsh{} Add legend (one entry per generation)}
    \PY{n}{ax}\PY{o}{.}\PY{n}{legend}\PY{p}{(}
        \PY{n}{handles}\PY{o}{=}\PY{p}{[}\PY{n}{generation\PYZus{}handles}\PY{p}{[}\PY{n}{g}\PY{p}{]} \PY{k}{for} \PY{n}{g} \PY{o+ow}{in} \PY{n+nb}{sorted}\PY{p}{(}\PY{n}{generation\PYZus{}handles}\PY{p}{)}\PY{p}{]}\PY{p}{,}
        \PY{n}{labels}\PY{o}{=}\PY{p}{[}\PY{l+s+sa}{f}\PY{l+s+s2}{\PYZdq{}}\PY{l+s+s2}{Generation }\PY{l+s+si}{\PYZob{}}\PY{n}{g}\PY{l+s+si}{\PYZcb{}}\PY{l+s+s2}{\PYZdq{}} \PY{k}{for} \PY{n}{g} \PY{o+ow}{in} \PY{n+nb}{sorted}\PY{p}{(}\PY{n}{generation\PYZus{}handles}\PY{p}{)}\PY{p}{]}\PY{p}{,}
        \PY{n}{title}\PY{o}{=}\PY{l+s+s2}{\PYZdq{}}\PY{l+s+s2}{Generations}\PY{l+s+s2}{\PYZdq{}}\PY{p}{,}
        \PY{n}{loc}\PY{o}{=}\PY{l+s+s2}{\PYZdq{}}\PY{l+s+s2}{upper right}\PY{l+s+s2}{\PYZdq{}}
    \PY{p}{)}

    \PY{c+c1}{\PYZsh{} Labels}
    \PY{n}{ax}\PY{o}{.}\PY{n}{set\PYZus{}title}\PY{p}{(}\PY{l+s+s2}{\PYZdq{}}\PY{l+s+s2}{Neutron Branching Paths by Generation}\PY{l+s+s2}{\PYZdq{}}\PY{p}{)}
    \PY{n}{ax}\PY{o}{.}\PY{n}{set\PYZus{}xlabel}\PY{p}{(}\PY{l+s+s2}{\PYZdq{}}\PY{l+s+s2}{x (m)}\PY{l+s+s2}{\PYZdq{}}\PY{p}{)}
    \PY{n}{ax}\PY{o}{.}\PY{n}{set\PYZus{}ylabel}\PY{p}{(}\PY{l+s+s2}{\PYZdq{}}\PY{l+s+s2}{y (m)}\PY{l+s+s2}{\PYZdq{}}\PY{p}{)}
    \PY{n}{ax}\PY{o}{.}\PY{n}{set\PYZus{}zlabel}\PY{p}{(}\PY{l+s+s2}{\PYZdq{}}\PY{l+s+s2}{z (m)}\PY{l+s+s2}{\PYZdq{}}\PY{p}{)}
    
    \PY{c+c1}{\PYZsh{} Draw the box as a rectangular prism}
    \PY{n}{draw\PYZus{}3d\PYZus{}box}\PY{p}{(}\PY{n}{ax}\PY{p}{,} \PY{n}{np}\PY{o}{.}\PY{n}{zeros}\PY{p}{(}\PY{l+m+mi}{3}\PY{p}{)}\PY{p}{,} \PY{n}{dimensions}\PY{p}{)}
    
    \PY{c+c1}{\PYZsh{} Limits}
    \PY{n}{ax}\PY{o}{.}\PY{n}{set\PYZus{}xlim}\PY{p}{(}\PY{o}{\PYZhy{}}\PY{n}{padding}\PY{p}{,} \PY{n}{dimensions}\PY{p}{[}\PY{l+m+mi}{0}\PY{p}{]} \PY{o}{+} \PY{n}{padding}\PY{p}{)}
    \PY{n}{ax}\PY{o}{.}\PY{n}{set\PYZus{}ylim}\PY{p}{(}\PY{o}{\PYZhy{}}\PY{n}{padding}\PY{p}{,} \PY{n}{dimensions}\PY{p}{[}\PY{l+m+mi}{1}\PY{p}{]} \PY{o}{+} \PY{n}{padding}\PY{p}{)}
    \PY{n}{ax}\PY{o}{.}\PY{n}{set\PYZus{}zlim}\PY{p}{(}\PY{o}{\PYZhy{}}\PY{n}{padding}\PY{p}{,} \PY{n}{dimensions}\PY{p}{[}\PY{l+m+mi}{2}\PY{p}{]} \PY{o}{+} \PY{n}{padding}\PY{p}{)}
    
    \PY{k}{return} \PY{n}{save\PYZus{}graph\PYZus{}and\PYZus{}close}\PY{p}{(}\PY{p}{)}
\end{Verbatim}
\end{tcolorbox}

    \textbf{Code Block Summary:} (\texttt{simulate}) Performs a singular
simulation given a maximal tree depth/number of generations, box
dimensions \texttt{(w,h)}, and number of trees/initial neutrons.

    \begin{tcolorbox}[breakable, size=fbox, boxrule=1pt, pad at break*=1mm,colback=cellbackground, colframe=cellborder]
\prompt{In}{incolor}{19}{\boxspacing}
\begin{Verbatim}[commandchars=\\\{\}]
\PY{k}{def} \PY{n+nf}{simulate}\PY{p}{(}\PY{n}{num\PYZus{}generations}\PY{p}{:} \PY{n+nb}{int}\PY{p}{,} \PY{n}{dimensions}\PY{p}{:} \PY{n+nb}{tuple}\PY{p}{[}\PY{n+nb}{float}\PY{p}{]}\PY{p}{,} 
             \PY{n}{num\PYZus{}initial\PYZus{}neutrons}\PY{p}{:} \PY{n+nb}{int} \PY{o}{=} \PY{l+m+mi}{1}\PY{p}{,} \PY{n}{plot}\PY{p}{:} \PY{n+nb}{bool} \PY{o}{=} \PY{k+kc}{False}\PY{p}{)}\PY{p}{:}
\PY{+w}{    }\PY{l+s+sd}{\PYZdq{}\PYZdq{}\PYZdq{}}
\PY{l+s+sd}{    Performs a singular, complete simulation based on the conditions}
\PY{l+s+sd}{    of num\PYZus{}generations (maximum tree depth), }
\PY{l+s+sd}{       dimensions (size of the box), and }
\PY{l+s+sd}{       num\PYZus{}initial\PYZus{}neutrons (the number of trees).}
\PY{l+s+sd}{    \PYZdq{}\PYZdq{}\PYZdq{}}
    \PY{n}{result\PYZus{}trees} \PY{o}{=} \PY{p}{[}\PY{n}{generate\PYZus{}tree}\PY{p}{(}\PY{n}{num\PYZus{}generations}\PY{p}{,} \PY{n}{dimensions}\PY{p}{)} \PYZbs{}
                    \PY{k}{for} \PY{n}{i} \PY{o+ow}{in} \PY{n+nb}{range}\PY{p}{(}\PY{n}{num\PYZus{}initial\PYZus{}neutrons}\PY{p}{)}\PY{p}{]}
    
    \PY{n}{graph}\PY{o}{=}\PY{k+kc}{None}
    \PY{k}{if} \PY{n}{plot}\PY{p}{:}
        \PY{n}{graph} \PY{o}{=} \PY{n}{plot\PYZus{}trajectories}\PY{p}{(}\PY{n}{result\PYZus{}trees}\PY{p}{,} \PY{n}{num\PYZus{}generations}\PY{p}{,} \PY{n}{dimensions}\PY{p}{)}
    
    \PY{k}{return} \PY{n}{SimulationResult}\PY{p}{(}
        \PY{n}{trajectories\PYZus{}graph} \PY{o}{=} \PY{n}{graph}\PY{p}{,}
        \PY{n}{result\PYZus{}trees}       \PY{o}{=} \PY{n}{result\PYZus{}trees}
    \PY{p}{)}
\end{Verbatim}
\end{tcolorbox}

    \hypertarget{test}{%
\subsubsection{Test}\label{test}}

    \textbf{Code Block Summary:} This is a test of the full simulation code
up to this point given the conditions of exploring 3 generations
(technically four, not counting the starting generation 0) for two
starting neutrons given a rectangular prism box with a width 8 meters,
length of 6 meters, and height of 10 meters.

    \begin{tcolorbox}[breakable, size=fbox, boxrule=1pt, pad at break*=1mm,colback=cellbackground, colframe=cellborder]
\prompt{In}{incolor}{20}{\boxspacing}
\begin{Verbatim}[commandchars=\\\{\}]
\PY{n}{test} \PY{o}{=} \PY{n}{simulate}\PY{p}{(}
    \PY{n}{num\PYZus{}generations}      \PY{o}{=} \PY{l+m+mi}{3}\PY{p}{,} 
    \PY{n}{dimensions}           \PY{o}{=} \PY{p}{(}\PY{l+m+mi}{8}\PY{p}{,}\PY{l+m+mi}{6}\PY{p}{,}\PY{l+m+mi}{10}\PY{p}{)}\PY{p}{,} 
    \PY{n}{num\PYZus{}initial\PYZus{}neutrons} \PY{o}{=} \PY{l+m+mi}{4}\PY{p}{,} 
    \PY{n}{plot} \PY{o}{=} \PY{k+kc}{True}\PY{p}{)}
\end{Verbatim}
\end{tcolorbox}

    Then, this same information can be displayed using a 2-D graph as well:

    \begin{tcolorbox}[breakable, size=fbox, boxrule=1pt, pad at break*=1mm,colback=cellbackground, colframe=cellborder]
\prompt{In}{incolor}{21}{\boxspacing}
\begin{Verbatim}[commandchars=\\\{\}]
\PY{n}{display}\PY{p}{(}\PY{n}{test}\PY{o}{.}\PY{n}{trajectories\PYZus{}graph}\PY{p}{)}
\end{Verbatim}
\end{tcolorbox}

    \begin{center}
    \adjustimage{max size={0.9\linewidth}{0.9\paperheight}}{Project 3_files/Project 3_46_0.png}
    \end{center}
    { \hspace*{\fill} \\}
    
    \textbf{Figure 1.} A plot of the Neutron trajectories given the initial
conditions of 4 starting neutrons, an \(8 \times 6 \times 10\) meter
box, and a maximal generation depth of 3. The mechanics of a singular
simulation can be verified by observing the tree-like structure of the
neutron paths. Each starting neutron produces a distinct tree of
furthered neutrons, but neutrons that end beyond the limits of the box
at any generation are untracked and all remaining trees are ended at the
maximal generation.

    \hypertarget{simulation-analysis}{%
\subsection{Simulation Analysis}\label{simulation-analysis}}

    \textbf{Code Block Summary:}
(\texttt{get\_neutrons\_in\_each\_generation}) Calculates the total
number of neutrons in each generation by using the previous traversal
method \texttt{traverse\_tree} and building a histogram based on the
generation index. Electrons that are lost are not considered in this
count.

    \begin{tcolorbox}[breakable, size=fbox, boxrule=1pt, pad at break*=1mm,colback=cellbackground, colframe=cellborder]
\prompt{In}{incolor}{22}{\boxspacing}
\begin{Verbatim}[commandchars=\\\{\}]
\PY{k}{def} \PY{n+nf}{get\PYZus{}neutrons\PYZus{}in\PYZus{}each\PYZus{}generation}\PY{p}{(}\PY{n}{tree}\PY{p}{:} \PY{n}{NeutronNode}\PY{p}{)} \PY{o}{\PYZhy{}}\PY{o}{\PYZgt{}} \PY{n}{np}\PY{o}{.}\PY{n}{array}\PY{p}{:}
\PY{+w}{    }\PY{l+s+sd}{\PYZdq{}\PYZdq{}\PYZdq{}}
\PY{l+s+sd}{    Counts the number of neutrons in each generation}
\PY{l+s+sd}{    \PYZdq{}\PYZdq{}\PYZdq{}}
    \PY{n}{counts} \PY{o}{=} \PY{p}{[}\PY{p}{]}

    \PY{k}{for} \PY{n}{neutron}\PY{p}{,} \PY{n}{generation} \PY{o+ow}{in} \PY{n}{traverse\PYZus{}tree}\PY{p}{(}\PY{n}{tree}\PY{p}{)}\PY{p}{:}
        \PY{k}{if} \PY{n+nb}{len}\PY{p}{(}\PY{n}{counts}\PY{p}{)} \PY{o}{\PYZlt{}}\PY{o}{=} \PY{n}{generation}\PY{p}{:}
            \PY{n}{counts}\PY{o}{.}\PY{n}{append}\PY{p}{(}\PY{l+m+mi}{0}\PY{p}{)}
        \PY{k}{if} \PY{o+ow}{not} \PY{n}{neutron}\PY{o}{.}\PY{n}{lost}\PY{p}{:}
            \PY{n}{counts}\PY{p}{[}\PY{n}{generation}\PY{p}{]} \PY{o}{+}\PY{o}{=} \PY{l+m+mi}{1}

    \PY{k}{return} \PY{n}{np}\PY{o}{.}\PY{n}{array}\PY{p}{(}\PY{n}{counts}\PY{p}{)}
\end{Verbatim}
\end{tcolorbox}

    \textbf{Code Block Summary:} (\texttt{calculate\_k\_values}) Calculates
the replication factor \(k\) for each generation based on the list of
neutrons in each generation for a given neutron tree using the formula
\(k_{i} = \frac{n_{i}}{n_{i-1}}\) where \(n_i\) is the number of
neutrons for generation \(i\).

    \begin{tcolorbox}[breakable, size=fbox, boxrule=1pt, pad at break*=1mm,colback=cellbackground, colframe=cellborder]
\prompt{In}{incolor}{23}{\boxspacing}
\begin{Verbatim}[commandchars=\\\{\}]
\PY{k}{def} \PY{n+nf}{calculate\PYZus{}k\PYZus{}values}\PY{p}{(}\PY{n}{tree}\PY{p}{:} \PY{n}{NeutronNode}\PY{p}{)}\PY{p}{:}
\PY{+w}{    }\PY{l+s+sd}{\PYZdq{}\PYZdq{}\PYZdq{}}
\PY{l+s+sd}{    From a neutron tree, return an array of k }
\PY{l+s+sd}{    values (n\PYZus{}next / n\PYZus{}current) for each generation. }
\PY{l+s+sd}{    \PYZdq{}\PYZdq{}\PYZdq{}}
    \PY{n}{counts} \PY{o}{=} \PY{n}{get\PYZus{}neutrons\PYZus{}in\PYZus{}each\PYZus{}generation}\PY{p}{(}\PY{n}{tree}\PY{p}{)}
    \PY{k}{return} \PY{n}{counts}\PY{p}{[}\PY{l+m+mi}{1}\PY{p}{:}\PY{p}{]} \PY{o}{/} \PY{n}{counts}\PY{p}{[}\PY{p}{:}\PY{o}{\PYZhy{}}\PY{l+m+mi}{1}\PY{p}{]}
\end{Verbatim}
\end{tcolorbox}

    \textbf{Code Block Summary:} (\texttt{simulate\_all}) This code repeats
a given simulation and calculates the mean, standard deviation, and
standard error for \(k\) alongside a bar plot of \(k\) values for each
generation obtained for given number of replications \texttt{m\_reps},
and this graph can be used for validation against the produced \(k\)
values based on Monte-Carlo expectations.

    \begin{tcolorbox}[breakable, size=fbox, boxrule=1pt, pad at break*=1mm,colback=cellbackground, colframe=cellborder]
\prompt{In}{incolor}{24}{\boxspacing}
\begin{Verbatim}[commandchars=\\\{\}]
\PY{n+nd}{@dataclass}
\PY{k}{class} \PY{n+nc}{GenerationData}\PY{p}{:}
\PY{+w}{    }\PY{l+s+sd}{\PYZdq{}\PYZdq{}\PYZdq{}}
\PY{l+s+sd}{    Helper structure for storing k value }
\PY{l+s+sd}{    data and a histogram for each generation}
\PY{l+s+sd}{    \PYZdq{}\PYZdq{}\PYZdq{}}
    \PY{n}{k\PYZus{}values\PYZus{}raw}\PY{p}{:}  \PY{n}{np}\PY{o}{.}\PY{n}{array}   \PY{o}{=} \PY{k+kc}{None}
    \PY{n}{k\PYZus{}mean}\PY{p}{:}        \PY{n}{np}\PY{o}{.}\PY{n}{float64} \PY{o}{=} \PY{n}{np}\PY{o}{.}\PY{n}{nan}
    \PY{n}{k\PYZus{}std\PYZus{}dev}\PY{p}{:}     \PY{n}{np}\PY{o}{.}\PY{n}{float64} \PY{o}{=} \PY{n}{np}\PY{o}{.}\PY{n}{nan}
    \PY{n}{k\PYZus{}std\PYZus{}err}\PY{p}{:}     \PY{n}{np}\PY{o}{.}\PY{n}{float64} \PY{o}{=} \PY{n}{np}\PY{o}{.}\PY{n}{nan}
    \PY{n}{rel\PYZus{}k\PYZus{}std\PYZus{}err}\PY{p}{:} \PY{n}{np}\PY{o}{.}\PY{n}{float64} \PY{o}{=} \PY{n}{np}\PY{o}{.}\PY{n}{nan}
    
    \PY{n}{histogram}\PY{p}{:} \PY{n}{Image} \PY{o}{=} \PY{k+kc}{None}

\PY{n+nd}{@dataclass}
\PY{k}{class} \PY{n+nc}{MonteCarloResults}\PY{p}{:}
    \PY{n}{generations}\PY{p}{:}     \PY{n+nb}{list}\PY{p}{[}\PY{n}{GenerationData}\PY{p}{]} \PY{o}{=} \PY{k+kc}{None}
    \PY{n}{mean\PYZus{}k\PYZus{}values}\PY{p}{:}   \PY{n}{np}\PY{o}{.}\PY{n}{array} \PY{o}{=} \PY{k+kc}{None}
    \PY{n}{k\PYZus{}value\PYZus{}uncerts}\PY{p}{:} \PY{n}{np}\PY{o}{.}\PY{n}{array} \PY{o}{=} \PY{k+kc}{None}
    \PY{n}{k\PYZus{}value\PYZus{}plot}\PY{p}{:}    \PY{n}{Image}    \PY{o}{=} \PY{k+kc}{None}
\end{Verbatim}
\end{tcolorbox}

    \begin{tcolorbox}[breakable, size=fbox, boxrule=1pt, pad at break*=1mm,colback=cellbackground, colframe=cellborder]
\prompt{In}{incolor}{25}{\boxspacing}
\begin{Verbatim}[commandchars=\\\{\}]
\PY{k}{def} \PY{n+nf}{monte\PYZus{}carlo}\PY{p}{(}\PY{n}{dimensions}\PY{p}{:} \PY{n+nb}{list}\PY{p}{[}\PY{n+nb}{float}\PY{p}{]}\PY{p}{,} \PY{n}{num\PYZus{}initial\PYZus{}neutrons}\PY{p}{:} \PY{n+nb}{int}\PY{p}{,} 
                \PY{n}{num\PYZus{}replications}\PY{p}{:} \PY{n+nb}{int}\PY{p}{,}   \PY{n}{max\PYZus{}generation}\PY{p}{:} \PY{n+nb}{int}\PY{p}{,} 
                \PY{n}{plot}\PY{p}{:} \PY{n+nb}{bool} \PY{o}{=} \PY{k+kc}{False}\PY{p}{,}      \PY{n}{debug\PYZus{}print}\PY{p}{:} \PY{n+nb}{bool} \PY{o}{=} \PY{k+kc}{False}\PY{p}{)}\PY{p}{:}
\PY{+w}{    }\PY{l+s+sd}{\PYZdq{}\PYZdq{}\PYZdq{}}
\PY{l+s+sd}{    Performs a Monte Carlo simulation by repeating a simulation of}
\PY{l+s+sd}{    a set of initial conditions (`dimensions`, `num\PYZus{}initial\PYZus{}neutrons`, }
\PY{l+s+sd}{    `max\PYZus{}generation`) for `num\PYZus{}replications`.}
\PY{l+s+sd}{    \PYZdq{}\PYZdq{}\PYZdq{}}
    \PY{c+c1}{\PYZsh{} saves the data}
    \PY{n}{generation\PYZus{}data} \PY{o}{=} \PY{p}{[}\PY{p}{[}\PY{p}{]} \PY{k}{for} \PY{n}{\PYZus{}} \PY{o+ow}{in} \PY{n+nb}{range}\PY{p}{(}\PY{n}{max\PYZus{}generation}\PY{p}{)}\PY{p}{]}
    \PY{n}{start\PYZus{}time} \PY{o}{=} \PY{n}{time}\PY{o}{.}\PY{n}{time}\PY{p}{(}\PY{p}{)}
    
    \PY{c+c1}{\PYZsh{} index and interval for debugging}
    \PY{n}{interval} \PY{o}{=} \PY{n+nb}{max}\PY{p}{(}\PY{n}{num\PYZus{}replications} \PY{o}{/}\PY{o}{/} \PY{l+m+mi}{10}\PY{p}{,} \PY{l+m+mi}{1}\PY{p}{)}
    
    \PY{c+c1}{\PYZsh{} repeats the simulation for `num\PYZus{}replications`}
    \PY{k}{for} \PY{n}{i} \PY{o+ow}{in} \PY{n+nb}{range}\PY{p}{(}\PY{n}{num\PYZus{}replications}\PY{p}{)}\PY{p}{:}
        \PY{n}{simulation\PYZus{}result} \PY{o}{=} \PY{n}{simulate}\PY{p}{(}
            \PY{n}{max\PYZus{}generation}\PY{p}{,} 
            \PY{n}{dimensions}\PY{p}{,}   
            \PY{n}{num\PYZus{}initial\PYZus{}neutrons}\PY{p}{,} 
            \PY{n}{plot}\PY{o}{=}\PY{k+kc}{False}
        \PY{p}{)}\PY{o}{.}\PY{n}{result\PYZus{}trees}
        
        \PY{c+c1}{\PYZsh{} then append the results for each tree}
        \PY{k}{for} \PY{n}{tree} \PY{o+ow}{in} \PY{n}{simulation\PYZus{}result}\PY{p}{:}
            \PY{n}{k\PYZus{}values} \PY{o}{=} \PY{n}{calculate\PYZus{}k\PYZus{}values}\PY{p}{(}\PY{n}{tree}\PY{p}{)}
            \PY{k}{for} \PY{n}{j}\PY{p}{,} \PY{n}{k} \PY{o+ow}{in} \PY{n+nb}{enumerate}\PY{p}{(}\PY{n}{k\PYZus{}values}\PY{p}{)}\PY{p}{:}
                \PY{n}{generation\PYZus{}data}\PY{p}{[}\PY{n}{j}\PY{p}{]}\PY{o}{.}\PY{n}{append}\PY{p}{(}\PY{n}{k}\PY{p}{)}
                
        \PY{c+c1}{\PYZsh{} Debug Print}
        \PY{k}{if} \PY{n}{debug\PYZus{}print} \PY{o+ow}{and} \PY{p}{(}\PY{n}{i}\PY{o}{+}\PY{l+m+mi}{1}\PY{p}{)} \PY{o}{\PYZpc{}} \PY{n}{interval} \PY{o}{==} \PY{l+m+mi}{0}\PY{p}{:}
            \PY{n+nb}{print}\PY{p}{(}\PY{l+s+sa}{f}\PY{l+s+s1}{\PYZsq{}}\PY{l+s+si}{\PYZob{}}\PY{p}{(}\PY{p}{(}\PY{n}{i}\PY{o}{+}\PY{l+m+mi}{1}\PY{p}{)}\PY{+w}{ }\PY{o}{/}\PY{+w}{ }\PY{n}{num\PYZus{}replications}\PY{p}{)}\PY{+w}{ }\PY{o}{*}\PY{+w}{ }\PY{l+m+mi}{100}\PY{l+s+si}{:}\PY{l+s+s1}{.0f}\PY{l+s+si}{\PYZcb{}}\PY{l+s+s1}{\PYZpc{} \PYZhy{} Replication }\PY{l+s+si}{\PYZob{}}\PY{n}{i}\PY{o}{+}\PY{l+m+mi}{1}\PY{l+s+si}{\PYZcb{}}\PY{l+s+s1}{/}\PY{l+s+si}{\PYZob{}}\PY{n}{num\PYZus{}replications}\PY{l+s+si}{\PYZcb{}}\PY{l+s+s1}{\PYZsq{}}\PY{p}{)}

    \PY{n}{results} \PY{o}{=} \PY{n}{MonteCarloResults}\PY{p}{(}\PY{p}{)}
    \PY{n}{results}\PY{o}{.}\PY{n}{generations}     \PY{o}{=} \PY{p}{[}\PY{p}{]}
    \PY{n}{results}\PY{o}{.}\PY{n}{mean\PYZus{}k\PYZus{}values}   \PY{o}{=} \PY{p}{[}\PY{p}{]}
    \PY{n}{results}\PY{o}{.}\PY{n}{k\PYZus{}value\PYZus{}uncerts} \PY{o}{=} \PY{p}{[}\PY{p}{]}
    
    \PY{k}{for} \PY{n}{gen} \PY{o+ow}{in} \PY{n+nb}{range}\PY{p}{(}\PY{n+nb}{len}\PY{p}{(}\PY{n}{generation\PYZus{}data}\PY{p}{)}\PY{p}{)}\PY{p}{:}
        \PY{n}{data} \PY{o}{=} \PY{n}{GenerationData}\PY{p}{(}\PY{p}{)}
        
        \PY{n}{data}\PY{o}{.}\PY{n}{k\PYZus{}values\PYZus{}raw} \PY{o}{=} \PY{n}{np}\PY{o}{.}\PY{n}{array}\PY{p}{(}\PY{n}{generation\PYZus{}data}\PY{p}{[}\PY{n}{gen}\PY{p}{]}\PY{p}{)}
        
        \PY{k}{if} \PY{n}{data}\PY{o}{.}\PY{n}{k\PYZus{}values\PYZus{}raw}\PY{o}{.}\PY{n}{size} \PY{o}{\PYZgt{}} \PY{l+m+mi}{0}\PY{p}{:}
            \PY{n}{data}\PY{o}{.}\PY{n}{k\PYZus{}mean}    \PY{o}{=} \PY{n}{np}\PY{o}{.}\PY{n}{mean}\PY{p}{(}\PY{n}{data}\PY{o}{.}\PY{n}{k\PYZus{}values\PYZus{}raw}\PY{p}{)}
            \PY{n}{data}\PY{o}{.}\PY{n}{k\PYZus{}std\PYZus{}dev} \PY{o}{=} \PY{n}{np}\PY{o}{.}\PY{n}{std}\PY{p}{(}\PY{n}{data}\PY{o}{.}\PY{n}{k\PYZus{}values\PYZus{}raw}\PY{p}{)}
            \PY{n}{data}\PY{o}{.}\PY{n}{k\PYZus{}std\PYZus{}err} \PY{o}{=} \PY{n}{data}\PY{o}{.}\PY{n}{k\PYZus{}std\PYZus{}dev} \PY{o}{/} \PY{n}{np}\PY{o}{.}\PY{n}{sqrt}\PY{p}{(}\PY{n}{data}\PY{o}{.}\PY{n}{k\PYZus{}values\PYZus{}raw}\PY{o}{.}\PY{n}{size}\PY{p}{)}
            \PY{n}{data}\PY{o}{.}\PY{n}{rel\PYZus{}k\PYZus{}std\PYZus{}err} \PY{o}{=} \PY{n}{data}\PY{o}{.}\PY{n}{k\PYZus{}std\PYZus{}err} \PY{o}{/} \PY{n}{data}\PY{o}{.}\PY{n}{k\PYZus{}mean} \PYZbs{}
                \PY{k}{if} \PY{n}{data}\PY{o}{.}\PY{n}{k\PYZus{}mean} \PY{o}{\PYZgt{}} \PY{l+m+mi}{0} \PY{k}{else} \PY{l+m+mi}{0}
            
            \PY{n}{results}\PY{o}{.}\PY{n}{mean\PYZus{}k\PYZus{}values}\PY{o}{.}\PY{n}{append}\PY{p}{(}\PY{n}{data}\PY{o}{.}\PY{n}{k\PYZus{}mean}\PY{p}{)}
            \PY{n}{results}\PY{o}{.}\PY{n}{k\PYZus{}value\PYZus{}uncerts}\PY{o}{.}\PY{n}{append}\PY{p}{(}\PY{n}{data}\PY{o}{.}\PY{n}{k\PYZus{}std\PYZus{}err}\PY{p}{)}
        
        \PY{k}{if} \PY{n}{plot}\PY{p}{:}
            \PY{n}{plt}\PY{o}{.}\PY{n}{figure}\PY{p}{(}\PY{n}{figsize}\PY{o}{=}\PY{p}{(}\PY{l+m+mi}{6}\PY{p}{,} \PY{l+m+mi}{4}\PY{p}{)}\PY{p}{)}

            \PY{c+c1}{\PYZsh{} Histogram}
            \PY{n}{plt}\PY{o}{.}\PY{n}{hist}\PY{p}{(}
                \PY{n}{data}\PY{o}{.}\PY{n}{k\PYZus{}values\PYZus{}raw}\PY{p}{,}
                \PY{n}{bins}\PY{o}{=}\PY{l+m+mi}{10}\PY{p}{,}
                \PY{n}{alpha}\PY{o}{=}\PY{l+m+mf}{0.6}\PY{p}{,}
                \PY{n}{edgecolor}\PY{o}{=}\PY{l+s+s2}{\PYZdq{}}\PY{l+s+s2}{black}\PY{l+s+s2}{\PYZdq{}}\PY{p}{,}
                \PY{n}{color}\PY{o}{=}\PY{n}{plt}\PY{o}{.}\PY{n}{get\PYZus{}cmap}\PY{p}{(}\PY{l+s+s2}{\PYZdq{}}\PY{l+s+s2}{viridis}\PY{l+s+s2}{\PYZdq{}}\PY{p}{)}\PY{p}{(}\PY{n}{gen} \PY{o}{/} \PY{n+nb}{max}\PY{p}{(}\PY{l+m+mi}{1}\PY{p}{,} \PY{n}{max\PYZus{}generation}\PY{o}{\PYZhy{}}\PY{l+m+mi}{1}\PY{p}{)}\PY{p}{)}
            \PY{p}{)}

            \PY{c+c1}{\PYZsh{} Vertical line for mean}
            \PY{n}{plt}\PY{o}{.}\PY{n}{axvline}\PY{p}{(}\PY{n}{data}\PY{o}{.}\PY{n}{k\PYZus{}mean}\PY{p}{,} \PY{n}{color}\PY{o}{=}\PY{l+s+s1}{\PYZsq{}}\PY{l+s+s1}{red}\PY{l+s+s1}{\PYZsq{}}\PY{p}{,} \PY{n}{linestyle}\PY{o}{=}\PY{l+s+s2}{\PYZdq{}}\PY{l+s+s2}{\PYZhy{}\PYZhy{}}\PY{l+s+s2}{\PYZdq{}}\PY{p}{,} \PY{n}{linewidth}\PY{o}{=}\PY{l+m+mf}{1.5}\PY{p}{)}

            \PY{c+c1}{\PYZsh{} Titles and labels}
            \PY{n}{plt}\PY{o}{.}\PY{n}{title}\PY{p}{(}\PY{l+s+sa}{f}\PY{l+s+s2}{\PYZdq{}}\PY{l+s+s2}{Generation }\PY{l+s+si}{\PYZob{}}\PY{n}{gen}\PY{l+s+si}{\PYZcb{}}\PY{l+s+s2}{ to }\PY{l+s+si}{\PYZob{}}\PY{n}{gen}\PY{o}{+}\PY{l+m+mi}{1}\PY{l+s+si}{\PYZcb{}}\PY{l+s+s2}{\PYZdq{}}\PY{p}{)}
            \PY{n}{plt}\PY{o}{.}\PY{n}{xlabel}\PY{p}{(}\PY{l+s+s2}{\PYZdq{}}\PY{l+s+s2}{k value}\PY{l+s+s2}{\PYZdq{}}\PY{p}{)}
            \PY{n}{plt}\PY{o}{.}\PY{n}{ylabel}\PY{p}{(}\PY{l+s+s2}{\PYZdq{}}\PY{l+s+s2}{Frequency}\PY{l+s+s2}{\PYZdq{}}\PY{p}{)}
            \PY{n}{plt}\PY{o}{.}\PY{n}{grid}\PY{p}{(}\PY{n}{alpha}\PY{o}{=}\PY{l+m+mf}{0.3}\PY{p}{)}

            \PY{c+c1}{\PYZsh{} Text box with stats}
            \PY{n}{plt}\PY{o}{.}\PY{n}{text}\PY{p}{(}
                \PY{l+m+mf}{0.95}\PY{p}{,} \PY{l+m+mf}{0.95}\PY{p}{,}
                \PY{l+s+sa}{f}\PY{l+s+s2}{\PYZdq{}}\PY{l+s+s2}{\PYZdl{}}\PY{l+s+se}{\PYZbs{}\PYZbs{}}\PY{l+s+s2}{bar}\PY{l+s+se}{\PYZob{}\PYZob{}}\PY{l+s+s2}{k}\PY{l+s+se}{\PYZcb{}\PYZcb{}}\PY{l+s+s2}{\PYZdl{}=}\PY{l+s+si}{\PYZob{}}\PY{n}{data}\PY{o}{.}\PY{n}{k\PYZus{}mean}\PY{l+s+si}{:}\PY{l+s+s2}{.3f}\PY{l+s+si}{\PYZcb{}}\PY{l+s+se}{\PYZbs{}n}\PY{l+s+s2}{±}\PY{l+s+si}{\PYZob{}}\PY{n}{data}\PY{o}{.}\PY{n}{k\PYZus{}std\PYZus{}err}\PY{l+s+si}{:}\PY{l+s+s2}{.3f}\PY{l+s+si}{\PYZcb{}}\PY{l+s+s2}{ (stderr)}\PY{l+s+se}{\PYZbs{}n}\PY{l+s+s2}{σ=}\PY{l+s+si}{\PYZob{}}\PY{n}{data}\PY{o}{.}\PY{n}{k\PYZus{}std\PYZus{}dev}\PY{l+s+si}{:}\PY{l+s+s2}{.3f}\PY{l+s+si}{\PYZcb{}}\PY{l+s+s2}{\PYZdq{}}\PY{p}{,}
                \PY{n}{verticalalignment}\PY{o}{=}\PY{l+s+s1}{\PYZsq{}}\PY{l+s+s1}{top}\PY{l+s+s1}{\PYZsq{}}\PY{p}{,}
                \PY{n}{horizontalalignment}\PY{o}{=}\PY{l+s+s1}{\PYZsq{}}\PY{l+s+s1}{right}\PY{l+s+s1}{\PYZsq{}}\PY{p}{,}
                \PY{n}{transform}\PY{o}{=}\PY{n}{plt}\PY{o}{.}\PY{n}{gca}\PY{p}{(}\PY{p}{)}\PY{o}{.}\PY{n}{transAxes}\PY{p}{,}
                \PY{n}{fontsize}\PY{o}{=}\PY{l+m+mi}{10}\PY{p}{,}
                \PY{n}{bbox}\PY{o}{=}\PY{n+nb}{dict}\PY{p}{(}\PY{n}{facecolor}\PY{o}{=}\PY{l+s+s1}{\PYZsq{}}\PY{l+s+s1}{white}\PY{l+s+s1}{\PYZsq{}}\PY{p}{,} \PY{n}{alpha}\PY{o}{=}\PY{l+m+mf}{0.6}\PY{p}{,} \PY{n}{edgecolor}\PY{o}{=}\PY{l+s+s1}{\PYZsq{}}\PY{l+s+s1}{gray}\PY{l+s+s1}{\PYZsq{}}\PY{p}{)}
            \PY{p}{)}

            \PY{n}{plt}\PY{o}{.}\PY{n}{tight\PYZus{}layout}\PY{p}{(}\PY{p}{)}
            \PY{n}{data}\PY{o}{.}\PY{n}{histogram} \PY{o}{=} \PY{n}{save\PYZus{}graph\PYZus{}and\PYZus{}close}\PY{p}{(}\PY{p}{)}
            
        \PY{n}{results}\PY{o}{.}\PY{n}{generations}\PY{o}{.}\PY{n}{append}\PY{p}{(}\PY{n}{data}\PY{p}{)}
    
    \PY{n}{results}\PY{o}{.}\PY{n}{mean\PYZus{}k\PYZus{}values}   \PY{o}{=} \PY{n}{np}\PY{o}{.}\PY{n}{asarray}\PY{p}{(}\PY{n}{results}\PY{o}{.}\PY{n}{mean\PYZus{}k\PYZus{}values}\PY{p}{)}
    \PY{n}{results}\PY{o}{.}\PY{n}{k\PYZus{}value\PYZus{}uncerts} \PY{o}{=} \PY{n}{np}\PY{o}{.}\PY{n}{asarray}\PY{p}{(}\PY{n}{results}\PY{o}{.}\PY{n}{k\PYZus{}value\PYZus{}uncerts}\PY{p}{)}
    
    \PY{k}{if} \PY{n}{plot}\PY{p}{:}
        \PY{n}{plt}\PY{o}{.}\PY{n}{figure}\PY{p}{(}\PY{p}{)}
        \PY{n}{plt}\PY{o}{.}\PY{n}{title}\PY{p}{(}\PY{l+s+s1}{\PYZsq{}}\PY{l+s+s1}{Mean Replication Rate by Generation}\PY{l+s+s1}{\PYZsq{}}\PY{p}{)}
        \PY{n}{plt}\PY{o}{.}\PY{n}{errorbar}\PY{p}{(}
            \PY{n}{x}\PY{o}{=}\PY{n}{np}\PY{o}{.}\PY{n}{arange}\PY{p}{(}\PY{n}{results}\PY{o}{.}\PY{n}{mean\PYZus{}k\PYZus{}values}\PY{o}{.}\PY{n}{size}\PY{p}{)}\PY{p}{,}
            \PY{n}{y}\PY{o}{=}\PY{n}{results}\PY{o}{.}\PY{n}{mean\PYZus{}k\PYZus{}values}\PY{p}{,} 
            \PY{n}{yerr}\PY{o}{=}\PY{n}{results}\PY{o}{.}\PY{n}{k\PYZus{}value\PYZus{}uncerts}\PY{p}{,}
            \PY{n}{fmt}\PY{o}{=}\PY{l+s+s1}{\PYZsq{}}\PY{l+s+s1}{bo\PYZhy{}\PYZhy{}}\PY{l+s+s1}{\PYZsq{}}\PY{p}{,}
            \PY{n}{label}\PY{o}{=}\PY{l+s+s1}{\PYZsq{}}\PY{l+s+s1}{k}\PY{l+s+s1}{\PYZsq{}}\PY{p}{)}
        \PY{n}{plt}\PY{o}{.}\PY{n}{xticks}\PY{p}{(}\PY{n}{np}\PY{o}{.}\PY{n}{arange}\PY{p}{(}\PY{n}{results}\PY{o}{.}\PY{n}{mean\PYZus{}k\PYZus{}values}\PY{o}{.}\PY{n}{size}\PY{p}{)}\PY{p}{)}
        \PY{n}{plt}\PY{o}{.}\PY{n}{xlabel}\PY{p}{(}\PY{l+s+s1}{\PYZsq{}}\PY{l+s+s1}{Generation}\PY{l+s+s1}{\PYZsq{}}\PY{p}{)}
        \PY{n}{plt}\PY{o}{.}\PY{n}{ylabel}\PY{p}{(}\PY{l+s+s1}{\PYZsq{}}\PY{l+s+s1}{Replication Rate}\PY{l+s+s1}{\PYZsq{}}\PY{p}{)}
        \PY{n}{results}\PY{o}{.}\PY{n}{k\PYZus{}value\PYZus{}plot} \PY{o}{=} \PY{n}{save\PYZus{}graph\PYZus{}and\PYZus{}close}\PY{p}{(}\PY{p}{)}
        
    \PY{n}{end\PYZus{}time} \PY{o}{=} \PY{n}{time}\PY{o}{.}\PY{n}{time}\PY{p}{(}\PY{p}{)}
    \PY{n}{elapsed\PYZus{}time} \PY{o}{=} \PY{n}{end\PYZus{}time} \PY{o}{\PYZhy{}} \PY{n}{start\PYZus{}time}
    
    \PY{k}{if} \PY{n}{debug\PYZus{}print}\PY{p}{:}
        \PY{n+nb}{print}\PY{p}{(}\PY{l+s+sa}{f}\PY{l+s+s2}{\PYZdq{}}\PY{l+s+s2}{Executed in }\PY{l+s+si}{\PYZob{}}\PY{n+nb}{int}\PY{p}{(}\PY{n}{elapsed\PYZus{}time}\PY{p}{)}\PY{l+s+si}{\PYZcb{}}\PY{l+s+s2}{ s }\PY{l+s+si}{\PYZob{}}\PY{n+nb}{int}\PY{p}{(}\PY{p}{(}\PY{n}{elapsed\PYZus{}time}\PY{+w}{ }\PY{o}{\PYZhy{}}\PY{+w}{ }\PY{n+nb}{int}\PY{p}{(}\PY{n}{elapsed\PYZus{}time}\PY{p}{)}\PY{p}{)}\PY{+w}{ }\PY{o}{*}\PY{+w}{ }\PY{l+m+mi}{1000}\PY{p}{)}\PY{l+s+si}{\PYZcb{}}\PY{l+s+s2}{ ms}\PY{l+s+s2}{\PYZdq{}}\PY{p}{)}
    
    \PY{k}{return} \PY{n}{results}
\end{Verbatim}
\end{tcolorbox}

    \hypertarget{example-monte-carlo-test}{%
\subsection{Example Monte Carlo Test}\label{example-monte-carlo-test}}

    \textbf{Code Block Summary:} This is an example Monte Carlo simulation
for 3 generations (with a singular starting neutron), 250 initial
neutrons, 300 replications, and a \(10 \times 10 \times 10\) box.

    \begin{tcolorbox}[breakable, size=fbox, boxrule=1pt, pad at break*=1mm,colback=cellbackground, colframe=cellborder]
\prompt{In}{incolor}{26}{\boxspacing}
\begin{Verbatim}[commandchars=\\\{\}]
\PY{n}{example\PYZus{}mc\PYZus{}results} \PY{o}{=} \PY{n}{monte\PYZus{}carlo}\PY{p}{(}
    \PY{n}{dimensions}            \PY{o}{=} \PY{p}{(}\PY{l+m+mi}{10}\PY{p}{,}\PY{l+m+mi}{10}\PY{p}{,}\PY{l+m+mi}{10}\PY{p}{)}\PY{p}{,} 
    \PY{n}{num\PYZus{}initial\PYZus{}neutrons}  \PY{o}{=} \PY{l+m+mi}{250}\PY{p}{,} 
    \PY{n}{num\PYZus{}replications}      \PY{o}{=} \PY{l+m+mi}{300}\PY{p}{,}
    \PY{n}{max\PYZus{}generation}        \PY{o}{=} \PY{l+m+mi}{3}\PY{p}{,}
    \PY{n}{plot}        \PY{o}{=} \PY{k+kc}{True}\PY{p}{,}
    \PY{n}{debug\PYZus{}print} \PY{o}{=} \PY{k+kc}{True}
\PY{p}{)}
\end{Verbatim}
\end{tcolorbox}

    \begin{Verbatim}[commandchars=\\\{\}]
10\% - Replication 30/300
20\% - Replication 60/300
30\% - Replication 90/300
40\% - Replication 120/300
50\% - Replication 150/300
60\% - Replication 180/300
70\% - Replication 210/300
80\% - Replication 240/300
90\% - Replication 270/300
100\% - Replication 300/300
Executed in 9 s 439 ms
    \end{Verbatim}

    \begin{tcolorbox}[breakable, size=fbox, boxrule=1pt, pad at break*=1mm,colback=cellbackground, colframe=cellborder]
\prompt{In}{incolor}{27}{\boxspacing}
\begin{Verbatim}[commandchars=\\\{\}]
\PY{n}{display}\PY{p}{(}\PY{n}{example\PYZus{}mc\PYZus{}results}\PY{o}{.}\PY{n}{k\PYZus{}value\PYZus{}plot}\PY{p}{)}
\end{Verbatim}
\end{tcolorbox}

    \begin{center}
    \adjustimage{max size={0.9\linewidth}{0.9\paperheight}}{Project 3_files/Project 3_59_0.png}
    \end{center}
    { \hspace*{\fill} \\}
    
    \textbf{Figure 2.} A clear upward trend in mean replication rate as a
function of generation can be seen, and this seems counterintuitive at
first, as initial Neutrons are statistically most likely to be toward
the exact center of the box, so later Neutrons become more likely to
spread outward radially (which implies the opposite trend). However,
this trend of increase is likely the result of the mathematical spectrum
of possible \(k\) values that can be produced at each neutron level. The
first generation can only ever replicate 0, 1, or 2 neutrons exactly, so
an individual neutron \(i\) will have exactly \(k_i = 0, 1, 2\): a
discrete range. So, in short, because the range of possible \(k\) values
for a given neutron is so limited, the negative contribution from low
\(k\) values will be much higher.

    \begin{tcolorbox}[breakable, size=fbox, boxrule=1pt, pad at break*=1mm,colback=cellbackground, colframe=cellborder]
\prompt{In}{incolor}{28}{\boxspacing}
\begin{Verbatim}[commandchars=\\\{\}]
\PY{n}{display}\PY{p}{(}\PY{n}{example\PYZus{}mc\PYZus{}results}\PY{o}{.}\PY{n}{generations}\PY{p}{[}\PY{l+m+mi}{0}\PY{p}{]}\PY{o}{.}\PY{n}{histogram}\PY{p}{)}
\end{Verbatim}
\end{tcolorbox}

    \begin{center}
    \adjustimage{max size={0.9\linewidth}{0.9\paperheight}}{Project 3_files/Project 3_61_0.png}
    \end{center}
    { \hspace*{\fill} \\}
    
    \textbf{Figure 3.} The first generation will usually be able to
multiplying into one or two more neutrons rather than producing none due
to the nature of always having to start out within the box (and being
more likely to begin closer to the middle than closer to the edge as a
result).

    \begin{tcolorbox}[breakable, size=fbox, boxrule=1pt, pad at break*=1mm,colback=cellbackground, colframe=cellborder]
\prompt{In}{incolor}{29}{\boxspacing}
\begin{Verbatim}[commandchars=\\\{\}]
\PY{n}{display}\PY{p}{(}\PY{n}{example\PYZus{}mc\PYZus{}results}\PY{o}{.}\PY{n}{generations}\PY{p}{[}\PY{l+m+mi}{1}\PY{p}{]}\PY{o}{.}\PY{n}{histogram}\PY{p}{)}
\end{Verbatim}
\end{tcolorbox}

    \begin{center}
    \adjustimage{max size={0.9\linewidth}{0.9\paperheight}}{Project 3_files/Project 3_63_0.png}
    \end{center}
    { \hspace*{\fill} \\}
    
    \textbf{Figure 4.} For the next generation, the \(k\)-values begin to
spread out as the successive neutrons become more likely to be in a
variety of places within the box. For example, the first neutron may be
most likely to begin closer to the middle, but subsequent neutrons
become more likely to become closer to the edges of the box. However,
the mean \(k\) value is slowly increasing as compared to the first
generation simply due to the number of total children that can possibly
be generated in each successive generation increasing by
\(2^n - 2^{n-1}\) for each generation \(n\), and furthermore, this also
means that as the generations grow, the standard deviation \(\sigma\)
should decrease due to the sheer number of neutrons making individual
neutrons being lost have less effect on the overall \(k\) value.

    \begin{tcolorbox}[breakable, size=fbox, boxrule=1pt, pad at break*=1mm,colback=cellbackground, colframe=cellborder]
\prompt{In}{incolor}{30}{\boxspacing}
\begin{Verbatim}[commandchars=\\\{\}]
\PY{n}{display}\PY{p}{(}\PY{n}{example\PYZus{}mc\PYZus{}results}\PY{o}{.}\PY{n}{generations}\PY{p}{[}\PY{l+m+mi}{2}\PY{p}{]}\PY{o}{.}\PY{n}{histogram}\PY{p}{)}
\end{Verbatim}
\end{tcolorbox}

    \begin{center}
    \adjustimage{max size={0.9\linewidth}{0.9\paperheight}}{Project 3_files/Project 3_65_0.png}
    \end{center}
    { \hspace*{\fill} \\}
    
    \textbf{Figure 5.} The aforementioned pattern in Figure 3. is continued
here, as the spread of \(k\) values continues to further out. As
expected, \(k\) has increased and the standard deviation has decreased
due to the sheer number of neutrons that exist at this generation.

    \hypertarget{determining-the-minimal-replication-rate-m-for-producing-accurate-aggregate-multiplication-rates}{%
\section{\texorpdfstring{Determining the Minimal Replication Rate \(m\)
for Producing Accurate Aggregate Multiplication
Rates}{Determining the Minimal Replication Rate m for Producing Accurate Aggregate Multiplication Rates}}\label{determining-the-minimal-replication-rate-m-for-producing-accurate-aggregate-multiplication-rates}}

    Before we determine the generation depth to work with, it'll be
extremely important to consider the minimum number of replications to
use to have accurate \(k\)-values. The way we can do this is based on
how the relative standard error (the standard error divided by the
multiplication rate) of a system changes based on our number of
replications, and in particular, we'll want the relative standard error
at whatever generation ends up being our minimal stable generation to be
reasonably low. To do this, we can just pick some generation depth
(which I'll refer to as the `base' generation) that we know will be
lower than minimal stable generation (as this should have a higher
standard error) then pick a replication rate that minimizes the relative
standard error for a case simulation.

    \textbf{Side Note on the Number of Initial Neutrons:} Before determining
the minimum-stable generation \(g\) or the mimimum-accurate number of
replications \(m\), we need to set a number of initial neutrons to
perform Monte-Carlo simulations with. It doesn't particularly matter
what number we set this to, as the number of initial neutrons with which
we perform the simulation with is exactly the same as performing that
simulation for each initial neutron (so each additional starting neutron
is like another replication anyway). This means that the total number of
replications, in reality, can be looked at as being the number of
replications \emph{multiplied by} the number of initial neutrons, so as
we will determine later, we're actually really performing 25,000
replications for each phase coordinate if we set the number of initial
neutrons at 250, which I'll keep consistent for the rest of this
investigation.

    As a good base generation, let's just consider going as far as
generation 5. Now, we'll want to pick a considerably nontrivial reactor
geometry (more intuitively stable, so higher volumes and more even
aspect ratios) and a trivial geometry to try and compare them.

    To begin, let's consider a nearly-flat plane (\(R=30\)) with a small
volume (\(V=1\)) as our unstable case:

    \begin{tcolorbox}[breakable, size=fbox, boxrule=1pt, pad at break*=1mm,colback=cellbackground, colframe=cellborder]
\prompt{In}{incolor}{31}{\boxspacing}
\begin{Verbatim}[commandchars=\\\{\}]
\PY{n}{min\PYZus{}rep\PYZus{}unstable\PYZus{}case} \PY{o}{=} \PY{n}{monte\PYZus{}carlo}\PY{p}{(}
    \PY{n}{dimensions}            \PY{o}{=} \PY{n}{calculate\PYZus{}dimensions}\PY{p}{(}\PY{l+m+mi}{30}\PY{p}{,} \PY{l+m+mi}{1}\PY{p}{)}\PY{p}{,} 
    \PY{n}{num\PYZus{}initial\PYZus{}neutrons}  \PY{o}{=} \PY{l+m+mi}{250}\PY{p}{,} 
    \PY{n}{num\PYZus{}replications}      \PY{o}{=} \PY{l+m+mi}{100}\PY{p}{,}
    \PY{n}{max\PYZus{}generation}        \PY{o}{=} \PY{l+m+mi}{5}\PY{p}{,}
    \PY{n}{plot}                  \PY{o}{=} \PY{k+kc}{True}\PY{p}{,}
    \PY{n}{debug\PYZus{}print}           \PY{o}{=} \PY{k+kc}{True}
\PY{p}{)}
\end{Verbatim}
\end{tcolorbox}

    \begin{Verbatim}[commandchars=\\\{\}]
10\% - Replication 10/100
20\% - Replication 20/100
30\% - Replication 30/100
40\% - Replication 40/100
50\% - Replication 50/100
60\% - Replication 60/100
70\% - Replication 70/100
80\% - Replication 80/100
90\% - Replication 90/100
100\% - Replication 100/100
Executed in 1 s 312 ms
    \end{Verbatim}

    \begin{tcolorbox}[breakable, size=fbox, boxrule=1pt, pad at break*=1mm,colback=cellbackground, colframe=cellborder]
\prompt{In}{incolor}{32}{\boxspacing}
\begin{Verbatim}[commandchars=\\\{\}]
\PY{n}{display}\PY{p}{(}\PY{n}{min\PYZus{}rep\PYZus{}unstable\PYZus{}case}\PY{o}{.}\PY{n}{k\PYZus{}value\PYZus{}plot}\PY{p}{)}
\end{Verbatim}
\end{tcolorbox}

    \begin{center}
    \adjustimage{max size={0.9\linewidth}{0.9\paperheight}}{Project 3_files/Project 3_73_0.png}
    \end{center}
    { \hspace*{\fill} \\}
    
    \textbf{Figure 6.} The progression didn't even make it to 5 generations,
so this geometry is trivially unstable, and there would be absolutely no
uncertainty in the long-run \(k\)-value as being completely zero.

    Now, let's consider a cube (\(R=1\)) with a large volume (\(V=2000\)
\(m^3\)) as our stable case:

    \begin{tcolorbox}[breakable, size=fbox, boxrule=1pt, pad at break*=1mm,colback=cellbackground, colframe=cellborder]
\prompt{In}{incolor}{33}{\boxspacing}
\begin{Verbatim}[commandchars=\\\{\}]
\PY{n}{min\PYZus{}rep\PYZus{}stable\PYZus{}case} \PY{o}{=} \PY{n}{monte\PYZus{}carlo}\PY{p}{(}
    \PY{n}{dimensions}            \PY{o}{=} \PY{n}{calculate\PYZus{}dimensions}\PY{p}{(}\PY{l+m+mi}{1}\PY{p}{,} \PY{l+m+mi}{2000}\PY{p}{)}\PY{p}{,} 
    \PY{n}{num\PYZus{}initial\PYZus{}neutrons}  \PY{o}{=} \PY{l+m+mi}{250}\PY{p}{,} 
    \PY{n}{num\PYZus{}replications}      \PY{o}{=} \PY{l+m+mi}{100}\PY{p}{,}
    \PY{n}{max\PYZus{}generation}        \PY{o}{=} \PY{l+m+mi}{5}\PY{p}{,}
    \PY{n}{plot}                  \PY{o}{=} \PY{k+kc}{True}\PY{p}{,}
    \PY{n}{debug\PYZus{}print}           \PY{o}{=} \PY{k+kc}{True}
\PY{p}{)}
\end{Verbatim}
\end{tcolorbox}

    \begin{Verbatim}[commandchars=\\\{\}]
10\% - Replication 10/100
20\% - Replication 20/100
30\% - Replication 30/100
40\% - Replication 40/100
50\% - Replication 50/100
60\% - Replication 60/100
70\% - Replication 70/100
80\% - Replication 80/100
90\% - Replication 90/100
100\% - Replication 100/100
Executed in 9 s 927 ms
    \end{Verbatim}

    \begin{tcolorbox}[breakable, size=fbox, boxrule=1pt, pad at break*=1mm,colback=cellbackground, colframe=cellborder]
\prompt{In}{incolor}{34}{\boxspacing}
\begin{Verbatim}[commandchars=\\\{\}]
\PY{n}{display}\PY{p}{(}\PY{n}{min\PYZus{}rep\PYZus{}stable\PYZus{}case}\PY{o}{.}\PY{n}{k\PYZus{}value\PYZus{}plot}\PY{p}{)}
\end{Verbatim}
\end{tcolorbox}

    \begin{center}
    \adjustimage{max size={0.9\linewidth}{0.9\paperheight}}{Project 3_files/Project 3_77_0.png}
    \end{center}
    { \hspace*{\fill} \\}
    
    \textbf{Figure 7.} The standard error does seem to decrease with the
number of successive generations.

Based off of this, we can calculate the relative standard error for the
last generation, as the relative standard error for whatever our final
generation should be should be longer:

    \begin{tcolorbox}[breakable, size=fbox, boxrule=1pt, pad at break*=1mm,colback=cellbackground, colframe=cellborder]
\prompt{In}{incolor}{35}{\boxspacing}
\begin{Verbatim}[commandchars=\\\{\}]
\PY{n+nb}{print}\PY{p}{(}\PY{l+s+sa}{f}\PY{l+s+s1}{\PYZsq{}}\PY{l+s+s1}{Relative Standard Error at Generation 4: }\PY{l+s+si}{\PYZob{}}\PY{n}{min\PYZus{}rep\PYZus{}stable\PYZus{}case}\PY{o}{.}\PY{n}{generations}\PY{p}{[}\PY{l+m+mi}{4}\PY{p}{]}\PY{o}{.}\PY{n}{rel\PYZus{}k\PYZus{}std\PYZus{}err}\PY{+w}{ }\PY{o}{*}\PY{+w}{ }\PY{l+m+mi}{100}\PY{l+s+si}{:}\PY{l+s+s1}{.1f}\PY{l+s+si}{\PYZcb{}}\PY{l+s+s1}{\PYZpc{}}\PY{l+s+s1}{\PYZsq{}}\PY{p}{)}
\end{Verbatim}
\end{tcolorbox}

    \begin{Verbatim}[commandchars=\\\{\}]
Relative Standard Error at Generation 4: 0.2\%
    \end{Verbatim}

    As the relative standard error becomes about 0.2\%, we can reasonably
say that it's small enough for the simulation to be accurate at a
replication rate of 100, so we'll just use \(m=100\) replications for
each point.

    \hypertarget{determining-the-minimal-generation-depth-g-for-producing-stable-system-multiplication-rates}{%
\section{\texorpdfstring{Determining the Minimal Generation Depth \(g\)
for Producing Stable-System Multiplication
Rates}{Determining the Minimal Generation Depth g for Producing Stable-System Multiplication Rates}}\label{determining-the-minimal-generation-depth-g-for-producing-stable-system-multiplication-rates}}

    As stated previously, each box geometry \((R,V)\) should eventually
converge to some characteristic long-run replication rate after
performing a number of generations of replications, with the purposes of
the programs utilized in this exploration to produce the mapping from a
box geometry to its characteristic replication rate
\[(R,V) \mapsto \overline{k}.\] However, the exact number of generations
required to converge (or, for a physical interpretation, for the reactor
to reach a stable state) isn't a well-defined or easily calculable
value. What can be reasonably assumed, however, is that different box
geometries should alter in how many generations they take to converge by
a predictable pattern, and then we can leverage this pattern to
determine which geometry in our intended phase space requires the
longest generation depth to converge.

    To begin, we can reasonably assume that trivially unstable systems
should converge to a characteristic replication rate of \(k=0\), meaning
that after some number of generations, the reaction stops completely (so
the reactor is completely unstable). We can re-examine figures 6 and 7
to aid in this: the trivially-unstable geometry quickly converges to a
zero replication rate, whereas the stable geometry is unable continues
to grow towards its stable replication rate over the course of the
simulation. Therefore, we'll only focus on stable systems for
determining the minimal generation depth for producing a stable long-run
value across all of our phase space.

In this simulation, I'm planning to explore a phase space of completely
even aspect ratios at \(R=1\) to skewed, more flat aspect ratios at
\(R=20\), and volumes from \(V=1\) \(m^3\) to \(200\) \(m^3\):
\[D = \{ 1 \le R \le 20 \} \otimes \{ 1 \le V \le 200 \},\] so with this
in mind, the most stable configuration should apriori be \(R=1,V=200\).
We can use this, pick some large generation number, then observe its
\(k\)-progression graph to pick a point at which it becomes roughly
stable.

    \begin{tcolorbox}[breakable, size=fbox, boxrule=1pt, pad at break*=1mm,colback=cellbackground, colframe=cellborder]
\prompt{In}{incolor}{36}{\boxspacing}
\begin{Verbatim}[commandchars=\\\{\}]
\PY{n}{min\PYZus{}gen\PYZus{}case} \PY{o}{=} \PY{n}{monte\PYZus{}carlo}\PY{p}{(}
    \PY{n}{dimensions}            \PY{o}{=} \PY{n}{calculate\PYZus{}dimensions}\PY{p}{(}\PY{l+m+mi}{1}\PY{p}{,} \PY{l+m+mi}{200}\PY{p}{)}\PY{p}{,} 
    \PY{n}{num\PYZus{}initial\PYZus{}neutrons}  \PY{o}{=} \PY{l+m+mi}{250}\PY{p}{,} 
    \PY{n}{num\PYZus{}replications}      \PY{o}{=} \PY{l+m+mi}{100}\PY{p}{,}
    \PY{n}{max\PYZus{}generation}        \PY{o}{=} \PY{l+m+mi}{25}\PY{p}{,}
    \PY{n}{plot}                  \PY{o}{=} \PY{k+kc}{True}\PY{p}{,}
    \PY{n}{debug\PYZus{}print}           \PY{o}{=} \PY{k+kc}{True}
\PY{p}{)}
\end{Verbatim}
\end{tcolorbox}

    \begin{Verbatim}[commandchars=\\\{\}]
10\% - Replication 10/100
20\% - Replication 20/100
30\% - Replication 30/100
40\% - Replication 40/100
50\% - Replication 50/100
60\% - Replication 60/100
70\% - Replication 70/100
80\% - Replication 80/100
90\% - Replication 90/100
100\% - Replication 100/100
Executed in 144 s 101 ms
    \end{Verbatim}

    \begin{tcolorbox}[breakable, size=fbox, boxrule=1pt, pad at break*=1mm,colback=cellbackground, colframe=cellborder]
\prompt{In}{incolor}{37}{\boxspacing}
\begin{Verbatim}[commandchars=\\\{\}]
\PY{n}{display}\PY{p}{(}\PY{n}{min\PYZus{}gen\PYZus{}case}\PY{o}{.}\PY{n}{k\PYZus{}value\PYZus{}plot}\PY{p}{)}
\end{Verbatim}
\end{tcolorbox}

    \begin{center}
    \adjustimage{max size={0.9\linewidth}{0.9\paperheight}}{Project 3_files/Project 3_85_0.png}
    \end{center}
    { \hspace*{\fill} \\}
    
    \textbf{Figure 8.} It's difficult to exactly pinpoint where this system
can be fully said to be stable, but we can be fairly confident that the
system has reached long-term stability around generation 13 or 14, so to
be absolutely sure, we can set the minimum-stable generation depth as
\(g=15\).

    \hypertarget{phase-investigation}{%
\section{Phase Investigation}\label{phase-investigation}}

    \textbf{Code Block Summary:} This is the actual phase investigation,
which performs \texttt{monte\_carlo} for each of the box sizes produced
given a domain of aspect ratios \((R_{min}, R_{max})\) and a domain of
areas \((V_{min}, V_{max})\), alongside a number of starting neutrons,
and number of replications per condition coordinate \((R,V)\).

    \begin{tcolorbox}[breakable, size=fbox, boxrule=1pt, pad at break*=1mm,colback=cellbackground, colframe=cellborder]
\prompt{In}{incolor}{38}{\boxspacing}
\begin{Verbatim}[commandchars=\\\{\}]
\PY{n+nd}{@dataclass}
\PY{k}{class} \PY{n+nc}{ContourResults}\PY{p}{:}
    \PY{n}{contour\PYZus{}plot}\PY{p}{:} \PY{n}{Image} \PY{o}{=} \PY{k+kc}{None}
    \PY{n}{uncertainty\PYZus{}contour\PYZus{}plot}\PY{p}{:} \PY{n}{Image} \PY{o}{=} \PY{k+kc}{None}

\PY{k}{def} \PY{n+nf}{generate\PYZus{}contour}\PY{p}{(}\PY{n}{aspect\PYZus{}ratio\PYZus{}range}\PY{p}{:} \PY{n+nb}{tuple}\PY{p}{,} \PY{n}{volume\PYZus{}range}\PY{p}{:} \PY{n+nb}{tuple}\PY{p}{,} \PY{n}{max\PYZus{}generation}\PY{p}{:} \PY{n+nb}{int} \PY{o}{=} \PY{l+m+mi}{3}\PY{p}{,}
                     \PY{n}{num\PYZus{}initial\PYZus{}neutrons}\PY{p}{:} \PY{n+nb}{int} \PY{o}{=} \PY{l+m+mi}{1}\PY{p}{,} \PY{n}{num\PYZus{}replications}\PY{p}{:} \PY{n+nb}{int} \PY{o}{=} \PY{l+m+mi}{1}\PY{p}{,}
                     \PY{n}{debug\PYZus{}print}\PY{p}{:} \PY{n+nb}{bool} \PY{o}{=} \PY{k+kc}{False}\PY{p}{)}\PY{p}{:}
\PY{+w}{    }\PY{l+s+sd}{\PYZdq{}\PYZdq{}\PYZdq{}}
\PY{l+s+sd}{    Explores the phase space of (R,V) and generates the overall contour for mean k across generations}
\PY{l+s+sd}{    based on calling monte\PYZus{}carlo for each phase coordinate.}
\PY{l+s+sd}{    \PYZdq{}\PYZdq{}\PYZdq{}}
    \PY{n}{start\PYZus{}time} \PY{o}{=} \PY{n}{time}\PY{o}{.}\PY{n}{time}\PY{p}{(}\PY{p}{)}
    
    \PY{c+c1}{\PYZsh{} Store some of the parameters to explore the contour}
    \PY{n}{contour\PYZus{}params} \PY{o}{=} \PY{p}{(}\PY{n}{max\PYZus{}generation}\PY{p}{,} \PY{n}{num\PYZus{}initial\PYZus{}neutrons}\PY{p}{,} \PY{n}{num\PYZus{}replications}\PY{p}{,} \PY{n}{aspect\PYZus{}ratio\PYZus{}range}\PY{p}{,} \PY{n}{volume\PYZus{}range}\PY{p}{)}
    
    \PY{c+c1}{\PYZsh{} Build the phase}
    \PY{n}{aspect\PYZus{}ratio\PYZus{}domain} \PY{o}{=} \PY{n}{np}\PY{o}{.}\PY{n}{linspace}\PY{p}{(}\PY{o}{*}\PY{n}{aspect\PYZus{}ratio\PYZus{}range}\PY{p}{)}
    \PY{n}{volume\PYZus{}domain} \PY{o}{=} \PY{n}{np}\PY{o}{.}\PY{n}{linspace}\PY{p}{(}\PY{o}{*}\PY{n}{volume\PYZus{}range}\PY{p}{)}
    
    \PY{n}{R}\PY{p}{,} \PY{n}{V} \PY{o}{=} \PY{n}{np}\PY{o}{.}\PY{n}{meshgrid}\PY{p}{(}\PY{n}{aspect\PYZus{}ratio\PYZus{}domain}\PY{p}{,} \PY{n}{volume\PYZus{}domain}\PY{p}{)}
    \PY{n}{K} \PY{o}{=} \PY{n}{np}\PY{o}{.}\PY{n}{zeros}\PY{p}{(}\PY{p}{(}\PY{n}{aspect\PYZus{}ratio\PYZus{}domain}\PY{o}{.}\PY{n}{size}\PY{p}{,} \PY{n}{volume\PYZus{}domain}\PY{o}{.}\PY{n}{size}\PY{p}{)}\PY{p}{,} \PY{n}{dtype}\PY{o}{=}\PY{n+nb}{float}\PY{p}{)}
    \PY{n}{K\PYZus{}uncert} \PY{o}{=} \PY{n}{np}\PY{o}{.}\PY{n}{zeros\PYZus{}like}\PY{p}{(}\PY{n}{K}\PY{p}{)}
    \PY{n}{total\PYZus{}sims} \PY{o}{=} \PY{n}{aspect\PYZus{}ratio\PYZus{}domain}\PY{o}{.}\PY{n}{size} \PY{o}{*} \PY{n}{volume\PYZus{}domain}\PY{o}{.}\PY{n}{size}
    
    \PY{c+c1}{\PYZsh{} index and interval for debugging}
    \PY{n}{idx} \PY{o}{=} \PY{l+m+mi}{0}
    \PY{n}{interval} \PY{o}{=} \PY{n+nb}{max}\PY{p}{(}\PY{n}{total\PYZus{}sims} \PY{o}{/}\PY{o}{/} \PY{l+m+mi}{10}\PY{p}{,} \PY{l+m+mi}{1}\PY{p}{)}
    
    \PY{c+c1}{\PYZsh{} Iterate over the Phase}
    \PY{k}{for} \PY{n}{i}\PY{p}{,} \PY{n}{aspect\PYZus{}ratio} \PY{o+ow}{in} \PY{n+nb}{enumerate}\PY{p}{(}\PY{n}{aspect\PYZus{}ratio\PYZus{}domain}\PY{p}{)}\PY{p}{:}
        \PY{k}{for} \PY{n}{j}\PY{p}{,} \PY{n}{volume} \PY{o+ow}{in} \PY{n+nb}{enumerate}\PY{p}{(}\PY{n}{volume\PYZus{}domain}\PY{p}{)}\PY{p}{:}
            \PY{n}{idx} \PY{o}{+}\PY{o}{=} \PY{l+m+mi}{1}
            
            \PY{n}{dimensions} \PY{o}{=} \PY{n}{calculate\PYZus{}dimensions}\PY{p}{(}\PY{n}{aspect\PYZus{}ratio}\PY{p}{,} \PY{n}{volume}\PY{p}{)}
            \PY{n}{simulation} \PY{o}{=} \PY{n}{monte\PYZus{}carlo}\PY{p}{(}\PY{n}{dimensions}\PY{p}{,} \PY{n}{num\PYZus{}initial\PYZus{}neutrons}\PY{p}{,}
                                     \PY{n}{num\PYZus{}replications}\PY{p}{,} \PY{n}{max\PYZus{}generation}\PY{p}{,} \PY{n}{plot}\PY{o}{=}\PY{k+kc}{False}\PY{p}{)}
            
            \PY{n}{K}\PY{p}{[}\PY{n}{j}\PY{p}{,}\PY{n}{i}\PY{p}{]} \PY{o}{=} \PY{n}{simulation}\PY{o}{.}\PY{n}{mean\PYZus{}k\PYZus{}values}\PY{p}{[}\PY{o}{\PYZhy{}}\PY{l+m+mi}{1}\PY{p}{]} \PY{k}{if} \PY{n}{simulation}\PY{o}{.}\PY{n}{mean\PYZus{}k\PYZus{}values}\PY{o}{.}\PY{n}{size} \PY{o}{\PYZgt{}} \PY{l+m+mi}{0} \PY{k}{else} \PY{l+m+mi}{0}
            \PY{n}{K\PYZus{}uncert}\PY{p}{[}\PY{n}{j}\PY{p}{,}\PY{n}{i}\PY{p}{]} \PY{o}{=} \PY{n}{simulation}\PY{o}{.}\PY{n}{k\PYZus{}value\PYZus{}uncerts}\PY{p}{[}\PY{o}{\PYZhy{}}\PY{l+m+mi}{1}\PY{p}{]} \PY{k}{if} \PY{n}{simulation}\PY{o}{.}\PY{n}{k\PYZus{}value\PYZus{}uncerts}\PY{o}{.}\PY{n}{size} \PY{o}{\PYZgt{}} \PY{l+m+mi}{0} \PY{k}{else} \PY{l+m+mi}{0}
            
            \PY{c+c1}{\PYZsh{} Debug Print}
            \PY{k}{if} \PY{n}{debug\PYZus{}print} \PY{o+ow}{and} \PY{n}{idx} \PY{o}{\PYZpc{}} \PY{n}{interval} \PY{o}{==} \PY{l+m+mi}{0}\PY{p}{:}
                \PY{n+nb}{print}\PY{p}{(}\PY{l+s+sa}{f}\PY{l+s+s1}{\PYZsq{}}\PY{l+s+si}{\PYZob{}}\PY{p}{(}\PY{n}{idx}\PY{+w}{ }\PY{o}{/}\PY{+w}{ }\PY{n}{total\PYZus{}sims}\PY{p}{)}\PY{+w}{ }\PY{o}{*}\PY{+w}{ }\PY{l+m+mi}{100}\PY{l+s+si}{:}\PY{l+s+s1}{.0f}\PY{l+s+si}{\PYZcb{}}\PY{l+s+s1}{\PYZpc{} \PYZhy{} Iteration }\PY{l+s+si}{\PYZob{}}\PY{n}{idx}\PY{l+s+si}{\PYZcb{}}\PY{l+s+s1}{/}\PY{l+s+si}{\PYZob{}}\PY{n}{total\PYZus{}sims}\PY{l+s+si}{\PYZcb{}}\PY{l+s+s1}{\PYZsq{}}\PY{p}{)}
    
    \PY{c+c1}{\PYZsh{} Plotting}
    \PY{n}{results} \PY{o}{=} \PY{n}{ContourResults}\PY{p}{(}\PY{p}{)}

    \PY{c+c1}{\PYZsh{} Main Figure}
    \PY{n}{plt}\PY{o}{.}\PY{n}{figure}\PY{p}{(}\PY{p}{)}
    
    \PY{n}{cp} \PY{o}{=} \PY{n}{plt}\PY{o}{.}\PY{n}{contourf}\PY{p}{(}\PY{n}{R}\PY{p}{,} \PY{n}{V}\PY{p}{,} \PY{n}{K}\PY{p}{,} \PY{n}{levels}\PY{o}{=}\PY{l+m+mi}{20}\PY{p}{,} \PY{n}{cmap}\PY{o}{=}\PY{l+s+s1}{\PYZsq{}}\PY{l+s+s1}{viridis}\PY{l+s+s1}{\PYZsq{}}\PY{p}{)}
    \PY{n}{plt}\PY{o}{.}\PY{n}{xlabel}\PY{p}{(}\PY{l+s+s1}{\PYZsq{}}\PY{l+s+s1}{Aspect Ratio \PYZdl{}w/h\PYZdl{}}\PY{l+s+s1}{\PYZsq{}}\PY{p}{)}
    \PY{n}{plt}\PY{o}{.}\PY{n}{ylabel}\PY{p}{(}\PY{l+s+s1}{\PYZsq{}}\PY{l+s+s1}{Volume (\PYZdl{}m\PYZca{}3\PYZdl{})}\PY{l+s+s1}{\PYZsq{}}\PY{p}{)}
    \PY{n}{plt}\PY{o}{.}\PY{n}{title}\PY{p}{(}\PY{l+s+sa}{f}\PY{l+s+s1}{\PYZsq{}}\PY{l+s+s1}{Generation }\PY{l+s+si}{\PYZob{}}\PY{n}{max\PYZus{}generation}\PY{o}{\PYZhy{}}\PY{l+m+mi}{1}\PY{l+s+si}{\PYZcb{}}\PY{l+s+s1}{ to }\PY{l+s+si}{\PYZob{}}\PY{n}{max\PYZus{}generation}\PY{l+s+si}{\PYZcb{}}\PY{l+s+s1}{: Mean \PYZdl{}k\PYZdl{}}\PY{l+s+s1}{\PYZsq{}}\PY{p}{)}
    \PY{n}{plt}\PY{o}{.}\PY{n}{colorbar}\PY{p}{(}\PY{n}{cp}\PY{p}{,} \PY{n}{orientation}\PY{o}{=}\PY{l+s+s1}{\PYZsq{}}\PY{l+s+s1}{vertical}\PY{l+s+s1}{\PYZsq{}}\PY{p}{,} \PY{n}{label}\PY{o}{=}\PY{l+s+s1}{\PYZsq{}}\PY{l+s+s1}{Mean Multiplication Rate}\PY{l+s+s1}{\PYZsq{}}\PY{p}{)}
    
    \PY{n}{results}\PY{o}{.}\PY{n}{contour\PYZus{}plot} \PY{o}{=} \PY{n}{save\PYZus{}graph\PYZus{}and\PYZus{}close}\PY{p}{(}\PY{p}{)}
    
    \PY{c+c1}{\PYZsh{} Relative Uncertainty Plot}
    \PY{n}{plt}\PY{o}{.}\PY{n}{figure}\PY{p}{(}\PY{p}{)}
    
    \PY{n}{cp} \PY{o}{=} \PY{n}{plt}\PY{o}{.}\PY{n}{contourf}\PY{p}{(}\PY{n}{R}\PY{p}{,} \PY{n}{V}\PY{p}{,} \PY{n}{K\PYZus{}uncert} \PY{o}{*} \PY{l+m+mi}{100}\PY{p}{,} \PY{n}{levels}\PY{o}{=}\PY{l+m+mi}{20}\PY{p}{,} \PY{n}{cmap}\PY{o}{=}\PY{l+s+s1}{\PYZsq{}}\PY{l+s+s1}{viridis}\PY{l+s+s1}{\PYZsq{}}\PY{p}{)}
    \PY{n}{plt}\PY{o}{.}\PY{n}{xlabel}\PY{p}{(}\PY{l+s+s1}{\PYZsq{}}\PY{l+s+s1}{Aspect Ratio \PYZdl{}w/h\PYZdl{}}\PY{l+s+s1}{\PYZsq{}}\PY{p}{)}
    \PY{n}{plt}\PY{o}{.}\PY{n}{ylabel}\PY{p}{(}\PY{l+s+s1}{\PYZsq{}}\PY{l+s+s1}{Volume (\PYZdl{}m\PYZca{}3\PYZdl{})}\PY{l+s+s1}{\PYZsq{}}\PY{p}{)}
    \PY{n}{plt}\PY{o}{.}\PY{n}{title}\PY{p}{(}\PY{l+s+sa}{f}\PY{l+s+s1}{\PYZsq{}}\PY{l+s+s1}{Generation }\PY{l+s+si}{\PYZob{}}\PY{n}{max\PYZus{}generation}\PY{o}{\PYZhy{}}\PY{l+m+mi}{1}\PY{l+s+si}{\PYZcb{}}\PY{l+s+s1}{ to }\PY{l+s+si}{\PYZob{}}\PY{n}{max\PYZus{}generation}\PY{l+s+si}{\PYZcb{}}\PY{l+s+s1}{: Relative Uncertainty}\PY{l+s+s1}{\PYZsq{}}\PY{p}{)}
    \PY{n}{plt}\PY{o}{.}\PY{n}{colorbar}\PY{p}{(}\PY{n}{cp}\PY{p}{,} \PY{n}{orientation}\PY{o}{=}\PY{l+s+s1}{\PYZsq{}}\PY{l+s+s1}{vertical}\PY{l+s+s1}{\PYZsq{}}\PY{p}{,} \PY{n}{label}\PY{o}{=}\PY{l+s+s1}{\PYZsq{}}\PY{l+s+s1}{Relative Uncertainty Multiplication Rate (}\PY{l+s+s1}{\PYZpc{}}\PY{l+s+s1}{)}\PY{l+s+s1}{\PYZsq{}}\PY{p}{)}
    
    \PY{n}{results}\PY{o}{.}\PY{n}{uncertainty\PYZus{}contour\PYZus{}plot} \PY{o}{=} \PY{n}{save\PYZus{}graph\PYZus{}and\PYZus{}close}\PY{p}{(}\PY{p}{)}
    
    \PY{n}{end\PYZus{}time} \PY{o}{=} \PY{n}{time}\PY{o}{.}\PY{n}{time}\PY{p}{(}\PY{p}{)}
    \PY{n}{elapsed\PYZus{}time} \PY{o}{=} \PY{n}{end\PYZus{}time} \PY{o}{\PYZhy{}} \PY{n}{start\PYZus{}time}
    
    \PY{k}{if} \PY{n}{debug\PYZus{}print}\PY{p}{:}
        \PY{n+nb}{print}\PY{p}{(}\PY{l+s+sa}{f}\PY{l+s+s2}{\PYZdq{}}\PY{l+s+s2}{Executed in }\PY{l+s+si}{\PYZob{}}\PY{n+nb}{int}\PY{p}{(}\PY{n}{elapsed\PYZus{}time}\PY{p}{)}\PY{+w}{ }\PY{o}{/}\PY{o}{/}\PY{+w}{ }\PY{l+m+mi}{60}\PY{l+s+si}{\PYZcb{}}\PY{l+s+s2}{ m }\PY{l+s+si}{\PYZob{}}\PY{n+nb}{int}\PY{p}{(}\PY{n}{elapsed\PYZus{}time}\PY{p}{)}\PY{+w}{ }\PY{o}{\PYZpc{}}\PY{+w}{ }\PY{l+m+mi}{60}\PY{l+s+si}{\PYZcb{}}\PY{l+s+s2}{ s }\PY{l+s+si}{\PYZob{}}\PY{n+nb}{int}\PY{p}{(}\PY{p}{(}\PY{n}{elapsed\PYZus{}time}\PY{+w}{ }\PY{o}{\PYZhy{}}\PY{+w}{ }\PY{n+nb}{int}\PY{p}{(}\PY{n}{elapsed\PYZus{}time}\PY{p}{)}\PY{p}{)}\PY{+w}{ }\PY{o}{*}\PY{+w}{ }\PY{l+m+mi}{1000}\PY{p}{)}\PY{l+s+si}{\PYZcb{}}\PY{l+s+s2}{ ms}\PY{l+s+s2}{\PYZdq{}}\PY{p}{)}
    
    \PY{k}{return} \PY{n}{results}\PY{p}{,} \PY{n}{contour\PYZus{}params}
\end{Verbatim}
\end{tcolorbox}

    \textbf{Code Block Summary:} This is the exploration code for a range of
10 aspect ratios from \(R=1\) to \(R=20\) and 10 areas from \(A=1\) to
\(A=200\), given 250 starting neutrons, 100 replications, and 15
generations. To lower the runtime, I'll only use a phase resolution of
400 points, with 20 slices for each dimension.

    \begin{tcolorbox}[breakable, size=fbox, boxrule=1pt, pad at break*=1mm,colback=cellbackground, colframe=cellborder]
\prompt{In}{incolor}{39}{\boxspacing}
\begin{Verbatim}[commandchars=\\\{\}]
\PY{n}{contour\PYZus{}results}\PY{p}{,} \PY{n}{contour\PYZus{}params} \PY{o}{=} \PY{n}{generate\PYZus{}contour}\PY{p}{(}
    \PY{n}{aspect\PYZus{}ratio\PYZus{}range}   \PY{o}{=} \PY{p}{(}\PY{l+m+mi}{1}\PY{p}{,} \PY{l+m+mi}{20}\PY{p}{,}  \PY{l+m+mi}{20}\PY{p}{)}\PY{p}{,} 
    \PY{n}{volume\PYZus{}range}         \PY{o}{=} \PY{p}{(}\PY{l+m+mi}{1}\PY{p}{,} \PY{l+m+mi}{200}\PY{p}{,} \PY{l+m+mi}{20}\PY{p}{)}\PY{p}{,}
    \PY{n}{max\PYZus{}generation}       \PY{o}{=} \PY{l+m+mi}{15}\PY{p}{,}
    \PY{n}{num\PYZus{}initial\PYZus{}neutrons} \PY{o}{=} \PY{l+m+mi}{250}\PY{p}{,} 
    \PY{n}{num\PYZus{}replications}     \PY{o}{=} \PY{l+m+mi}{100}\PY{p}{,}
    \PY{n}{debug\PYZus{}print}          \PY{o}{=} \PY{k+kc}{True}
\PY{p}{)}
\end{Verbatim}
\end{tcolorbox}

    \begin{Verbatim}[commandchars=\\\{\}]
10\% - Iteration 40/400
20\% - Iteration 80/400
30\% - Iteration 120/400
40\% - Iteration 160/400
50\% - Iteration 200/400
60\% - Iteration 240/400
70\% - Iteration 280/400
80\% - Iteration 320/400
90\% - Iteration 360/400
100\% - Iteration 400/400
Executed in 14 m 26 s 942 ms
    \end{Verbatim}

    \begin{tcolorbox}[breakable, size=fbox, boxrule=1pt, pad at break*=1mm,colback=cellbackground, colframe=cellborder]
\prompt{In}{incolor}{40}{\boxspacing}
\begin{Verbatim}[commandchars=\\\{\}]
\PY{n}{display}\PY{p}{(}\PY{n}{contour\PYZus{}results}\PY{o}{.}\PY{n}{contour\PYZus{}plot}\PY{p}{)}
\end{Verbatim}
\end{tcolorbox}

    \begin{center}
    \adjustimage{max size={0.9\linewidth}{0.9\paperheight}}{Project 3_files/Project 3_92_0.png}
    \end{center}
    { \hspace*{\fill} \\}
    
    \textbf{Figure 9.} In terms of aspect ratio, the model clearly favors
more cubic boxes around with aspect ratios of \(1:1\) or \(R=1\), and in
terms of volume, higher volumes are much more favorable across the
board. More skewed (unbalanced aspect ratios) and smaller volumes
generally result in worse multiplication rates, with the worst
multiplication rates (near 0, so no replications on average) are
extremely skewed and very small boxes.

    \begin{tcolorbox}[breakable, size=fbox, boxrule=1pt, pad at break*=1mm,colback=cellbackground, colframe=cellborder]
\prompt{In}{incolor}{41}{\boxspacing}
\begin{Verbatim}[commandchars=\\\{\}]
\PY{n}{display}\PY{p}{(}\PY{n}{contour\PYZus{}results}\PY{o}{.}\PY{n}{uncertainty\PYZus{}contour\PYZus{}plot}\PY{p}{)}
\end{Verbatim}
\end{tcolorbox}

    \begin{center}
    \adjustimage{max size={0.9\linewidth}{0.9\paperheight}}{Project 3_files/Project 3_94_0.png}
    \end{center}
    { \hspace*{\fill} \\}
    
    \textbf{Figure 10.} The random bouts of high or low multiplication rates
that can be observed in figure 9 are best explained by examining the
regions of highest uncertainty. We're least certain of the
validity/accuracy of the measurements made along the main positive
diagonal of the graph (from the bottom left to the top right), and this
coincides with the region in figure 9 which shows the most random
behavior.

    \hypertarget{phase-exploration-cross-sectional-analysis}{%
\subsection{Phase Exploration: Cross-Sectional
Analysis}\label{phase-exploration-cross-sectional-analysis}}

    \textbf{Code Block Summary:} This code allows us to explore
cross-sections of the phase to view patterns in how the multiplication
rate changes with respect to a given parameter by holding the other
parameters constant.

    \begin{tcolorbox}[breakable, size=fbox, boxrule=1pt, pad at break*=1mm,colback=cellbackground, colframe=cellborder]
\prompt{In}{incolor}{42}{\boxspacing}
\begin{Verbatim}[commandchars=\\\{\}]
\PY{n+nd}{@dataclass}
\PY{k}{class} \PY{n+nc}{CrossSectionData}\PY{p}{:}
    \PY{n}{k\PYZus{}value\PYZus{}plot}\PY{p}{:} \PY{n}{Image} \PY{o}{=} \PY{k+kc}{None}

\PY{k}{def} \PY{n+nf}{examine\PYZus{}cross\PYZus{}section}\PY{p}{(}\PY{n}{contour\PYZus{}params}\PY{p}{,} \PY{n}{aspect\PYZus{}ratio} \PY{o}{=} \PY{k+kc}{None}\PY{p}{,} \PY{n}{volume}\PY{o}{=}\PY{k+kc}{None}\PY{p}{,} \PY{n}{debug\PYZus{}print}\PY{o}{=}\PY{k+kc}{False}\PY{p}{)}\PY{p}{:}
\PY{+w}{    }\PY{l+s+sd}{\PYZdq{}\PYZdq{}\PYZdq{}}
\PY{l+s+sd}{    Performs the Monte\PYZhy{}Carlo simulation along a given cross\PYZhy{}section based on the parameter given.}
\PY{l+s+sd}{    \PYZdq{}\PYZdq{}\PYZdq{}}
    \PY{k}{if} \PY{n}{aspect\PYZus{}ratio} \PY{o+ow}{is} \PY{o+ow}{not} \PY{k+kc}{None} \PY{o+ow}{and} \PY{n}{volume} \PY{o+ow}{is} \PY{o+ow}{not} \PY{k+kc}{None}\PY{p}{:}
        \PY{k}{raise} \PY{n+ne}{ValueError}\PY{p}{(}\PY{l+s+s1}{\PYZsq{}}\PY{l+s+s1}{At least one parameter must be unspecified!}\PY{l+s+s1}{\PYZsq{}}\PY{p}{)}
    
    \PY{n}{max\PYZus{}generation}\PY{p}{,} \PY{n}{num\PYZus{}initial\PYZus{}neutrons}\PY{p}{,} \PY{n}{num\PYZus{}replications}\PY{p}{,} \PY{n}{aspect\PYZus{}ratio\PYZus{}range}\PY{p}{,} \PY{n}{volume\PYZus{}range} \PY{o}{=} \PY{n}{contour\PYZus{}params}
    
    \PY{n}{start\PYZus{}time} \PY{o}{=} \PY{n}{time}\PY{o}{.}\PY{n}{time}\PY{p}{(}\PY{p}{)}
    
    \PY{n}{aspect\PYZus{}ratio\PYZus{}domain} \PY{o}{=} \PY{n}{np}\PY{o}{.}\PY{n}{linspace}\PY{p}{(}\PY{o}{*}\PY{n}{aspect\PYZus{}ratio\PYZus{}range}\PY{p}{)}
    \PY{n}{volume\PYZus{}domain} \PY{o}{=} \PY{n}{np}\PY{o}{.}\PY{n}{linspace}\PY{p}{(}\PY{o}{*}\PY{n}{volume\PYZus{}range}\PY{p}{)}
    
    \PY{n}{x\PYZus{}label}  \PY{o}{=} \PY{k+kc}{None}
    \PY{n}{x\PYZus{}values} \PY{o}{=} \PY{p}{[}\PY{p}{]}
    
    \PY{k}{if} \PY{n}{aspect\PYZus{}ratio} \PY{o+ow}{is} \PY{o+ow}{not} \PY{k+kc}{None}\PY{p}{:} \PY{c+c1}{\PYZsh{} Defined Aspect Ratio}
        \PY{n}{aspect\PYZus{}ratio\PYZus{}domain} \PY{o}{=} \PY{n}{np}\PY{o}{.}\PY{n}{array}\PY{p}{(}\PY{p}{[}\PY{n}{aspect\PYZus{}ratio}\PY{p}{]}\PY{p}{)}
        \PY{n}{x\PYZus{}label} \PY{o}{=} \PY{l+s+s1}{\PYZsq{}}\PY{l+s+s1}{Volume}\PY{l+s+s1}{\PYZsq{}}
    \PY{k}{else}\PY{p}{:}
        \PY{n}{volume\PYZus{}domain} \PY{o}{=} \PY{n}{np}\PY{o}{.}\PY{n}{array}\PY{p}{(}\PY{p}{[}\PY{n}{volume}\PY{p}{]}\PY{p}{)}
        \PY{n}{x\PYZus{}label} \PY{o}{=} \PY{l+s+s1}{\PYZsq{}}\PY{l+s+s1}{Aspect Ratio}\PY{l+s+s1}{\PYZsq{}}
    
    \PY{n}{total\PYZus{}sims} \PY{o}{=} \PY{n}{aspect\PYZus{}ratio\PYZus{}domain}\PY{o}{.}\PY{n}{size} \PY{o}{*} \PY{n}{volume\PYZus{}domain}\PY{o}{.}\PY{n}{size}
    
    \PY{c+c1}{\PYZsh{} index and interval for debugging}
    \PY{n}{idx} \PY{o}{=} \PY{l+m+mi}{0}
    \PY{n}{interval} \PY{o}{=} \PY{n+nb}{max}\PY{p}{(}\PY{n}{total\PYZus{}sims} \PY{o}{/}\PY{o}{/} \PY{l+m+mi}{10}\PY{p}{,} \PY{l+m+mi}{1}\PY{p}{)}
    
    \PY{n}{k\PYZus{}values} \PY{o}{=} \PY{p}{[}\PY{p}{]}
    \PY{n}{k\PYZus{}uncert} \PY{o}{=} \PY{p}{[}\PY{p}{]}
    
    \PY{c+c1}{\PYZsh{} Iterate over the Phase}
    \PY{k}{for} \PY{n}{i}\PY{p}{,} \PY{n}{aspect\PYZus{}ratio\PYZus{}i} \PY{o+ow}{in} \PY{n+nb}{enumerate}\PY{p}{(}\PY{n}{aspect\PYZus{}ratio\PYZus{}domain}\PY{p}{)}\PY{p}{:}
        \PY{k}{for} \PY{n}{j}\PY{p}{,} \PY{n}{volume\PYZus{}i} \PY{o+ow}{in} \PY{n+nb}{enumerate}\PY{p}{(}\PY{n}{volume\PYZus{}domain}\PY{p}{)}\PY{p}{:}
            \PY{n}{idx} \PY{o}{+}\PY{o}{=} \PY{l+m+mi}{1}
            
            \PY{n}{dimensions} \PY{o}{=} \PY{n}{calculate\PYZus{}dimensions}\PY{p}{(}\PY{n}{aspect\PYZus{}ratio\PYZus{}i}\PY{p}{,} \PY{n}{volume\PYZus{}i}\PY{p}{)}
            \PY{n}{simulation} \PY{o}{=} \PY{n}{monte\PYZus{}carlo}\PY{p}{(}\PY{n}{dimensions}\PY{p}{,} \PY{n}{num\PYZus{}initial\PYZus{}neutrons}\PY{p}{,}
                                     \PY{n}{num\PYZus{}replications}\PY{p}{,} \PY{n}{max\PYZus{}generation}\PY{p}{,} \PY{n}{plot}\PY{o}{=}\PY{k+kc}{False}\PY{p}{)}
            
            \PY{k}{if} \PY{n}{aspect\PYZus{}ratio} \PY{o+ow}{is} \PY{o+ow}{not} \PY{k+kc}{None}\PY{p}{:}
                \PY{n}{x\PYZus{}values}\PY{o}{.}\PY{n}{append}\PY{p}{(}\PY{n}{volume\PYZus{}i}\PY{p}{)}
            \PY{k}{else}\PY{p}{:}
                \PY{n}{x\PYZus{}values}\PY{o}{.}\PY{n}{append}\PY{p}{(}\PY{n}{aspect\PYZus{}ratio\PYZus{}i}\PY{p}{)}
            
            \PY{n}{k\PYZus{}values}\PY{o}{.}\PY{n}{append}\PY{p}{(}\PY{n}{simulation}\PY{o}{.}\PY{n}{mean\PYZus{}k\PYZus{}values}\PY{p}{[}\PY{o}{\PYZhy{}}\PY{l+m+mi}{1}\PY{p}{]}\PY{p}{)}
            \PY{n}{k\PYZus{}uncert}\PY{o}{.}\PY{n}{append}\PY{p}{(}\PY{n}{simulation}\PY{o}{.}\PY{n}{k\PYZus{}value\PYZus{}uncerts}\PY{p}{[}\PY{o}{\PYZhy{}}\PY{l+m+mi}{1}\PY{p}{]}\PY{p}{)}
            
            \PY{c+c1}{\PYZsh{} Debug Print}
            \PY{k}{if} \PY{n}{debug\PYZus{}print} \PY{o+ow}{and} \PY{n}{idx} \PY{o}{\PYZpc{}} \PY{n}{interval} \PY{o}{==} \PY{l+m+mi}{0}\PY{p}{:}
                \PY{n+nb}{print}\PY{p}{(}\PY{l+s+sa}{f}\PY{l+s+s1}{\PYZsq{}}\PY{l+s+si}{\PYZob{}}\PY{p}{(}\PY{n}{idx}\PY{+w}{ }\PY{o}{/}\PY{+w}{ }\PY{n}{total\PYZus{}sims}\PY{p}{)}\PY{+w}{ }\PY{o}{*}\PY{+w}{ }\PY{l+m+mi}{100}\PY{l+s+si}{:}\PY{l+s+s1}{.0f}\PY{l+s+si}{\PYZcb{}}\PY{l+s+s1}{\PYZpc{} \PYZhy{} Iteration }\PY{l+s+si}{\PYZob{}}\PY{n}{idx}\PY{l+s+si}{\PYZcb{}}\PY{l+s+s1}{/}\PY{l+s+si}{\PYZob{}}\PY{n}{total\PYZus{}sims}\PY{l+s+si}{\PYZcb{}}\PY{l+s+s1}{\PYZsq{}}\PY{p}{)}
              
    \PY{n}{results} \PY{o}{=} \PY{n}{CrossSectionData}\PY{p}{(}\PY{p}{)}
            
    \PY{n}{plt}\PY{o}{.}\PY{n}{figure}\PY{p}{(}\PY{p}{)}
    \PY{n}{plt}\PY{o}{.}\PY{n}{title}\PY{p}{(}\PY{l+s+sa}{f}\PY{l+s+s1}{\PYZsq{}}\PY{l+s+s1}{Mean Replication Rate vs. }\PY{l+s+si}{\PYZob{}}\PY{n}{x\PYZus{}label}\PY{l+s+si}{\PYZcb{}}\PY{l+s+s1}{\PYZsq{}}\PY{p}{)}
    \PY{n}{plt}\PY{o}{.}\PY{n}{errorbar}\PY{p}{(}
        \PY{n}{x}\PY{o}{=}\PY{n}{x\PYZus{}values}\PY{p}{,}
        \PY{n}{y}\PY{o}{=}\PY{n}{k\PYZus{}values}\PY{p}{,} 
        \PY{n}{yerr}\PY{o}{=}\PY{n}{k\PYZus{}uncert}\PY{p}{,}
        \PY{n}{fmt}\PY{o}{=}\PY{l+s+s1}{\PYZsq{}}\PY{l+s+s1}{bo\PYZhy{}\PYZhy{}}\PY{l+s+s1}{\PYZsq{}}\PY{p}{,}
        \PY{n}{label}\PY{o}{=}\PY{l+s+s1}{\PYZsq{}}\PY{l+s+s1}{k}\PY{l+s+s1}{\PYZsq{}}\PY{p}{)}
    \PY{n}{plt}\PY{o}{.}\PY{n}{xlabel}\PY{p}{(}\PY{n}{x\PYZus{}label}\PY{p}{)}
    \PY{n}{plt}\PY{o}{.}\PY{n}{ylabel}\PY{p}{(}\PY{l+s+s1}{\PYZsq{}}\PY{l+s+s1}{Mean Replication Rate}\PY{l+s+s1}{\PYZsq{}}\PY{p}{)}
    \PY{n}{results}\PY{o}{.}\PY{n}{k\PYZus{}value\PYZus{}plot} \PY{o}{=} \PY{n}{save\PYZus{}graph\PYZus{}and\PYZus{}close}\PY{p}{(}\PY{p}{)}
    
    \PY{n}{end\PYZus{}time} \PY{o}{=} \PY{n}{time}\PY{o}{.}\PY{n}{time}\PY{p}{(}\PY{p}{)}
    \PY{n}{elapsed\PYZus{}time} \PY{o}{=} \PY{n}{end\PYZus{}time} \PY{o}{\PYZhy{}} \PY{n}{start\PYZus{}time}
    
    \PY{k}{if} \PY{n}{debug\PYZus{}print}\PY{p}{:}
        \PY{n+nb}{print}\PY{p}{(}\PY{l+s+sa}{f}\PY{l+s+s2}{\PYZdq{}}\PY{l+s+s2}{Executed in }\PY{l+s+si}{\PYZob{}}\PY{n+nb}{int}\PY{p}{(}\PY{n}{elapsed\PYZus{}time}\PY{p}{)}\PY{+w}{ }\PY{o}{/}\PY{o}{/}\PY{+w}{ }\PY{l+m+mi}{60}\PY{l+s+si}{\PYZcb{}}\PY{l+s+s2}{ m }\PY{l+s+si}{\PYZob{}}\PY{n+nb}{int}\PY{p}{(}\PY{n}{elapsed\PYZus{}time}\PY{p}{)}\PY{+w}{ }\PY{o}{\PYZpc{}}\PY{+w}{ }\PY{l+m+mi}{60}\PY{l+s+si}{\PYZcb{}}\PY{l+s+s2}{ s }\PY{l+s+si}{\PYZob{}}\PY{n+nb}{int}\PY{p}{(}\PY{p}{(}\PY{n}{elapsed\PYZus{}time}\PY{+w}{ }\PY{o}{\PYZhy{}}\PY{+w}{ }\PY{n+nb}{int}\PY{p}{(}\PY{n}{elapsed\PYZus{}time}\PY{p}{)}\PY{p}{)}\PY{+w}{ }\PY{o}{*}\PY{+w}{ }\PY{l+m+mi}{1000}\PY{p}{)}\PY{l+s+si}{\PYZcb{}}\PY{l+s+s2}{ ms}\PY{l+s+s2}{\PYZdq{}}\PY{p}{)}
    
    \PY{k}{return} \PY{n}{results}
\end{Verbatim}
\end{tcolorbox}

    In order to try and get a very good picture of the relationship between
how each individual parameter alters the multiplication rate, we can
utilize a few cross-sectional analyses to help us, in the form of
visualizing
\[f(R,V=V_0) \to \mathbf{k} \text{ or } f(R=R_0,V) \to \mathbf{k}.\] To
begin, we'll consider just the volume, holding the aspect ratio constant
at \(R_0=1\):

    \begin{tcolorbox}[breakable, size=fbox, boxrule=1pt, pad at break*=1mm,colback=cellbackground, colframe=cellborder]
\prompt{In}{incolor}{43}{\boxspacing}
\begin{Verbatim}[commandchars=\\\{\}]
\PY{n}{volume\PYZus{}cross\PYZus{}section} \PY{o}{=} \PY{n}{examine\PYZus{}cross\PYZus{}section}\PY{p}{(}\PY{n}{contour\PYZus{}params}\PY{p}{,} \PY{n}{aspect\PYZus{}ratio}\PY{o}{=}\PY{l+m+mi}{1}\PY{p}{,} \PY{n}{debug\PYZus{}print}\PY{o}{=}\PY{k+kc}{True}\PY{p}{)}
\end{Verbatim}
\end{tcolorbox}

    \begin{Verbatim}[commandchars=\\\{\}]
10\% - Iteration 2/20
20\% - Iteration 4/20
30\% - Iteration 6/20
40\% - Iteration 8/20
50\% - Iteration 10/20
60\% - Iteration 12/20
70\% - Iteration 14/20
80\% - Iteration 16/20
90\% - Iteration 18/20
100\% - Iteration 20/20
Executed in 3 m 18 s 937 ms
    \end{Verbatim}

    \begin{tcolorbox}[breakable, size=fbox, boxrule=1pt, pad at break*=1mm,colback=cellbackground, colframe=cellborder]
\prompt{In}{incolor}{44}{\boxspacing}
\begin{Verbatim}[commandchars=\\\{\}]
\PY{n}{display}\PY{p}{(}\PY{n}{volume\PYZus{}cross\PYZus{}section}\PY{o}{.}\PY{n}{k\PYZus{}value\PYZus{}plot}\PY{p}{)}
\end{Verbatim}
\end{tcolorbox}

    \begin{center}
    \adjustimage{max size={0.9\linewidth}{0.9\paperheight}}{Project 3_files/Project 3_101_0.png}
    \end{center}
    { \hspace*{\fill} \\}
    
    \textbf{Figure 11.} For the most stable aspect ratio (cubic), there's a
positive relation between the volume and the mean multiplication rate
with diminishing returns (levels out as the volume increases). Notably,
the uncertainty also decreases significantly as volume is increased.

    Now, we can consider the aspect ratio, holding the volume constant at
\(V_0=200\):

    \begin{tcolorbox}[breakable, size=fbox, boxrule=1pt, pad at break*=1mm,colback=cellbackground, colframe=cellborder]
\prompt{In}{incolor}{45}{\boxspacing}
\begin{Verbatim}[commandchars=\\\{\}]
\PY{n}{aspect\PYZus{}ratio\PYZus{}cross\PYZus{}section} \PY{o}{=} \PY{n}{examine\PYZus{}cross\PYZus{}section}\PY{p}{(}\PY{n}{contour\PYZus{}params}\PY{p}{,} \PY{n}{volume}\PY{o}{=}\PY{l+m+mi}{200}\PY{p}{,} \PY{n}{debug\PYZus{}print}\PY{o}{=}\PY{k+kc}{True}\PY{p}{)}
\end{Verbatim}
\end{tcolorbox}

    \begin{Verbatim}[commandchars=\\\{\}]
10\% - Iteration 2/20
20\% - Iteration 4/20
30\% - Iteration 6/20
40\% - Iteration 8/20
50\% - Iteration 10/20
60\% - Iteration 12/20
70\% - Iteration 14/20
80\% - Iteration 16/20
90\% - Iteration 18/20
100\% - Iteration 20/20
Executed in 1 m 34 s 825 ms
    \end{Verbatim}

    \begin{tcolorbox}[breakable, size=fbox, boxrule=1pt, pad at break*=1mm,colback=cellbackground, colframe=cellborder]
\prompt{In}{incolor}{46}{\boxspacing}
\begin{Verbatim}[commandchars=\\\{\}]
\PY{n}{display}\PY{p}{(}\PY{n}{aspect\PYZus{}ratio\PYZus{}cross\PYZus{}section}\PY{o}{.}\PY{n}{k\PYZus{}value\PYZus{}plot}\PY{p}{)}
\end{Verbatim}
\end{tcolorbox}

    \begin{center}
    \adjustimage{max size={0.9\linewidth}{0.9\paperheight}}{Project 3_files/Project 3_105_0.png}
    \end{center}
    { \hspace*{\fill} \\}
    
    \textbf{Figure 12.} As the aspect ratio is increased from an even ratio
(cubic, most stable) to a highly skewed ratio, the mean multiplication
rate decreases and the uncertainty greatly increases.

    \hypertarget{phase-exploration-edge-cases}{%
\subsection{Phase Exploration: Edge
Cases}\label{phase-exploration-edge-cases}}

    \textbf{Code Block Summary:} This code will allow us to explore the
phase by selecting individual coordinates, and that way, we can
understand the results we are seeing. It works by taking the parameters
we selected for our contour previously and performing the simulation for
given coordinates \((R,V)\), and it returns an example simulation result
alongside a full Monte Carlo simulation at that coordinate.

    \begin{tcolorbox}[breakable, size=fbox, boxrule=1pt, pad at break*=1mm,colback=cellbackground, colframe=cellborder]
\prompt{In}{incolor}{47}{\boxspacing}
\begin{Verbatim}[commandchars=\\\{\}]
\PY{n+nd}{@dataclass}
\PY{k}{class} \PY{n+nc}{PhaseCoordinateData}\PY{p}{:}
    \PY{n}{example\PYZus{}simulation}\PY{p}{:} \PY{n}{SimulationResult}
    \PY{n}{monte\PYZus{}carlo\PYZus{}result}\PY{p}{:} \PY{n+nb}{list}\PY{p}{[}\PY{n}{GenerationData}\PY{p}{]}

\PY{k}{def} \PY{n+nf}{examine\PYZus{}contour\PYZus{}coordinate}\PY{p}{(}\PY{n}{aspect\PYZus{}ratio}\PY{p}{,} \PY{n}{area}\PY{p}{,} \PY{n}{contour\PYZus{}params}\PY{p}{)}\PY{p}{:}
\PY{+w}{    }\PY{l+s+sd}{\PYZdq{}\PYZdq{}\PYZdq{}}
\PY{l+s+sd}{    Performs the Monte\PYZhy{}Carlo simulation for a phase coordinate}
\PY{l+s+sd}{    (R,V) over again, and also provides an example simulation}
\PY{l+s+sd}{    alongside the Monte\PYZhy{}Carlo results so that we can observe}
\PY{l+s+sd}{    what is going on at a certain coordinate.}
\PY{l+s+sd}{    \PYZdq{}\PYZdq{}\PYZdq{}}
    \PY{n}{max\PYZus{}generation}\PY{p}{,} \PY{n}{num\PYZus{}initial\PYZus{}neutrons}\PY{p}{,} \PY{n}{num\PYZus{}replications}\PY{p}{,} \PY{n}{\PYZus{}}\PY{p}{,} \PY{n}{\PYZus{}} \PY{o}{=} \PY{n}{contour\PYZus{}params}
    \PY{n}{dimensions} \PY{o}{=} \PY{n}{calculate\PYZus{}dimensions}\PY{p}{(}\PY{n}{aspect\PYZus{}ratio}\PY{p}{,} \PY{n}{area}\PY{p}{)}
    \PY{n}{example\PYZus{}simulation} \PY{o}{=} \PY{n}{simulate}\PY{p}{(}\PY{l+m+mi}{5}\PY{p}{,} \PY{n}{dimensions}\PY{p}{,} \PY{l+m+mi}{5}\PY{p}{,} \PY{n}{plot} \PY{o}{=} \PY{k+kc}{True}\PY{p}{)}
    
    \PY{c+c1}{\PYZsh{} num\PYZus{}replications can be doubled to make this Monte\PYZhy{}Carlo simulation more accurate}
    \PY{n}{monte\PYZus{}carlo\PYZus{}result} \PY{o}{=} \PY{n}{monte\PYZus{}carlo}\PY{p}{(}\PY{n}{dimensions}\PY{p}{,} \PY{n}{num\PYZus{}initial\PYZus{}neutrons}\PY{p}{,} \PY{n}{num\PYZus{}replications} \PY{o}{*} \PY{l+m+mi}{2}\PY{p}{,} \PY{n}{max\PYZus{}generation}\PY{p}{,} \PY{n}{plot}\PY{o}{=}\PY{k+kc}{True}\PY{p}{)}
    
    \PY{k}{return} \PY{n}{PhaseCoordinateData}\PY{p}{(}\PY{n}{example\PYZus{}simulation}\PY{p}{,} \PY{n}{monte\PYZus{}carlo\PYZus{}result}\PY{p}{)}
\end{Verbatim}
\end{tcolorbox}

    \hypertarget{edge-case-1-symmetric-aspect-ratio-large-volume-rv-1-200}{%
\subsubsection{\texorpdfstring{Edge Case 1: Symmetric Aspect Ratio,
Large Volume:
\((R,V) = (1, 200)\)}{Edge Case 1: Symmetric Aspect Ratio, Large Volume: (R,V) = (1, 200)}}\label{edge-case-1-symmetric-aspect-ratio-large-volume-rv-1-200}}

    \textbf{Code Block Summary:} Now, we can explore the phase a bit to
understand what is going on in the contour, and why it seems to favor
wide boxes over tall boxes. We can begin with a phase coordinate that is
clearly more representative of more stable geometry:
\((R,V) \approx (1, 200)\)

    \begin{tcolorbox}[breakable, size=fbox, boxrule=1pt, pad at break*=1mm,colback=cellbackground, colframe=cellborder]
\prompt{In}{incolor}{48}{\boxspacing}
\begin{Verbatim}[commandchars=\\\{\}]
\PY{n}{edge\PYZus{}case\PYZus{}1} \PY{o}{=} \PY{n}{examine\PYZus{}contour\PYZus{}coordinate}\PY{p}{(}\PY{l+m+mi}{1}\PY{p}{,} \PY{l+m+mi}{200}\PY{p}{,} \PY{n}{contour\PYZus{}params}\PY{p}{)}
\end{Verbatim}
\end{tcolorbox}

    \begin{tcolorbox}[breakable, size=fbox, boxrule=1pt, pad at break*=1mm,colback=cellbackground, colframe=cellborder]
\prompt{In}{incolor}{49}{\boxspacing}
\begin{Verbatim}[commandchars=\\\{\}]
\PY{n}{display}\PY{p}{(}\PY{n}{edge\PYZus{}case\PYZus{}1}\PY{o}{.}\PY{n}{example\PYZus{}simulation}\PY{o}{.}\PY{n}{trajectories\PYZus{}graph}\PY{p}{)}
\end{Verbatim}
\end{tcolorbox}

    \begin{center}
    \adjustimage{max size={0.9\linewidth}{0.9\paperheight}}{Project 3_files/Project 3_113_0.png}
    \end{center}
    { \hspace*{\fill} \\}
    
    \textbf{Figure 13.} The sheer density and variety in color of the inside
of this box shows that this model is capable of producing many neutrons
across all tracked generations, so without a doubt, this simulation was
extremely deep-reaching.

    \begin{tcolorbox}[breakable, size=fbox, boxrule=1pt, pad at break*=1mm,colback=cellbackground, colframe=cellborder]
\prompt{In}{incolor}{50}{\boxspacing}
\begin{Verbatim}[commandchars=\\\{\}]
\PY{n}{display}\PY{p}{(}\PY{n}{edge\PYZus{}case\PYZus{}1}\PY{o}{.}\PY{n}{monte\PYZus{}carlo\PYZus{}result}\PY{o}{.}\PY{n}{k\PYZus{}value\PYZus{}plot}\PY{p}{)}
\end{Verbatim}
\end{tcolorbox}

    \begin{center}
    \adjustimage{max size={0.9\linewidth}{0.9\paperheight}}{Project 3_files/Project 3_115_0.png}
    \end{center}
    { \hspace*{\fill} \\}
    
    \textbf{Figure 14.} Here, we can see that the multiplication rate, once
again, trends towards a stable, nonzero value. In particular, the
multiplication rate is near but above \(k=1\), so a physical
interpretation is that the number of free neutrons within the reactor is
increasing with respect to the number being lost, making this reactor
supercritical.

    \hypertarget{edge-case-2-skewed-aspect-ratio-and-small-volume-rv201}{%
\subsubsection{\texorpdfstring{Edge Case 2: Skewed Aspect Ratio and
Small Volume:
\((R,V)=(20,1)\)}{Edge Case 2: Skewed Aspect Ratio and Small Volume: (R,V)=(20,1)}}\label{edge-case-2-skewed-aspect-ratio-and-small-volume-rv201}}

    \begin{tcolorbox}[breakable, size=fbox, boxrule=1pt, pad at break*=1mm,colback=cellbackground, colframe=cellborder]
\prompt{In}{incolor}{51}{\boxspacing}
\begin{Verbatim}[commandchars=\\\{\}]
\PY{n}{edge\PYZus{}case\PYZus{}2} \PY{o}{=} \PY{n}{examine\PYZus{}contour\PYZus{}coordinate}\PY{p}{(}\PY{l+m+mi}{20}\PY{p}{,} \PY{l+m+mi}{1}\PY{p}{,} \PY{n}{contour\PYZus{}params}\PY{p}{)}
\end{Verbatim}
\end{tcolorbox}

    \begin{tcolorbox}[breakable, size=fbox, boxrule=1pt, pad at break*=1mm,colback=cellbackground, colframe=cellborder]
\prompt{In}{incolor}{52}{\boxspacing}
\begin{Verbatim}[commandchars=\\\{\}]
\PY{n}{display}\PY{p}{(}\PY{n}{edge\PYZus{}case\PYZus{}2}\PY{o}{.}\PY{n}{example\PYZus{}simulation}\PY{o}{.}\PY{n}{trajectories\PYZus{}graph}\PY{p}{)}
\end{Verbatim}
\end{tcolorbox}

    \begin{center}
    \adjustimage{max size={0.9\linewidth}{0.9\paperheight}}{Project 3_files/Project 3_119_0.png}
    \end{center}
    { \hspace*{\fill} \\}
    
    \textbf{Figure 15.} Here we can see that all of the neutrons in the
starting generation were lost. This geometry is clearly unstable as the
extremely small height caused all neutrons propagating vertically in any
large way to be completely ejected from the box, and the simulation
didn't progress past the very first generation.

    \begin{tcolorbox}[breakable, size=fbox, boxrule=1pt, pad at break*=1mm,colback=cellbackground, colframe=cellborder]
\prompt{In}{incolor}{53}{\boxspacing}
\begin{Verbatim}[commandchars=\\\{\}]
\PY{n}{display}\PY{p}{(}\PY{n}{edge\PYZus{}case\PYZus{}2}\PY{o}{.}\PY{n}{monte\PYZus{}carlo\PYZus{}result}\PY{o}{.}\PY{n}{k\PYZus{}value\PYZus{}plot}\PY{p}{)}
\end{Verbatim}
\end{tcolorbox}

    \begin{center}
    \adjustimage{max size={0.9\linewidth}{0.9\paperheight}}{Project 3_files/Project 3_121_0.png}
    \end{center}
    { \hspace*{\fill} \\}
    
    \textbf{Figure 16.} Again, as expected, this geometry was so unstable
that its multiplication rate trends to zero quickly, so the physical
interpretation from this is that this reactor geometry is subcritical.

    \hypertarget{edge-case-2-medium-skewed-aspect-ratio-and-medium-sized-volume-center-of-the-phase-domain-rv10100-and-rrv17.5175}{%
\subsubsection{\texorpdfstring{Edge Case 2: Medium-Skewed Aspect Ratio
and Medium-Sized Volume (Center of the Phase Domain): \((R,V)=(10,100)\)
and
\(R(R,V)=(17.5,175)\)}{Edge Case 2: Medium-Skewed Aspect Ratio and Medium-Sized Volume (Center of the Phase Domain): (R,V)=(10,100) and R(R,V)=(17.5,175)}}\label{edge-case-2-medium-skewed-aspect-ratio-and-medium-sized-volume-center-of-the-phase-domain-rv10100-and-rrv17.5175}}

    Lastly, to try and understand what is going on along the region of high
uncertainty, we can consider two points along the line \(V=10R\):
\((10,100)\) and \((17.5,175)\). First, we can consider \((10,100)\):

    \begin{tcolorbox}[breakable, size=fbox, boxrule=1pt, pad at break*=1mm,colback=cellbackground, colframe=cellborder]
\prompt{In}{incolor}{54}{\boxspacing}
\begin{Verbatim}[commandchars=\\\{\}]
\PY{n}{edge\PYZus{}case\PYZus{}3} \PY{o}{=} \PY{n}{examine\PYZus{}contour\PYZus{}coordinate}\PY{p}{(}\PY{l+m+mi}{10}\PY{p}{,} \PY{l+m+mi}{100}\PY{p}{,} \PY{n}{contour\PYZus{}params}\PY{p}{)}
\end{Verbatim}
\end{tcolorbox}

    \begin{tcolorbox}[breakable, size=fbox, boxrule=1pt, pad at break*=1mm,colback=cellbackground, colframe=cellborder]
\prompt{In}{incolor}{55}{\boxspacing}
\begin{Verbatim}[commandchars=\\\{\}]
\PY{n}{display}\PY{p}{(}\PY{n}{edge\PYZus{}case\PYZus{}3}\PY{o}{.}\PY{n}{monte\PYZus{}carlo\PYZus{}result}\PY{o}{.}\PY{n}{k\PYZus{}value\PYZus{}plot}\PY{p}{)}
\end{Verbatim}
\end{tcolorbox}

    \begin{center}
    \adjustimage{max size={0.9\linewidth}{0.9\paperheight}}{Project 3_files/Project 3_126_0.png}
    \end{center}
    { \hspace*{\fill} \\}
    
    \textbf{Figure 17.} So, oddly, higher generation counts actually
increase the uncertainty in the mean \(k\) value at this point without
obviously trending towards zero, which is the opposite of the trend seen
for the most stable geometry (large cubes).

    Now, we can repeat this with \((17.5,175)\) to see if this is a
consistent pattern:

    \begin{tcolorbox}[breakable, size=fbox, boxrule=1pt, pad at break*=1mm,colback=cellbackground, colframe=cellborder]
\prompt{In}{incolor}{56}{\boxspacing}
\begin{Verbatim}[commandchars=\\\{\}]
\PY{n}{edge\PYZus{}case\PYZus{}4} \PY{o}{=} \PY{n}{examine\PYZus{}contour\PYZus{}coordinate}\PY{p}{(}\PY{l+m+mf}{17.5}\PY{p}{,} \PY{l+m+mi}{175}\PY{p}{,} \PY{n}{contour\PYZus{}params}\PY{p}{)}
\end{Verbatim}
\end{tcolorbox}

    \begin{tcolorbox}[breakable, size=fbox, boxrule=1pt, pad at break*=1mm,colback=cellbackground, colframe=cellborder]
\prompt{In}{incolor}{57}{\boxspacing}
\begin{Verbatim}[commandchars=\\\{\}]
\PY{n}{display}\PY{p}{(}\PY{n}{edge\PYZus{}case\PYZus{}4}\PY{o}{.}\PY{n}{monte\PYZus{}carlo\PYZus{}result}\PY{o}{.}\PY{n}{k\PYZus{}value\PYZus{}plot}\PY{p}{)}
\end{Verbatim}
\end{tcolorbox}

    \begin{center}
    \adjustimage{max size={0.9\linewidth}{0.9\paperheight}}{Project 3_files/Project 3_130_0.png}
    \end{center}
    { \hspace*{\fill} \\}
    
    \textbf{Figure 18.} Again, we see the same trend: the uncertainty
multiplication rate actually increases with respect to the generation
for these points, and no obvious trend can be seen in the progression
pattern for these points (so the \(k\) value obtained from them can be
effectively said to be random). This begs the question of if the
geometries along this line converge to a specific long-run
multiplication rate whatsoever (which, if false, would counteract the
main assumption of this previous investigation), but seeing as the
considerations of this experiment are largely only focused on the trends
observed along trivial cross-sections for even aspect ratios or large
volumes rather than this section, further investigation into this region
can be left for future exploration.

    \hypertarget{summary-of-findings}{%
\subsection{Summary of Findings}\label{summary-of-findings}}

    \hypertarget{conclusions}{%
\subsubsection{Conclusions}\label{conclusions}}

    Altogether, there is a clear preference in stable multiplication rates
for even aspect ratios and larger volumes. Higher skew (aspect ratio
\(R\)) and lower volumes decrease the multiplication rate, whereas lower
skews and higher volumes increase the multiplication rate. For both
cases, however, very stable systems or very unstable systems both trend
to extremely certain measurements, and the uncertainty in the
measurement is maximal for geometries that are only somewhat skewed and
somewhat large in the middle of the phase (more ``traditionally''
rectangular geometries).

    \hypertarget{limitations}{%
\subsubsection{Limitations}\label{limitations}}

    \begin{itemize}
\tightlist
\item
  This model doesn't consider length dependence whatsoever, and length
  is produced as a calculation between the width and height in order to
  keep the aspect-ratio 2-dimensional (and similarly, having a
  2-dimensional contour).
\item
  We assume that these reactors are able to approach long-term stability
  whatsoever, and that the trends we've pinpointed for determining the
  minimal-stable generational depth are thorough enough to accurately
  determine the stable multiplication rate for each reactor geometry.
  This exploration wasn't thoroughly considered for the points along the
  positive diagonal that result in seemingly unpredictable behavior,
  which questions if this assumption is truly correct.
\item
  Only rectangular reactor geometries were considered, but in the real
  world, reactors are generally produced in cylindrical or spherical
  shapes.
\end{itemize}

    \hypertarget{future-work}{%
\subsubsection{Future Work}\label{future-work}}

    This research begets the following questions:

\begin{enumerate}
\def\labelenumi{\arabic{enumi}.}
\tightlist
\item
  How do the progression patterns in multiplication rate change for
  non-rectangular geometries, like spheres, cylinders, or polyhedrons?
\item
  What characteristics can be seen in the progression in multiplication
  rate for the maximally-uncertain geometries along the positive main
  diagonal of the contour?
\end{enumerate}

    \hypertarget{bibliography}{%
\subsection{Bibliography}\label{bibliography}}

    \begin{enumerate}
\def\labelenumi{\arabic{enumi}.}
\tightlist
\item
  https://docs.python.org/3/library/dataclasses.html
\item
  https://docs.python.org/3/library/collections.html\#collections.deque
\item
  https://pypi.org/project/pillow/
\item
  https://docs.python.org/3/library/io.html
\item
  https://cs110.students.cs.ubc.ca/reference/design-recipes.html\#GenRec
\item
  https://docs.python.org/3/library/itertools.html
\item
  https://matplotlib.org/3.3.3/api/\_as\_gen/mpl\_toolkits.mplot3d.axes3d.Axes3D.html
\end{enumerate}

    \hypertarget{appendix}{%
\section{Appendix}\label{appendix}}

    \hypertarget{a.-validation-tasks}{%
\subsection{A. Validation Tasks}\label{a.-validation-tasks}}

    \hypertarget{a-1-situational-intuition}{%
\subsubsection{A-1 Situational
Intuition}\label{a-1-situational-intuition}}

    Certain basic principles should apriori be true, and we can use this
intuition to help us validate the results of this simulation. For
example, an ``infinite sized'' box should lose absolutely no neutrons.
We obviously cannot make an infinite box, but we can make a box much,
much larger than the mean-free-path of a neutron.

A cubic box with a volume of, say, \(10^{10}\) \(m^3\) volume should do
the trick, and to keep the simulation simple, we can just model 8
generations (as every neutron will be modeled, so we want to keep this
low such that we don't have to do too many computations). Then, we can
just measure the number of neutrons that remain at the end, and if it is
exactly \(2^8\), then we know that the simulation works:

    \begin{tcolorbox}[breakable, size=fbox, boxrule=1pt, pad at break*=1mm,colback=cellbackground, colframe=cellborder]
\prompt{In}{incolor}{69}{\boxspacing}
\begin{Verbatim}[commandchars=\\\{\}]
\PY{n}{validation\PYZus{}1} \PY{o}{=} \PY{n}{simulate}\PY{p}{(}
    \PY{n}{num\PYZus{}generations}      \PY{o}{=} \PY{l+m+mi}{8}\PY{p}{,} 
    \PY{n}{dimensions}           \PY{o}{=} \PY{n}{calculate\PYZus{}dimensions}\PY{p}{(}\PY{l+m+mi}{1}\PY{p}{,}\PY{l+m+mf}{1e6}\PY{p}{)}\PY{p}{,} 
    \PY{n}{num\PYZus{}initial\PYZus{}neutrons} \PY{o}{=} \PY{l+m+mi}{1}\PY{p}{,} 
    \PY{n}{plot} \PY{o}{=} \PY{k+kc}{False}
\PY{p}{)}
\end{Verbatim}
\end{tcolorbox}

    \begin{tcolorbox}[breakable, size=fbox, boxrule=1pt, pad at break*=1mm,colback=cellbackground, colframe=cellborder]
\prompt{In}{incolor}{70}{\boxspacing}
\begin{Verbatim}[commandchars=\\\{\}]
\PY{n}{valid\PYZus{}1\PYZus{}neutrons} \PY{o}{=} \PY{n}{get\PYZus{}neutrons\PYZus{}in\PYZus{}each\PYZus{}generation}\PY{p}{(}\PY{n}{validation\PYZus{}1}\PY{o}{.}\PY{n}{result\PYZus{}trees}\PY{p}{[}\PY{l+m+mi}{0}\PY{p}{]}\PY{p}{)}
\PY{n+nb}{print}\PY{p}{(}\PY{l+s+sa}{f}\PY{l+s+s1}{\PYZsq{}}\PY{l+s+s1}{Neutron Count by Generation: }\PY{l+s+si}{\PYZob{}}\PY{n}{valid\PYZus{}1\PYZus{}neutrons}\PY{l+s+si}{\PYZcb{}}\PY{l+s+s1}{\PYZsq{}}\PY{p}{)}
\PY{n+nb}{print}\PY{p}{(}\PY{l+s+sa}{f}\PY{l+s+s1}{\PYZsq{}}\PY{l+s+s1}{Base 2 Logarithm of Neutrons in Generation 8: }\PY{l+s+si}{\PYZob{}}\PY{n}{np}\PY{o}{.}\PY{n}{log2}\PY{p}{(}\PY{n}{valid\PYZus{}1\PYZus{}neutrons}\PY{p}{[}\PY{o}{\PYZhy{}}\PY{l+m+mi}{1}\PY{p}{]}\PY{p}{)}\PY{l+s+si}{\PYZcb{}}\PY{l+s+s1}{\PYZsq{}}\PY{p}{)}
\end{Verbatim}
\end{tcolorbox}

    \begin{Verbatim}[commandchars=\\\{\}]
Neutron Count by Generation: [  1   2   4   8  16  32  64 128 256]
Base 2 Logarithm of Neutrons in Generation 8: 8.0
    \end{Verbatim}

    So, therefore, we know that no neutrons were lost over the course of the
simulation, so this absolutely worked. We can now try the next case,
where we have a box that with a volume that is approximately zero. This
should lose all of its neutrons absolutely, and shouldn't even progress
past the first generation. For this simulation, we can approximate this
by using a volume that is very close to zero but nonzero, like
\(V=10^{-6}\) \(m^3\):

    \begin{tcolorbox}[breakable, size=fbox, boxrule=1pt, pad at break*=1mm,colback=cellbackground, colframe=cellborder]
\prompt{In}{incolor}{71}{\boxspacing}
\begin{Verbatim}[commandchars=\\\{\}]
\PY{n}{validation\PYZus{}2} \PY{o}{=} \PY{n}{simulate}\PY{p}{(}
    \PY{n}{num\PYZus{}generations}      \PY{o}{=} \PY{l+m+mi}{8}\PY{p}{,} 
    \PY{n}{dimensions}           \PY{o}{=} \PY{n}{calculate\PYZus{}dimensions}\PY{p}{(}\PY{l+m+mi}{1}\PY{p}{,}\PY{l+m+mf}{1e\PYZhy{}6}\PY{p}{)}\PY{p}{,} 
    \PY{n}{num\PYZus{}initial\PYZus{}neutrons} \PY{o}{=} \PY{l+m+mi}{1}\PY{p}{,} 
    \PY{n}{plot} \PY{o}{=} \PY{k+kc}{False}
\PY{p}{)}
\end{Verbatim}
\end{tcolorbox}

    \begin{tcolorbox}[breakable, size=fbox, boxrule=1pt, pad at break*=1mm,colback=cellbackground, colframe=cellborder]
\prompt{In}{incolor}{72}{\boxspacing}
\begin{Verbatim}[commandchars=\\\{\}]
\PY{n}{valid\PYZus{}2\PYZus{}neutrons} \PY{o}{=} \PY{n}{get\PYZus{}neutrons\PYZus{}in\PYZus{}each\PYZus{}generation}\PY{p}{(}\PY{n}{validation\PYZus{}2}\PY{o}{.}\PY{n}{result\PYZus{}trees}\PY{p}{[}\PY{l+m+mi}{0}\PY{p}{]}\PY{p}{)}
\PY{n+nb}{print}\PY{p}{(}\PY{l+s+sa}{f}\PY{l+s+s1}{\PYZsq{}}\PY{l+s+s1}{Neutron Count by Generation: }\PY{l+s+si}{\PYZob{}}\PY{n}{valid\PYZus{}2\PYZus{}neutrons}\PY{l+s+si}{\PYZcb{}}\PY{l+s+s1}{\PYZsq{}}\PY{p}{)}
\end{Verbatim}
\end{tcolorbox}

    \begin{Verbatim}[commandchars=\\\{\}]
Neutron Count by Generation: [0]
    \end{Verbatim}

    As we can see, the simulation entirely doesn't progress. The first,
starting neutron isn't even counted as being successfully within the box
as it immediately exits the box on its first movement.

    Further, we also know that the same behavior of the neutrons being
immediately lost should occur for any velocity with infinitesimal width
(infinitely skewed ratio, or \(R \to \infty\)). We can test this,
similarly, by pairing \(R=10^{6}\) with \(V=10^{6}\):

    \begin{tcolorbox}[breakable, size=fbox, boxrule=1pt, pad at break*=1mm,colback=cellbackground, colframe=cellborder]
\prompt{In}{incolor}{73}{\boxspacing}
\begin{Verbatim}[commandchars=\\\{\}]
\PY{n}{validation\PYZus{}3} \PY{o}{=} \PY{n}{simulate}\PY{p}{(}
    \PY{n}{num\PYZus{}generations}      \PY{o}{=} \PY{l+m+mi}{8}\PY{p}{,} 
    \PY{n}{dimensions}           \PY{o}{=} \PY{n}{calculate\PYZus{}dimensions}\PY{p}{(}\PY{l+m+mf}{1e6}\PY{p}{,}\PY{l+m+mf}{1e6}\PY{p}{)}\PY{p}{,} 
    \PY{n}{num\PYZus{}initial\PYZus{}neutrons} \PY{o}{=} \PY{l+m+mi}{1}\PY{p}{,} 
    \PY{n}{plot} \PY{o}{=} \PY{k+kc}{False}
\PY{p}{)}
\end{Verbatim}
\end{tcolorbox}

    \begin{tcolorbox}[breakable, size=fbox, boxrule=1pt, pad at break*=1mm,colback=cellbackground, colframe=cellborder]
\prompt{In}{incolor}{74}{\boxspacing}
\begin{Verbatim}[commandchars=\\\{\}]
\PY{n}{valid\PYZus{}3\PYZus{}neutrons} \PY{o}{=} \PY{n}{get\PYZus{}neutrons\PYZus{}in\PYZus{}each\PYZus{}generation}\PY{p}{(}\PY{n}{validation\PYZus{}3}\PY{o}{.}\PY{n}{result\PYZus{}trees}\PY{p}{[}\PY{l+m+mi}{0}\PY{p}{]}\PY{p}{)}
\PY{n+nb}{print}\PY{p}{(}\PY{l+s+sa}{f}\PY{l+s+s1}{\PYZsq{}}\PY{l+s+s1}{Neutron Count by Generation: }\PY{l+s+si}{\PYZob{}}\PY{n}{valid\PYZus{}3\PYZus{}neutrons}\PY{l+s+si}{\PYZcb{}}\PY{l+s+s1}{\PYZsq{}}\PY{p}{)}
\end{Verbatim}
\end{tcolorbox}

    \begin{Verbatim}[commandchars=\\\{\}]
Neutron Count by Generation: [0]
    \end{Verbatim}

    As expected, even despite having the exact same volume as our ``infinite
volume'' validation case, all of the neutrons are still lost due to
having an infinitesimal width/infinite skew.

    \hypertarget{a-2-flat-printing-trees}{%
\subsubsection{A-2 Flat-Printing Trees}\label{a-2-flat-printing-trees}}

    Flat-printing trees was utilized quite often throughout this project,
and it has proven to be quite useful for understanding the behavior of a
tree for a given simulation without the chaos of working with
three-dimensional graphics. It's easy to quickly verify basic mechanics,
like that the ending position of a parent is the starting position of
all of its children, and I could quickly see which neutron trees were
lost and when/why.

    \hypertarget{a-3-unit-tests}{%
\subsubsection{A-3 Unit Tests}\label{a-3-unit-tests}}

    As usual, I've utilized unit testing quite thoroughly throughout this
project, especially to verify the behavior of core functions. For more
in-depth unit tests that required a lot of trivial testing (like testing
array shapes and datatypes), I could get away with simply generating
them with generative AI, but for very specific/core tests, I've made
sure to manually write them.

    \begin{tcolorbox}[breakable, size=fbox, boxrule=1pt, pad at break*=1mm,colback=cellbackground, colframe=cellborder]
\prompt{In}{incolor}{64}{\boxspacing}
\begin{Verbatim}[commandchars=\\\{\}]
\PY{k}{def} \PY{n+nf}{test\PYZus{}get\PYZus{}next\PYZus{}neutron\PYZus{}position}\PY{p}{(}\PY{n}{num\PYZus{}samples}\PY{o}{=}\PY{l+m+mi}{100\PYZus{}000}\PY{p}{,} \PY{n}{tol}\PY{o}{=}\PY{l+m+mf}{0.05}\PY{p}{)}\PY{p}{:}
\PY{+w}{    }\PY{l+s+sd}{\PYZdq{}\PYZdq{}\PYZdq{}}
\PY{l+s+sd}{    Tests `get\PYZus{}next\PYZus{}neutron\PYZus{}position`}
\PY{l+s+sd}{    \PYZdq{}\PYZdq{}\PYZdq{}}
    \PY{n}{np}\PY{o}{.}\PY{n}{random}\PY{o}{.}\PY{n}{seed}\PY{p}{(}\PY{l+m+mi}{0}\PY{p}{)}
    \PY{n}{current\PYZus{}position} \PY{o}{=} \PY{n}{np}\PY{o}{.}\PY{n}{zeros}\PY{p}{(}\PY{l+m+mi}{3}\PY{p}{)}
    
    \PY{n}{positions} \PY{o}{=} \PY{n}{np}\PY{o}{.}\PY{n}{array}\PY{p}{(}\PY{p}{[}\PY{n}{get\PYZus{}next\PYZus{}neutron\PYZus{}position}\PY{p}{(}\PY{n}{current\PYZus{}position}\PY{p}{)} \PY{k}{for} \PY{n}{\PYZus{}} \PY{o+ow}{in} \PY{n+nb}{range}\PY{p}{(}\PY{n}{num\PYZus{}samples}\PY{p}{)}\PY{p}{]}\PY{p}{)}
    \PY{n}{displacements} \PY{o}{=} \PY{n}{positions} \PY{o}{\PYZhy{}} \PY{n}{current\PYZus{}position}
    \PY{n}{distances} \PY{o}{=} \PY{n}{np}\PY{o}{.}\PY{n}{linalg}\PY{o}{.}\PY{n}{norm}\PY{p}{(}\PY{n}{displacements}\PY{p}{,} \PY{n}{axis}\PY{o}{=}\PY{l+m+mi}{1}\PY{p}{)}
    
    \PY{c+c1}{\PYZsh{} \PYZhy{}\PYZhy{}\PYZhy{} 1. Output shape check \PYZhy{}\PYZhy{}\PYZhy{}}
    \PY{k}{assert} \PY{n}{positions}\PY{o}{.}\PY{n}{shape} \PY{o}{==} \PY{p}{(}\PY{n}{num\PYZus{}samples}\PY{p}{,} \PY{l+m+mi}{3}\PY{p}{)}\PY{p}{,} \PY{l+s+s2}{\PYZdq{}}\PY{l+s+s2}{Output should be 3D positions}\PY{l+s+s2}{\PYZdq{}}
    
    \PY{c+c1}{\PYZsh{} \PYZhy{}\PYZhy{}\PYZhy{} 2. Mean displacement = mean free path \PYZhy{}\PYZhy{}\PYZhy{}}
    \PY{n}{mean\PYZus{}distance} \PY{o}{=} \PY{n}{distances}\PY{o}{.}\PY{n}{mean}\PY{p}{(}\PY{p}{)}
    \PY{k}{assert} \PY{n+nb}{abs}\PY{p}{(}\PY{n}{mean\PYZus{}distance} \PY{o}{\PYZhy{}} \PY{n}{MEAN\PYZus{}FREE\PYZus{}PATH}\PY{p}{)} \PY{o}{/} \PY{n}{MEAN\PYZus{}FREE\PYZus{}PATH} \PY{o}{\PYZlt{}} \PY{n}{tol}\PY{p}{,} \PYZbs{}
        \PY{l+s+sa}{f}\PY{l+s+s2}{\PYZdq{}}\PY{l+s+s2}{Mean distance }\PY{l+s+si}{\PYZob{}}\PY{n}{mean\PYZus{}distance}\PY{l+s+si}{:}\PY{l+s+s2}{.2f}\PY{l+s+si}{\PYZcb{}}\PY{l+s+s2}{ differs too much from expected }\PY{l+s+si}{\PYZob{}}\PY{n}{MEAN\PYZus{}FREE\PYZus{}PATH}\PY{l+s+si}{:}\PY{l+s+s2}{.2f}\PY{l+s+si}{\PYZcb{}}\PY{l+s+s2}{\PYZdq{}}
    
    \PY{c+c1}{\PYZsh{} \PYZhy{}\PYZhy{}\PYZhy{} 3. Direction uniformity check (mean of cos(theta) and sin(theta) near 0) \PYZhy{}\PYZhy{}\PYZhy{}}
    \PY{n}{directions} \PY{o}{=} \PY{n}{displacements} \PY{o}{/} \PY{n}{distances}\PY{p}{[}\PY{p}{:}\PY{p}{,} \PY{k+kc}{None}\PY{p}{]}
    \PY{n}{mean\PYZus{}dir} \PY{o}{=} \PY{n}{np}\PY{o}{.}\PY{n}{mean}\PY{p}{(}\PY{n}{directions}\PY{p}{,} \PY{n}{axis}\PY{o}{=}\PY{l+m+mi}{0}\PY{p}{)}
    \PY{k}{assert} \PY{n}{np}\PY{o}{.}\PY{n}{allclose}\PY{p}{(}\PY{n}{mean\PYZus{}dir}\PY{p}{,} \PY{p}{[}\PY{l+m+mi}{0}\PY{p}{,} \PY{l+m+mi}{0}\PY{p}{,} \PY{l+m+mi}{0}\PY{p}{]}\PY{p}{,} \PY{n}{atol}\PY{o}{=}\PY{l+m+mf}{0.02}\PY{p}{)}\PY{p}{,} \PYZbs{}
        \PY{l+s+sa}{f}\PY{l+s+s2}{\PYZdq{}}\PY{l+s+s2}{Directions are not uniform (mean direction = }\PY{l+s+si}{\PYZob{}}\PY{n}{mean\PYZus{}dir}\PY{l+s+si}{\PYZcb{}}\PY{l+s+s2}{)}\PY{l+s+s2}{\PYZdq{}}
    
    \PY{c+c1}{\PYZsh{} \PYZhy{}\PYZhy{}\PYZhy{} 4. Magnitude distribution sanity check (exponential shape) \PYZhy{}\PYZhy{}\PYZhy{}}
    \PY{c+c1}{\PYZsh{} Compare empirical mean to theoretical mean of exponential}
    \PY{n}{empirical\PYZus{}std} \PY{o}{=} \PY{n}{distances}\PY{o}{.}\PY{n}{std}\PY{p}{(}\PY{p}{)}
    \PY{n}{theoretical\PYZus{}std} \PY{o}{=} \PY{n}{MEAN\PYZus{}FREE\PYZus{}PATH}
    \PY{k}{assert} \PY{n+nb}{abs}\PY{p}{(}\PY{n}{empirical\PYZus{}std} \PY{o}{\PYZhy{}} \PY{n}{theoretical\PYZus{}std}\PY{p}{)} \PY{o}{/} \PY{n}{theoretical\PYZus{}std} \PY{o}{\PYZlt{}} \PY{l+m+mf}{0.1}\PY{p}{,} \PYZbs{}
        \PY{l+s+sa}{f}\PY{l+s+s2}{\PYZdq{}}\PY{l+s+s2}{Standard deviation }\PY{l+s+si}{\PYZob{}}\PY{n}{empirical\PYZus{}std}\PY{l+s+si}{:}\PY{l+s+s2}{.2f}\PY{l+s+si}{\PYZcb{}}\PY{l+s+s2}{ should be near }\PY{l+s+si}{\PYZob{}}\PY{n}{theoretical\PYZus{}std}\PY{l+s+si}{:}\PY{l+s+s2}{.2f}\PY{l+s+si}{\PYZcb{}}\PY{l+s+s2}{\PYZdq{}}
    
    \PY{n+nb}{print}\PY{p}{(}\PY{l+s+s1}{\PYZsq{}}\PY{l+s+s1}{Test Passed!}\PY{l+s+s1}{\PYZsq{}}\PY{p}{)}
    
\PY{n}{test\PYZus{}get\PYZus{}next\PYZus{}neutron\PYZus{}position}\PY{p}{(}\PY{p}{)}
\end{Verbatim}
\end{tcolorbox}

    \begin{Verbatim}[commandchars=\\\{\}]
Test Passed!
    \end{Verbatim}

    \begin{tcolorbox}[breakable, size=fbox, boxrule=1pt, pad at break*=1mm,colback=cellbackground, colframe=cellborder]
\prompt{In}{incolor}{65}{\boxspacing}
\begin{Verbatim}[commandchars=\\\{\}]
\PY{k}{def} \PY{n+nf}{test\PYZus{}traverse\PYZus{}tree}\PY{p}{(}\PY{n}{debug\PYZus{}print}\PY{p}{:} \PY{n+nb}{bool} \PY{o}{=} \PY{k+kc}{False}\PY{p}{)}\PY{p}{:}
    \PY{k}{if} \PY{n}{debug\PYZus{}print}\PY{p}{:}
        \PY{k}{for} \PY{n}{neutron}\PY{p}{,} \PY{n}{generation} \PY{o+ow}{in} \PY{n}{traverse\PYZus{}tree}\PY{p}{(}\PY{n}{EXAMPLE\PYZus{}TREE}\PY{p}{)}\PY{p}{:}
            \PY{n+nb}{print}\PY{p}{(}\PY{n}{generation}\PY{p}{,} \PY{n}{neutron}\PY{o}{.}\PY{n}{start\PYZus{}pos}\PY{p}{,} \PY{n}{neutron}\PY{o}{.}\PY{n}{end\PYZus{}pos}\PY{p}{)}
    
    \PY{k}{assert} \PY{n+nb}{len}\PY{p}{(}\PY{n+nb}{list}\PY{p}{(}\PY{n}{traverse\PYZus{}tree}\PY{p}{(}\PY{n}{EXAMPLE\PYZus{}TREE}\PY{p}{)}\PY{p}{)}\PY{p}{)} \PY{o}{==} \PY{l+m+mi}{5}
    \PY{n+nb}{print}\PY{p}{(}\PY{l+s+s2}{\PYZdq{}}\PY{l+s+s2}{Test Passed!}\PY{l+s+s2}{\PYZdq{}}\PY{p}{)}
    
\PY{n}{test\PYZus{}traverse\PYZus{}tree}\PY{p}{(}\PY{p}{)}
\end{Verbatim}
\end{tcolorbox}

    \begin{Verbatim}[commandchars=\\\{\}]
Test Passed!
    \end{Verbatim}

    \begin{tcolorbox}[breakable, size=fbox, boxrule=1pt, pad at break*=1mm,colback=cellbackground, colframe=cellborder]
\prompt{In}{incolor}{66}{\boxspacing}
\begin{Verbatim}[commandchars=\\\{\}]
\PY{k}{def} \PY{n+nf}{test\PYZus{}random\PYZus{}position\PYZus{}in\PYZus{}box}\PY{p}{(}\PY{p}{)}\PY{p}{:}
    \PY{n}{dim} \PY{o}{=} \PY{p}{(}\PY{l+m+mi}{10}\PY{p}{,}\PY{l+m+mi}{5}\PY{p}{,}\PY{l+m+mi}{3}\PY{p}{)}
    \PY{k}{assert} \PY{n}{random\PYZus{}position\PYZus{}in\PYZus{}box}\PY{p}{(}\PY{n}{dim}\PY{p}{)}\PY{o}{.}\PY{n}{size} \PY{o}{==} \PY{l+m+mi}{3}

    \PY{n}{pos} \PY{o}{=} \PY{n}{random\PYZus{}position\PYZus{}in\PYZus{}box}\PY{p}{(}\PY{n}{dim}\PY{p}{)}
    \PY{k}{assert} \PY{n}{bounded\PYZus{}in\PYZus{}dimensions}\PY{p}{(}\PY{n}{pos}\PY{p}{,} \PY{n}{dim}\PY{p}{)}
    
    \PY{n+nb}{print}\PY{p}{(}\PY{l+s+s1}{\PYZsq{}}\PY{l+s+s1}{Test Passed!}\PY{l+s+s1}{\PYZsq{}}\PY{p}{)}
    
\PY{n}{test\PYZus{}random\PYZus{}position\PYZus{}in\PYZus{}box}\PY{p}{(}\PY{p}{)}
\end{Verbatim}
\end{tcolorbox}

    \begin{Verbatim}[commandchars=\\\{\}]
Test Passed!
    \end{Verbatim}

    \begin{tcolorbox}[breakable, size=fbox, boxrule=1pt, pad at break*=1mm,colback=cellbackground, colframe=cellborder]
\prompt{In}{incolor}{67}{\boxspacing}
\begin{Verbatim}[commandchars=\\\{\}]
\PY{k}{def} \PY{n+nf}{test\PYZus{}is\PYZus{}outside\PYZus{}box}\PY{p}{(}\PY{p}{)}\PY{p}{:}
    \PY{n}{dim} \PY{o}{=} \PY{p}{(}\PY{l+m+mi}{10}\PY{p}{,}\PY{l+m+mi}{5}\PY{p}{)}
    
    \PY{k}{assert} \PY{n}{is\PYZus{}outside\PYZus{}box}\PY{p}{(}\PY{n}{random\PYZus{}position\PYZus{}in\PYZus{}box}\PY{p}{(}\PY{n}{dim}\PY{p}{)}\PY{p}{,} \PY{n}{dim}\PY{p}{)} \PY{o}{==} \PY{k+kc}{False}
    \PY{k}{assert} \PY{n}{is\PYZus{}outside\PYZus{}box}\PY{p}{(}\PY{n}{np}\PY{o}{.}\PY{n}{array}\PY{p}{(}\PY{p}{[}\PY{l+m+mi}{0}\PY{p}{,}   \PY{l+m+mi}{0}\PY{p}{]}\PY{p}{)}\PY{p}{,} \PY{n}{dim}\PY{p}{)} \PY{o}{==} \PY{k+kc}{False}
    \PY{k}{assert} \PY{n}{is\PYZus{}outside\PYZus{}box}\PY{p}{(}\PY{n}{np}\PY{o}{.}\PY{n}{array}\PY{p}{(}\PY{p}{[}\PY{l+m+mi}{10}\PY{p}{,}  \PY{l+m+mi}{5}\PY{p}{]}\PY{p}{)}\PY{p}{,} \PY{n}{dim}\PY{p}{)} \PY{o}{==} \PY{k+kc}{False}
    \PY{k}{assert} \PY{n}{is\PYZus{}outside\PYZus{}box}\PY{p}{(}\PY{n}{np}\PY{o}{.}\PY{n}{array}\PY{p}{(}\PY{p}{[}\PY{l+m+mi}{11}\PY{p}{,}  \PY{l+m+mi}{5}\PY{p}{]}\PY{p}{)}\PY{p}{,} \PY{n}{dim}\PY{p}{)} \PY{o}{==} \PY{k+kc}{True}
    \PY{k}{assert} \PY{n}{is\PYZus{}outside\PYZus{}box}\PY{p}{(}\PY{n}{np}\PY{o}{.}\PY{n}{array}\PY{p}{(}\PY{p}{[}\PY{o}{\PYZhy{}}\PY{l+m+mi}{1}\PY{p}{,} \PY{o}{\PYZhy{}}\PY{l+m+mi}{1}\PY{p}{]}\PY{p}{)}\PY{p}{,} \PY{n}{dim}\PY{p}{)} \PY{o}{==} \PY{k+kc}{True}
    
    \PY{n+nb}{print}\PY{p}{(}\PY{l+s+s1}{\PYZsq{}}\PY{l+s+s1}{Test Passed!}\PY{l+s+s1}{\PYZsq{}}\PY{p}{)}
    
\PY{n}{test\PYZus{}is\PYZus{}outside\PYZus{}box}\PY{p}{(}\PY{p}{)}
\end{Verbatim}
\end{tcolorbox}

    \begin{Verbatim}[commandchars=\\\{\}]
Test Passed!
    \end{Verbatim}

    \begin{tcolorbox}[breakable, size=fbox, boxrule=1pt, pad at break*=1mm,colback=cellbackground, colframe=cellborder]
\prompt{In}{incolor}{68}{\boxspacing}
\begin{Verbatim}[commandchars=\\\{\}]
\PY{k+kn}{import} \PY{n+nn}{math}

\PY{k}{def} \PY{n+nf}{test\PYZus{}generate\PYZus{}tree\PYZus{}structure}\PY{p}{(}\PY{p}{)}\PY{p}{:}
\PY{+w}{    }\PY{l+s+sd}{\PYZdq{}\PYZdq{}\PYZdq{}Root should be a NeutronNode with valid positions.\PYZdq{}\PYZdq{}\PYZdq{}}
    \PY{n}{root} \PY{o}{=} \PY{n}{generate\PYZus{}tree}\PY{p}{(}\PY{l+m+mi}{3}\PY{p}{,} \PY{p}{(}\PY{l+m+mi}{10}\PY{p}{,} \PY{l+m+mi}{10}\PY{p}{,} \PY{l+m+mi}{10}\PY{p}{)}\PY{p}{)}

    \PY{k}{assert} \PY{n+nb}{isinstance}\PY{p}{(}\PY{n}{root}\PY{p}{,} \PY{n}{NeutronNode}\PY{p}{)}\PY{p}{,} \PY{l+s+s2}{\PYZdq{}}\PY{l+s+s2}{Root must be NeutronNode}\PY{l+s+s2}{\PYZdq{}}
    \PY{k}{assert} \PY{n+nb}{hasattr}\PY{p}{(}\PY{n}{root}\PY{p}{,} \PY{l+s+s2}{\PYZdq{}}\PY{l+s+s2}{children}\PY{l+s+s2}{\PYZdq{}}\PY{p}{)}\PY{p}{,} \PY{l+s+s2}{\PYZdq{}}\PY{l+s+s2}{Root must have children list}\PY{l+s+s2}{\PYZdq{}}

    \PY{k}{assert} \PY{n+nb}{len}\PY{p}{(}\PY{n}{root}\PY{o}{.}\PY{n}{start\PYZus{}pos}\PY{p}{)} \PY{o}{==} \PY{l+m+mi}{3}\PY{p}{,} \PY{l+s+s2}{\PYZdq{}}\PY{l+s+s2}{Start position must be 3D}\PY{l+s+s2}{\PYZdq{}}
    \PY{k}{assert} \PY{n+nb}{len}\PY{p}{(}\PY{n}{root}\PY{o}{.}\PY{n}{end\PYZus{}pos}\PY{p}{)} \PY{o}{==} \PY{l+m+mi}{3}\PY{p}{,} \PY{l+s+s2}{\PYZdq{}}\PY{l+s+s2}{End position must be 3D}\PY{l+s+s2}{\PYZdq{}}

    \PY{n+nb}{print}\PY{p}{(}\PY{l+s+s2}{\PYZdq{}}\PY{l+s+s2}{test\PYZus{}generate\PYZus{}tree\PYZus{}structure passed.}\PY{l+s+s2}{\PYZdq{}}\PY{p}{)}


\PY{k}{def} \PY{n+nf}{test\PYZus{}positions\PYZus{}validity}\PY{p}{(}\PY{p}{)}\PY{p}{:}
\PY{+w}{    }\PY{l+s+sd}{\PYZdq{}\PYZdq{}\PYZdq{}}
\PY{l+s+sd}{    Start must be inside the box; end may be inside or outside.}
\PY{l+s+sd}{    \PYZdq{}\PYZdq{}\PYZdq{}}
    \PY{n}{dims} \PY{o}{=} \PY{p}{(}\PY{l+m+mi}{10}\PY{p}{,} \PY{l+m+mi}{10}\PY{p}{,} \PY{l+m+mi}{10}\PY{p}{)}
    \PY{n}{root} \PY{o}{=} \PY{n}{generate\PYZus{}tree}\PY{p}{(}\PY{l+m+mi}{4}\PY{p}{,} \PY{n}{dims}\PY{p}{)}

    \PY{k}{assert} \PY{n}{bounded\PYZus{}in\PYZus{}dimensions}\PY{p}{(}\PY{n}{root}\PY{o}{.}\PY{n}{start\PYZus{}pos}\PY{p}{,} \PY{n}{dims}\PY{p}{)}\PY{p}{,} \PYZbs{}
        \PY{l+s+s2}{\PYZdq{}}\PY{l+s+s2}{Initial position must be inside the box}\PY{l+s+s2}{\PYZdq{}}

    \PY{n+nb}{print}\PY{p}{(}\PY{l+s+s2}{\PYZdq{}}\PY{l+s+s2}{test\PYZus{}positions\PYZus{}validity passed.}\PY{l+s+s2}{\PYZdq{}}\PY{p}{)}


\PY{k}{def} \PY{n+nf}{test\PYZus{}tree\PYZus{}depth}\PY{p}{(}\PY{p}{)}\PY{p}{:}
\PY{+w}{    }\PY{l+s+sd}{\PYZdq{}\PYZdq{}\PYZdq{}}
\PY{l+s+sd}{    Tree depth should not exceed num\PYZus{}generations.}
\PY{l+s+sd}{    \PYZdq{}\PYZdq{}\PYZdq{}}
    \PY{n}{dims} \PY{o}{=} \PY{p}{(}\PY{l+m+mi}{10}\PY{p}{,} \PY{l+m+mi}{10}\PY{p}{,} \PY{l+m+mi}{10}\PY{p}{)}
    \PY{n}{gens} \PY{o}{=} \PY{l+m+mi}{4}
    \PY{n}{root} \PY{o}{=} \PY{n}{generate\PYZus{}tree}\PY{p}{(}\PY{n}{gens}\PY{p}{,} \PY{n}{dims}\PY{p}{)}

    \PY{c+c1}{\PYZsh{} BFS to compute max depth}
    \PY{n}{queue} \PY{o}{=} \PY{p}{[}\PY{p}{(}\PY{n}{root}\PY{p}{,} \PY{l+m+mi}{0}\PY{p}{)}\PY{p}{]}
    \PY{n}{max\PYZus{}depth} \PY{o}{=} \PY{l+m+mi}{0}

    \PY{k}{while} \PY{n}{queue}\PY{p}{:}
        \PY{n}{node}\PY{p}{,} \PY{n}{depth} \PY{o}{=} \PY{n}{queue}\PY{o}{.}\PY{n}{pop}\PY{p}{(}\PY{l+m+mi}{0}\PY{p}{)}
        \PY{n}{max\PYZus{}depth} \PY{o}{=} \PY{n+nb}{max}\PY{p}{(}\PY{n}{max\PYZus{}depth}\PY{p}{,} \PY{n}{depth}\PY{p}{)}

        \PY{k}{for} \PY{n}{child} \PY{o+ow}{in} \PY{n}{node}\PY{o}{.}\PY{n}{children}\PY{p}{:}
            \PY{n}{queue}\PY{o}{.}\PY{n}{append}\PY{p}{(}\PY{p}{(}\PY{n}{child}\PY{p}{,} \PY{n}{depth} \PY{o}{+} \PY{l+m+mi}{1}\PY{p}{)}\PY{p}{)}

    \PY{k}{assert} \PY{n}{max\PYZus{}depth} \PY{o}{\PYZlt{}}\PY{o}{=} \PY{n}{gens}\PY{p}{,} \PY{l+s+s2}{\PYZdq{}}\PY{l+s+s2}{Tree depth exceeds num\PYZus{}generations}\PY{l+s+s2}{\PYZdq{}}

    \PY{n+nb}{print}\PY{p}{(}\PY{l+s+s2}{\PYZdq{}}\PY{l+s+s2}{test\PYZus{}tree\PYZus{}depth passed.}\PY{l+s+s2}{\PYZdq{}}\PY{p}{)}


\PY{k}{def} \PY{n+nf}{test\PYZus{}lost\PYZus{}flag\PYZus{}correctness}\PY{p}{(}\PY{p}{)}\PY{p}{:}
\PY{+w}{    }\PY{l+s+sd}{\PYZdq{}\PYZdq{}\PYZdq{}}
\PY{l+s+sd}{    lost == True if and only if end\PYZus{}pos is outside box.}
\PY{l+s+sd}{    \PYZdq{}\PYZdq{}\PYZdq{}}
    \PY{n}{dims} \PY{o}{=} \PY{p}{(}\PY{l+m+mi}{10}\PY{p}{,} \PY{l+m+mi}{10}\PY{p}{,} \PY{l+m+mi}{10}\PY{p}{)}
    \PY{n}{root} \PY{o}{=} \PY{n}{generate\PYZus{}tree}\PY{p}{(}\PY{l+m+mi}{1}\PY{p}{,} \PY{n}{dims}\PY{p}{)}

    \PY{n}{expected} \PY{o}{=} \PY{n}{is\PYZus{}outside\PYZus{}box}\PY{p}{(}\PY{n}{root}\PY{o}{.}\PY{n}{end\PYZus{}pos}\PY{p}{,} \PY{n}{dims}\PY{p}{)}
    \PY{k}{assert} \PY{n}{root}\PY{o}{.}\PY{n}{lost} \PY{o}{==} \PY{n}{expected}\PY{p}{,} \PY{l+s+s2}{\PYZdq{}}\PY{l+s+s2}{Lost flag does not match outside\PYZhy{}box test}\PY{l+s+s2}{\PYZdq{}}

    \PY{n+nb}{print}\PY{p}{(}\PY{l+s+s2}{\PYZdq{}}\PY{l+s+s2}{test\PYZus{}lost\PYZus{}flag\PYZus{}correctness passed.}\PY{l+s+s2}{\PYZdq{}}\PY{p}{)}


\PY{k}{def} \PY{n+nf}{test\PYZus{}multiple\PYZus{}generations\PYZus{}not\PYZus{}empty}\PY{p}{(}\PY{p}{)}\PY{p}{:}
\PY{+w}{    }\PY{l+s+sd}{\PYZdq{}\PYZdq{}\PYZdq{}}
\PY{l+s+sd}{    If num\PYZus{}generations \PYZgt{} 0, root }
\PY{l+s+sd}{    should potentially have children.}
\PY{l+s+sd}{    \PYZdq{}\PYZdq{}\PYZdq{}}
    \PY{n}{dims} \PY{o}{=} \PY{p}{(}\PY{l+m+mi}{10}\PY{p}{,} \PY{l+m+mi}{10}\PY{p}{,} \PY{l+m+mi}{10}\PY{p}{)}
    \PY{n}{root} \PY{o}{=} \PY{n}{generate\PYZus{}tree}\PY{p}{(}\PY{l+m+mi}{3}\PY{p}{,} \PY{n}{dims}\PY{p}{)}

    \PY{c+c1}{\PYZsh{} Children may be zero if neutron is immediately lost,}
    \PY{c+c1}{\PYZsh{} but if not lost, we expect children}
    \PY{k}{if} \PY{o+ow}{not} \PY{n}{root}\PY{o}{.}\PY{n}{lost}\PY{p}{:}
        \PY{k}{assert} \PY{n+nb}{len}\PY{p}{(}\PY{n}{root}\PY{o}{.}\PY{n}{children}\PY{p}{)} \PY{o}{\PYZgt{}} \PY{l+m+mi}{0}\PY{p}{,} \PYZbs{}
            \PY{l+s+s2}{\PYZdq{}}\PY{l+s+s2}{Root is not lost, but has no children!}\PY{l+s+s2}{\PYZdq{}}

    \PY{n+nb}{print}\PY{p}{(}\PY{l+s+s2}{\PYZdq{}}\PY{l+s+s2}{test\PYZus{}multiple\PYZus{}generations\PYZus{}not\PYZus{}empty passed.}\PY{l+s+s2}{\PYZdq{}}\PY{p}{)}


\PY{k}{def} \PY{n+nf}{test\PYZus{}generate\PYZus{}tree}\PY{p}{(}\PY{p}{)}\PY{p}{:}
    \PY{n}{test\PYZus{}generate\PYZus{}tree\PYZus{}structure}\PY{p}{(}\PY{p}{)}
    \PY{n}{test\PYZus{}positions\PYZus{}validity}\PY{p}{(}\PY{p}{)}
    \PY{n}{test\PYZus{}tree\PYZus{}depth}\PY{p}{(}\PY{p}{)}
    \PY{n}{test\PYZus{}lost\PYZus{}flag\PYZus{}correctness}\PY{p}{(}\PY{p}{)}
    \PY{n}{test\PYZus{}multiple\PYZus{}generations\PYZus{}not\PYZus{}empty}\PY{p}{(}\PY{p}{)}
    \PY{n+nb}{print}\PY{p}{(}\PY{l+s+s2}{\PYZdq{}}\PY{l+s+se}{\PYZbs{}n}\PY{l+s+s2}{ALL TESTS PASSED!}\PY{l+s+s2}{\PYZdq{}}\PY{p}{)}


\PY{c+c1}{\PYZsh{} Run them}
\PY{n}{test\PYZus{}generate\PYZus{}tree}\PY{p}{(}\PY{p}{)}
\end{Verbatim}
\end{tcolorbox}

    \begin{Verbatim}[commandchars=\\\{\}]
test\_generate\_tree\_structure passed.
test\_positions\_validity passed.
test\_tree\_depth passed.
test\_lost\_flag\_correctness passed.
test\_multiple\_generations\_not\_empty passed.

ALL TESTS PASSED!
    \end{Verbatim}

    \hypertarget{b.-reflection-questions}{%
\subsection{B. Reflection Questions}\label{b.-reflection-questions}}

    \hypertarget{b-1.-coding-approaches-a}{%
\subsubsection{B-1. Coding Approaches
(A)}\label{b-1.-coding-approaches-a}}

    \begin{quote}
How well did you apply and extend your coding knowledge in this project?
Consider steps you took to make the code more efficient, readable and/or
concise.
\end{quote}

    I had to go back and reference some programming theory from CPSC 110, a
course I took last year, to treat the neutron-branching as a tree
generation problem using generative recursion. This problem can be done
in other ways, but a implementations of tree traversal and generation
algorithms undoubtedly serves as the cleanest and simplest solution to
this problem.

    \begin{quote}
Discuss any new-to-you coding techniques, functions or python packages
that you learned how to use.
\end{quote}

I had to use three-dimensional plotting, which was definitely rough to
implement (to say the least). I felt that, in this instance, no other
kind of plot would have been effective to display the trajectories, as,
for example, cross-sections leave out key dimensional information.

\begin{quote}
Reflect on any unforeseen coding challenges you faced in completing this
project.
\end{quote}

Unlike past projects, functions in this project build upon each other a
lot. If one function fails, then all future functions quickly fail. It
became even more critical on this project than ever before that certain
subroutines are guaranteed to work, and I struggled a bit on this
project with maintaining that fault tolerance.

    \hypertarget{b-2.-coding-approaches-b}{%
\subsubsection{B-2. Coding Approaches
(B)}\label{b-2.-coding-approaches-b}}

\begin{quote}
Highlight an aspect of your code that you feel you did particularily
well.
\end{quote}

    I'm particularly proud of my work on the function
\texttt{traverse\_tree} to simplify tree traversal, as this function
allows for the generalization of the tree data structure with a much
simpler interface.

    \begin{quote}
Discuss an aspect of your code that would benefit the most from further
effort.
\end{quote}

I feel somewhat unhappy with the code for generating the histograms for
each generation, as this may be better as one multi-panel plot rather
than separate plots, but I couldn't figure out how to dynamically
generate this like I was able to on the contour plotting.

    \hypertarget{b-3.-simulation-physics-and-investigation-a}{%
\subsubsection{B-3. Simulation Physics and Investigation
(A)}\label{b-3.-simulation-physics-and-investigation-a}}

\begin{quote}
How well did you apply and extend your physical modelling and scientific
investigation skills in this project? Consider the phase space you chose
to explore and how throroughly you explored it. Consider how you
translated physics into code and if appropriate any new physics you
learned or developed a more thorough understanding of.
\end{quote}

    The nature of the phase space I've explored is in an attempt to study
the relationship between the multiplication rate of the simulation as a
function of the geometry of the bounding box, and I feel as though given
the restrictions/research situation as I have framed it, I've been
successful at considering the trends produced. I feel as though my
investigation was sufficiently thorough, but my concern is that the
research situation as I've framed it is a little trivial. Again, I made
no consideration of length dependence (so this project is
three-dimensional computationally but only two-dimensionally in
parameters), and the results that I've produced are almost entirely
trivial/intuitive. As such, I didn't use particularly complicated (or
any new) physics on this project, but I feel quite happy with the
physics I did use.

    \hypertarget{b-4.-simulation-physics-and-investigation-b}{%
\subsubsection{B-4. Simulation Physics and Investigation
(B)}\label{b-4.-simulation-physics-and-investigation-b}}

\begin{quote}
Highlight something you feel you did particularily well in terms of the
context of your simulation, the physical modelling that you did or the
investigation you performed. Discuss an aspect of these dimensions of
your project that would benefit the most from further effort.
\end{quote}

    I believe my best section of physical modelling was in my calculation
and plotting of three-dimensional electron trajectories (or, in other
words, my full process for generating and displaying Neutron trees).
This was undoubtedly my favorite part of my project, not only in terms
of code-writing and mathematics but also in validating its success from
seeing the visualization.

    \hypertarget{b-5.-effectiveness-of-your-communication}{%
\subsubsection{B-5. Effectiveness of Your
Communication}\label{b-5.-effectiveness-of-your-communication}}

\begin{quote}
Highlight something you feel you did particularily well in your
visualizations or written communication. Discuss an aspect of your
visualizations or written communication that would benefit the most from
further effort.
\end{quote}

    I feel as though, again, that my 3-D plotting of the Neutron
trajectories was quite good, but my visualization of the aspect ratio
space is quite poor, because it really only shows the width dependence
rather than the height dependence as well, as the range where \(R > 1\)
is obviously greater than the range where \(0 \le R < 1\), and I
couldn't figure out how to rectify this.

    \hypertarget{acknowledgements}{%
\section{Acknowledgements}\label{acknowledgements}}

    I used ChatGPT quite a lot in writing this project, particularly for
code generation (trivial tests, simple implementations, etc.). For
example, due to my lack of knowledge in 3-D plotting, I provided ChatGPT
with the following stub:

\begin{Shaded}
\begin{Highlighting}[]
\KeywordTok{def}\NormalTok{ draw\_3d\_box(ax, origin, dimensions, edgecolor}\OperatorTok{=}\StringTok{\textquotesingle{}red\textquotesingle{}}\NormalTok{, linestyle}\OperatorTok{=}\StringTok{\textquotesingle{}dashed\textquotesingle{}}\NormalTok{, linewidth}\OperatorTok{=}\DecValTok{1}\NormalTok{):}
    \CommentTok{"""}
\CommentTok{    Draws a 3D wireframe box on a given 3D axes.}
\CommentTok{    """}
    \ControlFlowTok{return}
\end{Highlighting}
\end{Shaded}

and then tried (and tweaked) its responses until I got a solution that
seemed to work.

    \begin{tcolorbox}[breakable, size=fbox, boxrule=1pt, pad at break*=1mm,colback=cellbackground, colframe=cellborder]
\prompt{In}{incolor}{ }{\boxspacing}
\begin{Verbatim}[commandchars=\\\{\}]

\end{Verbatim}
\end{tcolorbox}


    % Add a bibliography block to the postdoc
    
    
    
\end{document}
