\documentclass[11pt]{article}

    \usepackage[breakable]{tcolorbox}
    \usepackage{parskip} % Stop auto-indenting (to mimic markdown behaviour)
    

    % Basic figure setup, for now with no caption control since it's done
    % automatically by Pandoc (which extracts ![](path) syntax from Markdown).
    \usepackage{graphicx}
    % Maintain compatibility with old templates. Remove in nbconvert 6.0
    \let\Oldincludegraphics\includegraphics
    % Ensure that by default, figures have no caption (until we provide a
    % proper Figure object with a Caption API and a way to capture that
    % in the conversion process - todo).
    \usepackage{caption}
    \DeclareCaptionFormat{nocaption}{}
    \captionsetup{format=nocaption,aboveskip=0pt,belowskip=0pt}

    \usepackage{float}
    \floatplacement{figure}{H} % forces figures to be placed at the correct location
    \usepackage{xcolor} % Allow colors to be defined
    \usepackage{enumerate} % Needed for markdown enumerations to work
    \usepackage{geometry} % Used to adjust the document margins
    \usepackage{amsmath} % Equations
    \usepackage{amssymb} % Equations
    \usepackage{textcomp} % defines textquotesingle
    % Hack from http://tex.stackexchange.com/a/47451/13684:
    \AtBeginDocument{%
        \def\PYZsq{\textquotesingle}% Upright quotes in Pygmentized code
    }
    \usepackage{upquote} % Upright quotes for verbatim code
    \usepackage{eurosym} % defines \euro

    \usepackage{iftex}
    \ifPDFTeX
        \usepackage[T1]{fontenc}
        \IfFileExists{alphabeta.sty}{
              \usepackage{alphabeta}
          }{
              \usepackage[mathletters]{ucs}
              \usepackage[utf8x]{inputenc}
          }
    \else
        \usepackage{fontspec}
        \usepackage{unicode-math}
    \fi

    \usepackage{fancyvrb} % verbatim replacement that allows latex
    \usepackage[Export]{adjustbox} % Used to constrain images to a maximum size
    \adjustboxset{max size={0.9\linewidth}{0.9\paperheight}}

    % The hyperref package gives us a pdf with properly built
    % internal navigation ('pdf bookmarks' for the table of contents,
    % internal cross-reference links, web links for URLs, etc.)
    \usepackage{hyperref}
    % The default LaTeX title has an obnoxious amount of whitespace. By default,
    % titling removes some of it. It also provides customization options.
    \usepackage{titling}
    \usepackage{longtable} % longtable support required by pandoc >1.10
    \usepackage{booktabs}  % table support for pandoc > 1.12.2
    \usepackage{array}     % table support for pandoc >= 2.11.3
    \usepackage{calc}      % table minipage width calculation for pandoc >= 2.11.1
    \usepackage[inline]{enumitem} % IRkernel/repr support (it uses the enumerate* environment)
    \usepackage[normalem]{ulem} % ulem is needed to support strikethroughs (\sout)
                                % normalem makes italics be italics, not underlines
    \usepackage{mathrsfs}
    

    
    % Colors for the hyperref package
    \definecolor{urlcolor}{rgb}{0,.145,.698}
    \definecolor{linkcolor}{rgb}{.71,0.21,0.01}
    \definecolor{citecolor}{rgb}{.12,.54,.11}

    % ANSI colors
    \definecolor{ansi-black}{HTML}{3E424D}
    \definecolor{ansi-black-intense}{HTML}{282C36}
    \definecolor{ansi-red}{HTML}{E75C58}
    \definecolor{ansi-red-intense}{HTML}{B22B31}
    \definecolor{ansi-green}{HTML}{00A250}
    \definecolor{ansi-green-intense}{HTML}{007427}
    \definecolor{ansi-yellow}{HTML}{DDB62B}
    \definecolor{ansi-yellow-intense}{HTML}{B27D12}
    \definecolor{ansi-blue}{HTML}{208FFB}
    \definecolor{ansi-blue-intense}{HTML}{0065CA}
    \definecolor{ansi-magenta}{HTML}{D160C4}
    \definecolor{ansi-magenta-intense}{HTML}{A03196}
    \definecolor{ansi-cyan}{HTML}{60C6C8}
    \definecolor{ansi-cyan-intense}{HTML}{258F8F}
    \definecolor{ansi-white}{HTML}{C5C1B4}
    \definecolor{ansi-white-intense}{HTML}{A1A6B2}
    \definecolor{ansi-default-inverse-fg}{HTML}{FFFFFF}
    \definecolor{ansi-default-inverse-bg}{HTML}{000000}

    % common color for the border for error outputs.
    \definecolor{outerrorbackground}{HTML}{FFDFDF}

    % commands and environments needed by pandoc snippets
    % extracted from the output of `pandoc -s`
    \providecommand{\tightlist}{%
      \setlength{\itemsep}{0pt}\setlength{\parskip}{0pt}}
    \DefineVerbatimEnvironment{Highlighting}{Verbatim}{commandchars=\\\{\}}
    % Add ',fontsize=\small' for more characters per line
    \newenvironment{Shaded}{}{}
    \newcommand{\KeywordTok}[1]{\textcolor[rgb]{0.00,0.44,0.13}{\textbf{{#1}}}}
    \newcommand{\DataTypeTok}[1]{\textcolor[rgb]{0.56,0.13,0.00}{{#1}}}
    \newcommand{\DecValTok}[1]{\textcolor[rgb]{0.25,0.63,0.44}{{#1}}}
    \newcommand{\BaseNTok}[1]{\textcolor[rgb]{0.25,0.63,0.44}{{#1}}}
    \newcommand{\FloatTok}[1]{\textcolor[rgb]{0.25,0.63,0.44}{{#1}}}
    \newcommand{\CharTok}[1]{\textcolor[rgb]{0.25,0.44,0.63}{{#1}}}
    \newcommand{\StringTok}[1]{\textcolor[rgb]{0.25,0.44,0.63}{{#1}}}
    \newcommand{\CommentTok}[1]{\textcolor[rgb]{0.38,0.63,0.69}{\textit{{#1}}}}
    \newcommand{\OtherTok}[1]{\textcolor[rgb]{0.00,0.44,0.13}{{#1}}}
    \newcommand{\AlertTok}[1]{\textcolor[rgb]{1.00,0.00,0.00}{\textbf{{#1}}}}
    \newcommand{\FunctionTok}[1]{\textcolor[rgb]{0.02,0.16,0.49}{{#1}}}
    \newcommand{\RegionMarkerTok}[1]{{#1}}
    \newcommand{\ErrorTok}[1]{\textcolor[rgb]{1.00,0.00,0.00}{\textbf{{#1}}}}
    \newcommand{\NormalTok}[1]{{#1}}

    % Additional commands for more recent versions of Pandoc
    \newcommand{\ConstantTok}[1]{\textcolor[rgb]{0.53,0.00,0.00}{{#1}}}
    \newcommand{\SpecialCharTok}[1]{\textcolor[rgb]{0.25,0.44,0.63}{{#1}}}
    \newcommand{\VerbatimStringTok}[1]{\textcolor[rgb]{0.25,0.44,0.63}{{#1}}}
    \newcommand{\SpecialStringTok}[1]{\textcolor[rgb]{0.73,0.40,0.53}{{#1}}}
    \newcommand{\ImportTok}[1]{{#1}}
    \newcommand{\DocumentationTok}[1]{\textcolor[rgb]{0.73,0.13,0.13}{\textit{{#1}}}}
    \newcommand{\AnnotationTok}[1]{\textcolor[rgb]{0.38,0.63,0.69}{\textbf{\textit{{#1}}}}}
    \newcommand{\CommentVarTok}[1]{\textcolor[rgb]{0.38,0.63,0.69}{\textbf{\textit{{#1}}}}}
    \newcommand{\VariableTok}[1]{\textcolor[rgb]{0.10,0.09,0.49}{{#1}}}
    \newcommand{\ControlFlowTok}[1]{\textcolor[rgb]{0.00,0.44,0.13}{\textbf{{#1}}}}
    \newcommand{\OperatorTok}[1]{\textcolor[rgb]{0.40,0.40,0.40}{{#1}}}
    \newcommand{\BuiltInTok}[1]{{#1}}
    \newcommand{\ExtensionTok}[1]{{#1}}
    \newcommand{\PreprocessorTok}[1]{\textcolor[rgb]{0.74,0.48,0.00}{{#1}}}
    \newcommand{\AttributeTok}[1]{\textcolor[rgb]{0.49,0.56,0.16}{{#1}}}
    \newcommand{\InformationTok}[1]{\textcolor[rgb]{0.38,0.63,0.69}{\textbf{\textit{{#1}}}}}
    \newcommand{\WarningTok}[1]{\textcolor[rgb]{0.38,0.63,0.69}{\textbf{\textit{{#1}}}}}


    % Define a nice break command that doesn't care if a line doesn't already
    % exist.
    \def\br{\hspace*{\fill} \\* }
    % Math Jax compatibility definitions
    \def\gt{>}
    \def\lt{<}
    \let\Oldtex\TeX
    \let\Oldlatex\LaTeX
    \renewcommand{\TeX}{\textrm{\Oldtex}}
    \renewcommand{\LaTeX}{\textrm{\Oldlatex}}
    % Document parameters
    % Document title
    \title{Project 1}
    
    
    
    
    
% Pygments definitions
\makeatletter
\def\PY@reset{\let\PY@it=\relax \let\PY@bf=\relax%
    \let\PY@ul=\relax \let\PY@tc=\relax%
    \let\PY@bc=\relax \let\PY@ff=\relax}
\def\PY@tok#1{\csname PY@tok@#1\endcsname}
\def\PY@toks#1+{\ifx\relax#1\empty\else%
    \PY@tok{#1}\expandafter\PY@toks\fi}
\def\PY@do#1{\PY@bc{\PY@tc{\PY@ul{%
    \PY@it{\PY@bf{\PY@ff{#1}}}}}}}
\def\PY#1#2{\PY@reset\PY@toks#1+\relax+\PY@do{#2}}

\@namedef{PY@tok@w}{\def\PY@tc##1{\textcolor[rgb]{0.73,0.73,0.73}{##1}}}
\@namedef{PY@tok@c}{\let\PY@it=\textit\def\PY@tc##1{\textcolor[rgb]{0.24,0.48,0.48}{##1}}}
\@namedef{PY@tok@cp}{\def\PY@tc##1{\textcolor[rgb]{0.61,0.40,0.00}{##1}}}
\@namedef{PY@tok@k}{\let\PY@bf=\textbf\def\PY@tc##1{\textcolor[rgb]{0.00,0.50,0.00}{##1}}}
\@namedef{PY@tok@kp}{\def\PY@tc##1{\textcolor[rgb]{0.00,0.50,0.00}{##1}}}
\@namedef{PY@tok@kt}{\def\PY@tc##1{\textcolor[rgb]{0.69,0.00,0.25}{##1}}}
\@namedef{PY@tok@o}{\def\PY@tc##1{\textcolor[rgb]{0.40,0.40,0.40}{##1}}}
\@namedef{PY@tok@ow}{\let\PY@bf=\textbf\def\PY@tc##1{\textcolor[rgb]{0.67,0.13,1.00}{##1}}}
\@namedef{PY@tok@nb}{\def\PY@tc##1{\textcolor[rgb]{0.00,0.50,0.00}{##1}}}
\@namedef{PY@tok@nf}{\def\PY@tc##1{\textcolor[rgb]{0.00,0.00,1.00}{##1}}}
\@namedef{PY@tok@nc}{\let\PY@bf=\textbf\def\PY@tc##1{\textcolor[rgb]{0.00,0.00,1.00}{##1}}}
\@namedef{PY@tok@nn}{\let\PY@bf=\textbf\def\PY@tc##1{\textcolor[rgb]{0.00,0.00,1.00}{##1}}}
\@namedef{PY@tok@ne}{\let\PY@bf=\textbf\def\PY@tc##1{\textcolor[rgb]{0.80,0.25,0.22}{##1}}}
\@namedef{PY@tok@nv}{\def\PY@tc##1{\textcolor[rgb]{0.10,0.09,0.49}{##1}}}
\@namedef{PY@tok@no}{\def\PY@tc##1{\textcolor[rgb]{0.53,0.00,0.00}{##1}}}
\@namedef{PY@tok@nl}{\def\PY@tc##1{\textcolor[rgb]{0.46,0.46,0.00}{##1}}}
\@namedef{PY@tok@ni}{\let\PY@bf=\textbf\def\PY@tc##1{\textcolor[rgb]{0.44,0.44,0.44}{##1}}}
\@namedef{PY@tok@na}{\def\PY@tc##1{\textcolor[rgb]{0.41,0.47,0.13}{##1}}}
\@namedef{PY@tok@nt}{\let\PY@bf=\textbf\def\PY@tc##1{\textcolor[rgb]{0.00,0.50,0.00}{##1}}}
\@namedef{PY@tok@nd}{\def\PY@tc##1{\textcolor[rgb]{0.67,0.13,1.00}{##1}}}
\@namedef{PY@tok@s}{\def\PY@tc##1{\textcolor[rgb]{0.73,0.13,0.13}{##1}}}
\@namedef{PY@tok@sd}{\let\PY@it=\textit\def\PY@tc##1{\textcolor[rgb]{0.73,0.13,0.13}{##1}}}
\@namedef{PY@tok@si}{\let\PY@bf=\textbf\def\PY@tc##1{\textcolor[rgb]{0.64,0.35,0.47}{##1}}}
\@namedef{PY@tok@se}{\let\PY@bf=\textbf\def\PY@tc##1{\textcolor[rgb]{0.67,0.36,0.12}{##1}}}
\@namedef{PY@tok@sr}{\def\PY@tc##1{\textcolor[rgb]{0.64,0.35,0.47}{##1}}}
\@namedef{PY@tok@ss}{\def\PY@tc##1{\textcolor[rgb]{0.10,0.09,0.49}{##1}}}
\@namedef{PY@tok@sx}{\def\PY@tc##1{\textcolor[rgb]{0.00,0.50,0.00}{##1}}}
\@namedef{PY@tok@m}{\def\PY@tc##1{\textcolor[rgb]{0.40,0.40,0.40}{##1}}}
\@namedef{PY@tok@gh}{\let\PY@bf=\textbf\def\PY@tc##1{\textcolor[rgb]{0.00,0.00,0.50}{##1}}}
\@namedef{PY@tok@gu}{\let\PY@bf=\textbf\def\PY@tc##1{\textcolor[rgb]{0.50,0.00,0.50}{##1}}}
\@namedef{PY@tok@gd}{\def\PY@tc##1{\textcolor[rgb]{0.63,0.00,0.00}{##1}}}
\@namedef{PY@tok@gi}{\def\PY@tc##1{\textcolor[rgb]{0.00,0.52,0.00}{##1}}}
\@namedef{PY@tok@gr}{\def\PY@tc##1{\textcolor[rgb]{0.89,0.00,0.00}{##1}}}
\@namedef{PY@tok@ge}{\let\PY@it=\textit}
\@namedef{PY@tok@gs}{\let\PY@bf=\textbf}
\@namedef{PY@tok@ges}{\let\PY@bf=\textbf\let\PY@it=\textit}
\@namedef{PY@tok@gp}{\let\PY@bf=\textbf\def\PY@tc##1{\textcolor[rgb]{0.00,0.00,0.50}{##1}}}
\@namedef{PY@tok@go}{\def\PY@tc##1{\textcolor[rgb]{0.44,0.44,0.44}{##1}}}
\@namedef{PY@tok@gt}{\def\PY@tc##1{\textcolor[rgb]{0.00,0.27,0.87}{##1}}}
\@namedef{PY@tok@err}{\def\PY@bc##1{{\setlength{\fboxsep}{\string -\fboxrule}\fcolorbox[rgb]{1.00,0.00,0.00}{1,1,1}{\strut ##1}}}}
\@namedef{PY@tok@kc}{\let\PY@bf=\textbf\def\PY@tc##1{\textcolor[rgb]{0.00,0.50,0.00}{##1}}}
\@namedef{PY@tok@kd}{\let\PY@bf=\textbf\def\PY@tc##1{\textcolor[rgb]{0.00,0.50,0.00}{##1}}}
\@namedef{PY@tok@kn}{\let\PY@bf=\textbf\def\PY@tc##1{\textcolor[rgb]{0.00,0.50,0.00}{##1}}}
\@namedef{PY@tok@kr}{\let\PY@bf=\textbf\def\PY@tc##1{\textcolor[rgb]{0.00,0.50,0.00}{##1}}}
\@namedef{PY@tok@bp}{\def\PY@tc##1{\textcolor[rgb]{0.00,0.50,0.00}{##1}}}
\@namedef{PY@tok@fm}{\def\PY@tc##1{\textcolor[rgb]{0.00,0.00,1.00}{##1}}}
\@namedef{PY@tok@vc}{\def\PY@tc##1{\textcolor[rgb]{0.10,0.09,0.49}{##1}}}
\@namedef{PY@tok@vg}{\def\PY@tc##1{\textcolor[rgb]{0.10,0.09,0.49}{##1}}}
\@namedef{PY@tok@vi}{\def\PY@tc##1{\textcolor[rgb]{0.10,0.09,0.49}{##1}}}
\@namedef{PY@tok@vm}{\def\PY@tc##1{\textcolor[rgb]{0.10,0.09,0.49}{##1}}}
\@namedef{PY@tok@sa}{\def\PY@tc##1{\textcolor[rgb]{0.73,0.13,0.13}{##1}}}
\@namedef{PY@tok@sb}{\def\PY@tc##1{\textcolor[rgb]{0.73,0.13,0.13}{##1}}}
\@namedef{PY@tok@sc}{\def\PY@tc##1{\textcolor[rgb]{0.73,0.13,0.13}{##1}}}
\@namedef{PY@tok@dl}{\def\PY@tc##1{\textcolor[rgb]{0.73,0.13,0.13}{##1}}}
\@namedef{PY@tok@s2}{\def\PY@tc##1{\textcolor[rgb]{0.73,0.13,0.13}{##1}}}
\@namedef{PY@tok@sh}{\def\PY@tc##1{\textcolor[rgb]{0.73,0.13,0.13}{##1}}}
\@namedef{PY@tok@s1}{\def\PY@tc##1{\textcolor[rgb]{0.73,0.13,0.13}{##1}}}
\@namedef{PY@tok@mb}{\def\PY@tc##1{\textcolor[rgb]{0.40,0.40,0.40}{##1}}}
\@namedef{PY@tok@mf}{\def\PY@tc##1{\textcolor[rgb]{0.40,0.40,0.40}{##1}}}
\@namedef{PY@tok@mh}{\def\PY@tc##1{\textcolor[rgb]{0.40,0.40,0.40}{##1}}}
\@namedef{PY@tok@mi}{\def\PY@tc##1{\textcolor[rgb]{0.40,0.40,0.40}{##1}}}
\@namedef{PY@tok@il}{\def\PY@tc##1{\textcolor[rgb]{0.40,0.40,0.40}{##1}}}
\@namedef{PY@tok@mo}{\def\PY@tc##1{\textcolor[rgb]{0.40,0.40,0.40}{##1}}}
\@namedef{PY@tok@ch}{\let\PY@it=\textit\def\PY@tc##1{\textcolor[rgb]{0.24,0.48,0.48}{##1}}}
\@namedef{PY@tok@cm}{\let\PY@it=\textit\def\PY@tc##1{\textcolor[rgb]{0.24,0.48,0.48}{##1}}}
\@namedef{PY@tok@cpf}{\let\PY@it=\textit\def\PY@tc##1{\textcolor[rgb]{0.24,0.48,0.48}{##1}}}
\@namedef{PY@tok@c1}{\let\PY@it=\textit\def\PY@tc##1{\textcolor[rgb]{0.24,0.48,0.48}{##1}}}
\@namedef{PY@tok@cs}{\let\PY@it=\textit\def\PY@tc##1{\textcolor[rgb]{0.24,0.48,0.48}{##1}}}

\def\PYZbs{\char`\\}
\def\PYZus{\char`\_}
\def\PYZob{\char`\{}
\def\PYZcb{\char`\}}
\def\PYZca{\char`\^}
\def\PYZam{\char`\&}
\def\PYZlt{\char`\<}
\def\PYZgt{\char`\>}
\def\PYZsh{\char`\#}
\def\PYZpc{\char`\%}
\def\PYZdl{\char`\$}
\def\PYZhy{\char`\-}
\def\PYZsq{\char`\'}
\def\PYZdq{\char`\"}
\def\PYZti{\char`\~}
% for compatibility with earlier versions
\def\PYZat{@}
\def\PYZlb{[}
\def\PYZrb{]}
\makeatother


    % For linebreaks inside Verbatim environment from package fancyvrb.
    \makeatletter
        \newbox\Wrappedcontinuationbox
        \newbox\Wrappedvisiblespacebox
        \newcommand*\Wrappedvisiblespace {\textcolor{red}{\textvisiblespace}}
        \newcommand*\Wrappedcontinuationsymbol {\textcolor{red}{\llap{\tiny$\m@th\hookrightarrow$}}}
        \newcommand*\Wrappedcontinuationindent {3ex }
        \newcommand*\Wrappedafterbreak {\kern\Wrappedcontinuationindent\copy\Wrappedcontinuationbox}
        % Take advantage of the already applied Pygments mark-up to insert
        % potential linebreaks for TeX processing.
        %        {, <, #, %, $, ' and ": go to next line.
        %        _, }, ^, &, >, - and ~: stay at end of broken line.
        % Use of \textquotesingle for straight quote.
        \newcommand*\Wrappedbreaksatspecials {%
            \def\PYGZus{\discretionary{\char`\_}{\Wrappedafterbreak}{\char`\_}}%
            \def\PYGZob{\discretionary{}{\Wrappedafterbreak\char`\{}{\char`\{}}%
            \def\PYGZcb{\discretionary{\char`\}}{\Wrappedafterbreak}{\char`\}}}%
            \def\PYGZca{\discretionary{\char`\^}{\Wrappedafterbreak}{\char`\^}}%
            \def\PYGZam{\discretionary{\char`\&}{\Wrappedafterbreak}{\char`\&}}%
            \def\PYGZlt{\discretionary{}{\Wrappedafterbreak\char`\<}{\char`\<}}%
            \def\PYGZgt{\discretionary{\char`\>}{\Wrappedafterbreak}{\char`\>}}%
            \def\PYGZsh{\discretionary{}{\Wrappedafterbreak\char`\#}{\char`\#}}%
            \def\PYGZpc{\discretionary{}{\Wrappedafterbreak\char`\%}{\char`\%}}%
            \def\PYGZdl{\discretionary{}{\Wrappedafterbreak\char`\$}{\char`\$}}%
            \def\PYGZhy{\discretionary{\char`\-}{\Wrappedafterbreak}{\char`\-}}%
            \def\PYGZsq{\discretionary{}{\Wrappedafterbreak\textquotesingle}{\textquotesingle}}%
            \def\PYGZdq{\discretionary{}{\Wrappedafterbreak\char`\"}{\char`\"}}%
            \def\PYGZti{\discretionary{\char`\~}{\Wrappedafterbreak}{\char`\~}}%
        }
        % Some characters . , ; ? ! / are not pygmentized.
        % This macro makes them "active" and they will insert potential linebreaks
        \newcommand*\Wrappedbreaksatpunct {%
            \lccode`\~`\.\lowercase{\def~}{\discretionary{\hbox{\char`\.}}{\Wrappedafterbreak}{\hbox{\char`\.}}}%
            \lccode`\~`\,\lowercase{\def~}{\discretionary{\hbox{\char`\,}}{\Wrappedafterbreak}{\hbox{\char`\,}}}%
            \lccode`\~`\;\lowercase{\def~}{\discretionary{\hbox{\char`\;}}{\Wrappedafterbreak}{\hbox{\char`\;}}}%
            \lccode`\~`\:\lowercase{\def~}{\discretionary{\hbox{\char`\:}}{\Wrappedafterbreak}{\hbox{\char`\:}}}%
            \lccode`\~`\?\lowercase{\def~}{\discretionary{\hbox{\char`\?}}{\Wrappedafterbreak}{\hbox{\char`\?}}}%
            \lccode`\~`\!\lowercase{\def~}{\discretionary{\hbox{\char`\!}}{\Wrappedafterbreak}{\hbox{\char`\!}}}%
            \lccode`\~`\/\lowercase{\def~}{\discretionary{\hbox{\char`\/}}{\Wrappedafterbreak}{\hbox{\char`\/}}}%
            \catcode`\.\active
            \catcode`\,\active
            \catcode`\;\active
            \catcode`\:\active
            \catcode`\?\active
            \catcode`\!\active
            \catcode`\/\active
            \lccode`\~`\~
        }
    \makeatother

    \let\OriginalVerbatim=\Verbatim
    \makeatletter
    \renewcommand{\Verbatim}[1][1]{%
        %\parskip\z@skip
        \sbox\Wrappedcontinuationbox {\Wrappedcontinuationsymbol}%
        \sbox\Wrappedvisiblespacebox {\FV@SetupFont\Wrappedvisiblespace}%
        \def\FancyVerbFormatLine ##1{\hsize\linewidth
            \vtop{\raggedright\hyphenpenalty\z@\exhyphenpenalty\z@
                \doublehyphendemerits\z@\finalhyphendemerits\z@
                \strut ##1\strut}%
        }%
        % If the linebreak is at a space, the latter will be displayed as visible
        % space at end of first line, and a continuation symbol starts next line.
        % Stretch/shrink are however usually zero for typewriter font.
        \def\FV@Space {%
            \nobreak\hskip\z@ plus\fontdimen3\font minus\fontdimen4\font
            \discretionary{\copy\Wrappedvisiblespacebox}{\Wrappedafterbreak}
            {\kern\fontdimen2\font}%
        }%

        % Allow breaks at special characters using \PYG... macros.
        \Wrappedbreaksatspecials
        % Breaks at punctuation characters . , ; ? ! and / need catcode=\active
        \OriginalVerbatim[#1,codes*=\Wrappedbreaksatpunct]%
    }
    \makeatother

    % Exact colors from NB
    \definecolor{incolor}{HTML}{303F9F}
    \definecolor{outcolor}{HTML}{D84315}
    \definecolor{cellborder}{HTML}{CFCFCF}
    \definecolor{cellbackground}{HTML}{F7F7F7}

    % prompt
    \makeatletter
    \newcommand{\boxspacing}{\kern\kvtcb@left@rule\kern\kvtcb@boxsep}
    \makeatother
    \newcommand{\prompt}[4]{
        {\ttfamily\llap{{\color{#2}[#3]:\hspace{3pt}#4}}\vspace{-\baselineskip}}
    }
    

    
    % Prevent overflowing lines due to hard-to-break entities
    \sloppy
    % Setup hyperref package
    \hypersetup{
      breaklinks=true,  % so long urls are correctly broken across lines
      colorlinks=true,
      urlcolor=urlcolor,
      linkcolor=linkcolor,
      citecolor=citecolor,
      }
    % Slightly bigger margins than the latex defaults
    
    \geometry{verbose,tmargin=1in,bmargin=1in,lmargin=1in,rmargin=1in}
    
    

\begin{document}
    
    \maketitle
    
    

    
    \hypertarget{physics-210-project-1-what-is-the-optimal-angle-and-initial-velocity-for-the-spaceship-to-escape-from-the-black-hole}{%
\section{Physics 210 Project 1: ``What is the optimal angle and initial
velocity for the spaceship to escape from the black
hole?''}\label{physics-210-project-1-what-is-the-optimal-angle-and-initial-velocity-for-the-spaceship-to-escape-from-the-black-hole}}

    \textbf{Name:} Mufaro Joseph Machaya

    \hypertarget{introduction}{%
\subsection{Introduction}\label{introduction}}

    \textbf{Context:} Given a three-body system containing a black hole
(with a mass \(M\)), a small planet made of JELL-O-like slime (with mass
\(m_p\)), and a spaceship (with mass \(m_s\)) accidentally headed
directly for the black hole, a potential way for the spaceship to avoid
being sucked into the black hole is to turn into and collide
(inelastically) with the planet such that it is successfully rebounded
with a velocity high enough to escape the gravitational pull of the
black hole. However, if it hits the planet with too much impulse during
the collision, it'll be destroyed by the planet (it's made out of
JELL-O, so it has surface tension).

\textbf{Research Question:} \emph{What is the optimal angle and initial
velocity for the space ship to escape from the black hole}? (Optimizing
for a combination of the ending distance from the black hole, the final
velocity, alongside the amount of fuel used at the end of the simulation
-- assuming that the ship isn't destroyed!)

    \begin{figure}
\centering
\includegraphics{image.png}
\caption{image.png}
\end{figure}

    \textbf{Figure 1:} A basic rendition of the \(3\)-body system
encompassed by this project. The general idea of this project is
modelling the possible collisions between the space ship and slime
planet given the presence of a gravitational field due to a fixed black
hole.

    \hypertarget{physics-model-and-assumptions}{%
\subsubsection{Physics Model and
Assumptions}\label{physics-model-and-assumptions}}

    \textbf{Model (``Additional Physics'')}

\begin{enumerate}
\def\labelenumi{\arabic{enumi}.}
\tightlist
\item
  3-Body System, Gravitational Vector Field
\end{enumerate}

Each of the three objects are modeled individually as point-masses. Each
object produces gravitational effects on all other objects by formula

\[{\bf F}_g = G \frac{m_1 m_2}{|{\bf r}|^2} \hat{r}\]

where \({\bf F}_g\) is the force of gravity,
\(G = 6.67 \times 10^{-11}\) \(m^3 kg^{-1} s^{-2}\) is the gravitational
constant, \(m_1\) and \(m_2\) are the masses of the objects, and
\({\bf r}\) is the displacement vector between them. From there, basic
kinematic equation of Newton's second law applies:

\[\frac{d^2 \bf{x}}{dt^2} = \frac{\sum \bf F}{m},\]

where \(\frac{d^2 \bf{x}}{dt^2}\) is the acceleration for the object of
mass \(m\) given the calculated net-force \({\sum \bf F}\). The system
is modeled in only two dimensions rather than three (but all equations
apply for each dimension)

    \begin{enumerate}
\def\labelenumi{\arabic{enumi}.}
\setcounter{enumi}{1}
\tightlist
\item
  Spaceship Turning
\end{enumerate}

As the ship is on a course for being sucked into the black hole, it
would be best for the crew members to turn the ship out of the direction
of the black hole. This turn may cause it to turn into the slime planet
or completely away from the black hole. In either case, this will
require the ship to apply some thrust to perform the turn, and this
thrust will require a certain amount of energy. Let's assume that the
ship starts with a certain amount of fuel on board (but it will require
fuel to accelerate/deccelerate during its journey).As the ship is on a
course for being sucked into the black hole, it would be best for the
crew members to turn the ship out of the direction of the black hole.
This turn may cause it to turn into the slime planet or completely away
from the black hole. In either case, this will require the ship to apply
some thrust to perform the turn, and this thrust will require a certain
amount of energy. Let's assume that the ship starts with a certain
amount of fuel on board (but it will require fuel to
accelerate/deccelerate during its journey).

The ship turns with thrust

\[{\bf T} = -\frac{dm}{dt} v_{ex} \hat{T},\]

where \(\mathbf{T}\) is the thrust force vector, \(-\frac{dm}{dt}\) is
the rate of mass consumption, \(v_{ex}\) is the exhaust speed, and
\(\hat{T}\) is the turning direction (which is calculated prior based on
the angle the spaceship is intended to turn at), after starting with an
amount of fuel \(m_0\). The ship will stop turning once the ship is in
the direction the turn intended to produce, i.e., if the ship begins
with a direction \(\theta\) and it seeks to turn by \(\Delta \theta\),
all turning will stop once the ship reaches \(\theta + \Delta \theta\),
and the ship will not resume turning even if it naturally turns off of
\(\theta + \Delta \theta\) afterward (due to gravity or otherwise). This
is determined by checking if the direction of velocity is equal to the
direction of thrust \(\hat{T}\):

\[\hat{T} = \mathbf{R}(\theta)\hat{v}_0 = \begin{pmatrix} \cos \theta & -\sin \theta \\ \sin \theta & \cos \theta \end{pmatrix} \hat{v}_0,\]

where \(\mathbf{R}(\theta)\) is the standard counterclockwise
two-dimensional rotation matrix by \(\theta\), as defined above, and
\(\hat{v}_0\) is the initial direction of the space ship.

At each point, the ship is designated to produce a constant magnitude of
thrust \(T_0\) at all times (as long as it can sustain the loss of fuel
mass, and in the singular frame where it will be unable to produce that
amount of thrust, then it will produce as much thrust as it can before
running out of fuel), so the amount of thrust produced adds to the total
amount of force that the ship can experience at all times before
breaking.

    \begin{enumerate}
\def\labelenumi{\arabic{enumi}.}
\setcounter{enumi}{2}
\tightlist
\item
  Inelastic Jello Collision
\end{enumerate}

The force from interacting with a gel-like substance will be modeled
using a Hunt-Crossley algorithm
\href{https://doi.org/10.1115/1.3423596}{{[}3{]}} with optional adhesion
and a plastic offset, i.e.,

\begin{itemize}
\tightlist
\item
  Contact when the overlap distance \(\delta > 0\), and the force for
  each frame is
  \[F_N = k \delta^n + \alpha \delta^n \frac{d \delta}{dt}.\]
\end{itemize}

    \textbf{Understanding the Hunt-Crossley Model}

The inclusion of the Hunt-Crossley nonlinear contact model is extremely
pivotal to the mechanics of this simulation, so it's pivotal to
understand both what this calculation is doing and why this model was
utilized.

3.1. How the Model Works

Put simply, the Hunt-Crossley contact model is a model for simulating
the contact between a rigid object and a soft object, and it assumes
that the rigid object is able to penetrate into the soft object by an
overlap distance \(\delta\), but as it goes further into the object, the
normal force \(F_N\) opposing the rigid object's penetration increases.
In particular, this model increases this normal force in two ways.
First, there is a elastic term \(k \delta^n\), and this is essentially
the primary driver of the force based on the stiffness \(k\) of the
object (which is related to the density of the material). The next term
\(\alpha \delta^n \frac{d \delta}{dt}\) is the viscous term that
accounts for the energy dissipation due to damping based on the damping
constant \(\alpha\) of the material and the speed of penetration
\(\frac{d \delta}{dt}\). Lastly, this model utilizes a non-linear
exponent that is, in real-world applications of the model, typically
fitted between \(3/2\) to \(2\) to best match the scenario, but in this
case, it essentially acts as a constant to describe the geometry of the
objects that are overlapping. As both of the objects in this scenario
are spherical, it is set to \(n=3/2\).

3.2. Why this Model was Chosen

This model was chosen largely because of its applicability and accuracy
when applied to real-world models. It's very efficient computationally
and allows for a much more accurate simulation as compared to more
simplified models (like linear, non-damping models) as the additional
parameters allows for more characterization of the simulation to be
directly encoded into the calculation (such as allowing an additional
damping parameter to further characterize the softbody and adding a
geometric exponent to characterize the type of geometry that is
colliding).

    \begin{enumerate}
\def\labelenumi{\arabic{enumi}.}
\setcounter{enumi}{3}
\tightlist
\item
  Ship Destruction
\end{enumerate}

The ship is designated to be destroyed upon experiencing, in any given
frame, a force greater than \(F_{max}\) Newtons, and as the magnitude of
force experienced during the collision is a function of the velocity of
the object and the magnitude of force experienced due to gravity is a
function of the radial distance between the object and other objects in
the simulation, there's an implicit cap on the minimum and maximum
velocities at which the object can safely travel (based on various other
conditions): - If the ship manages to collide with the slime planet at
too high of an initial velocity, it will certainly be destroyed - If the
velocity is too low, the ship will be sucked into the gravitational pull
of the black hole (and the collision between the objects isn't modeled
in this simulation, but it can be implied based on the final radial
distance between the ship and the black hole after the standard
simulation time of 10 seconds).

    \hypertarget{optimization}{%
\subsubsection{Optimization}\label{optimization}}

\emph{Therefore, this model acts as an optimization problem:} We're
attempting to see what combination \((\theta,v_0)\) will produce a
maximize the amount of fuel left in the ship, the radial distance of the
ship from the black hole, and the magnitude of final velocity of the
ship at the end of the standard simulation time, all without being
destroyed in the process.

    \hypertarget{a-broader-understanding-of-the-optimization}{%
\paragraph{A Broader Understanding of the
Optimization}\label{a-broader-understanding-of-the-optimization}}

Before explaining the exact details for how this optimization is
performed, it's key to understand how/why it's integrated into this
project in a very broad sense. To begin, consider the function
\texttt{simulate} defined below as some mathematical function
\(\mathbf{s}(v_0,\theta)\) which performs the simulation based on the
parameters of the initial speed \(v_0\) and the turning angle
\(\theta\).

Like any simulation, what happens is an initial state is built based on
these parameters (and other preset constant values we've defined before,
like the positions of the objects), and the simulation is time-evolving
this state until our desired final time, and the function then returns
three important values: 1. The final speed of the ship, \(v_0\), 2. The
final remaining mass of fuel in the ship, \(m_f\), and 3. The final
radial distance of the ship to the black hole, \(r_f\).

So therefore, \(\mathbf{s}\) is vector valued, and is defined as some
mapping like
\[\mathbf{s}(v_0, \theta) \mapsto \langle m_f, r_f, v_f \rangle.\]

Now, remember that the overall goal of this simulation is to find
parameter pairs \((v_0, \theta)\) within our domain of initial speeds
and angles that are optimal for the crew members on board, and that
means that we would want to maximize all three values: 1. We want to
maximize the final velocity \(v_f\) as it allows them to get away from
the black hole faster. 2. We want to maximize the remaining fuel \(m_f\)
as they are real astronauts, and may need fuel for other things later,
like dodging asteroids or launching escape pods or whatever else. 3. We
want to maximize the distance from the black hole \(r_f\) because the
astronauts want to avoid being sucked into the black hole at all costs.

Therefore, to do this, we can define as optimization function
\(U(\mathbf{s}) = U(m_f, r_f, v_f)\) which has the parameters of the
final fuel mass, distance, and speed from the simulation function
\(\mathbf{s}\) and produces some unitless scalar \(U\) that denotes how
favorable the simulation turned out in comparison to others, and we can
call this utility value the ``success'' score (or some similar term).
So, then, we get this pipeline:
\[(v_0,\theta) \to \mathbf{s}(v_0, \theta) \mapsto \langle m_f, r_f, v_f \rangle \to U(\mathbf{s}) \mapsto U\]
which, in the grand scheme of this experiment, simply means that using
this optimization function \(U\), we can associate each combination
\((v_0,\theta)\) with a ``success'' score \(U\) with the optimization
function and our simulation combined (\(U \circ \mathbf{s}\)),
\[(v_0,\theta) \mapsto U\] and this is how we can understand the
experiment.

The one caveat with this strategy, however, is what to do with
``failed'' states, where the ship is destroyed and the simulation ends
early. My solution was to just defined the output of a failed state as
\(\mathbf{s}_f = \vec{0}\), and from that, its utility score should be
defined trivially as zero.

    \hypertarget{exact-definition-of-the-optimization-function-u}{%
\paragraph{\texorpdfstring{Exact Definition of the Optimization
Function,
\(U\)}{Exact Definition of the Optimization Function, U}}\label{exact-definition-of-the-optimization-function-u}}

To keep things simple, the optimization method followed is simple linear
scalarization of each of the parameters
\href{https://en.wikipedia.org/wiki/Multi-objective_optimization\#Scalarizing}{{[}4{]}}.
Each component of the output vector is first normalized based on the
logarithmic range produced across all simulations for that output (a
log-scale is used due to the immense differences in magnitudes between
each of the values), so for a given value \(x\), it would be normalized
via a simple min-max expectation to
\[x' = \text{clamp}\left(0, \frac{\ln x - \ln x_{min}}{\ln x_{max} - \ln x_{min}}, 1\right),\]
then the resultant vector of normalized values is transformed into a
scalar by taking the weighted sum of each of the components using a
utility function \[U(\mathbf{s}) = w_1 m'_f + w_2 r' + w_3 v'_f,\] and
each of these weights will be determined based on the relative
importance of each of the variables (and these will be discussed later).
Due to using a log scale, of course, any zero values (including the
failed state \(\mathbf{0}\)) have to be only approximately zero, so
there is an error bound added to all values of
\(\varepsilon = 10^{-12}\), which should not make noticable differences
to the calculations.

    \hypertarget{determining-the-weights}{%
\paragraph{Determining the Weights}\label{determining-the-weights}}

Let the weights (or biases) be \(w_1, w_2,\) and \(w_3\) corresponding
to remaining fuel, radial distance, and speed respectively. These three
final variables are not realistically of equal importance, so these
weights shouldn't be equal. For this simulation, we assume that they
rank in importance in the following order (from most important to least
important: 1. Radial Distance 2. Final Speed 3. Remaining Fuel

This order is picked solely based on intuition. We consider the radial
distance to be the most important because the immediate goal of these
astronauts is to avoid the black hole at all costs (and nearly all of
their efforts through the simulation is to avoid being sucked into the
black hole). This is also why the final speed comes in second in
importance, as faster exit speeds are more likely to escape the event
horizon of the black hole. Then, the remaining fuel on board is
considered last as it's important that they maintain some fuel for
whatever endeavors they may need fuel for later, but it's not absolutely
critical to their survival now.

Therefore, our first criteria for the weights to satisfy is
\(w_2 > w_3 > w_1\). Next, we want there to be a hierarchical order, but
we don't want any particular weight to be too dominant, i.e., the
differences in the weights shouldn't be relatively large compared to the
weights themselves. The last criteria is that these weights should sum
to \(1\) (for the sake of clearly expressing that they are weighting
values for producing a utility. It doesn't effect the results for the
weights to sum to one, but it helps with being clear about their
relationship.) With these together, we can more or less pick any set of
weights that seems reasonable and satisfies these two qualities, so I
went with \(w_1=2/10, w_2=5/10,\) and \(w_3=3/10\).

    \hypertarget{assumptions}{%
\subsubsection{Assumptions}\label{assumptions}}

\begin{enumerate}
\def\labelenumi{\arabic{enumi}.}
\tightlist
\item
  \emph{The ship's destruction is immediate and summary.} We assume that
  at the exact moment at which the ship experiences a force above its
  maximum threshold \(F_i > F_{max}\), the rest of the information that
  would realistically occur would not be of use or interest.

  \begin{itemize}
  \tightlist
  \item
    \textbf{Justification:} In a realistic scenario, we can assume that
    the crew members on board, for instance, have some realistic
    limitation as to why any effects of the ship being destroyed in any
    fashion are entirely out-of-the-question. Imagine the destruction of
    the ship in any capacity could be far too damaging to very vital
    on-board components. The important data to be gained is for
    scenarios in which the ship remains in-tact, and the crew members
    really want to optimize for their best possible options under these
    circumstances.
  \end{itemize}
\item
  The slime planet is perfectly spherical.
\item
  The ship is able to produce a perfectly constant thrust force so long
  as it has enough fuel mass to do so.

  \begin{itemize}
  \tightlist
  \item
    \textbf{Justifications (2 and 3):} These are simply to make the
    calculations particularly easy, and we can assume that the effects
    without these assumptions are negligible. A true engine would have a
    limit to the amount of thrust it can produce, and a realistic analog
    for this situation is that these engines are producing their maximum
    thrust force and rapidly burning through their supply of fuel for
    the duration of the turn, so it can become effectively constant due
    to natural limitations by the engine (even despite other
    considerations).
  \end{itemize}
\end{enumerate}

    \begin{tcolorbox}[breakable, size=fbox, boxrule=1pt, pad at break*=1mm,colback=cellbackground, colframe=cellborder]
\prompt{In}{incolor}{1}{\boxspacing}
\begin{Verbatim}[commandchars=\\\{\}]
\PY{k+kn}{import} \PY{n+nn}{numpy} \PY{k}{as} \PY{n+nn}{np}
\PY{k+kn}{from} \PY{n+nn}{matplotlib} \PY{k+kn}{import} \PY{n}{pyplot} \PY{k}{as} \PY{n}{plt}
\PY{k+kn}{from} \PY{n+nn}{matplotlib}\PY{n+nn}{.}\PY{n+nn}{animation} \PY{k+kn}{import} \PY{n}{FuncAnimation}
\PY{k+kn}{import} \PY{n+nn}{matplotlib}\PY{n+nn}{.}\PY{n+nn}{patches} \PY{k}{as} \PY{n+nn}{patches}
\PY{k+kn}{from} \PY{n+nn}{dataclasses} \PY{k+kn}{import} \PY{n}{dataclass}
\PY{k+kn}{from} \PY{n+nn}{copy} \PY{k+kn}{import} \PY{n}{deepcopy}
\PY{k+kn}{import} \PY{n+nn}{io}
\PY{k+kn}{from} \PY{n+nn}{PIL} \PY{k+kn}{import} \PY{n}{Image}
\PY{k+kn}{from} \PY{n+nn}{IPython}\PY{n+nn}{.}\PY{n+nn}{display} \PY{k+kn}{import} \PY{n}{HTML}
\PY{k+kn}{import} \PY{n+nn}{tempfile}
\PY{k+kn}{import} \PY{n+nn}{time}
\end{Verbatim}
\end{tcolorbox}

    \textbf{Code Block Summary} This is just a simple functions for
calculating a unit vector, by
\[\hat{\bf u} = \frac{\vec{\bf u}}{|\vec{\bf u}|}\] for getting all of
the magnitudes for a list of vectors, like
\[\text{mag}(\{\mathbf{v}_1, \mathbf{v}_2, \dots, \mathbf{v}_n \}) = \{ |\mathbf{v}_1|, |\mathbf{v}_2|, \dots, |\mathbf{v}_n| \},\]
for rotating a vector by
\[\mathbf{R}(\mathbf{v},\theta) = \begin{pmatrix} \cos \theta & -\sin \theta \\ \sin \theta & \cos \theta \end{pmatrix} \begin{pmatrix} v_x \\ v_y \end{pmatrix},\]
for getting all of the unit vectors from a list of vectors, like
\[\text{units}(\{\mathbf{v}_1, \mathbf{v}_2, \dots, \mathbf{v}_n \}) = \{ \hat{v}_1, \hat{v}_2, \dots, \hat{v}_n \} = \left\{\frac{\mathbf{v}_1}{|\mathbf{v}_1|}, \frac{\mathbf{v}_2}{|\mathbf{v}_2|}, \dots, \frac{\mathbf{v}_n}{|\mathbf{v}_n|} \right\},\]
and for performing a clamp/clip, such that given a range
\([x_{\min}, x_{\max}]\) and a value \(x\),
\[\text{clamp}(x, x_{\max}, x_{\min}) = 
\begin{cases} 
    x & \text{ if } x \in [x_{\min}, x_{\max}] \\
    x_{\min} & \text{ if } x < x_{\min} \\
    x_{\max} & \text{ if } x > x_{\max}
\end{cases}.\]

    \begin{tcolorbox}[breakable, size=fbox, boxrule=1pt, pad at break*=1mm,colback=cellbackground, colframe=cellborder]
\prompt{In}{incolor}{2}{\boxspacing}
\begin{Verbatim}[commandchars=\\\{\}]
\PY{k}{def} \PY{n+nf}{unit\PYZus{}vector}\PY{p}{(}\PY{n}{v}\PY{p}{)}\PY{p}{:}
    \PY{k}{return} \PY{n}{np}\PY{o}{.}\PY{n}{array}\PY{p}{(}\PY{n}{v}\PY{p}{)} \PY{o}{/} \PY{n}{np}\PY{o}{.}\PY{n}{linalg}\PY{o}{.}\PY{n}{norm}\PY{p}{(}\PY{n}{v}\PY{p}{)}

\PY{k}{def} \PY{n+nf}{magnitudes}\PY{p}{(}\PY{n}{vector\PYZus{}list}\PY{p}{)}\PY{p}{:}
    \PY{k}{return} \PY{p}{[}\PY{n}{np}\PY{o}{.}\PY{n}{linalg}\PY{o}{.}\PY{n}{norm}\PY{p}{(}\PY{n}{vec}\PY{p}{)} \PY{k}{for} \PY{n}{vec} \PY{o+ow}{in} \PY{n}{vector\PYZus{}list}\PY{p}{]}

\PY{k}{def} \PY{n+nf}{rotate\PYZus{}vector\PYZus{}2d}\PY{p}{(}\PY{n}{vec}\PY{p}{,} \PY{n}{theta}\PY{p}{)}\PY{p}{:}
    \PY{n}{rotation\PYZus{}matrix} \PY{o}{=} \PYZbs{}
        \PY{n}{np}\PY{o}{.}\PY{n}{array}\PY{p}{(}\PY{p}{[}\PY{p}{[}\PY{n}{np}\PY{o}{.}\PY{n}{cos}\PY{p}{(}\PY{n}{theta}\PY{p}{)}\PY{p}{,} \PY{o}{\PYZhy{}}\PY{n}{np}\PY{o}{.}\PY{n}{sin}\PY{p}{(}\PY{n}{theta}\PY{p}{)}\PY{p}{]}\PY{p}{,}
                  \PY{p}{[}\PY{n}{np}\PY{o}{.}\PY{n}{sin}\PY{p}{(}\PY{n}{theta}\PY{p}{)}\PY{p}{,}  \PY{n}{np}\PY{o}{.}\PY{n}{cos}\PY{p}{(}\PY{n}{theta}\PY{p}{)}\PY{p}{]}\PY{p}{]}\PY{p}{)}
    \PY{k}{return} \PY{n}{rotation\PYZus{}matrix} \PY{o}{@} \PY{n}{vec}

\PY{k}{def} \PY{n+nf}{unit\PYZus{}vectors}\PY{p}{(}\PY{n}{s}\PY{p}{)}\PY{p}{:}
    \PY{k}{return} \PY{n}{np}\PY{o}{.}\PY{n}{array}\PY{p}{(}\PY{p}{[}\PY{n}{unit\PYZus{}vector}\PY{p}{(}\PY{n}{v}\PY{p}{)} \PY{k}{for} \PY{n}{v} \PY{o+ow}{in} \PY{n}{s}\PY{p}{]}\PY{p}{)}

\PY{k}{def} \PY{n+nf}{clamp}\PY{p}{(}\PY{n}{x}\PY{p}{,} \PY{n}{xmin}\PY{p}{,} \PY{n}{xmax}\PY{p}{)}\PY{p}{:} 
    \PY{k}{return} \PY{n+nb}{max}\PY{p}{(}\PY{n}{xmin}\PY{p}{,} \PY{n+nb}{min}\PY{p}{(}\PY{n}{x}\PY{p}{,} \PY{n}{xmax}\PY{p}{)}\PY{p}{)}
\end{Verbatim}
\end{tcolorbox}

    \hypertarget{simulation-parameters-invariant}{%
\subsection{Simulation Parameters
(Invariant)}\label{simulation-parameters-invariant}}

    \hypertarget{reference-values}{%
\subsubsection{Reference Values}\label{reference-values}}

    The following are the values that would be ideal to use (alongside their
justifications), and these aren't the values actually used in the
simulation, but rather, these are the values that the values used are
based off of due to computational limitations. These literature values
act as a guide, and then from there, I truncated and altered the values
to make it work reasonably.

    \textbf{Slime Planet:}

    \emph{Reference:} I'm assuming that the slime planet is made perfectly
of regular gelatin-like slime (similar to JELL-O), so starting with the
assumption of a density of \(1.14\) \(g/cm^3\) or \(1140\) \(kg/m^3\)
\href{https://www.aqua-calc.com/page/density-table/substance/gelatin-blank-desserts-coma-and-blank-dry-blank-mix-coma-and-blank-prepared-blank-with-blank-water}{{[}6{]}}
alongside a planetary radius equal to two-times that of Mars at 6.8 km.

    \begin{tcolorbox}[breakable, size=fbox, boxrule=1pt, pad at break*=1mm,colback=cellbackground, colframe=cellborder]
\prompt{In}{incolor}{3}{\boxspacing}
\begin{Verbatim}[commandchars=\\\{\}]
\PY{n}{SLIME\PYZus{}PLANET\PYZus{}RADIUS} \PY{o}{=} \PY{l+m+mf}{6.8e3}
\PY{n}{SLIME\PYZus{}DENSITY} \PY{o}{=} \PY{l+m+mi}{1140}
\PY{n}{SLIME\PYZus{}PLANET\PYZus{}VOLUME}\PY{o}{=} \PY{p}{(}\PY{l+m+mi}{4}\PY{o}{/}\PY{l+m+mi}{3}\PY{p}{)} \PY{o}{*} \PY{n}{np}\PY{o}{.}\PY{n}{pi} \PY{o}{*} \PY{n}{SLIME\PYZus{}PLANET\PYZus{}RADIUS} \PY{o}{*}\PY{o}{*} \PY{l+m+mi}{3}
\PY{n}{SLIME\PYZus{}PLANET\PYZus{}MASS}\PY{o}{=}\PY{n}{SLIME\PYZus{}DENSITY} \PY{o}{*} \PY{n}{SLIME\PYZus{}PLANET\PYZus{}VOLUME}
\end{Verbatim}
\end{tcolorbox}

    \emph{Actual:} For this simulation, I'm scaling down all of the units
into a micro-scale, so kilometers can reasonably just become meters.
Additionally, the density is going to remain the same despite the scale
being reduced as the space ship is a point-mass, so there is no
equivalent downscaling in the size of the space ship to compensate for
the reduction in the size of the slime planet. This means that it would
penetrate the same distance into the slime planet as before without
increasing the density, so without making the density quite high, it
could completely pass through the planet without much impulse (which
would make for a very boring simulation, as the ship would almost never
be destroyed).

    \begin{tcolorbox}[breakable, size=fbox, boxrule=1pt, pad at break*=1mm,colback=cellbackground, colframe=cellborder]
\prompt{In}{incolor}{4}{\boxspacing}
\begin{Verbatim}[commandchars=\\\{\}]
\PY{n}{SLIME\PYZus{}PLANET\PYZus{}RADIUS} \PY{o}{=} \PY{l+m+mf}{6.8}
\PY{n}{SLIME\PYZus{}DENSITY} \PY{o}{=} \PY{l+m+mi}{1140}
\PY{n}{SLIME\PYZus{}PLANET\PYZus{}VOLUME}\PY{o}{=} \PY{p}{(}\PY{l+m+mi}{4}\PY{o}{/}\PY{l+m+mi}{3}\PY{p}{)} \PY{o}{*} \PY{n}{np}\PY{o}{.}\PY{n}{pi} \PY{o}{*} \PY{n}{SLIME\PYZus{}PLANET\PYZus{}RADIUS} \PY{o}{*}\PY{o}{*} \PY{l+m+mi}{3}
\PY{n}{SLIME\PYZus{}PLANET\PYZus{}MASS}\PY{o}{=}\PY{n}{SLIME\PYZus{}DENSITY} \PY{o}{*} \PY{n}{SLIME\PYZus{}PLANET\PYZus{}VOLUME}
\end{Verbatim}
\end{tcolorbox}

    For the collision parameters of slime, there aren't clear literature
values. The stiffness and damping coefficients are unintuitive terms,
and I'll admit that it's hard to find any values that make sense here.
The only literature value is the growth coefficient \(n=3/2\) for the
spherical geometry of the slime planet. Otherwise, I just gave a guess,
assuming that the stiffness must be related to the density.

    \begin{tcolorbox}[breakable, size=fbox, boxrule=1pt, pad at break*=1mm,colback=cellbackground, colframe=cellborder]
\prompt{In}{incolor}{5}{\boxspacing}
\begin{Verbatim}[commandchars=\\\{\}]
\PY{n}{SLIME\PYZus{}STIFFNESS} \PY{o}{=} \PY{n}{SLIME\PYZus{}DENSITY} \PY{o}{*} \PY{l+m+mi}{100}
\PY{n}{SPHERICAL\PYZus{}GROWTH\PYZus{}EXPONENT} \PY{o}{=} \PY{l+m+mi}{3}\PY{o}{/}\PY{l+m+mi}{2}
\PY{n}{SLIME\PYZus{}DAMPING\PYZus{}COEFFICIENT} \PY{o}{=} \PY{n}{SLIME\PYZus{}DENSITY} \PY{o}{*} \PY{l+m+mi}{40}
\end{Verbatim}
\end{tcolorbox}

    \emph{Reference:} Once again, the black hole is situated at (0,0) for
the entirety of the simulation. With respect to this origin, the slime
planet will be located at the same \(y\)-value of 0 and at a radial
orbiting distance similar to something in our solar system as a
reference. In this case, I decided to go with something at roughly
one-fourth the distance with which Mercury is orbiting the Sun, or
\(14.5 \times 10^9\) meters
\href{https://en.wikipedia.org/wiki/Mercury_(planet)}{{[}7{]}}.

For this simulation, the planet is assumed to be not orbiting the black
hole, and the orbital/gravitational relationship with the black hole and
the slime planet does not need to be stable in any way whatsoever (as it
makes no difference to the astronauts), so I'll just arbitrarily pick a
speed of the planet as the orbital speed of Mercury of about 47 km/s
\href{https://science.nasa.gov/mercury/facts}{{[}8{]}}.

    \begin{tcolorbox}[breakable, size=fbox, boxrule=1pt, pad at break*=1mm,colback=cellbackground, colframe=cellborder]
\prompt{In}{incolor}{6}{\boxspacing}
\begin{Verbatim}[commandchars=\\\{\}]
\PY{n}{SLIME\PYZus{}PLANET\PYZus{}INITIAL\PYZus{}POSITION}\PY{o}{=}\PY{n}{np}\PY{o}{.}\PY{n}{array}\PY{p}{(}\PY{p}{[}\PY{l+m+mf}{58e9}\PY{p}{,}\PY{l+m+mi}{0}\PY{p}{]}\PY{p}{)}
\PY{n}{SLIME\PYZus{}PLANET\PYZus{}INITIAL\PYZus{}VELOCITY}\PY{o}{=}\PY{n}{np}\PY{o}{.}\PY{n}{array}\PY{p}{(}\PY{p}{[}\PY{l+m+mi}{0}\PY{p}{,}\PY{l+m+mf}{47e3}\PY{p}{]}\PY{p}{)}
\end{Verbatim}
\end{tcolorbox}

    \emph{Actual:} All of the objects in this simulation will be made
unrealistically close to the black hole for the sake of ease in the
simulation, and this is justifiable as the collision with the black hole
is not a consideration in the calculations as it's an end-state in
itself (i.e., it's actively minimized against). This approximation
allows me to specify the slime planet and space ship as being
unrealistically close with a focus purely on their initial positions
with relation to each other, and the inherent increase in the
gravitational force from making the black hole pointlike and close to
the objects necessarily requires a significant decrease in the mass of
the black hole.

    \begin{tcolorbox}[breakable, size=fbox, boxrule=1pt, pad at break*=1mm,colback=cellbackground, colframe=cellborder]
\prompt{In}{incolor}{7}{\boxspacing}
\begin{Verbatim}[commandchars=\\\{\}]
\PY{n}{SLIME\PYZus{}PLANET\PYZus{}INITIAL\PYZus{}POSITION}\PY{o}{=}\PY{n}{np}\PY{o}{.}\PY{n}{array}\PY{p}{(}\PY{p}{[}\PY{l+m+mi}{20}\PY{p}{,}\PY{l+m+mi}{0}\PY{p}{]}\PY{p}{)}
\PY{n}{SLIME\PYZus{}PLANET\PYZus{}INITIAL\PYZus{}VELOCITY}\PY{o}{=}\PY{n}{np}\PY{o}{.}\PY{n}{array}\PY{p}{(}\PY{p}{[}\PY{l+m+mi}{0}\PY{p}{,}\PY{o}{\PYZhy{}}\PY{l+m+mi}{3}\PY{p}{]}\PY{p}{)}
\end{Verbatim}
\end{tcolorbox}

    \textbf{Space Ship}

    \emph{Reference:} I'm modelling the spaceship as an Apollo-era Space
Shuttle, with mass \(2.03 \times 10^{6}\) kg
\href{https://en.wikipedia.org/wiki/Space_Shuttle}{{[}9{]}}. I'm using
the engine specifications of the first stage (just the external tank
without the boosters): a specific impulse of 4.46 km/s, a maximum thrust
of 1750 kN, containing a total of 1.6 million pounds (or roughly 700,000
kilograms) of total fuel
\href{https://www.nasa.gov/reference/the-space-shuttle/}{{[}10{]}}.
However, to keep things interesting, I'm going to assume they've already
used 6/7 of this amount, and that there is only 1/7 of the total initial
fuel remaining on board for 100,000 kilograms.

Lastly, it's not extremely clear as to what the maximum force the space
shuttle could take at one point is (because it's not a realistic
question to ask, in a real scenario it would be a maximal impulse over
some range of time), so I'll make the following (understandably insane)
assumption: the maximum payload for the ship to turn to each is 14,400
kg
\href{https://en.wikipedia.org/wiki/Space_Shuttle\#cite_note-woodcock-5}{{[}11{]}},
and then this can be translated into force based on \(F=mg_0\) to
roughly \(1\)

    \begin{tcolorbox}[breakable, size=fbox, boxrule=1pt, pad at break*=1mm,colback=cellbackground, colframe=cellborder]
\prompt{In}{incolor}{8}{\boxspacing}
\begin{Verbatim}[commandchars=\\\{\}]
\PY{n}{SPACE\PYZus{}SHIP\PYZus{}MASS}\PY{o}{=}\PY{l+m+mf}{2.03e6}
\PY{n}{SPACE\PYZus{}SHIP\PYZus{}INITIAL\PYZus{}FUEL\PYZus{}MASS}\PY{o}{=}\PY{l+m+mf}{1e6}
\PY{n}{SPACE\PYZus{}SHIP\PYZus{}REQUIRED\PYZus{}THRUST}\PY{o}{=}\PY{l+m+mf}{1750e3}
\PY{n}{SPACE\PYZus{}SHIP\PYZus{}SPECIFIC\PYZus{}IMPULSE}\PY{o}{=}\PY{l+m+mf}{4.46e3}
\PY{n}{SPACE\PYZus{}SHIP\PYZus{}MAXIMUM\PYZus{}IMPULSE}\PY{o}{=}\PY{l+m+mf}{2e10}
\end{Verbatim}
\end{tcolorbox}

    \emph{Actual:} The mass of the ship was just cut down by \(10^2\) (from
2 million kilograms to \(2 \times 10^4\) kilograms), the mass was cut by
the same scaling, and the specific impulse and thrust were reduced by a
factor of \(10^3\). The maximum, pre-destructive impulse was reduced by
\(10^5\) to compensate. I then just cut everything else down by a factor
of \(20\). I also gave them enough fuel to perform any turn that they
would like within the time scale, but we're still assuming that
overusing fuel is a bad idea for the astronauts regardless.

    \begin{tcolorbox}[breakable, size=fbox, boxrule=1pt, pad at break*=1mm,colback=cellbackground, colframe=cellborder]
\prompt{In}{incolor}{9}{\boxspacing}
\begin{Verbatim}[commandchars=\\\{\}]
\PY{n}{SPACE\PYZus{}SHIP\PYZus{}MASS}\PY{o}{=}\PY{l+m+mf}{2e4}
\PY{n}{SPACE\PYZus{}SHIP\PYZus{}INITIAL\PYZus{}FUEL\PYZus{}MASS}\PY{o}{=}\PY{l+m+mi}{10000}
\PY{n}{SPACE\PYZus{}SHIP\PYZus{}REQUIRED\PYZus{}THRUST}\PY{o}{=}\PY{l+m+mf}{875e2}
\PY{n}{SPACE\PYZus{}SHIP\PYZus{}SPECIFIC\PYZus{}IMPULSE}\PY{o}{=}\PY{l+m+mf}{2.23e2}
\PY{n}{SPACE\PYZus{}SHIP\PYZus{}MAXIMUM\PYZus{}IMPULSE}\PY{o}{=}\PY{l+m+mf}{3e7}
\end{Verbatim}
\end{tcolorbox}

    \emph{Reference:} The space ship is just designated as being roughly the
same distance from the black hole as the slime planet and actively
approaching the planet (to try and influence a collision in most of the
simulations).

    \begin{tcolorbox}[breakable, size=fbox, boxrule=1pt, pad at break*=1mm,colback=cellbackground, colframe=cellborder]
\prompt{In}{incolor}{10}{\boxspacing}
\begin{Verbatim}[commandchars=\\\{\}]
\PY{n}{SPACE\PYZus{}SHIP\PYZus{}INITIAL\PYZus{}POSITION}\PY{o}{=}\PY{n}{np}\PY{o}{.}\PY{n}{array}\PY{p}{(}\PY{p}{[}\PY{l+m+mf}{58e9}\PY{p}{,}\PY{o}{\PYZhy{}}\PY{l+m+mf}{2e7}\PY{p}{]}\PY{p}{)}
\PY{n}{SPACE\PYZus{}SHIP\PYZus{}INITIAL\PYZus{}DIRECTION}\PY{o}{=}\PY{n}{unit\PYZus{}vector}\PY{p}{(}\PY{p}{[}\PY{o}{\PYZhy{}}\PY{l+m+mi}{3}\PY{p}{,}\PY{l+m+mi}{10}\PY{p}{]}\PY{p}{)}
\end{Verbatim}
\end{tcolorbox}

    \emph{Actual:} For the same reasons as above, the distances are
unrealistically close to the black hole.

    \begin{tcolorbox}[breakable, size=fbox, boxrule=1pt, pad at break*=1mm,colback=cellbackground, colframe=cellborder]
\prompt{In}{incolor}{11}{\boxspacing}
\begin{Verbatim}[commandchars=\\\{\}]
\PY{n}{SPACE\PYZus{}SHIP\PYZus{}INITIAL\PYZus{}POSITION}\PY{o}{=}\PY{n}{np}\PY{o}{.}\PY{n}{array}\PY{p}{(}\PY{p}{[}\PY{l+m+mi}{20}\PY{p}{,}\PY{o}{\PYZhy{}}\PY{l+m+mi}{15}\PY{p}{]}\PY{p}{)}
\PY{n}{SPACE\PYZus{}SHIP\PYZus{}INITIAL\PYZus{}DIRECTION}\PY{o}{=}\PY{n}{unit\PYZus{}vector}\PY{p}{(}\PY{p}{[}\PY{o}{\PYZhy{}}\PY{l+m+mi}{3}\PY{p}{,}\PY{l+m+mi}{10}\PY{p}{]}\PY{p}{)}
\end{Verbatim}
\end{tcolorbox}

    \textbf{Black Hole}

    \emph{Reference:} This black hole is ideally a solar-mass black hole,
which would give it a mass of at least \(6\times 10^{30}\) kilograms (at
the minimum, about 3 times the mass of the sun). However, for this
simulation, we'll just assume the black hole is the same mass as the sun
for simplicity.

    \begin{tcolorbox}[breakable, size=fbox, boxrule=1pt, pad at break*=1mm,colback=cellbackground, colframe=cellborder]
\prompt{In}{incolor}{12}{\boxspacing}
\begin{Verbatim}[commandchars=\\\{\}]
\PY{n}{BLACK\PYZus{}HOLE\PYZus{}MASS}\PY{o}{=}\PY{l+m+mf}{2e30}
\PY{n}{BLACK\PYZus{}HOLE\PYZus{}INITIAL\PYZus{}POSITION}\PY{o}{=}\PY{n}{np}\PY{o}{.}\PY{n}{array}\PY{p}{(}\PY{p}{[}\PY{l+m+mi}{0}\PY{p}{,}\PY{l+m+mi}{0}\PY{p}{]}\PY{p}{)}
\end{Verbatim}
\end{tcolorbox}

    I've approximated the mass of the black hole as being roughly
\(2 \times 10^{10}\) kg for the sake of the simulation. This seems
reasonable as it maintains the large mass ratio
\(m_{bh} >>> m_{sp} >>> m_{ss}\) (as \(m_{sp} \approx 1.5 \times 10^6\)
kg)

    \begin{tcolorbox}[breakable, size=fbox, boxrule=1pt, pad at break*=1mm,colback=cellbackground, colframe=cellborder]
\prompt{In}{incolor}{13}{\boxspacing}
\begin{Verbatim}[commandchars=\\\{\}]
\PY{n}{BLACK\PYZus{}HOLE\PYZus{}MASS}\PY{o}{=}\PY{l+m+mf}{2e10}
\PY{n}{BLACK\PYZus{}HOLE\PYZus{}INITIAL\PYZus{}POSITION}\PY{o}{=}\PY{n}{np}\PY{o}{.}\PY{n}{array}\PY{p}{(}\PY{p}{[}\PY{l+m+mi}{0}\PY{p}{,}\PY{l+m+mi}{0}\PY{p}{]}\PY{p}{)}
\end{Verbatim}
\end{tcolorbox}

    \textbf{Code Block Summary:} This code block implements in the rigid and
soft bodies used for simulation. \texttt{RigidBody}-ies come with an
assumption of being a point-mass, so they don't hold a radius (or any
other size), but they're the base object in this simulation, with
tracker arrays for each of various simulation variables tied to the
object (like position, velocity, acceleration, and whatnot).

    \begin{tcolorbox}[breakable, size=fbox, boxrule=1pt, pad at break*=1mm,colback=cellbackground, colframe=cellborder]
\prompt{In}{incolor}{14}{\boxspacing}
\begin{Verbatim}[commandchars=\\\{\}]
\PY{n+nd}{@dataclass}
\PY{k}{class} \PY{n+nc}{RigidBody}\PY{p}{:}
    \PY{n}{initial\PYZus{}position}\PY{p}{:}     \PY{n+nb}{float} \PY{o}{=} \PY{l+m+mi}{0}
    \PY{n}{initial\PYZus{}velocity}\PY{p}{:}     \PY{n+nb}{float} \PY{o}{=} \PY{l+m+mi}{0}
    \PY{n}{initial\PYZus{}acceleration}\PY{p}{:} \PY{n+nb}{float} \PY{o}{=} \PY{l+m+mi}{0}
    
    \PY{n}{mass}\PY{p}{:}  \PY{n+nb}{float} \PY{o}{=} \PY{l+m+mi}{1}
    \PY{n}{color}\PY{p}{:} \PY{n+nb}{str}   \PY{o}{=} \PY{k+kc}{None}
    \PY{n}{name}\PY{p}{:}  \PY{n+nb}{str}   \PY{o}{=} \PY{k+kc}{None}

    \PY{n}{stationary}\PY{p}{:} \PY{n+nb}{bool} \PY{o}{=} \PY{k+kc}{False} 
    
    \PY{n}{position}\PY{p}{:}       \PY{n}{np}\PY{o}{.}\PY{n}{array} \PY{o}{=} \PY{k+kc}{None}
    \PY{n}{velocity}\PY{p}{:}       \PY{n}{np}\PY{o}{.}\PY{n}{array} \PY{o}{=} \PY{k+kc}{None}
    \PY{n}{acceleration}\PY{p}{:}   \PY{n}{np}\PY{o}{.}\PY{n}{array} \PY{o}{=} \PY{k+kc}{None}
    \PY{n}{force}\PY{p}{:}          \PY{n}{np}\PY{o}{.}\PY{n}{array} \PY{o}{=} \PY{k+kc}{None}
    \PY{n}{kinetic\PYZus{}energy}\PY{p}{:} \PY{n}{np}\PY{o}{.}\PY{n}{array} \PY{o}{=} \PY{k+kc}{None}
    \PY{n}{momentum}\PY{p}{:}       \PY{n}{np}\PY{o}{.}\PY{n}{array} \PY{o}{=} \PY{k+kc}{None}
    
    \PY{k}{def} \PY{n+nf}{init}\PY{p}{(}\PY{n+nb+bp}{self}\PY{p}{,} \PY{n}{n}\PY{p}{)}\PY{p}{:}
        \PY{n+nb+bp}{self}\PY{o}{.}\PY{n}{position}       \PY{o}{=} \PY{n}{np}\PY{o}{.}\PY{n}{zeros}\PY{p}{(}\PY{p}{(}\PY{n}{n}\PY{p}{,}\PY{l+m+mi}{2}\PY{p}{)}\PY{p}{)}
        \PY{n+nb+bp}{self}\PY{o}{.}\PY{n}{velocity}       \PY{o}{=} \PY{n}{np}\PY{o}{.}\PY{n}{zeros}\PY{p}{(}\PY{p}{(}\PY{n}{n}\PY{p}{,}\PY{l+m+mi}{2}\PY{p}{)}\PY{p}{)}
        \PY{n+nb+bp}{self}\PY{o}{.}\PY{n}{acceleration}   \PY{o}{=} \PY{n}{np}\PY{o}{.}\PY{n}{zeros}\PY{p}{(}\PY{p}{(}\PY{n}{n}\PY{p}{,}\PY{l+m+mi}{2}\PY{p}{)}\PY{p}{)}
        \PY{n+nb+bp}{self}\PY{o}{.}\PY{n}{force}          \PY{o}{=} \PY{n}{np}\PY{o}{.}\PY{n}{zeros}\PY{p}{(}\PY{p}{(}\PY{n}{n}\PY{p}{,}\PY{l+m+mi}{2}\PY{p}{)}\PY{p}{)}
        \PY{n+nb+bp}{self}\PY{o}{.}\PY{n}{kinetic\PYZus{}energy} \PY{o}{=} \PY{n}{np}\PY{o}{.}\PY{n}{zeros}\PY{p}{(}\PY{n}{n}\PY{p}{)}
        \PY{n+nb+bp}{self}\PY{o}{.}\PY{n}{momentum}       \PY{o}{=} \PY{n}{np}\PY{o}{.}\PY{n}{zeros}\PY{p}{(}\PY{p}{(}\PY{n}{n}\PY{p}{,}\PY{l+m+mi}{2}\PY{p}{)}\PY{p}{)}
        
        \PY{n+nb+bp}{self}\PY{o}{.}\PY{n}{position}\PY{p}{[}\PY{l+m+mi}{0}\PY{p}{,}\PY{p}{:}\PY{p}{]}     \PY{o}{=} \PY{n+nb+bp}{self}\PY{o}{.}\PY{n}{initial\PYZus{}position}
        \PY{n+nb+bp}{self}\PY{o}{.}\PY{n}{velocity}\PY{p}{[}\PY{l+m+mi}{0}\PY{p}{,}\PY{p}{:}\PY{p}{]}     \PY{o}{=} \PY{n+nb+bp}{self}\PY{o}{.}\PY{n}{initial\PYZus{}velocity}
        \PY{n+nb+bp}{self}\PY{o}{.}\PY{n}{acceleration}\PY{p}{[}\PY{l+m+mi}{0}\PY{p}{,}\PY{p}{:}\PY{p}{]} \PY{o}{=} \PY{n+nb+bp}{self}\PY{o}{.}\PY{n}{initial\PYZus{}acceleration}
        
    \PY{k}{def} \PY{n+nf+fm}{\PYZus{}\PYZus{}hash\PYZus{}\PYZus{}}\PY{p}{(}\PY{n+nb+bp}{self}\PY{p}{)}\PY{p}{:}
        \PY{k}{return} \PY{n+nb}{hash}\PY{p}{(}\PY{n+nb+bp}{self}\PY{o}{.}\PY{n}{name}\PY{p}{)}
        
    \PY{k}{def} \PY{n+nf}{distance\PYZus{}vec}\PY{p}{(}\PY{n+nb+bp}{self}\PY{p}{,} \PY{n}{other\PYZus{}body}\PY{p}{,} \PY{n}{i}\PY{p}{)}\PY{p}{:}
        \PY{k}{if} \PY{n}{other\PYZus{}body}\PY{o}{.}\PY{n}{position} \PY{o+ow}{is} \PY{k+kc}{None}\PY{p}{:}
            \PY{n+nb}{print}\PY{p}{(}\PY{n}{other\PYZus{}body}\PY{o}{.}\PY{n}{name}\PY{p}{)}
        \PY{k}{return} \PY{n+nb+bp}{self}\PY{o}{.}\PY{n}{position}\PY{p}{[}\PY{n}{i}\PY{p}{]} \PY{o}{\PYZhy{}} \PY{n}{other\PYZus{}body}\PY{o}{.}\PY{n}{position}\PY{p}{[}\PY{n}{i}\PY{p}{]}
    
    \PY{k}{def} \PY{n+nf}{near\PYZus{}barrier}\PY{p}{(}\PY{n+nb+bp}{self}\PY{p}{,} \PY{n}{bottom\PYZus{}barrier}\PY{p}{,} \PY{n}{top\PYZus{}barrier}\PY{p}{,} \PY{n}{collision\PYZus{}radius}\PY{p}{,} \PY{n}{i}\PY{p}{)}\PY{p}{:}
        \PY{k}{return} \PYZbs{}
            \PY{n+nb}{abs}\PY{p}{(}\PY{n+nb+bp}{self}\PY{o}{.}\PY{n}{position}\PY{p}{[}\PY{n}{i}\PY{p}{]} \PY{o}{\PYZhy{}} \PY{n}{bottom\PYZus{}barrier}\PY{p}{)} \PY{o}{\PYZlt{}}\PY{o}{=} \PY{n}{collision\PYZus{}radius} \PY{o+ow}{or} \PYZbs{}
            \PY{n+nb}{abs}\PY{p}{(}\PY{n+nb+bp}{self}\PY{o}{.}\PY{n}{position}\PY{p}{[}\PY{n}{i}\PY{p}{]} \PY{o}{\PYZhy{}} \PY{n}{top\PYZus{}barrier}\PY{p}{)} \PY{o}{\PYZlt{}}\PY{o}{=} \PY{n}{collision\PYZus{}radius}
    
\PY{n+nd}{@dataclass}
\PY{k}{class} \PY{n+nc}{Spaceship}\PY{p}{(}\PY{n}{RigidBody}\PY{p}{)}\PY{p}{:}
    \PY{n}{initial\PYZus{}fuel\PYZus{}mass}\PY{p}{:}       \PY{n+nb}{float} \PY{o}{=} \PY{l+m+mi}{0}
    \PY{n}{turning\PYZus{}angle}\PY{p}{:}           \PY{n+nb}{float} \PY{o}{=} \PY{l+m+mi}{0}
    \PY{n}{specific\PYZus{}impulse}\PY{p}{:}        \PY{n+nb}{float} \PY{o}{=} \PY{l+m+mi}{0}
    \PY{n}{desired\PYZus{}thrust\PYZus{}mag}\PY{p}{:}      \PY{n+nb}{float} \PY{o}{=} \PY{l+m+mi}{0}
    \PY{n}{maximum\PYZus{}impulse}\PY{p}{:}         \PY{n+nb}{float} \PY{o}{=} \PY{l+m+mi}{0}
    \PY{n}{total\PYZus{}collision\PYZus{}impulse}\PY{p}{:} \PY{n+nb}{float} \PY{o}{=} \PY{l+m+mi}{0}
    
    \PY{n}{destroyed}\PY{p}{:} \PY{n+nb}{bool} \PY{o}{=} \PY{k+kc}{False}
    
    \PY{n}{remaining\PYZus{}fuel\PYZus{}mass}\PY{p}{:} \PY{n}{np}\PY{o}{.}\PY{n}{array} \PY{o}{=} \PY{k+kc}{None}
    \PY{n}{thrust\PYZus{}dir}\PY{p}{:}          \PY{n}{np}\PY{o}{.}\PY{n}{array} \PY{o}{=} \PY{k+kc}{None}
    \PY{n}{collision\PYZus{}force}\PY{p}{:}     \PY{n+nb}{list}     \PY{o}{=} \PY{k+kc}{None}
    \PY{n}{collision\PYZus{}impulse}\PY{p}{:}   \PY{n+nb}{list}     \PY{o}{=} \PY{k+kc}{None}
    
    \PY{k}{def} \PY{n+nf}{init}\PY{p}{(}\PY{n+nb+bp}{self}\PY{p}{,} \PY{n}{n}\PY{p}{)}\PY{p}{:}
        \PY{n+nb}{super}\PY{p}{(}\PY{p}{)}\PY{o}{.}\PY{n}{init}\PY{p}{(}\PY{n}{n}\PY{p}{)}
        \PY{n+nb+bp}{self}\PY{o}{.}\PY{n}{remaining\PYZus{}fuel\PYZus{}mass} \PY{o}{=} \PY{n}{np}\PY{o}{.}\PY{n}{zeros}\PY{p}{(}\PY{n}{n}\PY{p}{)}
        \PY{n+nb+bp}{self}\PY{o}{.}\PY{n}{remaining\PYZus{}fuel\PYZus{}mass}\PY{p}{[}\PY{l+m+mi}{0}\PY{p}{]} \PY{o}{=} \PY{n+nb+bp}{self}\PY{o}{.}\PY{n}{initial\PYZus{}fuel\PYZus{}mass}
        \PY{n+nb+bp}{self}\PY{o}{.}\PY{n}{thrust\PYZus{}dir} \PY{o}{=} \PY{n}{rotate\PYZus{}vector\PYZus{}2d}\PY{p}{(}\PY{n}{unit\PYZus{}vector}\PY{p}{(}\PY{n+nb+bp}{self}\PY{o}{.}\PY{n}{initial\PYZus{}velocity}\PY{p}{)}\PY{p}{,} \PY{n+nb+bp}{self}\PY{o}{.}\PY{n}{turning\PYZus{}angle}\PY{p}{)}
        \PY{n+nb+bp}{self}\PY{o}{.}\PY{n}{collision\PYZus{}force} \PY{o}{=} \PY{p}{[}\PY{p}{]}
        \PY{n+nb+bp}{self}\PY{o}{.}\PY{n}{collision\PYZus{}impulse} \PY{o}{=} \PY{p}{[}\PY{p}{]}
    
\PY{n+nd}{@dataclass}
\PY{k}{class} \PY{n+nc}{SoftBody}\PY{p}{(}\PY{n}{RigidBody}\PY{p}{)}\PY{p}{:}
    \PY{n}{radius}\PY{p}{:}          \PY{n+nb}{float} \PY{o}{=} \PY{l+m+mi}{0}
    \PY{n}{stiffness}\PY{p}{:}       \PY{n+nb}{float} \PY{o}{=} \PY{l+m+mi}{0}
    \PY{n}{growth\PYZus{}exponent}\PY{p}{:} \PY{n+nb}{float} \PY{o}{=} \PY{l+m+mi}{0}
    \PY{n}{damping\PYZus{}coeff}\PY{p}{:}   \PY{n+nb}{float} \PY{o}{=} \PY{l+m+mi}{0}
    
    \PY{k}{def} \PY{n+nf}{radial\PYZus{}distance\PYZus{}to}\PY{p}{(}\PY{n+nb+bp}{self}\PY{p}{,} \PY{n}{rigid\PYZus{}body}\PY{p}{,} \PY{n}{i}\PY{p}{)}\PY{p}{:}
        \PY{k}{return} \PY{n+nb}{max}\PY{p}{(}\PY{l+m+mi}{0}\PY{p}{,} \PY{n}{np}\PY{o}{.}\PY{n}{linalg}\PY{o}{.}\PY{n}{norm}\PY{p}{(}\PY{n+nb+bp}{self}\PY{o}{.}\PY{n}{distance\PYZus{}vec}\PY{p}{(}\PY{n}{rigid\PYZus{}body}\PY{p}{,} \PY{n}{i}\PY{p}{)}\PY{p}{)} \PY{o}{\PYZhy{}} \PY{n+nb+bp}{self}\PY{o}{.}\PY{n}{radius}\PY{p}{)}
    
    \PY{k}{def} \PY{n+nf}{is\PYZus{}softbody\PYZus{}collision}\PY{p}{(}\PY{n+nb+bp}{self}\PY{p}{,} \PY{n}{rigid\PYZus{}body}\PY{p}{,} \PY{n}{i}\PY{p}{)}\PY{p}{:}
        \PY{k}{return} \PY{n}{np}\PY{o}{.}\PY{n}{linalg}\PY{o}{.}\PY{n}{norm}\PY{p}{(}\PY{n+nb+bp}{self}\PY{o}{.}\PY{n}{distance\PYZus{}vec}\PY{p}{(}\PY{n}{rigid\PYZus{}body}\PY{p}{,} \PY{n}{i}\PY{p}{)}\PY{p}{)} \PY{o}{\PYZlt{}}\PY{o}{=} \PY{n+nb+bp}{self}\PY{o}{.}\PY{n}{radius}
\end{Verbatim}
\end{tcolorbox}

    \textbf{Code Block Summary:} This is where the thrust calculations come
in. The space ship has a certain amount of fuel,
\texttt{remaining\_fuel\_mass}, in kilograms of burnable fuel (which is
approximately reliably converted into energy at a set rate). The
instantaneous thrust force produced is
\[{\bf F}_{thrust}(t) = T(t) \hat{\bf u}(t),\] where \(T(t)\) is the
magnitude of thrust at \(t\) and \(\hat{\bf u}(t)\) is the direction of
thrust. The magnitude of thrust for a given \(t\) is
\[T(t) = \left| \frac{dm_{f}}{dt}\right| I_{sp},\] where \(m_f\) is the
mass of the remaining fuel and \(I_{sp}\) is the specific impulse
\href{https://en.wikipedia.org/wiki/Specific_impulse}{{[}1{]}} and
\(g_0\) is the standard gravity
\href{https://en.wikipedia.org/wiki/Standard_gravity}{{[}2{]}}.

    \begin{tcolorbox}[breakable, size=fbox, boxrule=1pt, pad at break*=1mm,colback=cellbackground, colframe=cellborder]
\prompt{In}{incolor}{15}{\boxspacing}
\begin{Verbatim}[commandchars=\\\{\}]
\PY{k}{def} \PY{n+nf}{calculate\PYZus{}thrust}\PY{p}{(}\PY{n}{ship}\PY{p}{,} \PY{n}{dt}\PY{p}{,} \PY{n}{last\PYZus{}frame}\PY{p}{)}\PY{p}{:}
    \PY{c+c1}{\PYZsh{} force normalize thrust direction}
    \PY{n}{ship}\PY{o}{.}\PY{n}{thrust\PYZus{}dir} \PY{o}{=} \PY{n}{unit\PYZus{}vector}\PY{p}{(}\PY{n}{ship}\PY{o}{.}\PY{n}{thrust\PYZus{}dir}\PY{p}{)}
    
    \PY{k}{if} \PY{n}{np}\PY{o}{.}\PY{n}{allclose}\PY{p}{(}\PY{n}{ship}\PY{o}{.}\PY{n}{thrust\PYZus{}dir}\PY{p}{,} \PY{n}{unit\PYZus{}vector}\PY{p}{(}\PY{n}{ship}\PY{o}{.}\PY{n}{velocity}\PY{p}{[}\PY{n}{last\PYZus{}frame}\PY{p}{]}\PY{p}{)}\PY{p}{)}\PY{p}{:}
        \PY{c+c1}{\PYZsh{} don\PYZsq{}t need to continue turning}
        \PY{k}{return} \PY{l+m+mi}{0}\PY{p}{,}\PY{l+m+mi}{0}
    
    \PY{c+c1}{\PYZsh{} set the wanted thrust magnitude}
    \PY{n}{thrust\PYZus{}mag} \PY{o}{=} \PY{n+nb}{max}\PY{p}{(}\PY{l+m+mi}{0}\PY{p}{,} \PY{n}{ship}\PY{o}{.}\PY{n}{desired\PYZus{}thrust\PYZus{}mag}\PY{p}{)}
    
    \PY{k}{if} \PY{n}{thrust\PYZus{}mag} \PY{o}{==} \PY{l+m+mi}{0}\PY{p}{:}
        \PY{k}{return} \PY{n}{np}\PY{o}{.}\PY{n}{zeros}\PY{p}{(}\PY{l+m+mi}{2}\PY{p}{)}\PY{p}{,} \PY{l+m+mf}{0.0}
    
    \PY{n}{required\PYZus{}mass\PYZus{}deriv} \PY{o}{=} \PY{n}{thrust\PYZus{}mag} \PY{o}{/} \PY{n}{ship}\PY{o}{.}\PY{n}{specific\PYZus{}impulse}
    \PY{n}{required\PYZus{}mass} \PY{o}{=} \PY{n}{required\PYZus{}mass\PYZus{}deriv} \PY{o}{*} \PY{n}{dt}

    \PY{c+c1}{\PYZsh{} limit by available fuel}
    \PY{k}{if} \PY{n}{required\PYZus{}mass} \PY{o}{\PYZgt{}} \PY{n}{ship}\PY{o}{.}\PY{n}{remaining\PYZus{}fuel\PYZus{}mass}\PY{p}{[}\PY{n}{last\PYZus{}frame}\PY{p}{]}\PY{p}{:}
        \PY{n}{required\PYZus{}mass} \PY{o}{=} \PY{n}{ship}\PY{o}{.}\PY{n}{remaining\PYZus{}fuel\PYZus{}mass}\PY{p}{[}\PY{n}{last\PYZus{}frame}\PY{p}{]}
        \PY{n}{required\PYZus{}mass\PYZus{}deriv} \PY{o}{=} \PY{n}{required\PYZus{}mass} \PY{o}{/} \PY{n}{dt}
        \PY{n}{thrust\PYZus{}mag} \PY{o}{=} \PY{n}{required\PYZus{}mass\PYZus{}deriv} \PY{o}{*} \PY{n}{ship}\PY{o}{.}\PY{n}{specific\PYZus{}impulse}
        
    \PY{n}{thrust\PYZus{}force\PYZus{}vector} \PY{o}{=} \PY{n}{ship}\PY{o}{.}\PY{n}{thrust\PYZus{}dir} \PY{o}{*} \PY{n}{thrust\PYZus{}mag}
    
    \PY{k}{return} \PY{n}{thrust\PYZus{}force\PYZus{}vector}\PY{p}{,} \PY{n}{required\PYZus{}mass}
\end{Verbatim}
\end{tcolorbox}

    \textbf{Code Block Summary:} This is where the gravitational force
between two objects is calculated with
\[{\bf F}_g = G \frac{m_1 m_2}{|{\bf r}|^2} \hat{r}\] for each object
with mass \(m_1\) to each other object with mass \(m_2\), given a radial
distance \(\bf r\) between them.

    \begin{tcolorbox}[breakable, size=fbox, boxrule=1pt, pad at break*=1mm,colback=cellbackground, colframe=cellborder]
\prompt{In}{incolor}{16}{\boxspacing}
\begin{Verbatim}[commandchars=\\\{\}]
\PY{n}{GRAVITATIONAL\PYZus{}CONSTANT} \PY{o}{=} \PY{l+m+mf}{6.6743e\PYZhy{}11}

\PY{k}{def} \PY{n+nf}{calculate\PYZus{}gravitational\PYZus{}force}\PY{p}{(}\PY{n}{last\PYZus{}frame}\PY{p}{,} \PY{n}{body}\PY{p}{,} \PY{n}{other}\PY{p}{)}\PY{p}{:}
    \PY{n}{displacement} \PY{o}{=} \PY{o}{\PYZhy{}}\PY{n}{body}\PY{o}{.}\PY{n}{distance\PYZus{}vec}\PY{p}{(}\PY{n}{other}\PY{p}{,} \PY{n}{last\PYZus{}frame}\PY{p}{)}
    \PY{n}{distance}     \PY{o}{=} \PY{n}{np}\PY{o}{.}\PY{n}{linalg}\PY{o}{.}\PY{n}{norm}\PY{p}{(}\PY{n}{displacement}\PY{p}{)}
    \PY{k}{if} \PY{n}{distance} \PY{o}{==} \PY{l+m+mi}{0}\PY{p}{:}
        \PY{n}{distance} \PY{o}{=} \PY{l+m+mf}{1e\PYZhy{}10}
    \PY{n}{direction}    \PY{o}{=} \PY{n}{displacement} \PY{o}{/} \PY{n}{distance}

    \PY{k}{return} \PY{p}{(}\PY{n}{GRAVITATIONAL\PYZus{}CONSTANT} \PY{o}{*} \PY{n}{body}\PY{o}{.}\PY{n}{mass} \PY{o}{*} \PYZbs{}
             \PY{n}{other}\PY{o}{.}\PY{n}{mass} \PY{o}{*} \PY{n}{direction}\PY{p}{)} \PY{o}{/} \PY{n}{distance} \PY{o}{*}\PY{o}{*} \PY{l+m+mi}{2}
\end{Verbatim}
\end{tcolorbox}

    \textbf{Code Block Summary:} This is where the normal force during the
inelastic collision between the spaceship and the slime planet (in the
case that it happens) is calculated by the Hunt-Crossley non-linear
contact model \href{https://doi.org/10.1115/1.3423596}{3}
\[{\bf F}_N = k \delta^n + \alpha \delta^n \frac{d \delta}{dt},\] with
\(n=\frac{3}{2}\) for spherical geometry (and
\(\delta = R_1 + R_2 - |\vec{r}_1 - \vec{r}_2|\), but due to point-like
approximations, only the slime planet has a radius). (For a more
in-depth explanation, please see the introduction.)

    \begin{tcolorbox}[breakable, size=fbox, boxrule=1pt, pad at break*=1mm,colback=cellbackground, colframe=cellborder]
\prompt{In}{incolor}{17}{\boxspacing}
\begin{Verbatim}[commandchars=\\\{\}]
\PY{k}{def} \PY{n+nf}{calculate\PYZus{}soft\PYZus{}body\PYZus{}collision\PYZus{}normal\PYZus{}force}\PY{p}{(}\PY{n}{last\PYZus{}frame}\PY{p}{,} \PY{n}{body}\PY{p}{,} \PY{n}{soft\PYZus{}body}\PY{p}{)}\PY{p}{:}
    \PY{c+c1}{\PYZsh{} assume this object has 0 radius because}
    \PY{c+c1}{\PYZsh{} of pointlike approximations }
    \PY{n}{distance\PYZus{}vector} \PY{o}{=} \PY{n}{body}\PY{o}{.}\PY{n}{position}\PY{p}{[}\PY{n}{last\PYZus{}frame}\PY{p}{]} \PY{o}{\PYZhy{}} \PY{n}{soft\PYZus{}body}\PY{o}{.}\PY{n}{position}\PY{p}{[}\PY{n}{last\PYZus{}frame}\PY{p}{]}
    \PY{n}{overlap} \PY{o}{=} \PY{n}{soft\PYZus{}body}\PY{o}{.}\PY{n}{radius} \PY{o}{\PYZhy{}} \PY{n}{np}\PY{o}{.}\PY{n}{linalg}\PY{o}{.}\PY{n}{norm}\PY{p}{(}\PY{n}{distance\PYZus{}vector}\PY{p}{)}
    \PY{n}{direction} \PY{o}{=} \PY{n}{unit\PYZus{}vector}\PY{p}{(}\PY{n}{distance\PYZus{}vector}\PY{p}{)}
    \PY{n}{relative\PYZus{}velocity} \PY{o}{=} \PY{n}{body}\PY{o}{.}\PY{n}{velocity}\PY{p}{[}\PY{n}{last\PYZus{}frame}\PY{p}{]} \PY{o}{\PYZhy{}} \PY{n}{soft\PYZus{}body}\PY{o}{.}\PY{n}{velocity}\PY{p}{[}\PY{n}{last\PYZus{}frame}\PY{p}{]}
    
    \PY{n}{overlap\PYZus{}deriv} \PY{o}{=} \PY{o}{\PYZhy{}}\PY{n}{np}\PY{o}{.}\PY{n}{dot}\PY{p}{(}\PY{n}{relative\PYZus{}velocity}\PY{p}{,} \PY{n}{direction}\PY{p}{)}
    
    \PY{n}{contact\PYZus{}term} \PY{o}{=} \PY{n}{soft\PYZus{}body}\PY{o}{.}\PY{n}{stiffness} \PY{o}{*} \PY{n}{overlap} \PY{o}{*}\PY{o}{*} \PY{n}{soft\PYZus{}body}\PY{o}{.}\PY{n}{growth\PYZus{}exponent}
    \PY{n}{damping\PYZus{}term} \PY{o}{=} \PY{n}{soft\PYZus{}body}\PY{o}{.}\PY{n}{damping\PYZus{}coeff} \PY{o}{*} \PY{n}{overlap} \PY{o}{*}\PY{o}{*} \PY{n}{soft\PYZus{}body}\PY{o}{.}\PY{n}{growth\PYZus{}exponent} \PY{o}{*} \PY{n}{overlap\PYZus{}deriv}
    
    \PY{n}{normal\PYZus{}force\PYZus{}mag} \PY{o}{=} \PY{n}{contact\PYZus{}term} \PY{o}{+} \PY{n}{damping\PYZus{}term}
    \PY{n}{normal\PYZus{}force} \PY{o}{=} \PY{n}{normal\PYZus{}force\PYZus{}mag} \PY{o}{*} \PY{n}{direction}
    
    \PY{k}{return} \PY{n}{normal\PYZus{}force}
\end{Verbatim}
\end{tcolorbox}

    \textbf{Code Block Summary:} This is where the net force is calculated
as the sum of all of the gravitational forces to the other bodies, the
normal force if a collision is occuring, and the thrust force (for the
space ship).

    \begin{tcolorbox}[breakable, size=fbox, boxrule=1pt, pad at break*=1mm,colback=cellbackground, colframe=cellborder]
\prompt{In}{incolor}{18}{\boxspacing}
\begin{Verbatim}[commandchars=\\\{\}]
\PY{k}{def} \PY{n+nf}{calculate\PYZus{}net\PYZus{}force}\PY{p}{(}\PY{n}{last\PYZus{}frame}\PY{p}{,} \PY{n}{body}\PY{p}{,} \PY{n}{objects}\PY{p}{,} \PY{n}{dt}\PY{p}{,} \PY{n}{collision\PYZus{}frames}\PY{p}{,} 
                        \PY{n}{do\PYZus{}gravity}\PY{p}{,} \PY{n}{do\PYZus{}thrust}\PY{p}{,} \PY{n}{do\PYZus{}collision}\PY{p}{)}\PY{p}{:}
\PY{+w}{    }\PY{l+s+sd}{\PYZdq{}\PYZdq{}\PYZdq{}}
\PY{l+s+sd}{    A general function for calculating the net force}
\PY{l+s+sd}{    for the two moving objects in the simulation: the}
\PY{l+s+sd}{    space ship and the slime planet.}
\PY{l+s+sd}{    \PYZdq{}\PYZdq{}\PYZdq{}}
    
    \PY{n}{other\PYZus{}objects} \PY{o}{=} \PY{p}{[}\PY{n}{x} \PY{k}{for} \PY{n}{x} \PY{o+ow}{in} \PY{n}{objects} \PY{k}{if} \PY{n}{x} \PY{o+ow}{is} \PY{o+ow}{not} \PY{n}{body}\PY{p}{]}
    \PY{n}{total\PYZus{}force} \PY{o}{=} \PY{n}{np}\PY{o}{.}\PY{n}{zeros}\PY{p}{(}\PY{l+m+mi}{2}\PY{p}{)}
    
    \PY{k}{for} \PY{n}{other} \PY{o+ow}{in} \PY{n}{other\PYZus{}objects}\PY{p}{:}       
        \PY{c+c1}{\PYZsh{} gravitational force}
        \PY{k}{if} \PY{n}{do\PYZus{}gravity}\PY{p}{:}
            \PY{n}{gravitational\PYZus{}force} \PY{o}{=} \PY{n}{calculate\PYZus{}gravitational\PYZus{}force}\PY{p}{(}\PY{n}{last\PYZus{}frame}\PY{p}{,} \PY{n}{body}\PY{p}{,} \PY{n}{other}\PY{p}{)}
            \PY{n}{total\PYZus{}force} \PY{o}{=} \PY{n}{total\PYZus{}force} \PY{o}{+} \PY{n}{gravitational\PYZus{}force}
        
        \PY{c+c1}{\PYZsh{} basically just if it isnt the black hole}
        \PY{k}{if} \PY{o+ow}{not} \PY{n}{body}\PY{o}{.}\PY{n}{stationary} \PY{o+ow}{and} \PY{o+ow}{not} \PY{n}{other}\PY{o}{.}\PY{n}{stationary} \PY{o+ow}{and} \PY{n}{do\PYZus{}collision}\PY{p}{:}
            \PY{c+c1}{\PYZsh{} space ship \PYZhy{}\PYZgt{} slime\PYZus{}planet}
            \PY{k}{if} \PY{n+nb}{isinstance}\PY{p}{(}\PY{n}{other}\PY{p}{,} \PY{n}{SoftBody}\PY{p}{)} \PY{o+ow}{and} \PY{n}{other}\PY{o}{.}\PY{n}{is\PYZus{}softbody\PYZus{}collision}\PY{p}{(}\PY{n}{body}\PY{p}{,} \PY{n}{last\PYZus{}frame}\PY{p}{)}\PY{p}{:}
                \PY{n}{collision\PYZus{}frames}\PY{p}{[}\PY{n}{last\PYZus{}frame}\PY{o}{+}\PY{l+m+mi}{1}\PY{p}{]} \PY{o}{=} \PY{k+kc}{True}
                \PY{n}{normal\PYZus{}force} \PY{o}{=} \PY{n}{calculate\PYZus{}soft\PYZus{}body\PYZus{}collision\PYZus{}normal\PYZus{}force}\PY{p}{(}\PY{n}{last\PYZus{}frame}\PY{p}{,} \PY{n}{body}\PY{p}{,} \PY{n}{other}\PY{p}{)}
                \PY{n}{total\PYZus{}force} \PY{o}{=} \PY{n}{total\PYZus{}force} \PY{o}{+} \PY{n}{normal\PYZus{}force}
                
                \PY{k}{if} \PY{n+nb}{isinstance}\PY{p}{(}\PY{n}{body}\PY{p}{,} \PY{n}{Spaceship}\PY{p}{)}\PY{p}{:}
                    \PY{n}{ship} \PY{o}{=} \PY{n}{body}
                    \PY{n}{normal\PYZus{}force\PYZus{}mag} \PY{o}{=} \PY{n}{np}\PY{o}{.}\PY{n}{linalg}\PY{o}{.}\PY{n}{norm}\PY{p}{(}\PY{n}{normal\PYZus{}force}\PY{p}{)}
                    \PY{n}{ship}\PY{o}{.}\PY{n}{total\PYZus{}collision\PYZus{}impulse} \PY{o}{+}\PY{o}{=} \PY{n}{normal\PYZus{}force\PYZus{}mag}
                    \PY{n}{ship}\PY{o}{.}\PY{n}{collision\PYZus{}force}\PY{o}{.}\PY{n}{append}\PY{p}{(}\PY{n}{normal\PYZus{}force\PYZus{}mag}\PY{p}{)}
                    
                    \PY{k}{if} \PY{n+nb}{len}\PY{p}{(}\PY{n}{ship}\PY{o}{.}\PY{n}{collision\PYZus{}impulse}\PY{p}{)} \PY{o}{\PYZgt{}} \PY{l+m+mi}{0}\PY{p}{:}
                        \PY{n}{ship}\PY{o}{.}\PY{n}{collision\PYZus{}impulse}\PY{o}{.}\PY{n}{append}\PY{p}{(}\PY{n}{ship}\PY{o}{.}\PY{n}{collision\PYZus{}impulse}\PY{p}{[}\PY{o}{\PYZhy{}}\PY{l+m+mi}{1}\PY{p}{]} \PY{o}{+} \PY{n}{normal\PYZus{}force\PYZus{}mag}\PY{p}{)}
                    \PY{k}{else}\PY{p}{:}
                        \PY{n}{ship}\PY{o}{.}\PY{n}{collision\PYZus{}impulse}\PY{o}{.}\PY{n}{append}\PY{p}{(}\PY{n}{normal\PYZus{}force\PYZus{}mag}\PY{p}{)}
                
            \PY{c+c1}{\PYZsh{} slime planet \PYZhy{}\PYZgt{} space ship}
            \PY{k}{elif} \PY{n+nb}{isinstance}\PY{p}{(}\PY{n}{body}\PY{p}{,} \PY{n}{SoftBody}\PY{p}{)} \PY{o+ow}{and} \PY{n}{body}\PY{o}{.}\PY{n}{is\PYZus{}softbody\PYZus{}collision}\PY{p}{(}\PY{n}{other}\PY{p}{,} \PY{n}{last\PYZus{}frame}\PY{p}{)}\PY{p}{:}
                \PY{n}{collision\PYZus{}frames}\PY{p}{[}\PY{n}{last\PYZus{}frame}\PY{o}{+}\PY{l+m+mi}{1}\PY{p}{]} \PY{o}{=} \PY{k+kc}{True}
                \PY{n}{normal\PYZus{}force} \PY{o}{=} \PY{n}{calculate\PYZus{}soft\PYZus{}body\PYZus{}collision\PYZus{}normal\PYZus{}force}\PY{p}{(}\PY{n}{last\PYZus{}frame}\PY{p}{,} \PY{n}{other}\PY{p}{,} \PY{n}{body}\PY{p}{)}
                \PY{n}{total\PYZus{}force} \PY{o}{=} \PY{n}{total\PYZus{}force} \PY{o}{+} \PY{n}{normal\PYZus{}force}
    
    \PY{c+c1}{\PYZsh{} thrust force}
    \PY{k}{if} \PY{n+nb}{isinstance}\PY{p}{(}\PY{n}{body}\PY{p}{,} \PY{n}{Spaceship}\PY{p}{)} \PY{o+ow}{and} \PY{n}{do\PYZus{}thrust}\PY{p}{:}
        \PY{n}{ship} \PY{o}{=} \PY{n}{body}
        \PY{n}{thrust\PYZus{}force}\PY{p}{,} \PY{n}{fuel\PYZus{}mass\PYZus{}used} \PY{o}{=} \PY{n}{calculate\PYZus{}thrust}\PY{p}{(}\PY{n}{ship}\PY{p}{,} \PY{n}{dt}\PY{p}{,} \PY{n}{last\PYZus{}frame}\PY{p}{)}
        \PY{n}{ship}\PY{o}{.}\PY{n}{remaining\PYZus{}fuel\PYZus{}mass}\PY{p}{[}\PY{n}{last\PYZus{}frame} \PY{o}{+} \PY{l+m+mi}{1}\PY{p}{]} \PY{o}{=} \PY{n}{ship}\PY{o}{.}\PY{n}{remaining\PYZus{}fuel\PYZus{}mass}\PY{p}{[}\PY{n}{last\PYZus{}frame}\PY{p}{]} \PY{o}{\PYZhy{}} \PY{n}{fuel\PYZus{}mass\PYZus{}used}
        \PY{n}{total\PYZus{}force} \PY{o}{=} \PY{n}{total\PYZus{}force} \PY{o}{+} \PY{n}{thrust\PYZus{}force}
        
    \PY{k}{return} \PY{n}{total\PYZus{}force}
\end{Verbatim}
\end{tcolorbox}

    \textbf{Code Block Summary:} This section produces an animation of the
full simulation. It isn't the only means to verify that the simulation
is working as intended, but provides a quick and easy way to check for
glaring problems in the simulation.

    \begin{tcolorbox}[breakable, size=fbox, boxrule=1pt, pad at break*=1mm,colback=cellbackground, colframe=cellborder]
\prompt{In}{incolor}{19}{\boxspacing}
\begin{Verbatim}[commandchars=\\\{\}]
\PY{k}{def} \PY{n+nf}{produce\PYZus{}animation}\PY{p}{(}\PY{n}{simulation\PYZus{}result}\PY{p}{,} \PY{n}{canvas\PYZus{}size}\PY{o}{=}\PY{l+m+mi}{50}\PY{p}{,} \PY{n}{display\PYZus{}arrows}\PY{o}{=}\PY{k+kc}{False}\PY{p}{)}\PY{p}{:}
    \PY{c+c1}{\PYZsh{} aliases for the data}
    \PY{n}{time\PYZus{}data} \PY{o}{=} \PY{n}{simulation\PYZus{}result}\PY{o}{.}\PY{n}{time\PYZus{}data}
    \PY{n}{space\PYZus{}ship\PYZus{}trajectory} \PY{o}{=} \PY{n}{simulation\PYZus{}result}\PY{o}{.}\PY{n}{space\PYZus{}ship}\PY{o}{.}\PY{n}{position}
    \PY{n}{slime\PYZus{}planet\PYZus{}trajectory} \PY{o}{=} \PY{n}{simulation\PYZus{}result}\PY{o}{.}\PY{n}{slime\PYZus{}planet}\PY{o}{.}\PY{n}{position}
    \PY{n}{space\PYZus{}ship\PYZus{}velocity} \PY{o}{=} \PY{n}{simulation\PYZus{}result}\PY{o}{.}\PY{n}{space\PYZus{}ship}\PY{o}{.}\PY{n}{velocity}
    \PY{n}{slime\PYZus{}planet\PYZus{}velocity} \PY{o}{=} \PY{n}{simulation\PYZus{}result}\PY{o}{.}\PY{n}{slime\PYZus{}planet}\PY{o}{.}\PY{n}{velocity}
    \PY{n}{space\PYZus{}ship\PYZus{}acceleration} \PY{o}{=} \PY{n}{simulation\PYZus{}result}\PY{o}{.}\PY{n}{space\PYZus{}ship}\PY{o}{.}\PY{n}{acceleration}
    \PY{n}{slime\PYZus{}planet\PYZus{}acceleration} \PY{o}{=} \PY{n}{simulation\PYZus{}result}\PY{o}{.}\PY{n}{slime\PYZus{}planet}\PY{o}{.}\PY{n}{acceleration}
    \PY{n}{space\PYZus{}ship\PYZus{}fuel} \PY{o}{=} \PY{n}{simulation\PYZus{}result}\PY{o}{.}\PY{n}{space\PYZus{}ship}\PY{o}{.}\PY{n}{remaining\PYZus{}fuel\PYZus{}mass}
    \PY{n}{collision\PYZus{}directions} \PY{o}{=} \PY{n}{space\PYZus{}ship\PYZus{}trajectory} \PY{o}{\PYZhy{}} \PY{n}{slime\PYZus{}planet\PYZus{}trajectory}
    \PY{n}{collision\PYZus{}directions\PYZus{}units} \PY{o}{=} \PY{n}{unit\PYZus{}vectors}\PY{p}{(}\PY{n}{collision\PYZus{}directions}\PY{p}{)}
    \PY{n}{space\PYZus{}ship\PYZus{}force} \PY{o}{=} \PY{n}{magnitudes}\PY{p}{(}\PY{n}{simulation\PYZus{}result}\PY{o}{.}\PY{n}{space\PYZus{}ship}\PY{o}{.}\PY{n}{force}\PY{p}{)}
    
    \PY{c+c1}{\PYZsh{} a \PYZsq{}zero vector\PYZsq{} for when we don\PYZsq{}t want to plot stuff}
    \PY{n}{zero\PYZus{}vector} \PY{o}{=} \PY{n}{np}\PY{o}{.}\PY{n}{zeros\PYZus{}like}\PY{p}{(}\PY{n}{slime\PYZus{}planet\PYZus{}trajectory}\PY{p}{)}
    
    \PY{n}{fig}\PY{p}{,} \PY{n}{ax} \PY{o}{=} \PY{n}{plt}\PY{o}{.}\PY{n}{subplots}\PY{p}{(}\PY{n}{figsize}\PY{o}{=}\PY{p}{(}\PY{l+m+mi}{6}\PY{p}{,}\PY{l+m+mi}{6}\PY{p}{)}\PY{p}{)}
    \PY{n}{ax}\PY{o}{.}\PY{n}{set\PYZus{}aspect}\PY{p}{(}\PY{l+s+s1}{\PYZsq{}}\PY{l+s+s1}{equal}\PY{l+s+s1}{\PYZsq{}}\PY{p}{)}
    \PY{n}{ax}\PY{o}{.}\PY{n}{set\PYZus{}xlim}\PY{p}{(}\PY{o}{\PYZhy{}}\PY{n}{canvas\PYZus{}size}\PY{p}{,}\PY{n}{canvas\PYZus{}size}\PY{p}{)}
    \PY{n}{ax}\PY{o}{.}\PY{n}{set\PYZus{}ylim}\PY{p}{(}\PY{o}{\PYZhy{}}\PY{n}{canvas\PYZus{}size}\PY{p}{,}\PY{n}{canvas\PYZus{}size}\PY{p}{)}

    \PY{c+c1}{\PYZsh{} scatter points for each body}
    \PY{n}{space\PYZus{}ship\PYZus{}marker}\PY{p}{,}   \PY{o}{=} \PY{n}{ax}\PY{o}{.}\PY{n}{plot}\PY{p}{(}\PY{p}{[}\PY{p}{]}\PY{p}{,} \PY{p}{[}\PY{p}{]}\PY{p}{,} \PY{l+s+s1}{\PYZsq{}}\PY{l+s+s1}{ro}\PY{l+s+s1}{\PYZsq{}}\PY{p}{,} \PY{n}{label}\PY{o}{=}\PY{l+s+s1}{\PYZsq{}}\PY{l+s+s1}{Spaceship}\PY{l+s+s1}{\PYZsq{}}\PY{p}{)}
    \PY{n}{black\PYZus{}hole\PYZus{}marker}\PY{p}{,}   \PY{o}{=} \PY{n}{ax}\PY{o}{.}\PY{n}{plot}\PY{p}{(}\PY{p}{[}\PY{p}{]}\PY{p}{,} \PY{p}{[}\PY{p}{]}\PY{p}{,} \PY{l+s+s1}{\PYZsq{}}\PY{l+s+s1}{ko}\PY{l+s+s1}{\PYZsq{}}\PY{p}{,} \PY{n}{label}\PY{o}{=}\PY{l+s+s1}{\PYZsq{}}\PY{l+s+s1}{Black Hole}\PY{l+s+s1}{\PYZsq{}}\PY{p}{)}
    \PY{n}{ax}\PY{o}{.}\PY{n}{set\PYZus{}title}\PY{p}{(}\PY{l+s+sa}{f}\PY{l+s+s1}{\PYZsq{}}\PY{l+s+s1}{Simulation (θ=}\PY{l+s+si}{\PYZob{}}\PY{n}{simulation\PYZus{}result}\PY{o}{.}\PY{n}{angle}\PY{l+s+si}{:}\PY{l+s+s1}{.1f}\PY{l+s+si}{\PYZcb{}}\PY{l+s+s1}{ rad, \PYZdl{}v\PYZus{}0\PYZdl{}=}\PY{l+s+si}{\PYZob{}}\PY{n}{simulation\PYZus{}result}\PY{o}{.}\PY{n}{initial\PYZus{}speed}\PY{l+s+si}{:}\PY{l+s+s1}{.2e}\PY{l+s+si}{\PYZcb{}}\PY{l+s+s1}{ m/s)}\PY{l+s+s1}{\PYZsq{}}\PY{p}{)}
    
    \PY{n}{slime\PYZus{}planet\PYZus{}circle} \PY{o}{=} \PY{n}{patches}\PY{o}{.}\PY{n}{Circle}\PY{p}{(}
        \PY{p}{(}\PY{l+m+mi}{0}\PY{p}{,} \PY{l+m+mi}{0}\PY{p}{)}\PY{p}{,}
        \PY{n}{radius}\PY{o}{=}\PY{n}{simulation\PYZus{}result}\PY{o}{.}\PY{n}{slime\PYZus{}planet}\PY{o}{.}\PY{n}{radius}\PY{p}{,}  \PY{c+c1}{\PYZsh{} in same units as canvas\PYZus{}size}
        \PY{n}{color}\PY{o}{=}\PY{l+s+s1}{\PYZsq{}}\PY{l+s+s1}{green}\PY{l+s+s1}{\PYZsq{}}\PY{p}{,}
        \PY{n}{alpha}\PY{o}{=}\PY{l+m+mf}{0.5}\PY{p}{,}
        \PY{n}{label}\PY{o}{=}\PY{l+s+s1}{\PYZsq{}}\PY{l+s+s1}{Slime Planet}\PY{l+s+s1}{\PYZsq{}}
    \PY{p}{)}
    \PY{n}{ax}\PY{o}{.}\PY{n}{add\PYZus{}patch}\PY{p}{(}\PY{n}{slime\PYZus{}planet\PYZus{}circle}\PY{p}{)}
    
    \PY{c+c1}{\PYZsh{} remaining fuel label}
    \PY{n}{fuel\PYZus{}text} \PY{o}{=} \PY{n}{ax}\PY{o}{.}\PY{n}{text}\PY{p}{(}\PY{l+m+mf}{0.02}\PY{p}{,} \PY{l+m+mf}{0.98}\PY{p}{,} \PY{l+s+s1}{\PYZsq{}}\PY{l+s+s1}{\PYZsq{}}\PY{p}{,} \PY{n}{transform}\PY{o}{=}\PY{n}{ax}\PY{o}{.}\PY{n}{transAxes}\PY{p}{,}
                    \PY{n}{fontsize}\PY{o}{=}\PY{l+m+mi}{12}\PY{p}{,} \PY{n}{color}\PY{o}{=}\PY{l+s+s1}{\PYZsq{}}\PY{l+s+s1}{black}\PY{l+s+s1}{\PYZsq{}}\PY{p}{,} \PY{n}{va}\PY{o}{=}\PY{l+s+s1}{\PYZsq{}}\PY{l+s+s1}{top}\PY{l+s+s1}{\PYZsq{}}\PY{p}{)}
    \PY{n}{collision\PYZus{}indicator\PYZus{}text} \PY{o}{=} \PYZbs{}
        \PY{n}{ax}\PY{o}{.}\PY{n}{text}\PY{p}{(}\PY{l+m+mf}{0.02}\PY{p}{,} \PY{l+m+mf}{0.93}\PY{p}{,} \PY{l+s+s1}{\PYZsq{}}\PY{l+s+s1}{\PYZsq{}}\PY{p}{,} \PY{n}{transform}\PY{o}{=}\PY{n}{ax}\PY{o}{.}\PY{n}{transAxes}\PY{p}{,} \PY{n}{fontsize}\PY{o}{=}\PY{l+m+mi}{12}\PY{p}{,} \PY{n}{color}\PY{o}{=}\PY{l+s+s1}{\PYZsq{}}\PY{l+s+s1}{black}\PY{l+s+s1}{\PYZsq{}}\PY{p}{,} \PY{n}{va}\PY{o}{=}\PY{l+s+s1}{\PYZsq{}}\PY{l+s+s1}{top}\PY{l+s+s1}{\PYZsq{}}\PY{p}{)}
    \PY{n}{force\PYZus{}indicator\PYZus{}text} \PY{o}{=} \PYZbs{}
        \PY{n}{ax}\PY{o}{.}\PY{n}{text}\PY{p}{(}\PY{l+m+mf}{0.02}\PY{p}{,} \PY{l+m+mf}{0.88}\PY{p}{,} \PY{l+s+s1}{\PYZsq{}}\PY{l+s+s1}{\PYZsq{}}\PY{p}{,} \PY{n}{transform}\PY{o}{=}\PY{n}{ax}\PY{o}{.}\PY{n}{transAxes}\PY{p}{,} \PY{n}{fontsize}\PY{o}{=}\PY{l+m+mi}{12}\PY{p}{,} \PY{n}{color}\PY{o}{=}\PY{l+s+s1}{\PYZsq{}}\PY{l+s+s1}{black}\PY{l+s+s1}{\PYZsq{}}\PY{p}{,} \PY{n}{va}\PY{o}{=}\PY{l+s+s1}{\PYZsq{}}\PY{l+s+s1}{top}\PY{l+s+s1}{\PYZsq{}}\PY{p}{)}
    \PY{n}{time\PYZus{}text} \PY{o}{=} \PYZbs{}
        \PY{n}{ax}\PY{o}{.}\PY{n}{text}\PY{p}{(}\PY{l+m+mf}{0.02}\PY{p}{,} \PY{l+m+mf}{0.83}\PY{p}{,} \PY{l+s+s1}{\PYZsq{}}\PY{l+s+s1}{\PYZsq{}}\PY{p}{,} \PY{n}{transform}\PY{o}{=}\PY{n}{ax}\PY{o}{.}\PY{n}{transAxes}\PY{p}{,} \PY{n}{fontsize}\PY{o}{=}\PY{l+m+mi}{12}\PY{p}{,} \PY{n}{color}\PY{o}{=}\PY{l+s+s1}{\PYZsq{}}\PY{l+s+s1}{black}\PY{l+s+s1}{\PYZsq{}}\PY{p}{,} \PY{n}{va}\PY{o}{=}\PY{l+s+s1}{\PYZsq{}}\PY{l+s+s1}{top}\PY{l+s+s1}{\PYZsq{}}\PY{p}{)}
    
    \PY{c+c1}{\PYZsh{} quivery arrows for the velocities}
    \PY{n}{space\PYZus{}ship\PYZus{}quiver}  \PY{o}{=} \PY{n}{ax}\PY{o}{.}\PY{n}{quiver}\PY{p}{(}\PY{p}{[}\PY{l+m+mi}{0}\PY{p}{]}\PY{p}{,} \PY{p}{[}\PY{l+m+mi}{0}\PY{p}{]}\PY{p}{,} \PY{p}{[}\PY{l+m+mi}{0}\PY{p}{]}\PY{p}{,} \PY{p}{[}\PY{l+m+mi}{0}\PY{p}{]}\PY{p}{,}
                                   \PY{n}{color}\PY{o}{=}\PY{l+s+s1}{\PYZsq{}}\PY{l+s+s1}{black}\PY{l+s+s1}{\PYZsq{}}\PY{p}{,} \PY{n}{angles}\PY{o}{=}\PY{l+s+s1}{\PYZsq{}}\PY{l+s+s1}{xy}\PY{l+s+s1}{\PYZsq{}}\PY{p}{,}
                                   \PY{n}{scale\PYZus{}units}\PY{o}{=}\PY{l+s+s1}{\PYZsq{}}\PY{l+s+s1}{xy}\PY{l+s+s1}{\PYZsq{}}\PY{p}{,} \PY{n}{scale}\PY{o}{=}\PY{l+m+mi}{1}\PY{p}{,} \PY{n}{width}\PY{o}{=}\PY{l+m+mf}{0.005}\PY{p}{)}
    \PY{n}{space\PYZus{}ship\PYZus{}acceleration\PYZus{}quiver}  \PY{o}{=} \PY{n}{ax}\PY{o}{.}\PY{n}{quiver}\PY{p}{(}\PY{p}{[}\PY{l+m+mi}{0}\PY{p}{]}\PY{p}{,} \PY{p}{[}\PY{l+m+mi}{0}\PY{p}{]}\PY{p}{,} \PY{p}{[}\PY{l+m+mi}{0}\PY{p}{]}\PY{p}{,} \PY{p}{[}\PY{l+m+mi}{0}\PY{p}{]}\PY{p}{,}
                                   \PY{n}{color}\PY{o}{=}\PY{l+s+s1}{\PYZsq{}}\PY{l+s+s1}{orange}\PY{l+s+s1}{\PYZsq{}}\PY{p}{,} \PY{n}{angles}\PY{o}{=}\PY{l+s+s1}{\PYZsq{}}\PY{l+s+s1}{xy}\PY{l+s+s1}{\PYZsq{}}\PY{p}{,}
                                   \PY{n}{scale\PYZus{}units}\PY{o}{=}\PY{l+s+s1}{\PYZsq{}}\PY{l+s+s1}{xy}\PY{l+s+s1}{\PYZsq{}}\PY{p}{,} \PY{n}{scale}\PY{o}{=}\PY{l+m+mi}{1}\PY{p}{,} \PY{n}{width}\PY{o}{=}\PY{l+m+mf}{0.005}\PY{p}{)}
    \PY{n}{slime\PYZus{}planet\PYZus{}quiver} \PY{o}{=} \PY{n}{ax}\PY{o}{.}\PY{n}{quiver}\PY{p}{(}\PY{p}{[}\PY{l+m+mi}{0}\PY{p}{]}\PY{p}{,} \PY{p}{[}\PY{l+m+mi}{0}\PY{p}{]}\PY{p}{,} \PY{p}{[}\PY{l+m+mi}{0}\PY{p}{]}\PY{p}{,} \PY{p}{[}\PY{l+m+mi}{0}\PY{p}{]}\PY{p}{,}
                                   \PY{n}{color}\PY{o}{=}\PY{l+s+s1}{\PYZsq{}}\PY{l+s+s1}{black}\PY{l+s+s1}{\PYZsq{}}\PY{p}{,} \PY{n}{angles}\PY{o}{=}\PY{l+s+s1}{\PYZsq{}}\PY{l+s+s1}{xy}\PY{l+s+s1}{\PYZsq{}}\PY{p}{,}
                                   \PY{n}{scale\PYZus{}units}\PY{o}{=}\PY{l+s+s1}{\PYZsq{}}\PY{l+s+s1}{xy}\PY{l+s+s1}{\PYZsq{}}\PY{p}{,} \PY{n}{scale}\PY{o}{=}\PY{l+m+mi}{1}\PY{p}{,} \PY{n}{width}\PY{o}{=}\PY{l+m+mf}{0.005}\PY{p}{)}
    \PY{n}{slime\PYZus{}planet\PYZus{}acceleration\PYZus{}quiver} \PY{o}{=} \PY{n}{ax}\PY{o}{.}\PY{n}{quiver}\PY{p}{(}\PY{p}{[}\PY{l+m+mi}{0}\PY{p}{]}\PY{p}{,} \PY{p}{[}\PY{l+m+mi}{0}\PY{p}{]}\PY{p}{,} \PY{p}{[}\PY{l+m+mi}{0}\PY{p}{]}\PY{p}{,} \PY{p}{[}\PY{l+m+mi}{0}\PY{p}{]}\PY{p}{,}
                                   \PY{n}{color}\PY{o}{=}\PY{l+s+s1}{\PYZsq{}}\PY{l+s+s1}{orange}\PY{l+s+s1}{\PYZsq{}}\PY{p}{,} \PY{n}{angles}\PY{o}{=}\PY{l+s+s1}{\PYZsq{}}\PY{l+s+s1}{xy}\PY{l+s+s1}{\PYZsq{}}\PY{p}{,}
                                   \PY{n}{scale\PYZus{}units}\PY{o}{=}\PY{l+s+s1}{\PYZsq{}}\PY{l+s+s1}{xy}\PY{l+s+s1}{\PYZsq{}}\PY{p}{,} \PY{n}{scale}\PY{o}{=}\PY{l+m+mi}{1}\PY{p}{,} \PY{n}{width}\PY{o}{=}\PY{l+m+mf}{0.005}\PY{p}{)}
    \PY{n}{collision\PYZus{}direction\PYZus{}quiver} \PY{o}{=} \PY{n}{ax}\PY{o}{.}\PY{n}{quiver}\PY{p}{(}\PY{p}{[}\PY{l+m+mi}{0}\PY{p}{]}\PY{p}{,} \PY{p}{[}\PY{l+m+mi}{0}\PY{p}{]}\PY{p}{,} \PY{p}{[}\PY{l+m+mi}{0}\PY{p}{]}\PY{p}{,} \PY{p}{[}\PY{l+m+mi}{0}\PY{p}{]}\PY{p}{,}
                                   \PY{n}{color}\PY{o}{=}\PY{l+s+s1}{\PYZsq{}}\PY{l+s+s1}{lightblue}\PY{l+s+s1}{\PYZsq{}}\PY{p}{,} \PY{n}{angles}\PY{o}{=}\PY{l+s+s1}{\PYZsq{}}\PY{l+s+s1}{xy}\PY{l+s+s1}{\PYZsq{}}\PY{p}{,}
                                   \PY{n}{scale\PYZus{}units}\PY{o}{=}\PY{l+s+s1}{\PYZsq{}}\PY{l+s+s1}{xy}\PY{l+s+s1}{\PYZsq{}}\PY{p}{,} \PY{n}{scale}\PY{o}{=}\PY{l+m+mi}{1}\PY{p}{,} \PY{n}{width}\PY{o}{=}\PY{l+m+mf}{0.005}\PY{p}{)}
    \PY{n}{collision\PYZus{}force\PYZus{}quiver} \PY{o}{=} \PY{n}{ax}\PY{o}{.}\PY{n}{quiver}\PY{p}{(}\PY{p}{[}\PY{l+m+mi}{0}\PY{p}{]}\PY{p}{,} \PY{p}{[}\PY{l+m+mi}{0}\PY{p}{]}\PY{p}{,} \PY{p}{[}\PY{l+m+mi}{0}\PY{p}{]}\PY{p}{,} \PY{p}{[}\PY{l+m+mi}{0}\PY{p}{]}\PY{p}{,}
                                   \PY{n}{color}\PY{o}{=}\PY{l+s+s1}{\PYZsq{}}\PY{l+s+s1}{magenta}\PY{l+s+s1}{\PYZsq{}}\PY{p}{,} \PY{n}{angles}\PY{o}{=}\PY{l+s+s1}{\PYZsq{}}\PY{l+s+s1}{xy}\PY{l+s+s1}{\PYZsq{}}\PY{p}{,}
                                   \PY{n}{scale\PYZus{}units}\PY{o}{=}\PY{l+s+s1}{\PYZsq{}}\PY{l+s+s1}{xy}\PY{l+s+s1}{\PYZsq{}}\PY{p}{,} \PY{n}{scale}\PY{o}{=}\PY{l+m+mi}{1}\PY{p}{,} \PY{n}{width}\PY{o}{=}\PY{l+m+mf}{0.005}\PY{p}{)}
    
    \PY{n}{ax}\PY{o}{.}\PY{n}{legend}\PY{p}{(}\PY{p}{)}

    \PY{k}{def} \PY{n+nf}{init}\PY{p}{(}\PY{p}{)}\PY{p}{:}
\PY{+w}{        }\PY{l+s+sd}{\PYZdq{}\PYZdq{}\PYZdq{}}
\PY{l+s+sd}{        Initializes all of the quivers/markers as blank}
\PY{l+s+sd}{        \PYZdq{}\PYZdq{}\PYZdq{}}
        \PY{n}{space\PYZus{}ship\PYZus{}marker}\PY{o}{.}\PY{n}{set\PYZus{}data}\PY{p}{(}\PY{p}{[}\PY{p}{]}\PY{p}{,} \PY{p}{[}\PY{p}{]}\PY{p}{)}
        \PY{n}{black\PYZus{}hole\PYZus{}marker}\PY{o}{.}\PY{n}{set\PYZus{}data}\PY{p}{(}\PY{p}{[}\PY{p}{]}\PY{p}{,} \PY{p}{[}\PY{p}{]}\PY{p}{)}
        \PY{n}{slime\PYZus{}planet\PYZus{}circle}\PY{o}{.}\PY{n}{center} \PY{o}{=} \PY{p}{(}\PY{l+m+mi}{0}\PY{p}{,} \PY{l+m+mi}{0}\PY{p}{)}
        
        \PY{k}{def} \PY{n+nf}{set\PYZus{}quiver}\PY{p}{(}\PY{n}{q}\PY{p}{)}\PY{p}{:}
\PY{+w}{            }\PY{l+s+sd}{\PYZdq{}\PYZdq{}\PYZdq{}}
\PY{l+s+sd}{            Initializes a quiver as the zero vector.}
\PY{l+s+sd}{            \PYZdq{}\PYZdq{}\PYZdq{}}
            \PY{n}{q}\PY{o}{.}\PY{n}{set\PYZus{}offsets}\PY{p}{(}\PY{p}{[}\PY{p}{[}\PY{l+m+mi}{0}\PY{p}{,}\PY{l+m+mi}{0}\PY{p}{]}\PY{p}{]}\PY{p}{)}
            \PY{n}{q}\PY{o}{.}\PY{n}{set\PYZus{}UVC}\PY{p}{(}\PY{l+m+mi}{0}\PY{p}{,}\PY{l+m+mi}{0}\PY{p}{)}
        
        \PY{c+c1}{\PYZsh{} Init all of the quivers}
        \PY{n}{set\PYZus{}quiver}\PY{p}{(}\PY{n}{space\PYZus{}ship\PYZus{}quiver}\PY{p}{)}
        \PY{n}{set\PYZus{}quiver}\PY{p}{(}\PY{n}{slime\PYZus{}planet\PYZus{}quiver}\PY{p}{)}
        \PY{n}{set\PYZus{}quiver}\PY{p}{(}\PY{n}{space\PYZus{}ship\PYZus{}acceleration\PYZus{}quiver}\PY{p}{)}
        \PY{n}{set\PYZus{}quiver}\PY{p}{(}\PY{n}{slime\PYZus{}planet\PYZus{}acceleration\PYZus{}quiver}\PY{p}{)}
        \PY{n}{set\PYZus{}quiver}\PY{p}{(}\PY{n}{collision\PYZus{}direction\PYZus{}quiver}\PY{p}{)}
        \PY{n}{set\PYZus{}quiver}\PY{p}{(}\PY{n}{collision\PYZus{}force\PYZus{}quiver}\PY{p}{)}
        
        \PY{k}{return} \PY{n}{space\PYZus{}ship\PYZus{}marker}\PY{p}{,} \PY{n}{slime\PYZus{}planet\PYZus{}circle}\PY{p}{,} \PY{n}{black\PYZus{}hole\PYZus{}marker}\PY{p}{,} \PYZbs{}
            \PY{n}{space\PYZus{}ship\PYZus{}quiver}\PY{p}{,} \PY{n}{slime\PYZus{}planet\PYZus{}quiver}\PY{p}{,} \PY{n}{space\PYZus{}ship\PYZus{}acceleration\PYZus{}quiver}\PY{p}{,} \PYZbs{}
            \PY{n}{slime\PYZus{}planet\PYZus{}acceleration\PYZus{}quiver}\PY{p}{,} \PY{n}{fuel\PYZus{}text}\PY{p}{,} \PY{n}{collision\PYZus{}indicator\PYZus{}text}\PY{p}{,} \PYZbs{}
            \PY{n}{collision\PYZus{}direction\PYZus{}quiver}\PY{p}{,} \PY{n}{collision\PYZus{}force\PYZus{}quiver}\PY{p}{,} \PY{n}{force\PYZus{}indicator\PYZus{}text}\PY{p}{,} \PY{n}{time\PYZus{}text}

    \PY{k}{def} \PY{n+nf}{update}\PY{p}{(}\PY{n}{frame}\PY{p}{)}\PY{p}{:}
        \PY{k}{def} \PY{n+nf}{set\PYZus{}position}\PY{p}{(}\PY{n}{marker}\PY{p}{,} \PY{n}{trajectory}\PY{p}{)}\PY{p}{:}
            \PY{n}{marker}\PY{o}{.}\PY{n}{set\PYZus{}data}\PY{p}{(}\PY{p}{[}\PY{n}{trajectory}\PY{p}{[}\PY{n}{frame}\PY{p}{,}\PY{l+m+mi}{0}\PY{p}{]}\PY{p}{]}\PY{p}{,} \PY{p}{[}\PY{n}{trajectory}\PY{p}{[}\PY{n}{frame}\PY{p}{,}\PY{l+m+mi}{1}\PY{p}{]}\PY{p}{]}\PY{p}{)}
        \PY{k}{def} \PY{n+nf}{set\PYZus{}arrow}\PY{p}{(}\PY{n}{quiver}\PY{p}{,} \PY{n}{trajectory}\PY{p}{,} \PY{n}{vector}\PY{p}{)}\PY{p}{:}
            \PY{n}{quiver}\PY{o}{.}\PY{n}{set\PYZus{}offsets}\PY{p}{(}\PY{p}{[}\PY{p}{[}\PY{n}{trajectory}\PY{p}{[}\PY{n}{frame}\PY{p}{,}\PY{l+m+mi}{0}\PY{p}{]}\PY{p}{,} \PY{n}{trajectory}\PY{p}{[}\PY{n}{frame}\PY{p}{,}\PY{l+m+mi}{1}\PY{p}{]}\PY{p}{]}\PY{p}{]}\PY{p}{)}
            \PY{n}{quiver}\PY{o}{.}\PY{n}{set\PYZus{}UVC}\PY{p}{(}\PY{n}{vector}\PY{p}{[}\PY{n}{frame}\PY{p}{,}\PY{l+m+mi}{0}\PY{p}{]}\PY{p}{,} \PY{n}{vector}\PY{p}{[}\PY{n}{frame}\PY{p}{,}\PY{l+m+mi}{1}\PY{p}{]}\PY{p}{)}
            
        \PY{n}{set\PYZus{}position}\PY{p}{(}\PY{n}{space\PYZus{}ship\PYZus{}marker}\PY{p}{,}   \PY{n}{space\PYZus{}ship\PYZus{}trajectory}\PY{p}{)}
        \PY{n}{slime\PYZus{}planet\PYZus{}circle}\PY{o}{.}\PY{n}{center} \PY{o}{=} \PY{p}{(}\PY{n}{slime\PYZus{}planet\PYZus{}trajectory}\PY{p}{[}\PY{n}{frame}\PY{p}{,} \PY{l+m+mi}{0}\PY{p}{]}\PY{p}{,} \PY{n}{slime\PYZus{}planet\PYZus{}trajectory}\PY{p}{[}\PY{n}{frame}\PY{p}{,} \PY{l+m+mi}{1}\PY{p}{]}\PY{p}{)}
        \PY{n}{black\PYZus{}hole\PYZus{}marker}\PY{o}{.}\PY{n}{set\PYZus{}data}\PY{p}{(}\PY{p}{[}\PY{l+m+mi}{0}\PY{p}{]}\PY{p}{,}\PY{p}{[}\PY{l+m+mi}{0}\PY{p}{]}\PY{p}{)}
        
        \PY{k}{if} \PY{n}{display\PYZus{}arrows}\PY{p}{:}
            \PY{c+c1}{\PYZsh{} plot the velocity arrows}
            \PY{n}{set\PYZus{}arrow}\PY{p}{(}\PY{n}{space\PYZus{}ship\PYZus{}quiver}\PY{p}{,}   \PY{n}{space\PYZus{}ship\PYZus{}trajectory}\PY{p}{,}   \PY{n}{space\PYZus{}ship\PYZus{}velocity}\PY{p}{)}
            \PY{n}{set\PYZus{}arrow}\PY{p}{(}\PY{n}{slime\PYZus{}planet\PYZus{}quiver}\PY{p}{,} \PY{n}{slime\PYZus{}planet\PYZus{}trajectory}\PY{p}{,} \PY{n}{slime\PYZus{}planet\PYZus{}velocity}\PY{p}{)}

            \PY{c+c1}{\PYZsh{} plot the acceleration arrows}
            \PY{n}{set\PYZus{}arrow}\PY{p}{(}\PY{n}{space\PYZus{}ship\PYZus{}acceleration\PYZus{}quiver}\PY{p}{,}   \PY{n}{space\PYZus{}ship\PYZus{}trajectory}\PY{p}{,}   \PY{n}{space\PYZus{}ship\PYZus{}acceleration}\PY{p}{)}
            \PY{n}{set\PYZus{}arrow}\PY{p}{(}\PY{n}{slime\PYZus{}planet\PYZus{}acceleration\PYZus{}quiver}\PY{p}{,} \PY{n}{slime\PYZus{}planet\PYZus{}trajectory}\PY{p}{,} \PY{n}{slime\PYZus{}planet\PYZus{}acceleration}\PY{p}{)}
            
            \PY{c+c1}{\PYZsh{} plot the collision arrows}
            \PY{k}{if} \PY{n}{simulation\PYZus{}result}\PY{o}{.}\PY{n}{has\PYZus{}collision}\PY{p}{[}\PY{n}{frame}\PY{p}{]}\PY{p}{:}
                \PY{n}{set\PYZus{}arrow}\PY{p}{(}\PY{n}{collision\PYZus{}direction\PYZus{}quiver}\PY{p}{,} \PY{n}{slime\PYZus{}planet\PYZus{}trajectory}\PY{p}{,} \PY{n}{collision\PYZus{}directions}\PY{p}{)}
                \PY{n}{set\PYZus{}arrow}\PY{p}{(}\PY{n}{collision\PYZus{}force\PYZus{}quiver}\PY{p}{,}     \PY{n}{space\PYZus{}ship\PYZus{}trajectory}\PY{p}{,}   \PY{n}{collision\PYZus{}directions\PYZus{}units}\PY{p}{)}
            \PY{k}{else}\PY{p}{:}
                \PY{n}{set\PYZus{}arrow}\PY{p}{(}\PY{n}{collision\PYZus{}direction\PYZus{}quiver}\PY{p}{,} \PY{n}{slime\PYZus{}planet\PYZus{}trajectory}\PY{p}{,} \PY{n}{zero\PYZus{}vector}\PY{p}{)}
                \PY{n}{set\PYZus{}arrow}\PY{p}{(}\PY{n}{collision\PYZus{}force\PYZus{}quiver}\PY{p}{,}     \PY{n}{space\PYZus{}ship\PYZus{}trajectory}\PY{p}{,}   \PY{n}{zero\PYZus{}vector}\PY{p}{)}
        
        \PY{n}{fuel\PYZus{}text}\PY{o}{.}\PY{n}{set\PYZus{}text}\PY{p}{(}\PY{l+s+sa}{f}\PY{l+s+s1}{\PYZsq{}}\PY{l+s+s1}{Remaining Fuel: }\PY{l+s+si}{\PYZob{}}\PY{n}{space\PYZus{}ship\PYZus{}fuel}\PY{p}{[}\PY{n}{frame}\PY{p}{]}\PY{l+s+si}{:}\PY{l+s+s1}{.2e}\PY{l+s+si}{\PYZcb{}}\PY{l+s+s1}{ kg}\PY{l+s+s1}{\PYZsq{}}\PY{p}{)}
        \PY{n}{force\PYZus{}indicator\PYZus{}text}\PY{o}{.}\PY{n}{set\PYZus{}text}\PY{p}{(}\PY{l+s+sa}{f}\PY{l+s+s1}{\PYZsq{}}\PY{l+s+s1}{Force: }\PY{l+s+si}{\PYZob{}}\PY{n}{space\PYZus{}ship\PYZus{}force}\PY{p}{[}\PY{n}{frame}\PY{p}{]}\PY{l+s+si}{:}\PY{l+s+s1}{.2e}\PY{l+s+si}{\PYZcb{}}\PY{l+s+s1}{ N}\PY{l+s+s1}{\PYZsq{}}\PY{p}{)}
        \PY{n}{time\PYZus{}text}\PY{o}{.}\PY{n}{set\PYZus{}text}\PY{p}{(}\PY{l+s+sa}{f}\PY{l+s+s1}{\PYZsq{}}\PY{l+s+s1}{Time: }\PY{l+s+si}{\PYZob{}}\PY{n}{time\PYZus{}data}\PY{p}{[}\PY{n}{frame}\PY{p}{]}\PY{l+s+si}{:}\PY{l+s+s1}{.2f}\PY{l+s+si}{\PYZcb{}}\PY{l+s+s1}{ s}\PY{l+s+s1}{\PYZsq{}}\PY{p}{)}
        
        \PY{k}{if} \PY{n}{simulation\PYZus{}result}\PY{o}{.}\PY{n}{has\PYZus{}collision}\PY{p}{[}\PY{n}{frame}\PY{p}{]}\PY{p}{:}
            \PY{n}{collision\PYZus{}indicator\PYZus{}text}\PY{o}{.}\PY{n}{set\PYZus{}text}\PY{p}{(}\PY{l+s+s1}{\PYZsq{}}\PY{l+s+s1}{Collision: Yes}\PY{l+s+s1}{\PYZsq{}}\PY{p}{)}
        \PY{k}{else}\PY{p}{:}
            \PY{n}{collision\PYZus{}indicator\PYZus{}text}\PY{o}{.}\PY{n}{set\PYZus{}text}\PY{p}{(}\PY{l+s+s1}{\PYZsq{}}\PY{l+s+s1}{Collision: No}\PY{l+s+s1}{\PYZsq{}}\PY{p}{)}
        
        \PY{k}{return} \PY{n}{space\PYZus{}ship\PYZus{}marker}\PY{p}{,} \PY{n}{slime\PYZus{}planet\PYZus{}circle}\PY{p}{,} \PY{n}{black\PYZus{}hole\PYZus{}marker}\PY{p}{,} \PYZbs{}
            \PY{n}{space\PYZus{}ship\PYZus{}quiver}\PY{p}{,} \PY{n}{slime\PYZus{}planet\PYZus{}quiver}\PY{p}{,} \PY{n}{space\PYZus{}ship\PYZus{}acceleration\PYZus{}quiver}\PY{p}{,} \PYZbs{}
            \PY{n}{slime\PYZus{}planet\PYZus{}acceleration\PYZus{}quiver}\PY{p}{,} \PY{n}{fuel\PYZus{}text}\PY{p}{,} \PY{n}{collision\PYZus{}indicator\PYZus{}text}\PY{p}{,} \PYZbs{}
            \PY{n}{collision\PYZus{}direction\PYZus{}quiver}\PY{p}{,} \PY{n}{collision\PYZus{}force\PYZus{}quiver}\PY{p}{,} \PY{n}{time\PYZus{}text}

    \PY{n+nb}{print}\PY{p}{(}\PY{l+s+s1}{\PYZsq{}}\PY{l+s+s1}{Animating simulation..}\PY{l+s+s1}{\PYZsq{}}\PY{p}{)}
    
    \PY{n}{ani} \PY{o}{=} \PY{n}{FuncAnimation}\PY{p}{(}\PY{n}{fig}\PY{p}{,} \PY{n}{update}\PY{p}{,} \PY{n}{frames}\PY{o}{=}\PY{n+nb}{range}\PY{p}{(}\PY{l+m+mi}{0}\PY{p}{,} \PY{n+nb}{len}\PY{p}{(}\PY{n}{time\PYZus{}data}\PY{p}{)}\PY{p}{,} \PY{l+m+mi}{5}\PY{p}{)}\PY{p}{,}
                        \PY{n}{init\PYZus{}func}\PY{o}{=}\PY{n}{init}\PY{p}{,} \PY{n}{blit}\PY{o}{=}\PY{k+kc}{True}\PY{p}{,} \PY{n}{interval}\PY{o}{=}\PY{l+m+mi}{20}\PY{p}{)}

    \PY{n+nb}{print}\PY{p}{(}\PY{l+s+s1}{\PYZsq{}}\PY{l+s+s1}{Animation complete! Writing..}\PY{l+s+s1}{\PYZsq{}}\PY{p}{)}
    
    \PY{n}{start} \PY{o}{=} \PY{n}{time}\PY{o}{.}\PY{n}{time}\PY{p}{(}\PY{p}{)}
    \PY{n}{result} \PY{o}{=} \PY{n}{HTML}\PY{p}{(}\PY{n}{ani}\PY{o}{.}\PY{n}{to\PYZus{}html5\PYZus{}video}\PY{p}{(}\PY{p}{)}\PY{p}{)}
    \PY{n}{end} \PY{o}{=} \PY{n}{time}\PY{o}{.}\PY{n}{time}\PY{p}{(}\PY{p}{)}
    
    \PY{n+nb}{print}\PY{p}{(}\PY{l+s+sa}{f}\PY{l+s+s2}{\PYZdq{}}\PY{l+s+s2}{Elapsed time: }\PY{l+s+si}{\PYZob{}}\PY{n}{end}\PY{+w}{ }\PY{o}{\PYZhy{}}\PY{+w}{ }\PY{n}{start}\PY{l+s+si}{:}\PY{l+s+s2}{.1f}\PY{l+s+si}{\PYZcb{}}\PY{l+s+s2}{ seconds}\PY{l+s+s2}{\PYZdq{}}\PY{p}{)}
    
    \PY{n}{plt}\PY{o}{.}\PY{n}{close}\PY{p}{(}\PY{p}{)}
    
    \PY{k}{return} \PY{n}{result}
\end{Verbatim}
\end{tcolorbox}

    \textbf{Code Block Summary:} This block plots all of the results
\emph{of a singular simulation}: keeping track of the position (measured
as the scalar distance to the black hole), speed, acceleration, kinetic
energy, momentum, and remaining fuel mass.

    \begin{tcolorbox}[breakable, size=fbox, boxrule=1pt, pad at break*=1mm,colback=cellbackground, colframe=cellborder]
\prompt{In}{incolor}{20}{\boxspacing}
\begin{Verbatim}[commandchars=\\\{\}]
\PY{n+nd}{@dataclass}
\PY{k}{class} \PY{n+nc}{SimulationPlots}\PY{p}{:}
    \PY{n}{collision\PYZus{}graph}\PY{p}{:}                   \PY{n}{Image} \PY{o}{=} \PY{k+kc}{None}
    \PY{n}{distance\PYZus{}speed\PYZus{}acceleration\PYZus{}graph}\PY{p}{:} \PY{n}{Image} \PY{o}{=} \PY{k+kc}{None}
    \PY{n}{kinetic\PYZus{}energy\PYZus{}momentum\PYZus{}graph}\PY{p}{:}     \PY{n}{Image} \PY{o}{=} \PY{k+kc}{None}
    \PY{n}{trajectory\PYZus{}graph}\PY{p}{:}                  \PY{n}{Image} \PY{o}{=} \PY{k+kc}{None}
    
\PY{k}{def} \PY{n+nf}{save\PYZus{}graph\PYZus{}and\PYZus{}close}\PY{p}{(}\PY{p}{)}\PY{p}{:}
    \PY{n}{plt}\PY{o}{.}\PY{n}{tight\PYZus{}layout}\PY{p}{(}\PY{p}{)}
    \PY{n}{buf} \PY{o}{=} \PY{n}{io}\PY{o}{.}\PY{n}{BytesIO}\PY{p}{(}\PY{p}{)}
    \PY{n}{plt}\PY{o}{.}\PY{n}{savefig}\PY{p}{(}\PY{n}{buf}\PY{p}{,} \PY{n+nb}{format}\PY{o}{=}\PY{l+s+s1}{\PYZsq{}}\PY{l+s+s1}{png}\PY{l+s+s1}{\PYZsq{}}\PY{p}{)}
    \PY{n}{buf}\PY{o}{.}\PY{n}{seek}\PY{p}{(}\PY{l+m+mi}{0}\PY{p}{)}
    \PY{n}{graph} \PY{o}{=} \PY{n}{Image}\PY{o}{.}\PY{n}{open}\PY{p}{(}\PY{n}{buf}\PY{p}{)}
    \PY{n}{plt}\PY{o}{.}\PY{n}{close}\PY{p}{(}\PY{p}{)}
    \PY{k}{return} \PY{n}{graph}
    
\PY{k}{def} \PY{n+nf}{plot\PYZus{}results}\PY{p}{(}\PY{n}{angle}\PY{p}{,} \PY{n}{initial\PYZus{}speed}\PY{p}{,} \PY{n}{time\PYZus{}data}\PY{p}{,} \PY{n}{slime\PYZus{}planet}\PY{p}{,} \PY{n}{space\PYZus{}ship}\PY{p}{,} \PY{n}{black\PYZus{}hole}\PY{p}{,} \PY{n}{collision\PYZus{}frames}\PY{p}{)}\PY{p}{:}
    \PY{k}{def} \PY{n+nf}{set\PYZus{}axes}\PY{p}{(}\PY{n}{variable\PYZus{}name}\PY{p}{,} \PY{n}{units}\PY{p}{)}\PY{p}{:}
        \PY{n}{plt}\PY{o}{.}\PY{n}{xlabel}\PY{p}{(}\PY{l+s+s1}{\PYZsq{}}\PY{l+s+s1}{Time (s)}\PY{l+s+s1}{\PYZsq{}}\PY{p}{)}
        \PY{n}{plt}\PY{o}{.}\PY{n}{ylabel}\PY{p}{(}\PY{l+s+sa}{f}\PY{l+s+s1}{\PYZsq{}}\PY{l+s+si}{\PYZob{}}\PY{n}{variable\PYZus{}name}\PY{l+s+si}{\PYZcb{}}\PY{l+s+s1}{ (}\PY{l+s+si}{\PYZob{}}\PY{n}{units}\PY{l+s+si}{\PYZcb{}}\PY{l+s+s1}{)}\PY{l+s+s1}{\PYZsq{}}\PY{p}{)}
        \PY{n}{plt}\PY{o}{.}\PY{n}{title}\PY{p}{(}\PY{l+s+sa}{f}\PY{l+s+s1}{\PYZsq{}}\PY{l+s+s1}{Sim. (θ=}\PY{l+s+si}{\PYZob{}}\PY{n}{angle}\PY{l+s+si}{:}\PY{l+s+s1}{.1f}\PY{l+s+si}{\PYZcb{}}\PY{l+s+s1}{ rad, \PYZdl{}v\PYZus{}0\PYZdl{}=}\PY{l+s+si}{\PYZob{}}\PY{n}{initial\PYZus{}speed}\PY{l+s+si}{:}\PY{l+s+s1}{.1f}\PY{l+s+si}{\PYZcb{}}\PY{l+s+s1}{ m/s) \PYZhy{} }\PY{l+s+si}{\PYZob{}}\PY{n}{variable\PYZus{}name}\PY{l+s+si}{\PYZcb{}}\PY{l+s+s1}{ vs Time}\PY{l+s+s1}{\PYZsq{}}\PY{p}{)}
        \PY{n}{plt}\PY{o}{.}\PY{n}{legend}\PY{p}{(}\PY{p}{)}
    
    \PY{n}{plots}\PY{o}{=}\PY{n}{SimulationPlots}\PY{p}{(}\PY{p}{)}
    
    \PY{c+c1}{\PYZsh{} combined distance, speed, acceleration, and fuel graph}
    \PY{n}{fig}\PY{p}{,} \PY{n}{axs} \PY{o}{=} \PY{n}{plt}\PY{o}{.}\PY{n}{subplots}\PY{p}{(}\PY{l+m+mi}{4}\PY{p}{,}\PY{l+m+mi}{1}\PY{p}{,} \PY{n}{sharex}\PY{o}{=}\PY{k+kc}{True}\PY{p}{,} \PY{n}{figsize}\PY{o}{=}\PY{p}{(}\PY{l+m+mi}{8}\PY{p}{,}\PY{l+m+mi}{8}\PY{p}{)}\PY{p}{)}
    \PY{n}{fig}\PY{o}{.}\PY{n}{subplots\PYZus{}adjust}\PY{p}{(}\PY{n}{hspace}\PY{o}{=}\PY{l+m+mi}{0}\PY{p}{)}
    
    \PY{n}{axs}\PY{p}{[}\PY{l+m+mi}{0}\PY{p}{]}\PY{o}{.}\PY{n}{set\PYZus{}title}\PY{p}{(}\PY{l+s+s1}{\PYZsq{}}\PY{l+s+s1}{Radial Distance, Speed,}\PY{l+s+se}{\PYZbs{}n}\PY{l+s+s1}{Acceleration, and Fuel vs. Time}\PY{l+s+s1}{\PYZsq{}}\PY{p}{)}
    
    \PY{n}{axs}\PY{p}{[}\PY{l+m+mi}{0}\PY{p}{]}\PY{o}{.}\PY{n}{set\PYZus{}ylabel}\PY{p}{(}\PY{l+s+s1}{\PYZsq{}}\PY{l+s+s1}{Radial}\PY{l+s+se}{\PYZbs{}n}\PY{l+s+s1}{Distance (m)}\PY{l+s+s1}{\PYZsq{}}\PY{p}{)}
    \PY{n}{axs}\PY{p}{[}\PY{l+m+mi}{0}\PY{p}{]}\PY{o}{.}\PY{n}{plot}\PY{p}{(}\PY{n}{time\PYZus{}data}\PY{p}{,} \PY{n}{magnitudes}\PY{p}{(}\PY{n}{space\PYZus{}ship}\PY{o}{.}\PY{n}{position}\PY{p}{)}\PY{p}{,}   \PY{n}{label}\PY{o}{=}\PY{l+s+s1}{\PYZsq{}}\PY{l+s+s1}{Space Ship}\PY{l+s+s1}{\PYZsq{}}\PY{p}{)}
    \PY{n}{axs}\PY{p}{[}\PY{l+m+mi}{0}\PY{p}{]}\PY{o}{.}\PY{n}{plot}\PY{p}{(}\PY{n}{time\PYZus{}data}\PY{p}{,} \PY{n}{magnitudes}\PY{p}{(}\PY{n}{slime\PYZus{}planet}\PY{o}{.}\PY{n}{position}\PY{p}{)}\PY{p}{,} \PY{n}{label}\PY{o}{=}\PY{l+s+s1}{\PYZsq{}}\PY{l+s+s1}{Slime Planet}\PY{l+s+s1}{\PYZsq{}}\PY{p}{)}
    \PY{n}{axs}\PY{p}{[}\PY{l+m+mi}{0}\PY{p}{]}\PY{o}{.}\PY{n}{legend}\PY{p}{(}\PY{p}{)}
    
    \PY{n}{axs}\PY{p}{[}\PY{l+m+mi}{1}\PY{p}{]}\PY{o}{.}\PY{n}{set\PYZus{}ylabel}\PY{p}{(}\PY{l+s+s1}{\PYZsq{}}\PY{l+s+s1}{Speed (m/s)}\PY{l+s+s1}{\PYZsq{}}\PY{p}{)}
    \PY{n}{axs}\PY{p}{[}\PY{l+m+mi}{1}\PY{p}{]}\PY{o}{.}\PY{n}{plot}\PY{p}{(}\PY{n}{time\PYZus{}data}\PY{p}{,} \PY{n}{magnitudes}\PY{p}{(}\PY{n}{space\PYZus{}ship}\PY{o}{.}\PY{n}{velocity}\PY{p}{)}\PY{p}{,}   \PY{n}{label}\PY{o}{=}\PY{l+s+s1}{\PYZsq{}}\PY{l+s+s1}{Space Ship}\PY{l+s+s1}{\PYZsq{}}\PY{p}{)}
    \PY{n}{axs}\PY{p}{[}\PY{l+m+mi}{1}\PY{p}{]}\PY{o}{.}\PY{n}{plot}\PY{p}{(}\PY{n}{time\PYZus{}data}\PY{p}{,} \PY{n}{magnitudes}\PY{p}{(}\PY{n}{slime\PYZus{}planet}\PY{o}{.}\PY{n}{velocity}\PY{p}{)}\PY{p}{,} \PY{n}{label}\PY{o}{=}\PY{l+s+s1}{\PYZsq{}}\PY{l+s+s1}{Slime Planet}\PY{l+s+s1}{\PYZsq{}}\PY{p}{)}
    \PY{n}{axs}\PY{p}{[}\PY{l+m+mi}{1}\PY{p}{]}\PY{o}{.}\PY{n}{legend}\PY{p}{(}\PY{p}{)}
    
    \PY{n}{axs}\PY{p}{[}\PY{l+m+mi}{2}\PY{p}{]}\PY{o}{.}\PY{n}{set\PYZus{}ylabel}\PY{p}{(}\PY{l+s+s1}{\PYZsq{}}\PY{l+s+s1}{Magnitude of}\PY{l+s+se}{\PYZbs{}n}\PY{l+s+s1}{Acceleration (m/\PYZdl{}s\PYZca{}2\PYZdl{})}\PY{l+s+s1}{\PYZsq{}}\PY{p}{)}
    \PY{n}{axs}\PY{p}{[}\PY{l+m+mi}{2}\PY{p}{]}\PY{o}{.}\PY{n}{plot}\PY{p}{(}\PY{n}{time\PYZus{}data}\PY{p}{,} \PY{n}{magnitudes}\PY{p}{(}\PY{n}{space\PYZus{}ship}\PY{o}{.}\PY{n}{acceleration}\PY{p}{)}\PY{p}{,}   \PY{n}{label}\PY{o}{=}\PY{l+s+s1}{\PYZsq{}}\PY{l+s+s1}{Space Ship}\PY{l+s+s1}{\PYZsq{}}\PY{p}{)}
    \PY{n}{axs}\PY{p}{[}\PY{l+m+mi}{2}\PY{p}{]}\PY{o}{.}\PY{n}{plot}\PY{p}{(}\PY{n}{time\PYZus{}data}\PY{p}{,} \PY{n}{magnitudes}\PY{p}{(}\PY{n}{slime\PYZus{}planet}\PY{o}{.}\PY{n}{acceleration}\PY{p}{)}\PY{p}{,} \PY{n}{label}\PY{o}{=}\PY{l+s+s1}{\PYZsq{}}\PY{l+s+s1}{Slime Planet}\PY{l+s+s1}{\PYZsq{}}\PY{p}{)}
    \PY{n}{axs}\PY{p}{[}\PY{l+m+mi}{2}\PY{p}{]}\PY{o}{.}\PY{n}{legend}\PY{p}{(}\PY{p}{)}
    
    \PY{n}{axs}\PY{p}{[}\PY{l+m+mi}{3}\PY{p}{]}\PY{o}{.}\PY{n}{set\PYZus{}ylabel}\PY{p}{(}\PY{l+s+s1}{\PYZsq{}}\PY{l+s+s1}{Remaining Fuel (kg)}\PY{l+s+s1}{\PYZsq{}}\PY{p}{)}
    \PY{n}{axs}\PY{p}{[}\PY{l+m+mi}{3}\PY{p}{]}\PY{o}{.}\PY{n}{plot}\PY{p}{(}\PY{n}{time\PYZus{}data}\PY{p}{,} \PY{n}{space\PYZus{}ship}\PY{o}{.}\PY{n}{remaining\PYZus{}fuel\PYZus{}mass}\PY{p}{)}
    \PY{n}{axs}\PY{p}{[}\PY{l+m+mi}{3}\PY{p}{]}\PY{o}{.}\PY{n}{set\PYZus{}xlabel}\PY{p}{(}\PY{l+s+s1}{\PYZsq{}}\PY{l+s+s1}{Time (s)}\PY{l+s+s1}{\PYZsq{}}\PY{p}{)}
    
    \PY{n}{plots}\PY{o}{.}\PY{n}{distance\PYZus{}speed\PYZus{}acceleration\PYZus{}graph} \PY{o}{=} \PY{n}{save\PYZus{}graph\PYZus{}and\PYZus{}close}\PY{p}{(}\PY{p}{)}
    
    \PY{c+c1}{\PYZsh{} combined kinetic energy momentum graph}
    
    \PY{n}{fig}\PY{p}{,} \PY{n}{axs} \PY{o}{=} \PY{n}{plt}\PY{o}{.}\PY{n}{subplots}\PY{p}{(}\PY{l+m+mi}{2}\PY{p}{,}\PY{l+m+mi}{1}\PY{p}{,} \PY{n}{sharex}\PY{o}{=}\PY{k+kc}{False}\PY{p}{,} \PY{n}{figsize}\PY{o}{=}\PY{p}{(}\PY{l+m+mi}{8}\PY{p}{,}\PY{l+m+mi}{6}\PY{p}{)}\PY{p}{)}
    
    \PY{n}{axs}\PY{p}{[}\PY{l+m+mi}{0}\PY{p}{]}\PY{o}{.}\PY{n}{set\PYZus{}title}\PY{p}{(}\PY{l+s+s1}{\PYZsq{}}\PY{l+s+s1}{Kinetic Energy vs. Time}\PY{l+s+s1}{\PYZsq{}}\PY{p}{)}
    \PY{n}{axs}\PY{p}{[}\PY{l+m+mi}{0}\PY{p}{]}\PY{o}{.}\PY{n}{set\PYZus{}ylabel}\PY{p}{(}\PY{l+s+s1}{\PYZsq{}}\PY{l+s+s1}{Kinetic Energy (J)}\PY{l+s+s1}{\PYZsq{}}\PY{p}{)}
    \PY{n}{axs}\PY{p}{[}\PY{l+m+mi}{0}\PY{p}{]}\PY{o}{.}\PY{n}{plot}\PY{p}{(}\PY{n}{time\PYZus{}data}\PY{p}{,} \PY{n}{space\PYZus{}ship}\PY{o}{.}\PY{n}{kinetic\PYZus{}energy}\PY{p}{,}   \PY{n}{label}\PY{o}{=}\PY{l+s+s1}{\PYZsq{}}\PY{l+s+s1}{Space Ship}\PY{l+s+s1}{\PYZsq{}}\PY{p}{)}
    \PY{n}{axs}\PY{p}{[}\PY{l+m+mi}{0}\PY{p}{]}\PY{o}{.}\PY{n}{plot}\PY{p}{(}\PY{n}{time\PYZus{}data}\PY{p}{,} \PY{n}{slime\PYZus{}planet}\PY{o}{.}\PY{n}{kinetic\PYZus{}energy}\PY{p}{,} \PY{n}{label}\PY{o}{=}\PY{l+s+s1}{\PYZsq{}}\PY{l+s+s1}{Slime Planet}\PY{l+s+s1}{\PYZsq{}}\PY{p}{)}
    \PY{n}{axs}\PY{p}{[}\PY{l+m+mi}{0}\PY{p}{]}\PY{o}{.}\PY{n}{plot}\PY{p}{(}\PY{n}{time\PYZus{}data}\PY{p}{,} \PY{n}{space\PYZus{}ship}\PY{o}{.}\PY{n}{kinetic\PYZus{}energy} \PY{o}{+} \PY{n}{slime\PYZus{}planet}\PY{o}{.}\PY{n}{kinetic\PYZus{}energy}\PY{p}{,} \PY{n}{label}\PY{o}{=}\PY{l+s+s1}{\PYZsq{}}\PY{l+s+s1}{Total}\PY{l+s+s1}{\PYZsq{}}\PY{p}{)}
    \PY{n}{axs}\PY{p}{[}\PY{l+m+mi}{0}\PY{p}{]}\PY{o}{.}\PY{n}{legend}\PY{p}{(}\PY{p}{)}
    
    \PY{n}{axs}\PY{p}{[}\PY{l+m+mi}{1}\PY{p}{]}\PY{o}{.}\PY{n}{set\PYZus{}title}\PY{p}{(}\PY{l+s+s1}{\PYZsq{}}\PY{l+s+s1}{Magnitude of Momentum vs. Time}\PY{l+s+s1}{\PYZsq{}}\PY{p}{)}
    \PY{n}{axs}\PY{p}{[}\PY{l+m+mi}{1}\PY{p}{]}\PY{o}{.}\PY{n}{set\PYZus{}ylabel}\PY{p}{(}\PY{l+s+s1}{\PYZsq{}}\PY{l+s+s1}{Magnitude of}\PY{l+s+se}{\PYZbs{}n}\PY{l+s+s1}{Momentum (kg*m/s)}\PY{l+s+s1}{\PYZsq{}}\PY{p}{)}
    \PY{n}{axs}\PY{p}{[}\PY{l+m+mi}{1}\PY{p}{]}\PY{o}{.}\PY{n}{plot}\PY{p}{(}\PY{n}{time\PYZus{}data}\PY{p}{,} \PY{n}{magnitudes}\PY{p}{(}\PY{n}{space\PYZus{}ship}\PY{o}{.}\PY{n}{momentum}\PY{p}{)}\PY{p}{,}   \PY{n}{label}\PY{o}{=}\PY{l+s+s1}{\PYZsq{}}\PY{l+s+s1}{Space Ship}\PY{l+s+s1}{\PYZsq{}}\PY{p}{)}
    \PY{n}{axs}\PY{p}{[}\PY{l+m+mi}{1}\PY{p}{]}\PY{o}{.}\PY{n}{plot}\PY{p}{(}\PY{n}{time\PYZus{}data}\PY{p}{,} \PY{n}{magnitudes}\PY{p}{(}\PY{n}{slime\PYZus{}planet}\PY{o}{.}\PY{n}{momentum}\PY{p}{)}\PY{p}{,} \PY{n}{label}\PY{o}{=}\PY{l+s+s1}{\PYZsq{}}\PY{l+s+s1}{Slime Planet}\PY{l+s+s1}{\PYZsq{}}\PY{p}{)}
    \PY{n}{axs}\PY{p}{[}\PY{l+m+mi}{1}\PY{p}{]}\PY{o}{.}\PY{n}{plot}\PY{p}{(}\PY{n}{time\PYZus{}data}\PY{p}{,} \PY{n}{magnitudes}\PY{p}{(}\PY{n}{space\PYZus{}ship}\PY{o}{.}\PY{n}{momentum} \PY{o}{+} \PY{n}{slime\PYZus{}planet}\PY{o}{.}\PY{n}{momentum}\PY{p}{)}\PY{p}{,} \PY{n}{label}\PY{o}{=}\PY{l+s+s1}{\PYZsq{}}\PY{l+s+s1}{Total}\PY{l+s+s1}{\PYZsq{}}\PY{p}{)}
    \PY{n}{axs}\PY{p}{[}\PY{l+m+mi}{1}\PY{p}{]}\PY{o}{.}\PY{n}{set\PYZus{}xlabel}\PY{p}{(}\PY{l+s+s1}{\PYZsq{}}\PY{l+s+s1}{Time (s)}\PY{l+s+s1}{\PYZsq{}}\PY{p}{)}
    \PY{n}{axs}\PY{p}{[}\PY{l+m+mi}{1}\PY{p}{]}\PY{o}{.}\PY{n}{legend}\PY{p}{(}\PY{p}{)}
    
    \PY{n}{plots}\PY{o}{.}\PY{n}{kinetic\PYZus{}energy\PYZus{}momentum\PYZus{}graph} \PY{o}{=} \PY{n}{save\PYZus{}graph\PYZus{}and\PYZus{}close}\PY{p}{(}\PY{p}{)}
    
    \PY{c+c1}{\PYZsh{} collision plot}
    
    \PY{n}{collision\PYZus{}time} \PY{o}{=} \PY{n}{time\PYZus{}data}\PY{p}{[}\PY{n}{collision\PYZus{}frames}\PY{p}{]}
    \PY{n}{n\PYZus{}collision\PYZus{}fames} \PY{o}{=} \PY{n+nb}{len}\PY{p}{(}\PY{n}{collision\PYZus{}time}\PY{p}{)} 
    
    \PY{k}{if} \PY{n}{n\PYZus{}collision\PYZus{}fames} \PY{o}{\PYZgt{}} \PY{l+m+mi}{0}\PY{p}{:}
        \PY{n}{fig}\PY{p}{,} \PY{n}{axs} \PY{o}{=} \PY{n}{plt}\PY{o}{.}\PY{n}{subplots}\PY{p}{(}\PY{l+m+mi}{2}\PY{p}{,}\PY{l+m+mi}{1}\PY{p}{,} \PY{n}{sharex}\PY{o}{=}\PY{k+kc}{False}\PY{p}{,} \PY{n}{figsize}\PY{o}{=}\PY{p}{(}\PY{l+m+mi}{8}\PY{p}{,}\PY{l+m+mi}{8}\PY{p}{)}\PY{p}{)}

        \PY{n}{min\PYZus{}len} \PY{o}{=} \PY{n+nb}{min}\PY{p}{(}
            \PY{n+nb}{len}\PY{p}{(}\PY{n}{collision\PYZus{}time}\PY{p}{)}\PY{p}{,}
            \PY{n+nb}{len}\PY{p}{(}\PY{n}{space\PYZus{}ship}\PY{o}{.}\PY{n}{collision\PYZus{}force}\PY{p}{)}\PY{p}{,}
            \PY{n+nb}{len}\PY{p}{(}\PY{n}{space\PYZus{}ship}\PY{o}{.}\PY{n}{collision\PYZus{}impulse}\PY{p}{)}
        \PY{p}{)}
        \PY{n}{collision\PYZus{}time} \PY{o}{=} \PY{n}{collision\PYZus{}time}\PY{p}{[}\PY{p}{:}\PY{n}{min\PYZus{}len}\PY{p}{]}
        \PY{n}{space\PYZus{}ship}\PY{o}{.}\PY{n}{collision\PYZus{}force} \PY{o}{=} \PY{n}{space\PYZus{}ship}\PY{o}{.}\PY{n}{collision\PYZus{}force}\PY{p}{[}\PY{p}{:}\PY{n}{min\PYZus{}len}\PY{p}{]}
        \PY{n}{space\PYZus{}ship}\PY{o}{.}\PY{n}{collision\PYZus{}impulse} \PY{o}{=} \PY{n}{space\PYZus{}ship}\PY{o}{.}\PY{n}{collision\PYZus{}impulse}\PY{p}{[}\PY{p}{:}\PY{n}{min\PYZus{}len}\PY{p}{]}
        
        \PY{n}{axs}\PY{p}{[}\PY{l+m+mi}{0}\PY{p}{]}\PY{o}{.}\PY{n}{set\PYZus{}title}\PY{p}{(}\PY{l+s+s1}{\PYZsq{}}\PY{l+s+s1}{Collision: Force vs. Time}\PY{l+s+s1}{\PYZsq{}}\PY{p}{)}
        \PY{n}{axs}\PY{p}{[}\PY{l+m+mi}{0}\PY{p}{]}\PY{o}{.}\PY{n}{plot}\PY{p}{(}\PY{n}{collision\PYZus{}time}\PY{p}{,} \PY{n}{space\PYZus{}ship}\PY{o}{.}\PY{n}{collision\PYZus{}force}\PY{p}{[}\PY{p}{:}\PY{n}{n\PYZus{}collision\PYZus{}fames}\PY{p}{]}\PY{p}{,} \PY{n}{label}\PY{o}{=}\PY{l+s+s1}{\PYZsq{}}\PY{l+s+s1}{Normal Force}\PY{l+s+s1}{\PYZsq{}}\PY{p}{)}
        \PY{n}{axs}\PY{p}{[}\PY{l+m+mi}{0}\PY{p}{]}\PY{o}{.}\PY{n}{set\PYZus{}ylabel}\PY{p}{(}\PY{l+s+s1}{\PYZsq{}}\PY{l+s+s1}{Force (N)}\PY{l+s+s1}{\PYZsq{}}\PY{p}{)}
        \PY{n}{axs}\PY{p}{[}\PY{l+m+mi}{0}\PY{p}{]}\PY{o}{.}\PY{n}{set\PYZus{}xlabel}\PY{p}{(}\PY{l+s+s1}{\PYZsq{}}\PY{l+s+s1}{Time (s)}\PY{l+s+s1}{\PYZsq{}}\PY{p}{)}
        \PY{n}{axs}\PY{p}{[}\PY{l+m+mi}{0}\PY{p}{]}\PY{o}{.}\PY{n}{legend}\PY{p}{(}\PY{p}{)}
        
        \PY{n}{axs}\PY{p}{[}\PY{l+m+mi}{1}\PY{p}{]}\PY{o}{.}\PY{n}{set\PYZus{}title}\PY{p}{(}\PY{l+s+s1}{\PYZsq{}}\PY{l+s+s1}{Collision: Impulse vs. Time}\PY{l+s+s1}{\PYZsq{}}\PY{p}{)}
        \PY{n}{axs}\PY{p}{[}\PY{l+m+mi}{1}\PY{p}{]}\PY{o}{.}\PY{n}{plot}\PY{p}{(}\PY{n}{collision\PYZus{}time}\PY{p}{,} \PY{n}{space\PYZus{}ship}\PY{o}{.}\PY{n}{collision\PYZus{}impulse}\PY{p}{,} \PY{n}{label}\PY{o}{=}\PY{l+s+s1}{\PYZsq{}}\PY{l+s+s1}{Normal Force Impulse}\PY{l+s+s1}{\PYZsq{}}\PY{p}{)}
        \PY{n}{axs}\PY{p}{[}\PY{l+m+mi}{1}\PY{p}{]}\PY{o}{.}\PY{n}{plot}\PY{p}{(}\PY{n}{collision\PYZus{}time}\PY{p}{,} \PY{n}{np}\PY{o}{.}\PY{n}{full\PYZus{}like}\PY{p}{(}\PY{n}{collision\PYZus{}time}\PY{p}{,} \PY{n}{space\PYZus{}ship}\PY{o}{.}\PY{n}{maximum\PYZus{}impulse}\PY{p}{)}\PY{p}{,}
                    \PY{n}{linestyle}\PY{o}{=}\PY{l+s+s1}{\PYZsq{}}\PY{l+s+s1}{dashed}\PY{l+s+s1}{\PYZsq{}}\PY{p}{,}
                    \PY{n}{color}\PY{o}{=}\PY{l+s+s1}{\PYZsq{}}\PY{l+s+s1}{limegreen}\PY{l+s+s1}{\PYZsq{}}\PY{p}{,}
                    \PY{n}{label}\PY{o}{=}\PY{l+s+s1}{\PYZsq{}}\PY{l+s+s1}{Minimum Destructive Impulse}\PY{l+s+s1}{\PYZsq{}}\PY{p}{)}
        \PY{n}{axs}\PY{p}{[}\PY{l+m+mi}{1}\PY{p}{]}\PY{o}{.}\PY{n}{set\PYZus{}xlabel}\PY{p}{(}\PY{l+s+s1}{\PYZsq{}}\PY{l+s+s1}{Time (s)}\PY{l+s+s1}{\PYZsq{}}\PY{p}{)}
        \PY{n}{axs}\PY{p}{[}\PY{l+m+mi}{1}\PY{p}{]}\PY{o}{.}\PY{n}{set\PYZus{}ylabel}\PY{p}{(}\PY{l+s+s1}{\PYZsq{}}\PY{l+s+s1}{Impulse (Ns)}\PY{l+s+s1}{\PYZsq{}}\PY{p}{)}
        \PY{n}{axs}\PY{p}{[}\PY{l+m+mi}{1}\PY{p}{]}\PY{o}{.}\PY{n}{legend}\PY{p}{(}\PY{p}{)}
        
        \PY{n}{plots}\PY{o}{.}\PY{n}{collision\PYZus{}graph} \PY{o}{=} \PY{n}{save\PYZus{}graph\PYZus{}and\PYZus{}close}\PY{p}{(}\PY{p}{)}
    
    \PY{c+c1}{\PYZsh{} trajectory graph}
    \PY{n}{plt}\PY{o}{.}\PY{n}{figure}\PY{p}{(}\PY{p}{)}
    \PY{n}{plt}\PY{o}{.}\PY{n}{title}\PY{p}{(}\PY{l+s+s1}{\PYZsq{}}\PY{l+s+s1}{2D Cartesian Trajectories}\PY{l+s+s1}{\PYZsq{}}\PY{p}{)}
    \PY{n}{plt}\PY{o}{.}\PY{n}{scatter}\PY{p}{(}\PY{l+m+mi}{0}\PY{p}{,}\PY{l+m+mi}{0}\PY{p}{,}\PY{n}{label}\PY{o}{=}\PY{l+s+s1}{\PYZsq{}}\PY{l+s+s1}{Black Hole}\PY{l+s+s1}{\PYZsq{}}\PY{p}{)}
    
    \PY{k}{def} \PY{n+nf}{plot\PYZus{}trajectory}\PY{p}{(}\PY{n}{trajectory}\PY{p}{,} \PY{n}{color}\PY{p}{,} \PY{n}{label}\PY{p}{)}\PY{p}{:}
        \PY{n}{x}\PY{p}{,} \PY{n}{y} \PY{o}{=} \PY{n}{trajectory}\PY{p}{[}\PY{p}{:}\PY{p}{,} \PY{l+m+mi}{0}\PY{p}{]}\PY{p}{,} \PY{n}{trajectory}\PY{p}{[}\PY{p}{:}\PY{p}{,} \PY{l+m+mi}{1}\PY{p}{]}
        \PY{n}{plt}\PY{o}{.}\PY{n}{plot}\PY{p}{(}\PY{n}{x}\PY{p}{,} \PY{n}{y}\PY{p}{,} \PY{n}{linestyle}\PY{o}{=}\PY{l+s+s1}{\PYZsq{}}\PY{l+s+s1}{dashed}\PY{l+s+s1}{\PYZsq{}}\PY{p}{,} \PY{n}{color}\PY{o}{=}\PY{n}{color}\PY{p}{,} \PY{n}{label}\PY{o}{=}\PY{n}{label}\PY{p}{)}
        
        \PY{c+c1}{\PYZsh{} Add a dot where it begins}
        \PY{n}{plt}\PY{o}{.}\PY{n}{scatter}\PY{p}{(}\PY{n}{x}\PY{p}{[}\PY{l+m+mi}{0}\PY{p}{]}\PY{p}{,} \PY{n}{y}\PY{p}{[}\PY{l+m+mi}{0}\PY{p}{]}\PY{p}{,} \PY{n}{color}\PY{o}{=}\PY{n}{color}\PY{p}{,} \PY{n}{s}\PY{o}{=}\PY{l+m+mi}{50}\PY{p}{)}
        
        \PY{c+c1}{\PYZsh{} Add an arrowhead at the end}
        \PY{n}{plt}\PY{o}{.}\PY{n}{annotate}\PY{p}{(}
            \PY{l+s+s1}{\PYZsq{}}\PY{l+s+s1}{\PYZsq{}}\PY{p}{,} 
            \PY{n}{xy}\PY{o}{=}\PY{p}{(}\PY{n}{x}\PY{p}{[}\PY{o}{\PYZhy{}}\PY{l+m+mi}{1}\PY{p}{]}\PY{p}{,} \PY{n}{y}\PY{p}{[}\PY{o}{\PYZhy{}}\PY{l+m+mi}{1}\PY{p}{]}\PY{p}{)}\PY{p}{,}          \PY{c+c1}{\PYZsh{} arrow tip (end point)}
            \PY{n}{xytext}\PY{o}{=}\PY{p}{(}\PY{n}{x}\PY{p}{[}\PY{o}{\PYZhy{}}\PY{l+m+mi}{2}\PY{p}{]}\PY{p}{,} \PY{n}{y}\PY{p}{[}\PY{o}{\PYZhy{}}\PY{l+m+mi}{2}\PY{p}{]}\PY{p}{)}\PY{p}{,}      \PY{c+c1}{\PYZsh{} a little behind the tip}
            \PY{n}{arrowprops}\PY{o}{=}\PY{n+nb}{dict}\PY{p}{(}\PY{n}{arrowstyle}\PY{o}{=}\PY{l+s+s1}{\PYZsq{}}\PY{l+s+s1}{\PYZhy{}\PYZgt{}}\PY{l+s+s1}{\PYZsq{}}\PY{p}{,} \PY{n}{color}\PY{o}{=}\PY{n}{color}\PY{p}{,} \PY{n}{lw}\PY{o}{=}\PY{l+m+mi}{2}\PY{p}{)}
        \PY{p}{)}
    
    \PY{n}{plot\PYZus{}trajectory}\PY{p}{(}\PY{n}{space\PYZus{}ship}\PY{o}{.}\PY{n}{position}\PY{p}{,} \PY{n}{color}\PY{o}{=}\PY{l+s+s1}{\PYZsq{}}\PY{l+s+s1}{orange}\PY{l+s+s1}{\PYZsq{}}\PY{p}{,} \PY{n}{label}\PY{o}{=}\PY{l+s+s1}{\PYZsq{}}\PY{l+s+s1}{Space Ship}\PY{l+s+s1}{\PYZsq{}}\PY{p}{)}
    \PY{n}{plot\PYZus{}trajectory}\PY{p}{(}\PY{n}{slime\PYZus{}planet}\PY{o}{.}\PY{n}{position}\PY{p}{,} \PY{n}{color}\PY{o}{=}\PY{l+s+s1}{\PYZsq{}}\PY{l+s+s1}{green}\PY{l+s+s1}{\PYZsq{}}\PY{p}{,} \PY{n}{label}\PY{o}{=}\PY{l+s+s1}{\PYZsq{}}\PY{l+s+s1}{Slime Planet}\PY{l+s+s1}{\PYZsq{}}\PY{p}{)}
    \PY{n}{plt}\PY{o}{.}\PY{n}{legend}\PY{p}{(}\PY{p}{)}
    \PY{n}{plt}\PY{o}{.}\PY{n}{xlabel}\PY{p}{(}\PY{l+s+s1}{\PYZsq{}}\PY{l+s+s1}{x (m)}\PY{l+s+s1}{\PYZsq{}}\PY{p}{)}
    \PY{n}{plt}\PY{o}{.}\PY{n}{ylabel}\PY{p}{(}\PY{l+s+s1}{\PYZsq{}}\PY{l+s+s1}{y (m)}\PY{l+s+s1}{\PYZsq{}}\PY{p}{)}
    \PY{n}{plots}\PY{o}{.}\PY{n}{trajectory\PYZus{}graph} \PY{o}{=} \PY{n}{save\PYZus{}graph\PYZus{}and\PYZus{}close}\PY{p}{(}\PY{p}{)}
    
    \PY{k}{return} \PY{n}{plots}
\end{Verbatim}
\end{tcolorbox}

    \textbf{Code Block Summary:} This code runs the full simulation for one
combination pair of a spaceship turning angle \(\theta\) and an initial
speed \(v_0\). At every moment, it calculates the acceleration for each
moving object then updates all lower differentials accordingly.

    \begin{tcolorbox}[breakable, size=fbox, boxrule=1pt, pad at break*=1mm,colback=cellbackground, colframe=cellborder]
\prompt{In}{incolor}{21}{\boxspacing}
\begin{Verbatim}[commandchars=\\\{\}]
\PY{n+nd}{@dataclass}
\PY{k}{class} \PY{n+nc}{SimulationResult}\PY{p}{:}
    \PY{n}{frames}\PY{p}{:}         \PY{n+nb}{int}
    \PY{n}{angle}\PY{p}{:}          \PY{n+nb}{float}
    \PY{n}{initial\PYZus{}speed}\PY{p}{:}  \PY{n+nb}{float}
    \PY{n}{time\PYZus{}data}\PY{p}{:}      \PY{n}{np}\PY{o}{.}\PY{n}{array}
    \PY{n}{space\PYZus{}ship}\PY{p}{:}     \PY{n}{Spaceship}
    \PY{n}{black\PYZus{}hole}\PY{p}{:}     \PY{n}{RigidBody}
    \PY{n}{slime\PYZus{}planet}\PY{p}{:}   \PY{n}{SoftBody}
    \PY{n}{visualizations}\PY{p}{:} \PY{n}{SimulationPlots}
    \PY{n}{has\PYZus{}collision}\PY{p}{:}  \PY{n+nb}{list}

\PY{k}{def} \PY{n+nf}{simulate}\PY{p}{(}\PY{n}{time\PYZus{}range}\PY{p}{,} \PY{n}{dt}\PY{p}{,} \PY{n}{angle}\PY{p}{,} \PY{n}{initial\PYZus{}speed}\PY{p}{,} 
             \PY{n}{do\PYZus{}gravity}\PY{o}{=}\PY{k+kc}{True}\PY{p}{,} \PY{n}{do\PYZus{}thrust}\PY{o}{=}\PY{k+kc}{True}\PY{p}{,} \PY{n}{do\PYZus{}collision}\PY{o}{=}\PY{k+kc}{True}\PY{p}{,} \PY{n}{debug\PYZus{}print}\PY{o}{=}\PY{k+kc}{False}\PY{p}{)}\PY{p}{:}
    \PY{n}{time\PYZus{}data} \PY{o}{=} \PY{n}{np}\PY{o}{.}\PY{n}{arange}\PY{p}{(}\PY{n}{time\PYZus{}range}\PY{p}{[}\PY{l+m+mi}{0}\PY{p}{]}\PY{p}{,} \PY{n}{time\PYZus{}range}\PY{p}{[}\PY{l+m+mi}{1}\PY{p}{]}\PY{p}{,} \PY{n}{dt}\PY{p}{)}
    \PY{n}{total\PYZus{}frames} \PY{o}{=} \PY{n}{np}\PY{o}{.}\PY{n}{size}\PY{p}{(}\PY{n}{time\PYZus{}data}\PY{p}{)}
    
    \PY{n}{black\PYZus{}hole} \PY{o}{=} \PY{n}{RigidBody}\PY{p}{(}
        \PY{n}{initial\PYZus{}position}\PY{o}{=}\PY{n}{BLACK\PYZus{}HOLE\PYZus{}INITIAL\PYZus{}POSITION}\PY{p}{,}
        \PY{n}{position}\PY{o}{=}\PY{n}{np}\PY{o}{.}\PY{n}{repeat}\PY{p}{(}\PY{n}{BLACK\PYZus{}HOLE\PYZus{}INITIAL\PYZus{}POSITION}\PY{p}{,}\PY{n}{total\PYZus{}frames}\PY{p}{)}\PY{p}{,}
        \PY{n}{mass}\PY{o}{=}\PY{n}{BLACK\PYZus{}HOLE\PYZus{}MASS}\PY{p}{,}
        \PY{n}{stationary}\PY{o}{=}\PY{k+kc}{True}\PY{p}{,}
        \PY{n}{name}\PY{o}{=}\PY{l+s+s1}{\PYZsq{}}\PY{l+s+s1}{Black Hole}\PY{l+s+s1}{\PYZsq{}}
    \PY{p}{)}

    \PY{n}{space\PYZus{}ship} \PY{o}{=} \PY{n}{Spaceship}\PY{p}{(}
        \PY{n}{initial\PYZus{}position}\PY{o}{=}\PY{n}{SPACE\PYZus{}SHIP\PYZus{}INITIAL\PYZus{}POSITION}\PY{p}{,}
        \PY{n}{initial\PYZus{}velocity}\PY{o}{=}\PY{n}{SPACE\PYZus{}SHIP\PYZus{}INITIAL\PYZus{}DIRECTION} \PY{o}{*} \PY{n}{initial\PYZus{}speed}\PY{p}{,}
        \PY{n}{maximum\PYZus{}impulse}\PY{o}{=}\PY{n}{SPACE\PYZus{}SHIP\PYZus{}MAXIMUM\PYZus{}IMPULSE}\PY{p}{,}
        \PY{n}{turning\PYZus{}angle}\PY{o}{=}\PY{n}{angle}\PY{p}{,}
        \PY{n}{mass}\PY{o}{=}\PY{n}{SPACE\PYZus{}SHIP\PYZus{}MASS}\PY{p}{,}
        \PY{n}{initial\PYZus{}fuel\PYZus{}mass}\PY{o}{=}\PY{n}{SPACE\PYZus{}SHIP\PYZus{}INITIAL\PYZus{}FUEL\PYZus{}MASS}\PY{p}{,}
        \PY{n}{desired\PYZus{}thrust\PYZus{}mag}\PY{o}{=}\PY{n}{SPACE\PYZus{}SHIP\PYZus{}REQUIRED\PYZus{}THRUST}\PY{p}{,}
        \PY{n}{specific\PYZus{}impulse}\PY{o}{=}\PY{n}{SPACE\PYZus{}SHIP\PYZus{}SPECIFIC\PYZus{}IMPULSE}\PY{p}{,}
        \PY{n}{name}\PY{o}{=}\PY{l+s+s1}{\PYZsq{}}\PY{l+s+s1}{Space Ship}\PY{l+s+s1}{\PYZsq{}}
    \PY{p}{)}

    \PY{n}{slime\PYZus{}planet} \PY{o}{=} \PY{n}{SoftBody}\PY{p}{(}
        \PY{n}{initial\PYZus{}position}\PY{o}{=}\PY{n}{SLIME\PYZus{}PLANET\PYZus{}INITIAL\PYZus{}POSITION}\PY{p}{,}
        \PY{n}{initial\PYZus{}velocity}\PY{o}{=}\PY{n}{SLIME\PYZus{}PLANET\PYZus{}INITIAL\PYZus{}VELOCITY}\PY{p}{,}
        \PY{n}{mass}\PY{o}{=}\PY{n}{SLIME\PYZus{}PLANET\PYZus{}MASS}\PY{p}{,}
        \PY{n}{radius}\PY{o}{=}\PY{n}{SLIME\PYZus{}PLANET\PYZus{}RADIUS}\PY{p}{,}
        \PY{n}{stiffness}\PY{o}{=}\PY{n}{SLIME\PYZus{}STIFFNESS}\PY{p}{,}
        \PY{n}{growth\PYZus{}exponent}\PY{o}{=}\PY{n}{SPHERICAL\PYZus{}GROWTH\PYZus{}EXPONENT}\PY{p}{,}
        \PY{n}{damping\PYZus{}coeff}\PY{o}{=}\PY{n}{SLIME\PYZus{}DAMPING\PYZus{}COEFFICIENT}\PY{p}{,}
        \PY{n}{name}\PY{o}{=}\PY{l+s+s1}{\PYZsq{}}\PY{l+s+s1}{Slime Planet}\PY{l+s+s1}{\PYZsq{}}
    \PY{p}{)}
    
    \PY{n}{space\PYZus{}ship}\PY{o}{.}\PY{n}{init}\PY{p}{(}\PY{n}{total\PYZus{}frames}\PY{p}{)}
    \PY{n}{slime\PYZus{}planet}\PY{o}{.}\PY{n}{init}\PY{p}{(}\PY{n}{total\PYZus{}frames}\PY{p}{)}
    
    \PY{n}{collision\PYZus{}frames} \PY{o}{=} \PY{p}{[}\PY{k+kc}{False}\PY{p}{]} \PY{o}{*} \PY{n}{total\PYZus{}frames}
    
    \PY{c+c1}{\PYZsh{} core loop}
    \PY{n}{interval} \PY{o}{=} \PY{n}{total\PYZus{}frames} \PY{o}{/}\PY{o}{/} \PY{l+m+mi}{10}
    \PY{k}{for} \PY{n}{i} \PY{o+ow}{in} \PY{n+nb}{range}\PY{p}{(}\PY{l+m+mi}{1}\PY{p}{,} \PY{n}{total\PYZus{}frames}\PY{p}{)}\PY{p}{:}
        \PY{k}{if} \PY{n}{debug\PYZus{}print} \PY{o+ow}{and} \PY{n}{i} \PY{o}{\PYZpc{}} \PY{n}{interval} \PY{o}{==} \PY{l+m+mi}{0}\PY{p}{:}
            \PY{n+nb}{print}\PY{p}{(}\PY{l+s+sa}{f}\PY{l+s+s1}{\PYZsq{}}\PY{l+s+si}{\PYZob{}}\PY{p}{(}\PY{n}{i}\PY{+w}{ }\PY{o}{/}\PY{+w}{ }\PY{n}{total\PYZus{}frames}\PY{p}{)}\PY{+w}{ }\PY{o}{*}\PY{+w}{ }\PY{l+m+mi}{100}\PY{l+s+si}{:}\PY{l+s+s1}{.0f}\PY{l+s+si}{\PYZcb{}}\PY{l+s+s1}{\PYZpc{} \PYZhy{} Iteration }\PY{l+s+si}{\PYZob{}}\PY{n}{i}\PY{l+s+si}{\PYZcb{}}\PY{l+s+s1}{/}\PY{l+s+si}{\PYZob{}}\PY{n}{total\PYZus{}frames}\PY{l+s+si}{\PYZcb{}}\PY{l+s+s1}{\PYZsq{}}\PY{p}{)}
        \PY{k}{if} \PY{n}{space\PYZus{}ship}\PY{o}{.}\PY{n}{destroyed}\PY{p}{:}
            \PY{k}{break}
            
        \PY{c+c1}{\PYZsh{} solve the differential equations}
        \PY{k}{for} \PY{n}{body} \PY{o+ow}{in} \PY{p}{[}\PY{n}{space\PYZus{}ship}\PY{p}{,} \PY{n}{slime\PYZus{}planet}\PY{p}{]}\PY{p}{:}          
            \PY{n}{body}\PY{o}{.}\PY{n}{position}\PY{p}{[}\PY{n}{i}\PY{p}{]}     \PY{o}{=} \PY{n}{body}\PY{o}{.}\PY{n}{position}\PY{p}{[}\PY{n}{i}\PY{o}{\PYZhy{}}\PY{l+m+mi}{1}\PY{p}{]} \PY{o}{+} \PY{n}{body}\PY{o}{.}\PY{n}{velocity}\PY{p}{[}\PY{n}{i}\PY{o}{\PYZhy{}}\PY{l+m+mi}{1}\PY{p}{]}     \PY{o}{*} \PY{n}{dt}
            \PY{n}{body}\PY{o}{.}\PY{n}{velocity}\PY{p}{[}\PY{n}{i}\PY{p}{]}     \PY{o}{=} \PY{n}{body}\PY{o}{.}\PY{n}{velocity}\PY{p}{[}\PY{n}{i}\PY{o}{\PYZhy{}}\PY{l+m+mi}{1}\PY{p}{]} \PY{o}{+} \PY{n}{body}\PY{o}{.}\PY{n}{acceleration}\PY{p}{[}\PY{n}{i}\PY{o}{\PYZhy{}}\PY{l+m+mi}{1}\PY{p}{]} \PY{o}{*} \PY{n}{dt}
            \PY{n}{body}\PY{o}{.}\PY{n}{force}\PY{p}{[}\PY{n}{i}\PY{p}{]}        \PY{o}{=} \PY{n}{calculate\PYZus{}net\PYZus{}force}\PY{p}{(}\PY{n}{i}\PY{o}{\PYZhy{}}\PY{l+m+mi}{1}\PY{p}{,} \PY{n}{body}\PY{p}{,} \PY{p}{[}\PY{n}{space\PYZus{}ship}\PY{p}{,} \PY{n}{slime\PYZus{}planet}\PY{p}{,} \PY{n}{black\PYZus{}hole}\PY{p}{]}\PY{p}{,} \PY{n}{dt}\PY{p}{,}
                                                       \PY{n}{collision\PYZus{}frames}\PY{p}{,} \PY{n}{do\PYZus{}gravity}\PY{p}{,} \PY{n}{do\PYZus{}thrust}\PY{p}{,} \PY{n}{do\PYZus{}collision}\PY{p}{)}
            \PY{n}{body}\PY{o}{.}\PY{n}{acceleration}\PY{p}{[}\PY{n}{i}\PY{p}{]} \PY{o}{=} \PY{n}{body}\PY{o}{.}\PY{n}{force}\PY{p}{[}\PY{n}{i}\PY{p}{]} \PY{o}{/} \PY{n}{body}\PY{o}{.}\PY{n}{mass}
            
            \PY{n}{body}\PY{o}{.}\PY{n}{momentum}\PY{p}{[}\PY{n}{i}\PY{p}{]}       \PY{o}{=} \PY{n}{body}\PY{o}{.}\PY{n}{mass} \PY{o}{*} \PY{n}{body}\PY{o}{.}\PY{n}{velocity}\PY{p}{[}\PY{n}{i}\PY{p}{]}
            \PY{n}{body}\PY{o}{.}\PY{n}{kinetic\PYZus{}energy}\PY{p}{[}\PY{n}{i}\PY{p}{]} \PY{o}{=} \PY{p}{(}\PY{l+m+mi}{1}\PY{o}{/}\PY{l+m+mi}{2}\PY{p}{)} \PY{o}{*} \PY{n}{body}\PY{o}{.}\PY{n}{mass} \PY{o}{*} \PY{n}{np}\PY{o}{.}\PY{n}{linalg}\PY{o}{.}\PY{n}{norm}\PY{p}{(}\PY{n}{body}\PY{o}{.}\PY{n}{velocity}\PY{p}{[}\PY{n}{i}\PY{p}{]}\PY{p}{)} \PY{o}{*}\PY{o}{*} \PY{l+m+mi}{2}
            
            \PY{k}{if} \PY{n}{body} \PY{o+ow}{is} \PY{n}{space\PYZus{}ship} \PY{o+ow}{and} \PYZbs{}
                    \PY{n+nb}{len}\PY{p}{(}\PY{n}{space\PYZus{}ship}\PY{o}{.}\PY{n}{collision\PYZus{}impulse}\PY{p}{)} \PY{o}{\PYZgt{}} \PY{l+m+mi}{0} \PY{o+ow}{and} \PYZbs{}
                    \PY{n}{space\PYZus{}ship}\PY{o}{.}\PY{n}{collision\PYZus{}impulse}\PY{p}{[}\PY{o}{\PYZhy{}}\PY{l+m+mi}{1}\PY{p}{]} \PY{o}{\PYZgt{}} \PY{n}{space\PYZus{}ship}\PY{o}{.}\PY{n}{maximum\PYZus{}impulse}\PY{p}{:}
                \PY{n}{space\PYZus{}ship}\PY{o}{.}\PY{n}{destroyed} \PY{o}{=} \PY{k+kc}{True}
                \PY{k}{if} \PY{n}{debug\PYZus{}print}\PY{p}{:}
                    \PY{n+nb}{print}\PY{p}{(}\PY{l+s+s1}{\PYZsq{}}\PY{l+s+s1}{Space Ship Destroyed!}\PY{l+s+s1}{\PYZsq{}}\PY{p}{)}
                \PY{n}{total\PYZus{}frames} \PY{o}{=} \PY{n}{i}
                \PY{c+c1}{\PYZsh{} break}
    
    \PY{n}{collision\PYZus{}frames} \PY{o}{=} \PY{n}{np}\PY{o}{.}\PY{n}{array}\PY{p}{(}\PY{n}{collision\PYZus{}frames}\PY{p}{)}
    
    \PY{k}{if} \PY{n}{debug\PYZus{}print}\PY{p}{:}
        \PY{n+nb}{print}\PY{p}{(}\PY{l+s+sa}{f}\PY{l+s+s1}{\PYZsq{}}\PY{l+s+s1}{100\PYZpc{} \PYZhy{} Complete!}\PY{l+s+s1}{\PYZsq{}}\PY{p}{)}
    
    \PY{c+c1}{\PYZsh{} truncates all of the datasets}
    \PY{k}{if} \PY{n}{space\PYZus{}ship}\PY{o}{.}\PY{n}{destroyed}\PY{p}{:}
        \PY{n}{total\PYZus{}frames} \PY{o}{+}\PY{o}{=} \PY{l+m+mi}{1}
        \PY{k}{for} \PY{n}{body} \PY{o+ow}{in} \PY{p}{[}\PY{n}{space\PYZus{}ship}\PY{p}{,} \PY{n}{slime\PYZus{}planet}\PY{p}{]}\PY{p}{:}
            \PY{n}{time\PYZus{}data} \PY{o}{=} \PY{n}{time\PYZus{}data}\PY{p}{[}\PY{p}{:}\PY{n}{total\PYZus{}frames}\PY{p}{]}
            \PY{n}{body}\PY{o}{.}\PY{n}{position} \PY{o}{=} \PY{n}{body}\PY{o}{.}\PY{n}{position}\PY{p}{[}\PY{p}{:}\PY{n}{total\PYZus{}frames}\PY{p}{]}
            \PY{n}{body}\PY{o}{.}\PY{n}{velocity} \PY{o}{=} \PY{n}{body}\PY{o}{.}\PY{n}{velocity}\PY{p}{[}\PY{p}{:}\PY{n}{total\PYZus{}frames}\PY{p}{]}
            \PY{n}{body}\PY{o}{.}\PY{n}{acceleration} \PY{o}{=} \PY{n}{body}\PY{o}{.}\PY{n}{acceleration}\PY{p}{[}\PY{p}{:}\PY{n}{total\PYZus{}frames}\PY{p}{]}
            \PY{n}{body}\PY{o}{.}\PY{n}{force} \PY{o}{=} \PY{n}{body}\PY{o}{.}\PY{n}{force}\PY{p}{[}\PY{p}{:}\PY{n}{total\PYZus{}frames}\PY{p}{]}
            \PY{n}{body}\PY{o}{.}\PY{n}{momentum} \PY{o}{=} \PY{n}{body}\PY{o}{.}\PY{n}{momentum}\PY{p}{[}\PY{p}{:}\PY{n}{total\PYZus{}frames}\PY{p}{]}
            \PY{n}{body}\PY{o}{.}\PY{n}{kinetic\PYZus{}energy} \PY{o}{=} \PY{n}{body}\PY{o}{.}\PY{n}{kinetic\PYZus{}energy}\PY{p}{[}\PY{p}{:}\PY{n}{total\PYZus{}frames}\PY{p}{]}
            \PY{n}{collision\PYZus{}frames} \PY{o}{=} \PY{n}{collision\PYZus{}frames}\PY{p}{[}\PY{p}{:}\PY{n}{total\PYZus{}frames}\PY{p}{]}
            \PY{n}{space\PYZus{}ship}\PY{o}{.}\PY{n}{remaining\PYZus{}fuel\PYZus{}mass} \PY{o}{=} \PY{n}{space\PYZus{}ship}\PY{o}{.}\PY{n}{remaining\PYZus{}fuel\PYZus{}mass}\PY{p}{[}\PY{p}{:}\PY{n}{total\PYZus{}frames}\PY{p}{]}
        
    \PY{n}{plots} \PY{o}{=} \PY{n}{plot\PYZus{}results}\PY{p}{(}\PY{n}{angle}\PY{p}{,} \PY{n}{initial\PYZus{}speed}\PY{p}{,} \PY{n}{time\PYZus{}data}\PY{p}{,} \PY{n}{slime\PYZus{}planet}\PY{p}{,} 
                         \PY{n}{space\PYZus{}ship}\PY{p}{,} \PY{n}{black\PYZus{}hole}\PY{p}{,} \PY{n}{collision\PYZus{}frames}\PY{p}{)}
        
    \PY{c+c1}{\PYZsh{} returns all of the info from the simulation}
    \PY{k}{return} \PY{n}{SimulationResult}\PY{p}{(}
        \PY{n}{frames}\PY{o}{=}\PY{n}{total\PYZus{}frames}\PY{p}{,}
        \PY{n}{angle} \PY{o}{=} \PY{n}{angle}\PY{p}{,}
        \PY{n}{initial\PYZus{}speed} \PY{o}{=} \PY{n}{initial\PYZus{}speed}\PY{p}{,}
        \PY{n}{time\PYZus{}data}\PY{o}{=}\PY{n}{time\PYZus{}data}\PY{p}{,}
        \PY{n}{slime\PYZus{}planet}\PY{o}{=}\PY{n}{slime\PYZus{}planet}\PY{p}{,}
        \PY{n}{space\PYZus{}ship}\PY{o}{=}\PY{n}{space\PYZus{}ship}\PY{p}{,}
        \PY{n}{black\PYZus{}hole}\PY{o}{=}\PY{n}{black\PYZus{}hole}\PY{p}{,}
        \PY{n}{visualizations}\PY{o}{=}\PY{n}{plots}\PY{p}{,}
        \PY{n}{has\PYZus{}collision}\PY{o}{=}\PY{n}{collision\PYZus{}frames}
    \PY{p}{)}
\end{Verbatim}
\end{tcolorbox}

    \hypertarget{testing-the-system-with-one-simulation}{%
\section{Testing the System with
One-Simulation}\label{testing-the-system-with-one-simulation}}

    \textbf{Code Block Summary:} This code block sets up the basic
conditions for each of the objects in our simulation. These initial
parameters were determined to best show a collision within the system.

    \begin{tcolorbox}[breakable, size=fbox, boxrule=1pt, pad at break*=1mm,colback=cellbackground, colframe=cellborder]
\prompt{In}{incolor}{22}{\boxspacing}
\begin{Verbatim}[commandchars=\\\{\}]
\PY{n}{test} \PY{o}{=} \PY{n}{simulate}\PY{p}{(}\PY{n}{time\PYZus{}range}\PY{o}{=}\PY{p}{(}\PY{l+m+mi}{0}\PY{p}{,}\PY{l+m+mi}{10}\PY{p}{)}\PY{p}{,} \PY{n}{dt}\PY{o}{=}\PY{l+m+mf}{0.01}\PY{p}{,} 
                \PY{n}{angle}\PY{o}{=}\PY{o}{\PYZhy{}}\PY{n}{np}\PY{o}{.}\PY{n}{pi}\PY{o}{/}\PY{l+m+mi}{4}\PY{p}{,} \PY{n}{initial\PYZus{}speed}\PY{o}{=}\PY{l+m+mi}{2}\PY{p}{,} 
                \PY{n}{do\PYZus{}gravity}\PY{o}{=}\PY{k+kc}{True}\PY{p}{,} \PY{n}{do\PYZus{}thrust}\PY{o}{=}\PY{k+kc}{True}\PY{p}{,} \PY{n}{do\PYZus{}collision}\PY{o}{=}\PY{k+kc}{True}\PY{p}{,} \PY{n}{debug\PYZus{}print}\PY{o}{=}\PY{k+kc}{False}\PY{p}{)}
\end{Verbatim}
\end{tcolorbox}

    \begin{Verbatim}[commandchars=\\\{\}]
10\% - Iteration 100/1000
20\% - Iteration 200/1000
30\% - Iteration 300/1000
40\% - Iteration 400/1000
50\% - Iteration 500/1000
60\% - Iteration 600/1000
70\% - Iteration 700/1000
80\% - Iteration 800/1000
90\% - Iteration 900/1000
100\% - Complete!
    \end{Verbatim}

    \begin{tcolorbox}[breakable, size=fbox, boxrule=1pt, pad at break*=1mm,colback=cellbackground, colframe=cellborder]
\prompt{In}{incolor}{23}{\boxspacing}
\begin{Verbatim}[commandchars=\\\{\}]
\PY{n}{display}\PY{p}{(}\PY{n}{test}\PY{o}{.}\PY{n}{visualizations}\PY{o}{.}\PY{n}{distance\PYZus{}speed\PYZus{}acceleration\PYZus{}graph}\PY{p}{)}
\end{Verbatim}
\end{tcolorbox}

    \begin{center}
    \adjustimage{max size={0.9\linewidth}{0.9\paperheight}}{Project 1_files/Project 1_67_0.png}
    \end{center}
    { \hspace*{\fill} \\}
    
    \textbf{Figure 2.} A collision can be clearly observed based on
examining the radial distance, speed, and acceleration near \(t=2\)
seconds. During the collision, due to the force supplied on the object
being a function of both the overlap of the space ship and its velocity
into the slime planet, a clear parabolic shape is visible. Furthermore,
the initial seemingly-flat region of acceleration is the result of the
thrust of the ship as it turns by an angle \(\theta = -\pi/4\). We can
also observe that the fuel decreases linearly (as expected), but doesn't
seem to run out over the course of the simulation.

    \begin{tcolorbox}[breakable, size=fbox, boxrule=1pt, pad at break*=1mm,colback=cellbackground, colframe=cellborder]
\prompt{In}{incolor}{24}{\boxspacing}
\begin{Verbatim}[commandchars=\\\{\}]
\PY{n}{display}\PY{p}{(}\PY{n}{test}\PY{o}{.}\PY{n}{visualizations}\PY{o}{.}\PY{n}{trajectory\PYZus{}graph}\PY{p}{)}
\end{Verbatim}
\end{tcolorbox}

    \begin{center}
    \adjustimage{max size={0.9\linewidth}{0.9\paperheight}}{Project 1_files/Project 1_69_0.png}
    \end{center}
    { \hspace*{\fill} \\}
    
    \textbf{Figure 3.} For the test case, we can see how a collision with
the slime planet is able to divert the space ship far off course from
its initial path of destruction due to the Black Hole. The gravitational
effects of the black hole on the slime planet are clearly very small,
and this is to be expected because the slime planet has a large enough
mass where it is essentially unaffected by the gravitational pull of the
black hole (but an object like the space ship is small enough where this
is significant).

    \begin{tcolorbox}[breakable, size=fbox, boxrule=1pt, pad at break*=1mm,colback=cellbackground, colframe=cellborder]
\prompt{In}{incolor}{25}{\boxspacing}
\begin{Verbatim}[commandchars=\\\{\}]
\PY{n}{display}\PY{p}{(}\PY{n}{test}\PY{o}{.}\PY{n}{visualizations}\PY{o}{.}\PY{n}{collision\PYZus{}graph}\PY{p}{)}
\end{Verbatim}
\end{tcolorbox}

    \begin{center}
    \adjustimage{max size={0.9\linewidth}{0.9\paperheight}}{Project 1_files/Project 1_71_0.png}
    \end{center}
    { \hspace*{\fill} \\}
    
    \textbf{Figure 4.} The collision's total impulse comes quite close to
the minimum destructive impulse threshold, but maintains such a distance
that it doesn't cross the threshold and breaking the ship, so this
collision is successful. Ideally, for the people on the spaceship to
maximize their exit velocity, they would want to make the impulse from
the collision as high as possible without breaking, so this would likely
be a near-ideal collision for the spaceship. Furthermore, the force
vs.~time graph shows exactly where the ship comes the farthest into the
slime planet at around 1.4 seconds, and at this point, the force is the
highest.

    \hypertarget{multi-simulations-and-analysis-code}{%
\section{Multi-Simulations and Analysis
Code}\label{multi-simulations-and-analysis-code}}

    \textbf{Code Block Summary:} This is the utility function for performing
the multi-objective optimization described in the introduction.

    \begin{tcolorbox}[breakable, size=fbox, boxrule=1pt, pad at break*=1mm,colback=cellbackground, colframe=cellborder]
\prompt{In}{incolor}{26}{\boxspacing}
\begin{Verbatim}[commandchars=\\\{\}]
\PY{k}{def} \PY{n+nf}{normalize\PYZus{}parameter}\PY{p}{(}\PY{n}{x}\PY{p}{,} \PY{n}{xrange}\PY{p}{,} \PY{n}{eps}\PY{o}{=}\PY{l+m+mf}{1e\PYZhy{}12}\PY{p}{)}\PY{p}{:}
    \PY{n}{x} \PY{o}{=} \PY{n+nb}{max}\PY{p}{(}\PY{n}{x}\PY{p}{,} \PY{n}{eps}\PY{p}{)}
    \PY{n}{xmin}\PY{p}{,} \PY{n}{xmax} \PY{o}{=} \PY{n}{xrange}
    \PY{k}{if} \PY{o+ow}{not} \PY{n}{np}\PY{o}{.}\PY{n}{isfinite}\PY{p}{(}\PY{n}{xmin}\PY{p}{)}\PY{p}{:} \PY{n}{xmin} \PY{o}{=} \PY{n}{eps}
    \PY{k}{if} \PY{o+ow}{not} \PY{n}{np}\PY{o}{.}\PY{n}{isfinite}\PY{p}{(}\PY{n}{xmax}\PY{p}{)}\PY{p}{:} \PY{n}{xmax} \PY{o}{=} \PY{n}{eps} \PY{o}{*} \PY{l+m+mi}{10}
    \PY{k}{if} \PY{n}{xmax} \PY{o}{\PYZlt{}}\PY{o}{=} \PY{n}{xmin}\PY{p}{:}
        \PY{k}{return} \PY{n}{eps}  \PY{c+c1}{\PYZsh{} fallback if range is invalid}
    \PY{k}{return} \PY{n}{clamp}\PY{p}{(}\PY{p}{(}\PY{n}{np}\PY{o}{.}\PY{n}{log}\PY{p}{(}\PY{n}{x}\PY{p}{)} \PY{o}{\PYZhy{}} \PY{n}{np}\PY{o}{.}\PY{n}{log}\PY{p}{(}\PY{n}{xmin}\PY{p}{)}\PY{p}{)} \PY{o}{/} \PY{p}{(}\PY{n}{np}\PY{o}{.}\PY{n}{log}\PY{p}{(}\PY{n}{xmax}\PY{p}{)} \PY{o}{\PYZhy{}} \PY{n}{np}\PY{o}{.}\PY{n}{log}\PY{p}{(}\PY{n}{xmin}\PY{p}{)}\PY{p}{)}\PY{p}{,} \PY{l+m+mi}{0}\PY{p}{,} \PY{l+m+mi}{1}\PY{p}{)}

\PY{k}{def} \PY{n+nf}{scalarize}\PY{p}{(}\PY{n}{result\PYZus{}state}\PY{p}{,} \PY{n}{ranges}\PY{p}{,} \PY{n}{weights}\PY{p}{)}\PY{p}{:}
    \PY{n}{mf\PYZus{}n}\PY{p}{,} \PY{n}{rf\PYZus{}n}\PY{p}{,} \PY{n}{vf\PYZus{}n} \PY{o}{=} \PY{p}{[}\PY{n}{normalize\PYZus{}parameter}\PY{p}{(}\PY{n}{x}\PY{p}{,} \PY{n}{x\PYZus{}range}\PY{p}{)} \PY{k}{for} \PY{n}{x}\PY{p}{,} \PY{n}{x\PYZus{}range} \PY{o+ow}{in} \PY{n+nb}{zip}\PY{p}{(}\PY{n}{result\PYZus{}state}\PY{p}{,} \PY{n}{ranges}\PY{p}{)}\PY{p}{]}
    \PY{k}{return} \PY{p}{(}\PY{n}{mf\PYZus{}n} \PY{o}{*} \PY{n}{weights}\PY{p}{[}\PY{l+m+mi}{0}\PY{p}{]} \PY{o}{+} \PY{n}{rf\PYZus{}n} \PY{o}{*} \PY{n}{weights}\PY{p}{[}\PY{l+m+mi}{1}\PY{p}{]} \PY{o}{+} \PY{n}{vf\PYZus{}n} \PY{o}{*} \PY{n}{weights}\PY{p}{[}\PY{l+m+mi}{2}\PY{p}{]}\PY{p}{)}
\end{Verbatim}
\end{tcolorbox}

    \textbf{Code Block Summary:} This function performs all of the
simulations by running the above function across a range of initial
angles and initial speeds, producing an associative list mapping the
angle/speed pairs to the results of its simulation.

    \begin{tcolorbox}[breakable, size=fbox, boxrule=1pt, pad at break*=1mm,colback=cellbackground, colframe=cellborder]
\prompt{In}{incolor}{27}{\boxspacing}
\begin{Verbatim}[commandchars=\\\{\}]
\PY{n+nd}{@dataclass}
\PY{k}{class} \PY{n+nc}{RangedSimulationResults}\PY{p}{:}
    \PY{n}{results}\PY{p}{:} \PY{n+nb}{dict}
    \PY{n}{contour}\PY{p}{:} \PY{n+nb}{dict} 
    \PY{n}{contour\PYZus{}plot}\PY{p}{:} \PY{n}{Image}
    \PY{n}{optimal\PYZus{}pair}\PY{p}{:} \PY{n}{np}\PY{o}{.}\PY{n}{array}
    \PY{n}{input\PYZus{}domain}\PY{p}{:} \PY{n+nb}{tuple}
    \PY{n}{output\PYZus{}range}\PY{p}{:} \PY{n+nb}{tuple}

\PY{k}{def} \PY{n+nf}{simulate\PYZus{}all}\PY{p}{(}\PY{n}{time\PYZus{}range}\PY{p}{,} \PY{n}{dt}\PY{p}{,} \PY{n}{angle\PYZus{}range}\PY{p}{,} \PY{n}{speed\PYZus{}range}\PY{p}{,}
                 \PY{n}{fuel\PYZus{}bias}\PY{o}{=}\PY{l+m+mi}{1}\PY{o}{/}\PY{l+m+mi}{3}\PY{p}{,} \PY{n}{distance\PYZus{}bias}\PY{o}{=}\PY{l+m+mi}{1}\PY{o}{/}\PY{l+m+mi}{3}\PY{p}{,} \PY{n}{speed\PYZus{}bias}\PY{o}{=}\PY{l+m+mi}{1}\PY{o}{/}\PY{l+m+mi}{3}\PY{p}{,} \PY{n}{debug\PYZus{}print}\PY{o}{=}\PY{k+kc}{False}\PY{p}{)}\PY{p}{:}
\PY{+w}{    }\PY{l+s+sd}{\PYZdq{}\PYZdq{}\PYZdq{}}
\PY{l+s+sd}{    Performs all simulations within the angle and speed range, }
\PY{l+s+sd}{    then returns a dictionary mapping the angle/speed pairs to}
\PY{l+s+sd}{    the results of a given simulation.}
\PY{l+s+sd}{    \PYZdq{}\PYZdq{}\PYZdq{}}    
    \PY{n}{results} \PY{o}{=} \PY{n+nb}{dict}\PY{p}{(}\PY{p}{)}
    \PY{n}{contour} \PY{o}{=} \PY{n+nb}{dict}\PY{p}{(}\PY{p}{)}
    \PY{n}{failed\PYZus{}state} \PY{o}{=} \PY{n}{np}\PY{o}{.}\PY{n}{zeros}\PY{p}{(}\PY{l+m+mi}{3}\PY{p}{)}
    \PY{n}{weights} \PY{o}{=} \PY{n}{unit\PYZus{}vector}\PY{p}{(}\PY{n}{np}\PY{o}{.}\PY{n}{array}\PY{p}{(}\PY{p}{[}\PY{n}{fuel\PYZus{}bias}\PY{p}{,} \PY{n}{distance\PYZus{}bias}\PY{p}{,} \PY{n}{speed\PYZus{}bias}\PY{p}{]}\PY{p}{)}\PY{p}{)}
    
    \PY{n}{mf\PYZus{}range} \PY{o}{=} \PY{n}{np}\PY{o}{.}\PY{n}{array}\PY{p}{(}\PY{p}{[}\PY{n}{np}\PY{o}{.}\PY{n}{inf}\PY{p}{,} \PY{o}{\PYZhy{}}\PY{n}{np}\PY{o}{.}\PY{n}{inf}\PY{p}{]}\PY{p}{)}
    \PY{n}{rf\PYZus{}range} \PY{o}{=} \PY{n}{np}\PY{o}{.}\PY{n}{array}\PY{p}{(}\PY{p}{[}\PY{n}{np}\PY{o}{.}\PY{n}{inf}\PY{p}{,} \PY{o}{\PYZhy{}}\PY{n}{np}\PY{o}{.}\PY{n}{inf}\PY{p}{]}\PY{p}{)}
    \PY{n}{vf\PYZus{}range} \PY{o}{=} \PY{n}{np}\PY{o}{.}\PY{n}{array}\PY{p}{(}\PY{p}{[}\PY{n}{np}\PY{o}{.}\PY{n}{inf}\PY{p}{,} \PY{o}{\PYZhy{}}\PY{n}{np}\PY{o}{.}\PY{n}{inf}\PY{p}{]}\PY{p}{)}
    
    \PY{k}{def} \PY{n+nf}{adjust\PYZus{}range}\PY{p}{(}\PY{n}{x}\PY{p}{,} \PY{n}{xrange}\PY{p}{)}\PY{p}{:}
        \PY{k}{if} \PY{n}{x} \PY{o}{\PYZlt{}} \PY{n}{xrange}\PY{p}{[}\PY{l+m+mi}{0}\PY{p}{]}\PY{p}{:}
            \PY{n}{xrange}\PY{p}{[}\PY{l+m+mi}{0}\PY{p}{]} \PY{o}{=} \PY{n}{x}
        \PY{k}{if} \PY{n}{x} \PY{o}{\PYZgt{}} \PY{n}{xrange}\PY{p}{[}\PY{l+m+mi}{1}\PY{p}{]}\PY{p}{:}
            \PY{n}{xrange}\PY{p}{[}\PY{l+m+mi}{1}\PY{p}{]} \PY{o}{=} \PY{n}{x}
    
    \PY{n}{angle\PYZus{}domain} \PY{o}{=} \PY{n}{np}\PY{o}{.}\PY{n}{linspace}\PY{p}{(}\PY{n}{angle\PYZus{}range}\PY{p}{[}\PY{l+m+mi}{0}\PY{p}{]}\PY{p}{,} \PY{n}{angle\PYZus{}range}\PY{p}{[}\PY{l+m+mi}{1}\PY{p}{]}\PY{p}{,} \PY{n}{num}\PY{o}{=}\PY{n}{angle\PYZus{}range}\PY{p}{[}\PY{l+m+mi}{2}\PY{p}{]}\PY{p}{)}
    \PY{n}{speed\PYZus{}domain} \PY{o}{=} \PY{n}{np}\PY{o}{.}\PY{n}{linspace}\PY{p}{(}\PY{n}{speed\PYZus{}range}\PY{p}{[}\PY{l+m+mi}{0}\PY{p}{]}\PY{p}{,} \PY{n}{speed\PYZus{}range}\PY{p}{[}\PY{l+m+mi}{1}\PY{p}{]}\PY{p}{,} \PY{n}{num}\PY{o}{=}\PY{n}{speed\PYZus{}range}\PY{p}{[}\PY{l+m+mi}{2}\PY{p}{]}\PY{p}{)}
    
    \PY{n}{start} \PY{o}{=} \PY{n}{time}\PY{o}{.}\PY{n}{time}\PY{p}{(}\PY{p}{)}
    
    \PY{n}{sim\PYZus{}idx} \PY{o}{=} \PY{l+m+mi}{0}
    \PY{n}{total\PYZus{}sims} \PY{o}{=} \PY{n}{angle\PYZus{}domain}\PY{o}{.}\PY{n}{size} \PY{o}{*} \PY{n}{speed\PYZus{}domain}\PY{o}{.}\PY{n}{size}
    \PY{n}{interval} \PY{o}{=} \PY{n+nb}{max}\PY{p}{(}\PY{n}{total\PYZus{}sims} \PY{o}{/}\PY{o}{/} \PY{l+m+mi}{10}\PY{p}{,} \PY{l+m+mi}{1}\PY{p}{)}
    
    \PY{k}{for} \PY{n}{angle} \PY{o+ow}{in} \PY{n}{angle\PYZus{}domain}\PY{p}{:}
        \PY{k}{for} \PY{n}{speed} \PY{o+ow}{in} \PY{n}{speed\PYZus{}domain}\PY{p}{:}
            \PY{n}{sim\PYZus{}idx} \PY{o}{+}\PY{o}{=} \PY{l+m+mi}{1}
            
            \PY{n}{simulation\PYZus{}result} \PY{o}{=} \PY{n}{simulate}\PY{p}{(}\PY{n}{time\PYZus{}range}\PY{p}{,} \PY{n}{dt}\PY{p}{,} \PY{n}{angle}\PY{p}{,} \PY{n}{speed}\PY{p}{,} \PY{n}{debug\PYZus{}print}\PY{o}{=}\PY{n}{debug\PYZus{}print}\PY{p}{)}
            \PY{n}{ss} \PY{o}{=} \PY{n}{simulation\PYZus{}result}\PY{o}{.}\PY{n}{space\PYZus{}ship}
            
            \PY{c+c1}{\PYZsh{} Debug Print}
            \PY{k}{if} \PY{p}{(}\PY{n}{sim\PYZus{}idx} \PY{o}{\PYZpc{}} \PY{n}{interval}\PY{p}{)} \PY{o}{==} \PY{l+m+mi}{0}\PY{p}{:}
                \PY{n+nb}{print}\PY{p}{(}\PY{l+s+sa}{f}\PY{l+s+s1}{\PYZsq{}}\PY{l+s+si}{\PYZob{}}\PY{p}{(}\PY{n}{sim\PYZus{}idx}\PY{+w}{ }\PY{o}{/}\PY{+w}{ }\PY{n}{total\PYZus{}sims}\PY{p}{)}\PY{+w}{ }\PY{o}{*}\PY{+w}{ }\PY{l+m+mi}{100}\PY{l+s+si}{:}\PY{l+s+s1}{.0f}\PY{l+s+si}{\PYZcb{}}\PY{l+s+s1}{\PYZpc{} \PYZhy{} Iteration }\PY{l+s+si}{\PYZob{}}\PY{n}{sim\PYZus{}idx}\PY{l+s+si}{\PYZcb{}}\PY{l+s+s1}{/}\PY{l+s+si}{\PYZob{}}\PY{n}{total\PYZus{}sims}\PY{l+s+si}{\PYZcb{}}\PY{l+s+s1}{\PYZsq{}}\PY{p}{)}
            
            \PY{k}{if} \PY{n}{ss}\PY{o}{.}\PY{n}{destroyed}\PY{p}{:}
                \PY{n}{results}\PY{p}{[}\PY{p}{(}\PY{n}{angle}\PY{p}{,}\PY{n}{speed}\PY{p}{)}\PY{p}{]} \PY{o}{=} \PY{n}{failed\PYZus{}state}
                \PY{k}{continue}
            
            \PY{n}{mf} \PY{o}{=} \PY{n}{ss}\PY{o}{.}\PY{n}{remaining\PYZus{}fuel\PYZus{}mass}\PY{p}{[}\PY{o}{\PYZhy{}}\PY{l+m+mi}{1}\PY{p}{]}
            \PY{n}{rf} \PY{o}{=} \PY{n}{np}\PY{o}{.}\PY{n}{linalg}\PY{o}{.}\PY{n}{norm}\PY{p}{(}\PY{n}{ss}\PY{o}{.}\PY{n}{position}\PY{p}{[}\PY{o}{\PYZhy{}}\PY{l+m+mi}{1}\PY{p}{]}\PY{p}{)}
            \PY{n}{vf} \PY{o}{=} \PY{n}{np}\PY{o}{.}\PY{n}{linalg}\PY{o}{.}\PY{n}{norm}\PY{p}{(}\PY{n}{ss}\PY{o}{.}\PY{n}{velocity}\PY{p}{[}\PY{o}{\PYZhy{}}\PY{l+m+mi}{1}\PY{p}{]}\PY{p}{)}
            
            \PY{n}{adjust\PYZus{}range}\PY{p}{(}\PY{n}{mf}\PY{p}{,} \PY{n}{mf\PYZus{}range}\PY{p}{)}
            \PY{n}{adjust\PYZus{}range}\PY{p}{(}\PY{n}{rf}\PY{p}{,} \PY{n}{rf\PYZus{}range}\PY{p}{)}
            \PY{n}{adjust\PYZus{}range}\PY{p}{(}\PY{n}{vf}\PY{p}{,} \PY{n}{vf\PYZus{}range}\PY{p}{)}
            
            \PY{n}{results}\PY{p}{[}\PY{p}{(}\PY{n}{angle}\PY{p}{,}\PY{n}{speed}\PY{p}{)}\PY{p}{]} \PY{o}{=} \PY{p}{(}\PY{n}{mf}\PY{p}{,} \PY{n}{rf}\PY{p}{,} \PY{n}{vf}\PY{p}{)}
    
    \PY{n+nb}{print}\PY{p}{(}\PY{l+s+sa}{f}\PY{l+s+s1}{\PYZsq{}}\PY{l+s+s1}{100\PYZpc{} \PYZhy{} Completed.}\PY{l+s+s1}{\PYZsq{}}\PY{p}{)}
    
    \PY{n}{optimal\PYZus{}pair} \PY{o}{=} \PY{k+kc}{None}
    \PY{n}{optimal\PYZus{}utility} \PY{o}{=} \PY{l+m+mi}{0}
    \PY{k}{for} \PY{n}{angle} \PY{o+ow}{in} \PY{n}{angle\PYZus{}domain}\PY{p}{:}
        \PY{k}{for} \PY{n}{speed} \PY{o+ow}{in} \PY{n}{speed\PYZus{}domain}\PY{p}{:}
            \PY{n}{utility} \PY{o}{=} \PY{n}{scalarize}\PY{p}{(}\PY{n}{results}\PY{p}{[}\PY{p}{(}\PY{n}{angle}\PY{p}{,}\PY{n}{speed}\PY{p}{)}\PY{p}{]}\PY{p}{,} \PY{p}{[}\PY{n}{mf\PYZus{}range}\PY{p}{,} \PY{n}{rf\PYZus{}range}\PY{p}{,} \PY{n}{vf\PYZus{}range}\PY{p}{]}\PY{p}{,} \PY{n}{weights}\PY{p}{)}
            \PY{k}{if} \PY{n}{debug\PYZus{}print}\PY{p}{:}
                \PY{n}{mf\PYZus{}n}\PY{p}{,} \PY{n}{rf\PYZus{}n}\PY{p}{,} \PY{n}{vf\PYZus{}n} \PY{o}{=} \PY{n}{results}\PY{p}{[}\PY{p}{(}\PY{n}{angle}\PY{p}{,}\PY{n}{speed}\PY{p}{)}\PY{p}{]}
                \PY{n+nb}{print}\PY{p}{(}\PY{l+s+sa}{f}\PY{l+s+s1}{\PYZsq{}}\PY{l+s+s1}{Utility: (}\PY{l+s+si}{\PYZob{}}\PY{n}{angle}\PY{l+s+si}{:}\PY{l+s+s1}{.2f}\PY{l+s+si}{\PYZcb{}}\PY{l+s+s1}{,}\PY{l+s+si}{\PYZob{}}\PY{n}{speed}\PY{l+s+si}{:}\PY{l+s+s1}{.2f}\PY{l+s+si}{\PYZcb{}}\PY{l+s+s1}{) = \PYZlt{}}\PY{l+s+si}{\PYZob{}}\PY{n}{mf\PYZus{}n}\PY{l+s+si}{:}\PY{l+s+s1}{.2f}\PY{l+s+si}{\PYZcb{}}\PY{l+s+s1}{,}\PY{l+s+si}{\PYZob{}}\PY{n}{rf\PYZus{}n}\PY{l+s+si}{:}\PY{l+s+s1}{.2f}\PY{l+s+si}{\PYZcb{}}\PY{l+s+s1}{,}\PY{l+s+si}{\PYZob{}}\PY{n}{vf\PYZus{}n}\PY{l+s+si}{:}\PY{l+s+s1}{.2f}\PY{l+s+si}{\PYZcb{}}\PY{l+s+s1}{\PYZgt{} = }\PY{l+s+si}{\PYZob{}}\PY{n}{utility}\PY{l+s+si}{:}\PY{l+s+s1}{.1f}\PY{l+s+si}{\PYZcb{}}\PY{l+s+s1}{\PYZsq{}}\PY{p}{)}
            \PY{n}{contour}\PY{p}{[}\PY{p}{(}\PY{n}{angle}\PY{p}{,}\PY{n}{speed}\PY{p}{)}\PY{p}{]} \PY{o}{=} \PY{n}{utility}
            \PY{k}{if} \PY{n}{utility} \PY{o}{\PYZgt{}} \PY{n}{optimal\PYZus{}utility}\PY{p}{:}
                \PY{n}{optimal\PYZus{}pair} \PY{o}{=} \PY{n}{np}\PY{o}{.}\PY{n}{array}\PY{p}{(}\PY{p}{[}\PY{n}{angle}\PY{p}{,}\PY{n}{speed}\PY{p}{]}\PY{p}{)}
                \PY{n}{optimal\PYZus{}utility} \PY{o}{=} \PY{n}{utility}
            
    \PY{c+c1}{\PYZsh{} plot results to a contour plot}
    \PY{n}{A}\PY{p}{,} \PY{n}{S} \PY{o}{=} \PY{n}{np}\PY{o}{.}\PY{n}{meshgrid}\PY{p}{(}\PY{n}{np}\PY{o}{.}\PY{n}{rad2deg}\PY{p}{(}\PY{n}{angle\PYZus{}domain}\PY{p}{)}\PY{p}{,} \PY{n}{speed\PYZus{}domain}\PY{p}{,} \PY{n}{indexing}\PY{o}{=}\PY{l+s+s1}{\PYZsq{}}\PY{l+s+s1}{ij}\PY{l+s+s1}{\PYZsq{}}\PY{p}{)}
    \PY{n}{Z} \PY{o}{=} \PY{n}{np}\PY{o}{.}\PY{n}{zeros\PYZus{}like}\PY{p}{(}\PY{n}{A}\PY{p}{,} \PY{n}{dtype}\PY{o}{=}\PY{n+nb}{float}\PY{p}{)}
    \PY{k}{for} \PY{n}{i}\PY{p}{,} \PY{n}{angle} \PY{o+ow}{in} \PY{n+nb}{enumerate}\PY{p}{(}\PY{n}{angle\PYZus{}domain}\PY{p}{)}\PY{p}{:}
        \PY{k}{for} \PY{n}{j}\PY{p}{,} \PY{n}{speed} \PY{o+ow}{in} \PY{n+nb}{enumerate}\PY{p}{(}\PY{n}{speed\PYZus{}domain}\PY{p}{)}\PY{p}{:}
            \PY{n}{Z}\PY{p}{[}\PY{n}{i}\PY{p}{,}\PY{n}{j}\PY{p}{]} \PY{o}{=} \PY{n}{contour}\PY{p}{[}\PY{p}{(}\PY{n}{angle}\PY{p}{,}\PY{n}{speed}\PY{p}{)}\PY{p}{]}
    
    \PY{n}{plt}\PY{o}{.}\PY{n}{figure}\PY{p}{(}\PY{p}{)}
    \PY{n}{cp} \PY{o}{=} \PY{n}{plt}\PY{o}{.}\PY{n}{contourf}\PY{p}{(}\PY{n}{A}\PY{p}{,}\PY{n}{S}\PY{p}{,}\PY{n}{Z}\PY{p}{,} \PY{n}{levels}\PY{o}{=}\PY{l+m+mi}{20}\PY{p}{,} \PY{n}{cmap}\PY{o}{=}\PY{l+s+s1}{\PYZsq{}}\PY{l+s+s1}{viridis}\PY{l+s+s1}{\PYZsq{}}\PY{p}{)}
    \PY{n}{plt}\PY{o}{.}\PY{n}{colorbar}\PY{p}{(}\PY{n}{cp}\PY{p}{,} \PY{n}{label}\PY{o}{=}\PY{l+s+s1}{\PYZsq{}}\PY{l+s+s1}{Utility}\PY{l+s+s1}{\PYZsq{}}\PY{p}{)}
    \PY{n}{plt}\PY{o}{.}\PY{n}{xlabel}\PY{p}{(}\PY{l+s+s1}{\PYZsq{}}\PY{l+s+s1}{Angle (deg)}\PY{l+s+s1}{\PYZsq{}}\PY{p}{)}
    \PY{n}{plt}\PY{o}{.}\PY{n}{ylabel}\PY{p}{(}\PY{l+s+s1}{\PYZsq{}}\PY{l+s+s1}{Speed (m/s)}\PY{l+s+s1}{\PYZsq{}}\PY{p}{)}
    \PY{n}{plt}\PY{o}{.}\PY{n}{title}\PY{p}{(}\PY{l+s+s1}{\PYZsq{}}\PY{l+s+s1}{Scalarized Utility Contour}\PY{l+s+s1}{\PYZsq{}}\PY{p}{)}
    \PY{n}{contour\PYZus{}plot} \PY{o}{=} \PY{n}{save\PYZus{}graph\PYZus{}and\PYZus{}close}\PY{p}{(}\PY{p}{)}
        
    \PY{n}{end} \PY{o}{=} \PY{n}{time}\PY{o}{.}\PY{n}{time}\PY{p}{(}\PY{p}{)}
    
    \PY{n+nb}{print}\PY{p}{(}\PY{l+s+s1}{\PYZsq{}}\PY{l+s+s1}{Completed in }\PY{l+s+si}{\PYZpc{}.3f}\PY{l+s+s1}{ s.}\PY{l+s+s1}{\PYZsq{}} \PY{o}{\PYZpc{}} \PY{p}{(}\PY{n}{end} \PY{o}{\PYZhy{}} \PY{n}{start}\PY{p}{)}\PY{p}{)}
        
    \PY{k}{return} \PY{n}{RangedSimulationResults}\PY{p}{(}
        \PY{n}{results}\PY{o}{=}\PY{n}{results}\PY{p}{,}
        \PY{n}{contour}\PY{o}{=}\PY{n}{contour}\PY{p}{,}
        \PY{n}{contour\PYZus{}plot}\PY{o}{=}\PY{n}{contour\PYZus{}plot}\PY{p}{,}
        \PY{n}{optimal\PYZus{}pair}\PY{o}{=}\PY{n}{optimal\PYZus{}pair}\PY{p}{,}
        \PY{n}{input\PYZus{}domain}\PY{o}{=}\PY{p}{(}\PY{n}{angle\PYZus{}domain}\PY{p}{,} \PY{n}{speed\PYZus{}domain}\PY{p}{)}\PY{p}{,}
        \PY{n}{output\PYZus{}range}\PY{o}{=}\PY{p}{(}\PY{n}{mf\PYZus{}range}\PY{p}{,} \PY{n}{rf\PYZus{}range}\PY{p}{,} \PY{n}{vf\PYZus{}range}\PY{p}{)}\PY{p}{,}
    \PY{p}{)}
\end{Verbatim}
\end{tcolorbox}

    \textbf{Code Block Summary:} This is the full simulation plot, which
produces a contour of the utility function \(U(\mathbf{s}(\theta,v_0))\)
over the domain \(\theta \in \left\{-\pi/2, \dots, \pi/2 \right\}\) rad.
and \(v_0 \in \{ 5, \dots, 30\}\) m/s.

    \begin{tcolorbox}[breakable, size=fbox, boxrule=1pt, pad at break*=1mm,colback=cellbackground, colframe=cellborder]
\prompt{In}{incolor}{28}{\boxspacing}
\begin{Verbatim}[commandchars=\\\{\}]
\PY{n}{results} \PY{o}{=} \PY{n}{simulate\PYZus{}all}\PY{p}{(}
    \PY{n}{time\PYZus{}range}\PY{o}{=}\PY{p}{(}\PY{l+m+mi}{0}\PY{p}{,} \PY{l+m+mi}{10}\PY{p}{)}\PY{p}{,} \PY{n}{dt}\PY{o}{=}\PY{l+m+mf}{0.01}\PY{p}{,}
    \PY{n}{angle\PYZus{}range}\PY{o}{=}\PY{p}{(}\PY{o}{\PYZhy{}}\PY{n}{np}\PY{o}{.}\PY{n}{pi}\PY{o}{/}\PY{l+m+mi}{2}\PY{p}{,} \PY{n}{np}\PY{o}{.}\PY{n}{pi}\PY{o}{/}\PY{l+m+mi}{2}\PY{p}{,} \PY{l+m+mi}{10}\PY{p}{)}\PY{p}{,}
    \PY{n}{speed\PYZus{}range}\PY{o}{=}\PY{p}{(}\PY{l+m+mi}{1}\PY{p}{,} \PY{l+m+mi}{30}\PY{p}{,} \PY{l+m+mi}{10}\PY{p}{)}\PY{p}{,}
    \PY{n}{fuel\PYZus{}bias}\PY{o}{=}\PY{l+m+mi}{2}\PY{o}{/}\PY{l+m+mi}{10}\PY{p}{,}
    \PY{n}{distance\PYZus{}bias}\PY{o}{=}\PY{l+m+mi}{5}\PY{o}{/}\PY{l+m+mi}{10}\PY{p}{,}
    \PY{n}{speed\PYZus{}bias}\PY{o}{=}\PY{l+m+mi}{3}\PY{o}{/}\PY{l+m+mi}{10}\PY{p}{,}
    \PY{n}{debug\PYZus{}print}\PY{o}{=}\PY{k+kc}{False}
\PY{p}{)}
\end{Verbatim}
\end{tcolorbox}

    \begin{Verbatim}[commandchars=\\\{\}]
10\% - Iteration 10/100
20\% - Iteration 20/100
30\% - Iteration 30/100
40\% - Iteration 40/100
50\% - Iteration 50/100
60\% - Iteration 60/100
70\% - Iteration 70/100
80\% - Iteration 80/100
90\% - Iteration 90/100
100\% - Iteration 100/100
100\% - Completed.
Completed in 97.206 s.
    \end{Verbatim}

    This full phase contour is over an angular domain of a 90 (hard left) to
-90 (hard right) degree turn between initial velocities of 1 to 30 m/s.
As stated before, the simulation has produced a mapping of each phase
coordinate \((\theta,v_0)\) to a scalar utility (or ``success'' or
``optimization'') score \(U\), with higher utility scores being more
favorable simulations based on the model weights, and for this model,
the weights are \(w_{fuel}=\frac{2}{10},w_{distance}=\frac{5}{10},\) and
\(w_{speed}=\frac{3}{10}\).

    \begin{tcolorbox}[breakable, size=fbox, boxrule=1pt, pad at break*=1mm,colback=cellbackground, colframe=cellborder]
\prompt{In}{incolor}{29}{\boxspacing}
\begin{Verbatim}[commandchars=\\\{\}]
\PY{n}{display}\PY{p}{(}\PY{n}{results}\PY{o}{.}\PY{n}{contour\PYZus{}plot}\PY{p}{)}
\end{Verbatim}
\end{tcolorbox}

    \begin{center}
    \adjustimage{max size={0.9\linewidth}{0.9\paperheight}}{Project 1_files/Project 1_81_0.png}
    \end{center}
    { \hspace*{\fill} \\}
    
    \textbf{Figure 5.} From this plot, the most successful combinations come
at low velocities with high, leftward (positive) turning angles or high
speeds with low, rightward (negative) turning angles. This is evident in
the contour based on all high, rightward turning angles being
destructive (which is defined as a zero utility score), high leftward
angles at high speeds being terminal, and the safest seem to lie mostly
in the region of low speeds and low angles, going from -1.5 rad at low
speeds to 0.5 rad at higher speeds. This is likely because a rightward
turn is directly into the slime planet, so faster speeds will cause much
higher impulses with the planet, which would result in a crash. If the
ship is able to start at a fast enough velocity and minimizes the turn
(to minimize the amount of fuel used), it will successfully escape the
gravitational pull of the planet and the black hole while maximizing
speed and distance and minimizing fuel, which maximizes the contour.
(The optimal case directly shows all of this, see Appendix A2 -
Animations.)

    \hypertarget{the-optimal-pair}{%
\subsection{The Optimal Pair}\label{the-optimal-pair}}

    Now, we've reached the ``optimal'' phase coordinate for this simulation,
and this pair \((\theta,v_0)\) represents the turning angle and initial
velocity that maximizes the chance of long-term survival for the
astronauts on board this ship under the assumptions/opinions of our
optimization model. Taking a closer look into what occurs at this
coordinate will yield some valuable insight into this simulation,
because, for example, if this is a very trivial or otherwise poor result
(like in the case that we cannot explain its behavior), then our
optimization assumptions need to be reworked as the model is not
optimizing for a true best-case scenario.

    \begin{tcolorbox}[breakable, size=fbox, boxrule=1pt, pad at break*=1mm,colback=cellbackground, colframe=cellborder]
\prompt{In}{incolor}{30}{\boxspacing}
\begin{Verbatim}[commandchars=\\\{\}]
\PY{n+nb}{print}\PY{p}{(}\PY{l+s+sa}{f}\PY{l+s+s1}{\PYZsq{}}\PY{l+s+s1}{Optimal Angle = }\PY{l+s+si}{\PYZob{}}\PY{n}{np}\PY{o}{.}\PY{n}{rad2deg}\PY{p}{(}\PY{n}{results}\PY{o}{.}\PY{n}{optimal\PYZus{}pair}\PY{p}{[}\PY{l+m+mi}{0}\PY{p}{]}\PY{p}{)}\PY{l+s+si}{:}\PY{l+s+s1}{.1f}\PY{l+s+si}{\PYZcb{}}\PY{l+s+s1}{ deg.}\PY{l+s+s1}{\PYZsq{}}\PY{p}{)}
\PY{n+nb}{print}\PY{p}{(}\PY{l+s+sa}{f}\PY{l+s+s1}{\PYZsq{}}\PY{l+s+s1}{Optimal Starting Velocity = }\PY{l+s+si}{\PYZob{}}\PY{n}{results}\PY{o}{.}\PY{n}{optimal\PYZus{}pair}\PY{p}{[}\PY{l+m+mi}{1}\PY{p}{]}\PY{l+s+si}{:}\PY{l+s+s1}{.1f}\PY{l+s+si}{\PYZcb{}}\PY{l+s+s1}{ m/s}\PY{l+s+s1}{\PYZsq{}}\PY{p}{)}
\end{Verbatim}
\end{tcolorbox}

    \begin{Verbatim}[commandchars=\\\{\}]
Optimal Angle = 30.0 deg.
Optimal Starting Velocity = 30.0 m/s
    \end{Verbatim}

    In this case, the model obtained an ideal coordinate of
\(\theta=30\text{ deg}\) and \(v_0=30 \text{ m/s}\).

    \begin{tcolorbox}[breakable, size=fbox, boxrule=1pt, pad at break*=1mm,colback=cellbackground, colframe=cellborder]
\prompt{In}{incolor}{31}{\boxspacing}
\begin{Verbatim}[commandchars=\\\{\}]
\PY{n}{optimal\PYZus{}test} \PY{o}{=} \PY{n}{simulate}\PY{p}{(}
    \PY{n}{time\PYZus{}range}\PY{o}{=}\PY{p}{(}\PY{l+m+mi}{0}\PY{p}{,}\PY{l+m+mi}{10}\PY{p}{)}\PY{p}{,} 
    \PY{n}{dt}\PY{o}{=}\PY{l+m+mf}{0.01}\PY{p}{,} 
    \PY{n}{angle}\PY{o}{=}\PY{n}{results}\PY{o}{.}\PY{n}{optimal\PYZus{}pair}\PY{p}{[}\PY{l+m+mi}{0}\PY{p}{]}\PY{p}{,} 
    \PY{n}{initial\PYZus{}speed}\PY{o}{=}\PY{n}{results}\PY{o}{.}\PY{n}{optimal\PYZus{}pair}\PY{p}{[}\PY{l+m+mi}{1}\PY{p}{]}\PY{p}{,} 
    \PY{n}{do\PYZus{}gravity}\PY{o}{=}\PY{k+kc}{True}\PY{p}{,} 
    \PY{n}{do\PYZus{}thrust}\PY{o}{=}\PY{k+kc}{True}\PY{p}{,} 
    \PY{n}{do\PYZus{}collision}\PY{o}{=}\PY{k+kc}{True}\PY{p}{)}
\end{Verbatim}
\end{tcolorbox}

    \begin{tcolorbox}[breakable, size=fbox, boxrule=1pt, pad at break*=1mm,colback=cellbackground, colframe=cellborder]
\prompt{In}{incolor}{32}{\boxspacing}
\begin{Verbatim}[commandchars=\\\{\}]
\PY{n}{display}\PY{p}{(}\PY{n}{optimal\PYZus{}test}\PY{o}{.}\PY{n}{visualizations}\PY{o}{.}\PY{n}{trajectory\PYZus{}graph}\PY{p}{)}
\end{Verbatim}
\end{tcolorbox}

    \begin{center}
    \adjustimage{max size={0.9\linewidth}{0.9\paperheight}}{Project 1_files/Project 1_88_0.png}
    \end{center}
    { \hspace*{\fill} \\}
    
    \textbf{Figure 6.} In this case, a maximal speed and leftward turning
angle seems to be ideal because the ship is actually able to combine the
normal force from a collision with the slime planet and the
gravitational pull of the black hole to maximize its acceleration, and
keeping as small of a turning angle while doing this is ideal to
minimize fuel usage.

    \begin{tcolorbox}[breakable, size=fbox, boxrule=1pt, pad at break*=1mm,colback=cellbackground, colframe=cellborder]
\prompt{In}{incolor}{33}{\boxspacing}
\begin{Verbatim}[commandchars=\\\{\}]
\PY{n}{display}\PY{p}{(}\PY{n}{optimal\PYZus{}test}\PY{o}{.}\PY{n}{visualizations}\PY{o}{.}\PY{n}{distance\PYZus{}speed\PYZus{}acceleration\PYZus{}graph}\PY{p}{)}
\end{Verbatim}
\end{tcolorbox}

    \begin{center}
    \adjustimage{max size={0.9\linewidth}{0.9\paperheight}}{Project 1_files/Project 1_90_0.png}
    \end{center}
    { \hspace*{\fill} \\}
    
    \textbf{Figure 7.} As expected, the space ship spends nearly all of its
motion during the simulation accelerating away from the black hole with
very slight changes to acceleration, making it an ideal motion.

    \begin{tcolorbox}[breakable, size=fbox, boxrule=1pt, pad at break*=1mm,colback=cellbackground, colframe=cellborder]
\prompt{In}{incolor}{34}{\boxspacing}
\begin{Verbatim}[commandchars=\\\{\}]
\PY{n}{display}\PY{p}{(}\PY{n}{optimal\PYZus{}test}\PY{o}{.}\PY{n}{visualizations}\PY{o}{.}\PY{n}{kinetic\PYZus{}energy\PYZus{}momentum\PYZus{}graph}\PY{p}{)}
\end{Verbatim}
\end{tcolorbox}

    \begin{center}
    \adjustimage{max size={0.9\linewidth}{0.9\paperheight}}{Project 1_files/Project 1_92_0.png}
    \end{center}
    { \hspace*{\fill} \\}
    
    \textbf{Figure 8.} Furthermore, the energy and momentum of the system
also just increase with time due to the distance and velocity increasing
(but stabilize quite quickly due to the ship going very far very fast).

    \begin{tcolorbox}[breakable, size=fbox, boxrule=1pt, pad at break*=1mm,colback=cellbackground, colframe=cellborder]
\prompt{In}{incolor}{35}{\boxspacing}
\begin{Verbatim}[commandchars=\\\{\}]
\PY{n}{display}\PY{p}{(}\PY{n}{optimal\PYZus{}test}\PY{o}{.}\PY{n}{visualizations}\PY{o}{.}\PY{n}{collision\PYZus{}graph}\PY{p}{)}
\end{Verbatim}
\end{tcolorbox}

    \begin{center}
    \adjustimage{max size={0.9\linewidth}{0.9\paperheight}}{Project 1_files/Project 1_94_0.png}
    \end{center}
    { \hspace*{\fill} \\}
    
    \textbf{Figure 9.} The impulse in the situation best illustrates why
this is the optimal situation: the total normal force impulse comes the
closest out of all of the simulations to reaching the minimum
destructive impulse, so in other words, it is at the highest energy
combination that can successfully perform a collision at this angle. A
slight change to the angle or a higher velocity for this angle would
likely result in a collision.

    \hypertarget{weighting-alternatives-for-the-optimization}{%
\subsection{Weighting Alternatives for the
Optimization}\label{weighting-alternatives-for-the-optimization}}

    Additionally, we can consider other optimization cases to validate our
prior choice in weights. For example, we can pick a scenario where we
optimize entirely for the final ending speed and don't worry about
anything else. Let's see what this gets us:

    \begin{tcolorbox}[breakable, size=fbox, boxrule=1pt, pad at break*=1mm,colback=cellbackground, colframe=cellborder]
\prompt{In}{incolor}{36}{\boxspacing}
\begin{Verbatim}[commandchars=\\\{\}]
\PY{n}{results\PYZus{}2} \PY{o}{=} \PY{n}{simulate\PYZus{}all}\PY{p}{(}
    \PY{n}{time\PYZus{}range}\PY{o}{=}\PY{p}{(}\PY{l+m+mi}{0}\PY{p}{,} \PY{l+m+mi}{10}\PY{p}{)}\PY{p}{,} \PY{n}{dt}\PY{o}{=}\PY{l+m+mf}{0.01}\PY{p}{,}
    \PY{n}{angle\PYZus{}range}\PY{o}{=}\PY{p}{(}\PY{o}{\PYZhy{}}\PY{n}{np}\PY{o}{.}\PY{n}{pi}\PY{o}{/}\PY{l+m+mi}{2}\PY{p}{,} \PY{n}{np}\PY{o}{.}\PY{n}{pi}\PY{o}{/}\PY{l+m+mi}{2}\PY{p}{,} \PY{l+m+mi}{10}\PY{p}{)}\PY{p}{,}
    \PY{n}{speed\PYZus{}range}\PY{o}{=}\PY{p}{(}\PY{l+m+mi}{1}\PY{p}{,} \PY{l+m+mi}{30}\PY{p}{,} \PY{l+m+mi}{10}\PY{p}{)}\PY{p}{,}
    \PY{n}{fuel\PYZus{}bias}\PY{o}{=}\PY{l+m+mi}{0}\PY{p}{,}
    \PY{n}{distance\PYZus{}bias}\PY{o}{=}\PY{l+m+mi}{0}\PY{p}{,}
    \PY{n}{speed\PYZus{}bias}\PY{o}{=}\PY{l+m+mi}{1}\PY{p}{,}
    \PY{n}{debug\PYZus{}print}\PY{o}{=}\PY{k+kc}{False}
\PY{p}{)}
\end{Verbatim}
\end{tcolorbox}

    \begin{Verbatim}[commandchars=\\\{\}]
10\% - Iteration 10/100
20\% - Iteration 20/100
30\% - Iteration 30/100
40\% - Iteration 40/100
50\% - Iteration 50/100
60\% - Iteration 60/100
70\% - Iteration 70/100
80\% - Iteration 80/100
90\% - Iteration 90/100
100\% - Iteration 100/100
100\% - Completed.
Completed in 107.441 s.
    \end{Verbatim}

    \begin{tcolorbox}[breakable, size=fbox, boxrule=1pt, pad at break*=1mm,colback=cellbackground, colframe=cellborder]
\prompt{In}{incolor}{37}{\boxspacing}
\begin{Verbatim}[commandchars=\\\{\}]
\PY{n+nb}{print}\PY{p}{(}\PY{l+s+sa}{f}\PY{l+s+s1}{\PYZsq{}}\PY{l+s+s1}{Optimal Angle = }\PY{l+s+si}{\PYZob{}}\PY{n}{np}\PY{o}{.}\PY{n}{rad2deg}\PY{p}{(}\PY{n}{results\PYZus{}2}\PY{o}{.}\PY{n}{optimal\PYZus{}pair}\PY{p}{[}\PY{l+m+mi}{0}\PY{p}{]}\PY{p}{)}\PY{l+s+si}{:}\PY{l+s+s1}{.1f}\PY{l+s+si}{\PYZcb{}}\PY{l+s+s1}{ deg.}\PY{l+s+s1}{\PYZsq{}}\PY{p}{)}
\PY{n+nb}{print}\PY{p}{(}\PY{l+s+sa}{f}\PY{l+s+s1}{\PYZsq{}}\PY{l+s+s1}{Optimal Starting Velocity = }\PY{l+s+si}{\PYZob{}}\PY{n}{results\PYZus{}2}\PY{o}{.}\PY{n}{optimal\PYZus{}pair}\PY{p}{[}\PY{l+m+mi}{1}\PY{p}{]}\PY{l+s+si}{:}\PY{l+s+s1}{.1f}\PY{l+s+si}{\PYZcb{}}\PY{l+s+s1}{ m/s}\PY{l+s+s1}{\PYZsq{}}\PY{p}{)}
\end{Verbatim}
\end{tcolorbox}

    \begin{Verbatim}[commandchars=\\\{\}]
Optimal Angle = 30.0 deg.
Optimal Starting Velocity = 30.0 m/s
    \end{Verbatim}

    So, interestingly, we actually get the same results as with our previous
weights! This probably means that optimizing for speed naturally seems
to optimize the other two values, and in particular, the speed and
distance values are probably linked together (optimizing for one
optimizes the other). Let's see if this is true by optimizing solely for
distance:

    \begin{tcolorbox}[breakable, size=fbox, boxrule=1pt, pad at break*=1mm,colback=cellbackground, colframe=cellborder]
\prompt{In}{incolor}{38}{\boxspacing}
\begin{Verbatim}[commandchars=\\\{\}]
\PY{n}{results\PYZus{}3} \PY{o}{=} \PY{n}{simulate\PYZus{}all}\PY{p}{(}
    \PY{n}{time\PYZus{}range}\PY{o}{=}\PY{p}{(}\PY{l+m+mi}{0}\PY{p}{,} \PY{l+m+mi}{10}\PY{p}{)}\PY{p}{,} \PY{n}{dt}\PY{o}{=}\PY{l+m+mf}{0.01}\PY{p}{,}
    \PY{n}{angle\PYZus{}range}\PY{o}{=}\PY{p}{(}\PY{o}{\PYZhy{}}\PY{n}{np}\PY{o}{.}\PY{n}{pi}\PY{o}{/}\PY{l+m+mi}{2}\PY{p}{,} \PY{n}{np}\PY{o}{.}\PY{n}{pi}\PY{o}{/}\PY{l+m+mi}{2}\PY{p}{,} \PY{l+m+mi}{10}\PY{p}{)}\PY{p}{,}
    \PY{n}{speed\PYZus{}range}\PY{o}{=}\PY{p}{(}\PY{l+m+mi}{1}\PY{p}{,} \PY{l+m+mi}{30}\PY{p}{,} \PY{l+m+mi}{10}\PY{p}{)}\PY{p}{,}
    \PY{n}{fuel\PYZus{}bias}\PY{o}{=}\PY{l+m+mi}{0}\PY{p}{,}
    \PY{n}{distance\PYZus{}bias}\PY{o}{=}\PY{l+m+mi}{1}\PY{p}{,}
    \PY{n}{speed\PYZus{}bias}\PY{o}{=}\PY{l+m+mi}{0}\PY{p}{,}
    \PY{n}{debug\PYZus{}print}\PY{o}{=}\PY{k+kc}{False}
\PY{p}{)}
\end{Verbatim}
\end{tcolorbox}

    \begin{Verbatim}[commandchars=\\\{\}]
10\% - Iteration 10/100
20\% - Iteration 20/100
30\% - Iteration 30/100
40\% - Iteration 40/100
50\% - Iteration 50/100
60\% - Iteration 60/100
70\% - Iteration 70/100
80\% - Iteration 80/100
90\% - Iteration 90/100
100\% - Iteration 100/100
100\% - Completed.
Completed in 116.045 s.
    \end{Verbatim}

    \begin{tcolorbox}[breakable, size=fbox, boxrule=1pt, pad at break*=1mm,colback=cellbackground, colframe=cellborder]
\prompt{In}{incolor}{39}{\boxspacing}
\begin{Verbatim}[commandchars=\\\{\}]
\PY{n+nb}{print}\PY{p}{(}\PY{l+s+sa}{f}\PY{l+s+s1}{\PYZsq{}}\PY{l+s+s1}{Optimal Angle = }\PY{l+s+si}{\PYZob{}}\PY{n}{np}\PY{o}{.}\PY{n}{rad2deg}\PY{p}{(}\PY{n}{results\PYZus{}3}\PY{o}{.}\PY{n}{optimal\PYZus{}pair}\PY{p}{[}\PY{l+m+mi}{0}\PY{p}{]}\PY{p}{)}\PY{l+s+si}{:}\PY{l+s+s1}{.1f}\PY{l+s+si}{\PYZcb{}}\PY{l+s+s1}{ deg.}\PY{l+s+s1}{\PYZsq{}}\PY{p}{)}
\PY{n+nb}{print}\PY{p}{(}\PY{l+s+sa}{f}\PY{l+s+s1}{\PYZsq{}}\PY{l+s+s1}{Optimal Starting Velocity = }\PY{l+s+si}{\PYZob{}}\PY{n}{results\PYZus{}3}\PY{o}{.}\PY{n}{optimal\PYZus{}pair}\PY{p}{[}\PY{l+m+mi}{1}\PY{p}{]}\PY{l+s+si}{:}\PY{l+s+s1}{.1f}\PY{l+s+si}{\PYZcb{}}\PY{l+s+s1}{ m/s}\PY{l+s+s1}{\PYZsq{}}\PY{p}{)}
\end{Verbatim}
\end{tcolorbox}

    \begin{Verbatim}[commandchars=\\\{\}]
Optimal Angle = 30.0 deg.
Optimal Starting Velocity = 30.0 m/s
    \end{Verbatim}

    So, as the speed-only and distance-only optimization cases produced the
exact same results, we know without a doubt that they are co-optimized
parameters (optimizing for one optimizes the other). Let's see if this
trend is maintained in the case of optimizing only for fuel usage:

    \begin{tcolorbox}[breakable, size=fbox, boxrule=1pt, pad at break*=1mm,colback=cellbackground, colframe=cellborder]
\prompt{In}{incolor}{40}{\boxspacing}
\begin{Verbatim}[commandchars=\\\{\}]
\PY{n}{results\PYZus{}4} \PY{o}{=} \PY{n}{simulate\PYZus{}all}\PY{p}{(}
    \PY{n}{time\PYZus{}range}\PY{o}{=}\PY{p}{(}\PY{l+m+mi}{0}\PY{p}{,} \PY{l+m+mi}{10}\PY{p}{)}\PY{p}{,} \PY{n}{dt}\PY{o}{=}\PY{l+m+mf}{0.01}\PY{p}{,}
    \PY{n}{angle\PYZus{}range}\PY{o}{=}\PY{p}{(}\PY{o}{\PYZhy{}}\PY{n}{np}\PY{o}{.}\PY{n}{pi}\PY{o}{/}\PY{l+m+mi}{2}\PY{p}{,} \PY{n}{np}\PY{o}{.}\PY{n}{pi}\PY{o}{/}\PY{l+m+mi}{2}\PY{p}{,} \PY{l+m+mi}{10}\PY{p}{)}\PY{p}{,}
    \PY{n}{speed\PYZus{}range}\PY{o}{=}\PY{p}{(}\PY{l+m+mi}{1}\PY{p}{,} \PY{l+m+mi}{30}\PY{p}{,} \PY{l+m+mi}{10}\PY{p}{)}\PY{p}{,}
    \PY{n}{fuel\PYZus{}bias}\PY{o}{=}\PY{l+m+mi}{1}\PY{p}{,}
    \PY{n}{distance\PYZus{}bias}\PY{o}{=}\PY{l+m+mi}{0}\PY{p}{,}
    \PY{n}{speed\PYZus{}bias}\PY{o}{=}\PY{l+m+mi}{0}\PY{p}{,}
    \PY{n}{debug\PYZus{}print}\PY{o}{=}\PY{k+kc}{False}
\PY{p}{)}
\end{Verbatim}
\end{tcolorbox}

    \begin{Verbatim}[commandchars=\\\{\}]
10\% - Iteration 10/100
20\% - Iteration 20/100
30\% - Iteration 30/100
40\% - Iteration 40/100
50\% - Iteration 50/100
60\% - Iteration 60/100
70\% - Iteration 70/100
80\% - Iteration 80/100
90\% - Iteration 90/100
100\% - Iteration 100/100
100\% - Completed.
Completed in 111.872 s.
    \end{Verbatim}

    \begin{tcolorbox}[breakable, size=fbox, boxrule=1pt, pad at break*=1mm,colback=cellbackground, colframe=cellborder]
\prompt{In}{incolor}{41}{\boxspacing}
\begin{Verbatim}[commandchars=\\\{\}]
\PY{n+nb}{print}\PY{p}{(}\PY{l+s+sa}{f}\PY{l+s+s1}{\PYZsq{}}\PY{l+s+s1}{Optimal Angle = }\PY{l+s+si}{\PYZob{}}\PY{n}{np}\PY{o}{.}\PY{n}{rad2deg}\PY{p}{(}\PY{n}{results\PYZus{}4}\PY{o}{.}\PY{n}{optimal\PYZus{}pair}\PY{p}{[}\PY{l+m+mi}{0}\PY{p}{]}\PY{p}{)}\PY{l+s+si}{:}\PY{l+s+s1}{.1f}\PY{l+s+si}{\PYZcb{}}\PY{l+s+s1}{ deg.}\PY{l+s+s1}{\PYZsq{}}\PY{p}{)}
\PY{n+nb}{print}\PY{p}{(}\PY{l+s+sa}{f}\PY{l+s+s1}{\PYZsq{}}\PY{l+s+s1}{Optimal Starting Velocity = }\PY{l+s+si}{\PYZob{}}\PY{n}{results\PYZus{}4}\PY{o}{.}\PY{n}{optimal\PYZus{}pair}\PY{p}{[}\PY{l+m+mi}{1}\PY{p}{]}\PY{l+s+si}{:}\PY{l+s+s1}{.1f}\PY{l+s+si}{\PYZcb{}}\PY{l+s+s1}{ m/s}\PY{l+s+s1}{\PYZsq{}}\PY{p}{)}
\end{Verbatim}
\end{tcolorbox}

    \begin{Verbatim}[commandchars=\\\{\}]
Optimal Angle = 30.0 deg.
Optimal Starting Velocity = 7.4 m/s
    \end{Verbatim}

    Incredible. So, this trend was broken when we optimized only for fuel
usage. It yields the same turning angle, but a lower initial speed.
Let's see the trajectory to understand what is going on here:

    \begin{tcolorbox}[breakable, size=fbox, boxrule=1pt, pad at break*=1mm,colback=cellbackground, colframe=cellborder]
\prompt{In}{incolor}{43}{\boxspacing}
\begin{Verbatim}[commandchars=\\\{\}]
\PY{n}{irregular\PYZus{}case} \PY{o}{=} \PY{n}{simulate}\PY{p}{(}
    \PY{n}{time\PYZus{}range}\PY{o}{=}\PY{p}{(}\PY{l+m+mi}{0}\PY{p}{,}\PY{l+m+mi}{10}\PY{p}{)}\PY{p}{,} 
    \PY{n}{dt}\PY{o}{=}\PY{l+m+mf}{0.01}\PY{p}{,} 
    \PY{n}{angle}\PY{o}{=}\PY{n}{results\PYZus{}4}\PY{o}{.}\PY{n}{optimal\PYZus{}pair}\PY{p}{[}\PY{l+m+mi}{0}\PY{p}{]}\PY{p}{,} 
    \PY{n}{initial\PYZus{}speed}\PY{o}{=}\PY{n}{results\PYZus{}4}\PY{o}{.}\PY{n}{optimal\PYZus{}pair}\PY{p}{[}\PY{l+m+mi}{1}\PY{p}{]}\PY{p}{,} 
    \PY{n}{do\PYZus{}gravity}\PY{o}{=}\PY{k+kc}{True}\PY{p}{,} 
    \PY{n}{do\PYZus{}thrust}\PY{o}{=}\PY{k+kc}{True}\PY{p}{,} 
    \PY{n}{do\PYZus{}collision}\PY{o}{=}\PY{k+kc}{True}\PY{p}{)}
\end{Verbatim}
\end{tcolorbox}

    \begin{tcolorbox}[breakable, size=fbox, boxrule=1pt, pad at break*=1mm,colback=cellbackground, colframe=cellborder]
\prompt{In}{incolor}{44}{\boxspacing}
\begin{Verbatim}[commandchars=\\\{\}]
\PY{n}{display}\PY{p}{(}\PY{n}{irregular\PYZus{}case}\PY{o}{.}\PY{n}{visualizations}\PY{o}{.}\PY{n}{trajectory\PYZus{}graph}\PY{p}{)}
\end{Verbatim}
\end{tcolorbox}

    \begin{center}
    \adjustimage{max size={0.9\linewidth}{0.9\paperheight}}{Project 1_files/Project 1_108_0.png}
    \end{center}
    { \hspace*{\fill} \\}
    
    \textbf{Figure 10.} What is likely occuring here is that this angle is
the best for acceleration, period, so there is likely little fuel that
is actually required for the ship to ``turn'' into 30 degrees. This is
because there is so much non-thrust force that pulls it into this
direction: the gravitational pull of the black hole and the
normal/collision force from the slime planet. However, maximizing purely
for fuel causes the space ship to prefer a slower starting velocity most
likely because a higher velocity requires a larger amount of energy to
turn based on the inertia of the ship, so it's easier to make a turn
with a smaller momentum.

    \hypertarget{resultsconclusion}{%
\section{Results/Conclusion}\label{resultsconclusion}}

    Once again, the goal of this project has been to locate the safest
configuration of initial speed and turning angle for the spaceship to
maximize its chances of successful escape from the black hole, and the
simulation shows that the most successful option for the space ship is
the trivial solution: avoid collision altogether and just turn out of
the way as soon as possible by beginning with as small of a velocity as
possible. Even despite how much fuel this eats up, the time requirement
to perform the collision doesn't yield a net larger radial distance from
the black hole or final speed for the space ship as compared to just
turning as soon as possible and then spending the rest of the simulation
time accelerating out of the way of the black hole for the remaining
time. However, we also know that a collision is favorable for certain
speeds due to the ship not being able to turn out of the way, and this
is evident from the contour plot as at higher speeds, not turning at all
or even turning closer into the black hole to further encourage a
collision becomes more favorable.

    \hypertarget{limitations}{%
\subsubsection{Limitations}\label{limitations}}

This model has several limitations. There is little to no consideration
of realistic geometry and both the spaceship and the black hole are
approximated as pointlike. It could not be realistically applied in
scenarios requiring more complex chemical mixes for fuels due to the
heavy simplification on the specific impulse, and most importantly, the
optimization model used has the problem of having arbitrary weights
(without proper weight generation based on industry-standard methods),
leading to the seemingly trivial results that it generates for valid
optimal solutions.

The biggest limitation of this model, by far, is the parameters. A much
better simulation for this would focus on observing real-world values
for a simulation with, say, an analogous situation, but working with
real planetary-size values is not naturally possible without a much
stronger computing environment, so this simulation is completely limited
in the computational ability available in this situation.

    \hypertarget{next-steps}{%
\subsubsection{Next Steps}\label{next-steps}}

\begin{enumerate}
\def\labelenumi{\arabic{enumi}.}
\tightlist
\item
  \emph{More detail can be added to the model of the space ship and the
  black hole.} To begin, the spaceship can be actually modeled with more
  accurate geometry rather than being point-like, and the black hole can
  be given a radius, as the event horizon of the black hole is entirely
  not considered, and collision effects of the geometry of the spaceship
  aren't considered either.
\item
  \emph{The thrust/fuel model can be made much more realistic.} The
  spaceship's thrust model is heavily simplified to have a constant
  thrust force, but in reality, this would build up over time with
  respect to intricacies of the actual ship's multiple engines (rather
  than one). Furthermore, the fuel considered in this model is heavily
  simplified as well, as in reality, multiple fuels were used aboard the
  real Apollo-era space shuttles.
\item
  \emph{More objects can be added to this collision.} This project's
  space consideration is large enough to realistically have more
  gravity-effecting objects within the area, such as asteroids or other
  planets.
\end{enumerate}

    \hypertarget{acknowledgements}{%
\subsubsection{Acknowledgements}\label{acknowledgements}}

\begin{itemize}
\tightlist
\item
  ChatGPT has been used extensively for brainstorming. For example, I
  have never performed a multi-objective optimization before, so I had
  the following exchange:
\end{itemize}

\begin{quote}
I want to create a value for my simulation to optimize. I want to
simultaneously maximize the radial distance of the ship from the black
hole, the amount of fuel left in the ship, and it's velocity at the end
of the simulation. How can I combine these in a correct, weighted way?
\end{quote}

\begin{quote}
Nice --- that's a classic multi-objective setup. There are two clean
ways to proceed: Collapse to one scalar (single objective) --- easiest
to plug into ordinary optimizers or keep it multi-objective (Pareto
optimization) --- finds tradeoff/front between the three objectives.
Below I give a practical recipe for both, plus working Python code you
can drop into your simulation. \ldots{}
\end{quote}

\begin{itemize}
\tightlist
\item
  I also often used ChatGPT and StackOverFlow for bugfixing or function
  writing, such as when I was determining how to use \texttt{set\_UVC}
  for the animations (to draw arrows) and found the MatPlotLib
  documentation too vague:
  https://stackoverflow.com/questions/48911643/set-uvc-equivilent-for-a-3d-quiver-plot-in-matplotlib
\end{itemize}

    \hypertarget{bibliography}{%
\subsubsection{Bibliography}\label{bibliography}}

{[}1{]} https://en.wikipedia.org/wiki/Specific\_impulse

{[}2{]} https://en.wikipedia.org/wiki/Standard\_gravity

{[}3{]} Hunt, K. H., and Crossley, F. R. E. (June 1, 1975).
``Coefficient of Restitution Interpreted as Damping in Vibroimpact.''
ASME. J. Appl. Mech. June 1975; 42(2): 440--445.
https://doi.org/10.1115/1.3423596

{[}4{]}
https://en.wikipedia.org/wiki/Multi-objective\_optimization\#Scalarizing

{[}5{]}
https://www.aqua-calc.com/page/density-table/substance/gelatin-blank-desserts-coma-and-blank-dry-blank-mix-coma-and-blank-prepared-blank-with-blank-water

{[}6{]} https://en.wikipedia.org/wiki/Mars

{[}7{]} https://en.wikipedia.org/wiki/Mercury\_(planet)

{[}10{]} https://www.nasa.gov/reference/the-space-shuttle/

{[}11{]}
https://en.wikipedia.org/wiki/Space\_Shuttle\#cite\_note-woodcock-5

{[}12{]} https://en.wikipedia.org/wiki/Solar\_mass

    \hypertarget{appendix}{%
\section{Appendix}\label{appendix}}

    \hypertarget{a---validation}{%
\subsection{A - Validation}\label{a---validation}}

    \hypertarget{a-1-intuition}{%
\subsubsection{A-1: Intuition}\label{a-1-intuition}}

    \begin{tcolorbox}[breakable, size=fbox, boxrule=1pt, pad at break*=1mm,colback=cellbackground, colframe=cellborder]
\prompt{In}{incolor}{45}{\boxspacing}
\begin{Verbatim}[commandchars=\\\{\}]
\PY{k}{def} \PY{n+nf}{test\PYZus{}simulation}\PY{p}{(}\PY{n}{angle}\PY{p}{,} \PY{n}{speed}\PY{p}{)}\PY{p}{:}
    \PY{n}{simulation\PYZus{}result} \PY{o}{=} \PY{n}{simulate}\PY{p}{(}
        \PY{n}{time\PYZus{}range}\PY{o}{=}\PY{p}{(}\PY{l+m+mi}{0}\PY{p}{,}\PY{l+m+mi}{10}\PY{p}{)}\PY{p}{,} 
        \PY{n}{dt}\PY{o}{=}\PY{l+m+mf}{0.01}\PY{p}{,} 
        \PY{n}{angle}\PY{o}{=}\PY{n}{angle}\PY{p}{,} 
        \PY{n}{initial\PYZus{}speed}\PY{o}{=}\PY{n}{speed}\PY{p}{,} 
        \PY{n}{do\PYZus{}gravity}\PY{o}{=}\PY{k+kc}{True}\PY{p}{,} 
        \PY{n}{do\PYZus{}thrust}\PY{o}{=}\PY{k+kc}{True}\PY{p}{,} 
        \PY{n}{do\PYZus{}collision}\PY{o}{=}\PY{k+kc}{True}\PY{p}{,}
        \PY{n}{debug\PYZus{}print}\PY{o}{=}\PY{k+kc}{False}
    \PY{p}{)}
    \PY{n}{ss} \PY{o}{=} \PY{n}{simulation\PYZus{}result}\PY{o}{.}\PY{n}{space\PYZus{}ship}
            
    \PY{k}{if} \PY{n}{ss}\PY{o}{.}\PY{n}{destroyed}\PY{p}{:}
        \PY{n+nb}{print}\PY{p}{(}\PY{l+s+sa}{f}\PY{l+s+s1}{\PYZsq{}}\PY{l+s+s1}{(}\PY{l+s+si}{\PYZob{}}\PY{n}{angle}\PY{l+s+si}{:}\PY{l+s+s1}{.1f}\PY{l+s+si}{\PYZcb{}}\PY{l+s+s1}{ rad, }\PY{l+s+si}{\PYZob{}}\PY{n}{speed}\PY{l+s+si}{:}\PY{l+s+s1}{.1f}\PY{l+s+si}{\PYZcb{}}\PY{l+s+s1}{ m/s) \PYZhy{}\PYZgt{} Ship Destroyed!}\PY{l+s+s1}{\PYZsq{}}\PY{p}{)}
        \PY{k}{return}
            
    \PY{n}{mf} \PY{o}{=} \PY{n}{ss}\PY{o}{.}\PY{n}{remaining\PYZus{}fuel\PYZus{}mass}\PY{p}{[}\PY{o}{\PYZhy{}}\PY{l+m+mi}{1}\PY{p}{]}
    \PY{n}{rf} \PY{o}{=} \PY{n}{np}\PY{o}{.}\PY{n}{linalg}\PY{o}{.}\PY{n}{norm}\PY{p}{(}\PY{n}{ss}\PY{o}{.}\PY{n}{position}\PY{p}{[}\PY{o}{\PYZhy{}}\PY{l+m+mi}{1}\PY{p}{]}\PY{p}{)}
    \PY{n}{vf} \PY{o}{=} \PY{n}{np}\PY{o}{.}\PY{n}{linalg}\PY{o}{.}\PY{n}{norm}\PY{p}{(}\PY{n}{ss}\PY{o}{.}\PY{n}{velocity}\PY{p}{[}\PY{o}{\PYZhy{}}\PY{l+m+mi}{1}\PY{p}{]}\PY{p}{)}
            
    \PY{n+nb}{print}\PY{p}{(}\PY{l+s+sa}{f}\PY{l+s+s1}{\PYZsq{}}\PY{l+s+s1}{(}\PY{l+s+si}{\PYZob{}}\PY{n}{angle}\PY{l+s+si}{:}\PY{l+s+s1}{.1f}\PY{l+s+si}{\PYZcb{}}\PY{l+s+s1}{ rad, }\PY{l+s+si}{\PYZob{}}\PY{n}{speed}\PY{l+s+si}{:}\PY{l+s+s1}{.1f}\PY{l+s+si}{\PYZcb{}}\PY{l+s+s1}{ m/s) =\PYZgt{} (}\PY{l+s+si}{\PYZob{}}\PY{n}{mf}\PY{l+s+si}{:}\PY{l+s+s1}{.1f}\PY{l+s+si}{\PYZcb{}}\PY{l+s+s1}{ kg, }\PY{l+s+si}{\PYZob{}}\PY{n}{rf}\PY{l+s+si}{:}\PY{l+s+s1}{.1f}\PY{l+s+si}{\PYZcb{}}\PY{l+s+s1}{ m, }\PY{l+s+si}{\PYZob{}}\PY{n}{vf}\PY{l+s+si}{:}\PY{l+s+s1}{.1f}\PY{l+s+si}{\PYZcb{}}\PY{l+s+s1}{ m/s)}\PY{l+s+s1}{\PYZsq{}}\PY{p}{)}
    \PY{k}{if} \PY{n}{ss}\PY{o}{.}\PY{n}{total\PYZus{}collision\PYZus{}impulse} \PY{o}{\PYZgt{}} \PY{l+m+mi}{0}\PY{p}{:}
        \PY{n+nb}{print}\PY{p}{(}\PY{l+s+s1}{\PYZsq{}}\PY{l+s+s1}{Collision occured!}\PY{l+s+s1}{\PYZsq{}}\PY{p}{)}
    \PY{k}{else}\PY{p}{:}
        \PY{n+nb}{print}\PY{p}{(}\PY{l+s+s1}{\PYZsq{}}\PY{l+s+s1}{No collision recorded.}\PY{l+s+s1}{\PYZsq{}}\PY{p}{)}
\end{Verbatim}
\end{tcolorbox}

    We can also verify our results using situations that should obviously
yield good contour scores and bad contour scores. For example, going too
fast should naturally destroy the ship, so if we set the ship to move at
a speed of \(500\) m/s, then it should crash:

    \begin{tcolorbox}[breakable, size=fbox, boxrule=1pt, pad at break*=1mm,colback=cellbackground, colframe=cellborder]
\prompt{In}{incolor}{46}{\boxspacing}
\begin{Verbatim}[commandchars=\\\{\}]
\PY{n}{test\PYZus{}simulation}\PY{p}{(}\PY{l+m+mi}{0}\PY{p}{,} \PY{l+m+mi}{100}\PY{p}{)}
\end{Verbatim}
\end{tcolorbox}

    \begin{Verbatim}[commandchars=\\\{\}]
(0.0 rad, 100.0 m/s) -> Ship Destroyed!
    \end{Verbatim}

    Similarly, a velocity of zero absolutely should not crash! We can
reasonably assume that it should collide with the planet as well. (For
math reasons, we can't actually use 0 velocity as the velocity is
divided, so I'll use a very small one).

    \begin{tcolorbox}[breakable, size=fbox, boxrule=1pt, pad at break*=1mm,colback=cellbackground, colframe=cellborder]
\prompt{In}{incolor}{47}{\boxspacing}
\begin{Verbatim}[commandchars=\\\{\}]
\PY{n}{test\PYZus{}simulation}\PY{p}{(}\PY{l+m+mi}{0}\PY{p}{,} \PY{l+m+mf}{1e\PYZhy{}12}\PY{p}{)}
\end{Verbatim}
\end{tcolorbox}

    \begin{Verbatim}[commandchars=\\\{\}]
(0.0 rad, 0.0 m/s) => (6088.0 kg, 135.2 m, 35.9 m/s)
Collision occured!
    \end{Verbatim}

    Lastly, we know where we can find a reasonably good final result, a low
velocity and low, leftward turning angle. Let's just pick one: it should
result in a higher final distance as compared to a high velocity with a
low turning angle.

    \begin{tcolorbox}[breakable, size=fbox, boxrule=1pt, pad at break*=1mm,colback=cellbackground, colframe=cellborder]
\prompt{In}{incolor}{48}{\boxspacing}
\begin{Verbatim}[commandchars=\\\{\}]
\PY{n}{test\PYZus{}simulation}\PY{p}{(}\PY{o}{\PYZhy{}}\PY{n}{np}\PY{o}{.}\PY{n}{pi}\PY{o}{/}\PY{l+m+mi}{8}\PY{p}{,} \PY{l+m+mf}{1e\PYZhy{}12}\PY{p}{)}
\PY{n}{test\PYZus{}simulation}\PY{p}{(}\PY{o}{\PYZhy{}}\PY{n}{np}\PY{o}{.}\PY{n}{pi}\PY{o}{/}\PY{l+m+mi}{8}\PY{p}{,} \PY{l+m+mi}{7}\PY{p}{)}
\end{Verbatim}
\end{tcolorbox}

    \begin{Verbatim}[commandchars=\\\{\}]
(-0.4 rad, 0.0 m/s) => (6080.2 kg, 103.1 m, 29.7 m/s)
Collision occured!
(-0.4 rad, 7.0 m/s) => (6080.2 kg, 164.6 m, 38.6 m/s)
Collision occured!
    \end{Verbatim}

    As we can see, that's nearly double the final distance.

    \hypertarget{a-2-animations}{%
\subsection{A-2 Animations}\label{a-2-animations}}

    I can visually validate these scenarios using animations, because from
these, we can directly observe the behavior of the objects in the system
and apply our intuition of mechanics in the real world. For example, in
the first test/example case defined initially, it carries the following
animation:

    \begin{tcolorbox}[breakable, size=fbox, boxrule=1pt, pad at break*=1mm,colback=cellbackground, colframe=cellborder]
\prompt{In}{incolor}{49}{\boxspacing}
\begin{Verbatim}[commandchars=\\\{\}]
\PY{n}{display}\PY{p}{(}\PY{n}{produce\PYZus{}animation}\PY{p}{(}\PY{n}{test}\PY{p}{,} \PY{n}{display\PYZus{}arrows}\PY{o}{=}\PY{k+kc}{True}\PY{p}{)}\PY{p}{)}
\end{Verbatim}
\end{tcolorbox}

    \begin{Verbatim}[commandchars=\\\{\}]
Animating simulation..
Animation complete! Writing..
Elapsed time: 12.4 seconds
    \end{Verbatim}

    
    \begin{Verbatim}[commandchars=\\\{\}]
<IPython.core.display.HTML object>
    \end{Verbatim}

    
    And this simulation shows a nice, simple collision. Another case is the
optimal solution found at the end, which looks as so:

    \begin{tcolorbox}[breakable, size=fbox, boxrule=1pt, pad at break*=1mm,colback=cellbackground, colframe=cellborder]
\prompt{In}{incolor}{50}{\boxspacing}
\begin{Verbatim}[commandchars=\\\{\}]
\PY{n}{display}\PY{p}{(}\PY{n}{produce\PYZus{}animation}\PY{p}{(}\PY{n}{optimal\PYZus{}test}\PY{p}{,} \PY{n}{display\PYZus{}arrows}\PY{o}{=}\PY{k+kc}{True}\PY{p}{)}\PY{p}{)}
\end{Verbatim}
\end{tcolorbox}

    \begin{Verbatim}[commandchars=\\\{\}]
Animating simulation..
Animation complete! Writing..
Elapsed time: 14.2 seconds
    \end{Verbatim}

    
    \begin{Verbatim}[commandchars=\\\{\}]
<IPython.core.display.HTML object>
    \end{Verbatim}

    
    Both of these animations look, on visual observation, to be mechanically
correct. Even if individual numbers are incorrect, the how each of the
objects act relative to each other (especially the soft body collision)
seem perfectly in-line with real-world mechanics.

Note that we can also directly observe all of the characteristics
described in the caption of Figure 5, as a slight collision does occur
here, but it turns just enough to minimize the impulse such that it
doesn't crash, maximizes the exit velocity, maximizes the distance at
the end of the simulation, and minimizes the total fuel used by turning
only slightly. It's essentially a sweet spot: if the ship were to have
turned

    \hypertarget{a-3-by-hand-calculations}{%
\subsection{A-3 By-Hand Calculations}\label{a-3-by-hand-calculations}}

    Due to this simulation being a 3-body problem, accurate by-hand
calculations for every part would be neigh-impossible. What can be
realistically done, however, is verifying basic characteristics for
individual objects. For this, we can use our previous example case
\texttt{test} (\(v=2\) m/s and \(\theta = -0.8\) rad.). For example, we
can generally check that the magnitudes of force experienced
instantaneously by the space ship are generally reasonable in size.

    First, we can get the maximal force based on the force it feels when its
the closest to the center of the slime planet (so the highest overlap
distance \(\delta\)) as the radius of the slime planet minus their
minimum distance, \(\delta_{max} \approx 6.8 - 4.72 = 2.08 \text{ m}\)

    \begin{tcolorbox}[breakable, size=fbox, boxrule=1pt, pad at break*=1mm,colback=cellbackground, colframe=cellborder]
\prompt{In}{incolor}{51}{\boxspacing}
\begin{Verbatim}[commandchars=\\\{\}]
\PY{n}{distances} \PY{o}{=} \PY{n}{np}\PY{o}{.}\PY{n}{linalg}\PY{o}{.}\PY{n}{norm}\PY{p}{(}\PY{n}{test}\PY{o}{.}\PY{n}{space\PYZus{}ship}\PY{o}{.}\PY{n}{position} \PY{o}{\PYZhy{}} \PY{n}{test}\PY{o}{.}\PY{n}{slime\PYZus{}planet}\PY{o}{.}\PY{n}{position}\PY{p}{,} \PY{n}{axis}\PY{o}{=}\PY{l+m+mi}{1}\PY{p}{)}
\PY{n}{closest\PYZus{}distance\PYZus{}index} \PY{o}{=} \PY{n}{np}\PY{o}{.}\PY{n}{argmin}\PY{p}{(}\PY{n}{distances}\PY{p}{)}
\PY{n}{delta} \PY{o}{=} \PY{n}{SLIME\PYZus{}PLANET\PYZus{}RADIUS} \PY{o}{\PYZhy{}} \PY{n}{distances}\PY{p}{[}\PY{n}{closest\PYZus{}distance\PYZus{}index}\PY{p}{]}
\PY{n+nb}{print}\PY{p}{(}\PY{l+s+sa}{f}\PY{l+s+s1}{\PYZsq{}}\PY{l+s+s1}{Max(Delta) = }\PY{l+s+si}{\PYZob{}}\PY{n}{delta}\PY{l+s+si}{:}\PY{l+s+s1}{.2f}\PY{l+s+si}{\PYZcb{}}\PY{l+s+s1}{ m}\PY{l+s+s1}{\PYZsq{}}\PY{p}{)}
\end{Verbatim}
\end{tcolorbox}

    \begin{Verbatim}[commandchars=\\\{\}]
Max(Delta) = 2.13 m
    \end{Verbatim}

    Then, we can get the speed at this distance:

    \begin{tcolorbox}[breakable, size=fbox, boxrule=1pt, pad at break*=1mm,colback=cellbackground, colframe=cellborder]
\prompt{In}{incolor}{52}{\boxspacing}
\begin{Verbatim}[commandchars=\\\{\}]
\PY{n}{speeds} \PY{o}{=} \PY{n}{np}\PY{o}{.}\PY{n}{linalg}\PY{o}{.}\PY{n}{norm}\PY{p}{(}\PY{n}{test}\PY{o}{.}\PY{n}{space\PYZus{}ship}\PY{o}{.}\PY{n}{velocity}\PY{p}{[}\PY{l+m+mi}{50}\PY{p}{:}\PY{p}{]}\PY{p}{,} \PY{n}{axis}\PY{o}{=}\PY{l+m+mi}{1}\PY{p}{)}
\PY{n+nb}{print}\PY{p}{(}\PY{l+s+sa}{f}\PY{l+s+s1}{\PYZsq{}}\PY{l+s+s1}{Speed = }\PY{l+s+si}{\PYZob{}}\PY{n}{speeds}\PY{p}{[}\PY{n}{closest\PYZus{}distance\PYZus{}index}\PY{p}{]}\PY{l+s+si}{:}\PY{l+s+s1}{.2f}\PY{l+s+si}{\PYZcb{}}\PY{l+s+s1}{ m/s}\PY{l+s+s1}{\PYZsq{}}\PY{p}{)}
\end{Verbatim}
\end{tcolorbox}

    \begin{Verbatim}[commandchars=\\\{\}]
Speed = 6.24 m/s
    \end{Verbatim}

    Now, we know that the mechanics of this simulation are governed by
\[F(\delta,v) = 100(1140)\delta^\frac{3}{2} + 40(1140)\delta^\frac{3}{2} v,\]
(assuming that \(v = \frac{d \delta}{dt}\), which isn't necessarily true
in the 2-D simulation, but should be roughly close due to the direction
at which the ship is moving into the planet) so we can reasonably assume
that the maximal magnitude of force experienced purely due to this
collision should be around
\[F(2.08,5.44) \approx 1.1 \times 10^{6} \text{ N},\] so force values on
an order of magnitude of about \(0 \le F \le 10^{6}\) roughly seems to
make sense given these parameters. Now, let's see if this is what we
actually see:

    \begin{tcolorbox}[breakable, size=fbox, boxrule=1pt, pad at break*=1mm,colback=cellbackground, colframe=cellborder]
\prompt{In}{incolor}{53}{\boxspacing}
\begin{Verbatim}[commandchars=\\\{\}]
\PY{n}{forces} \PY{o}{=} \PY{n}{np}\PY{o}{.}\PY{n}{linalg}\PY{o}{.}\PY{n}{norm}\PY{p}{(}\PY{n}{test}\PY{o}{.}\PY{n}{space\PYZus{}ship}\PY{o}{.}\PY{n}{force}\PY{p}{,} \PY{n}{axis}\PY{o}{=}\PY{l+m+mi}{1}\PY{p}{)}
\PY{n+nb}{print}\PY{p}{(}\PY{l+s+sa}{f}\PY{l+s+s1}{\PYZsq{}}\PY{l+s+s1}{Max(Force) = }\PY{l+s+si}{\PYZob{}}\PY{n+nb}{max}\PY{p}{(}\PY{n}{forces}\PY{p}{)}\PY{l+s+si}{:}\PY{l+s+s1}{.2e}\PY{l+s+si}{\PYZcb{}}\PY{l+s+s1}{ N}\PY{l+s+s1}{\PYZsq{}}\PY{p}{)}
\end{Verbatim}
\end{tcolorbox}

    \begin{Verbatim}[commandchars=\\\{\}]
Max(Force) = 7.48e+05 N
    \end{Verbatim}

    So, we get a maxmimum instantaneous force on an order of magnitude of
about \(7.8 \times 10^{5}\), which is off by \(2.2 \times 10^{5}\), but
fairly close. The error can be best explained by other forces in the
simulation that aren't being accounted for (i.e., the magnitude of
gravity from the slime planet and the black hole), but in general, the
order of magnitude of this force seem entirely reasonable.

    Next, for extra verification, we can just verify that the magnitude of
kinetic energy at the beginnings and ends of the simulation make
realistic sense. The test case predicts an ending kinetic energy
beginning around \(4 \times 10^4\) Joules

    \begin{tcolorbox}[breakable, size=fbox, boxrule=1pt, pad at break*=1mm,colback=cellbackground, colframe=cellborder]
\prompt{In}{incolor}{54}{\boxspacing}
\begin{Verbatim}[commandchars=\\\{\}]
\PY{n+nb}{print}\PY{p}{(}\PY{l+s+sa}{f}\PY{l+s+s1}{\PYZsq{}}\PY{l+s+s1}{K=}\PY{l+s+si}{\PYZob{}}\PY{n}{test}\PY{o}{.}\PY{n}{space\PYZus{}ship}\PY{o}{.}\PY{n}{kinetic\PYZus{}energy}\PY{p}{[}\PY{l+m+mi}{1}\PY{p}{]}\PY{l+s+si}{:}\PY{l+s+s1}{.1e}\PY{l+s+si}{\PYZcb{}}\PY{l+s+s1}{ J}\PY{l+s+s1}{\PYZsq{}}\PY{p}{)}
\end{Verbatim}
\end{tcolorbox}

    \begin{Verbatim}[commandchars=\\\{\}]
K=4.0e+04 J
    \end{Verbatim}

    and ending at around \(2.8 \times 10^{5}\) Joules

    \begin{tcolorbox}[breakable, size=fbox, boxrule=1pt, pad at break*=1mm,colback=cellbackground, colframe=cellborder]
\prompt{In}{incolor}{55}{\boxspacing}
\begin{Verbatim}[commandchars=\\\{\}]
\PY{n+nb}{print}\PY{p}{(}\PY{l+s+sa}{f}\PY{l+s+s1}{\PYZsq{}}\PY{l+s+s1}{K=}\PY{l+s+si}{\PYZob{}}\PY{n}{test}\PY{o}{.}\PY{n}{space\PYZus{}ship}\PY{o}{.}\PY{n}{kinetic\PYZus{}energy}\PY{p}{[}\PY{o}{\PYZhy{}}\PY{l+m+mi}{1}\PY{p}{]}\PY{l+s+si}{:}\PY{l+s+s1}{.1e}\PY{l+s+si}{\PYZcb{}}\PY{l+s+s1}{ J}\PY{l+s+s1}{\PYZsq{}}\PY{p}{)}
\end{Verbatim}
\end{tcolorbox}

    \begin{Verbatim}[commandchars=\\\{\}]
K=1.4e+07 J
    \end{Verbatim}

    Knowing that the space ship has a mass of \(m=2 \times 10^4\) kilograms
and begins with a velocity of \(2\) m/s, we get
\[K_i = \frac{1}{2} mv^2 = \frac{1}{2} (2 \times 10^4)(2)^2 = 4 \times 10^4 \text{ J},\]
so the starting kinetic energy makes sense.

    For the ending kinetic energy, the space ship exits at a seemingly
constant velocity of about \(5.27\) m/s, so we can also verify that the
kinetic energy should be
\[K_f = \frac{1}{2} mv^2 = \frac{1}{2}(2 \times 10^4)(5.27)^2 = 2.78 \times 10^5 \text{ J},\]
and that shows that the calculations for the beginning and ending
kinetic energies are fine.

    \begin{tcolorbox}[breakable, size=fbox, boxrule=1pt, pad at break*=1mm,colback=cellbackground, colframe=cellborder]
\prompt{In}{incolor}{56}{\boxspacing}
\begin{Verbatim}[commandchars=\\\{\}]
\PY{n+nb}{print}\PY{p}{(}\PY{l+s+sa}{f}\PY{l+s+s1}{\PYZsq{}}\PY{l+s+s1}{V\PYZus{}end = }\PY{l+s+si}{\PYZob{}}\PY{n}{speeds}\PY{p}{[}\PY{o}{\PYZhy{}}\PY{l+m+mi}{1}\PY{p}{]}\PY{l+s+si}{:}\PY{l+s+s1}{.2f}\PY{l+s+si}{\PYZcb{}}\PY{l+s+s1}{\PYZsq{}}\PY{p}{)}
\end{Verbatim}
\end{tcolorbox}

    \begin{Verbatim}[commandchars=\\\{\}]
V\_end = 37.68
    \end{Verbatim}

    \hypertarget{b---reflection-questions}{%
\subsection{B - Reflection Questions}\label{b---reflection-questions}}

    \textbf{Reflection 1: Coding Approaches (A)}

\begin{quote}
(How well did you apply and extend your coding knowledge in this
project? Consider steps you took to make the code more efficient,
readable and/or concise. Discuss any new-to-you coding techniques,
functions or python packages that you learned how to use. Reflect on any
unforeseen coding challenges you faced in completing this project.)
\end{quote}

    I hadn't performed \(n\)-Body gravitational simulation before,
especially not in two-dimensions, so performing the vector calculus
necessary to simulate that was a real challenge. In particular, working
out the correct vectors for turning by thrust, determining when to stop
thrust, and calculating the correct vectors for the softbody collision
were complicated. I had to keep trying to strike a balance between
generality, the ability to potentially add new objects to this system or
remove objects from it, with also having code that was as simple and
easy as possible. None of the python packages I used were new to me, but
some of the functions (or, moreso, mathematical processes) I used were
completely new, like the thrust-fuel equation and the Hunt-Crossley
collision model.

    \textbf{Reflection 2: Coding Approaches (B)}

\begin{quote}
(Highlight an aspect of your code that you feel you did particularily
well. Discuss an aspect of your code that would benefit the most from
further effort.)
\end{quote}

    I'm quite proud of how I was able to structure the
\texttt{calculate\_net\_force} function in its entirety. The three main
categories of forces to be considered in this simulation (gravity,
thrust/turning, and collision) more or less formed the three main areas
on which I had to spend the majority of my time on this project. I was
able to write it such that generality is maintained in the rigid object
system (meaning that this can extent to an \(n\)-Body simulation with
minimal changes), and it's well-broken down into distinct parts, with a
few \texttt{if} statements determining whether or not each calculation
needs to be made. It allows for me to turn off certain force components
for bugfixing reasons alongside calculating each independently. I'm
particularly proud of

\begin{Shaded}
\begin{Highlighting}[]
\KeywordTok{def}\NormalTok{ calculate\_thrust(ship, dt, last\_frame):}
\end{Highlighting}
\end{Shaded}

because I found this part to be, by far, the most ``mathematical.''
Because of the complexities of the vector calculus combined with the new
physics of basic engine physics, getting this part to work was
incredibly difficult. On the other hand, the part of my code with the
weakest/sloppiest code is

\begin{Shaded}
\begin{Highlighting}[]
\KeywordTok{def}\NormalTok{ plot\_results(angle, initial\_speed, time\_data, slime\_planet, space\_ship, black\_hole):}
\end{Highlighting}
\end{Shaded}

as I'm more or less just rewriting code over and over, and most
importantly, this code isn't extensible to multiple objects, just the
space ship/slime planet system.

    \textbf{Reflection 3: Simulation phyiscs and investigation (A)}

\begin{quote}
(How well did you apply and extend your physical modelling and
scientific investigation skills in this project? Consider the phase
space you chose to explore and how throroughly you explored it. Consider
how you translated physics into code and if appropriate any new physics
you learned or developed a more thorough understanding of.)
\end{quote}

    I did a fairly shallow, non-exhaustive search over the ranges of
\(\theta \in [-\pi, \pi]\) and \(v_0 \in [1, 30]\) over the phase space
of \(\theta\) and \(v_0\), collecting only 10 evenly-spaced angle values
over the angular domain and 10 evenly-spaced speed values over the speed
domain. The raw physics of this simulation proved simple, it was
essentially an optimization (maximization) problem in the form
\[U(\theta, v_0) = U(\mathbf{s}(\theta, v_0)) = w_1 s(\theta, v_0)_1 + w_2 s(\theta, v_0)_2 + w_3 s(\theta, v_0)_3,\]
over \(\theta \in [-\pi/2,\pi/2]\) and \(v_0 \in [1,30]\)
\[\mathbf{s}(\theta, v_0) = \left\langle m_f - \int_{t_0}^{t_f} |\dot{m}_f| dt, |\mathbf{r}(t_f)|, |\frac{d \mathbf{r}}{dt}(t_f)| \right\rangle,\]
where \[\frac{d^2 \mathbf{r}}{dt^2}(t) = \frac{1}{m}\mathbf{F}(t).\] It
forms an overall fairly complicated mathematical structure in how each
piece interacts, but translating it to code was actually fairly
intuitive. Numerical integration, for example, was almost like
clockwork, reducing down to a simple for loop (similar to the
following):

\begin{Shaded}
\begin{Highlighting}[]
\KeywordTok{def}\NormalTok{ net\_force(t): ...}

\NormalTok{mass}\OperatorTok{=}\NormalTok{...}
\NormalTok{t0}\OperatorTok{=}\NormalTok{...}
\NormalTok{tf}\OperatorTok{=}\NormalTok{...}
\NormalTok{dt}\OperatorTok{=}\NormalTok{...}

\NormalTok{n }\OperatorTok{=}\NormalTok{ (tf }\OperatorTok{{-}}\NormalTok{ t0) }\OperatorTok{/}\NormalTok{ dt}

\NormalTok{time}\OperatorTok{=}\NormalTok{np.linspace(t0, tf, dt)}
\NormalTok{position}\OperatorTok{=}\NormalTok{np.zeros((n,}\DecValTok{2}\NormalTok{))}
\NormalTok{velocity}\OperatorTok{=}\NormalTok{np.zeros((n,}\DecValTok{2}\NormalTok{))}
\NormalTok{acceleration}\OperatorTok{=}\NormalTok{np.zeros((n,}\DecValTok{2}\NormalTok{))}

\ControlFlowTok{for}\NormalTok{ i }\KeywordTok{in} \BuiltInTok{range}\NormalTok{(}\DecValTok{1}\NormalTok{,n):}
\NormalTok{    position[i] }\OperatorTok{=}\NormalTok{ velocity[i}\OperatorTok{{-}}\DecValTok{1}\NormalTok{] }\OperatorTok{*}\NormalTok{ dt}
\NormalTok{    velocity[i] }\OperatorTok{=}\NormalTok{ acceleration[i}\OperatorTok{{-}}\DecValTok{1}\NormalTok{] }\OperatorTok{*}\NormalTok{ dt}
\NormalTok{    acceleration[i] }\OperatorTok{=}\NormalTok{ net\_force(time[i]) }\OperatorTok{/}\NormalTok{ mass}
\NormalTok{...}
\end{Highlighting}
\end{Shaded}

what I had to keep in mind, for example, is the distinction between
in-frame calculations and out-of-frame calculations. For example, a net
force calculation is in-frame: net force in a given frame directly
determines the acceleration of the same frame, but the velocity
measurement of a given frame doesn't determine the position measurement
of its own frame.

I also, obviously, learned quite a bit about basic planetary and rocket
physics! It's super cool! Working with the thrust equation, fuel engine
considerations, and so on are completely out of my comfort zone.

    \textbf{Reflection 4: Simulation phyiscs and investigation (B)}

\begin{quote}
(Highlight something you feel you did particularily well in terms of the
context of your simulation, the physical modelling that you did or the
investigation you performed. Discuss an aspect of these dimensions of
your project that would benefit the most from further effort.)
\end{quote}

    I feel as though I made my optimization section quite well! Deciding to
perform a multi-objective optimization (MOO) as opposed to optimizing
for a singular scalar quantity feels like it's the most natural way to
determine the true optimal scenario under this simulation. The
mathematics of this simulation are particularly new to me (as I'm only
just now learning basic multivariate calculus, and not even
multi-output). I'm particularly happy with how I wrote the weighting
system, as it allowed for bug-fixing on the fly, and this section may
benefit the most from further effort as implementing multiple potential
models for MOO could also allow me to see the strengths and weaknesses
for each model (as I only picked the simplest model I found rather than
anything with more nuance).

    \textbf{Reflection 5: Effectiveness of your communication}

\begin{quote}
(Highlight something you feel you did particularily well in your
visualizations or written communication. Discuss an aspect of your
visualizations or written communication that would benefit the most from
further effort.)
\end{quote}

    I feel as though my mathematical notation could be much, much better, as
right now, I'm worried that my lack of concrete knowledge of the math at
hand limits the applicability/correctness of this project, but
otherwise, I'm quite happy with how well I was able to communicate the
overall scope of the project (breaking down something this complicated
into smaller, simpler parts) in the introduction.

    \begin{tcolorbox}[breakable, size=fbox, boxrule=1pt, pad at break*=1mm,colback=cellbackground, colframe=cellborder]
\prompt{In}{incolor}{ }{\boxspacing}
\begin{Verbatim}[commandchars=\\\{\}]

\end{Verbatim}
\end{tcolorbox}


    % Add a bibliography block to the postdoc
    
    
    
\end{document}
